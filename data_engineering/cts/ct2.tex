\documentclass[12pt]{exam}
\usepackage[utf8]{inputenc}

\usepackage[margin=1in]{geometry}
\usepackage{amsmath,amssymb}
\usepackage{multicol}

\newcommand{\class}{CSE 817: Data Engineering}
\newcommand{\term}{Winter 2014}
\newcommand{\examnum}{CT 2}
\newcommand{\examdate}{06/03/2023}
\newcommand{\timelimit}{60 Minutes}

\pagestyle{head}
\firstpageheader{}{}{}
\runningheader{\class}{\hspace{1cm}\examnum\ - Page \thepage\ of \numpages}{\examdate}
\runningheadrule


\usepackage{tikz}
\usetikzlibrary{automata,positioning,shapes,arrows}

\usepackage{graphics,algorithmicx}
%\input{pdfinfo.tex}
 



\usepackage[english]{babel}
\usepackage{pgf,pgfarrows,pgfnodes,pgfautomata,pgfheaps}
\usepackage{amsmath,amssymb} 
\usepackage{listings}
%\usepackage{html,makeidx}
\usepackage{hyperref}

% Define block styles
\tikzstyle{decision} = [diamond, draw, fill=blue!20, text width=4.5em, text badly centered, node distance=3cm, inner sep=0pt]
\tikzstyle{block} = [rectangle, draw, fill=blue!20,  text width=5em, text badly centered, rounded corners, minimum height=2em]
\tikzstyle{block1} = [rectangle, draw, fill=blue!20, text badly centered, rounded corners,]


\tikzstyle{line} = [draw, -latex']
\tikzstyle{cloud} = [ellipse,fill=red!20, node distance=2cm, text width=5em, text badly centered, minimum height=2em]
\tikzstyle{info} = [ellipse, node distance=4cm, text width=5em, text badly centered, minimum height=2em]
\tikzstyle{prg} = [ ellipse, node distance=4cm, minimum height=2em]
\tikzstyle{prg2} = [ ellipse, node distance=2cm, minimum height=2em]
\tikzstyle{prg3} = [ ellipse, node distance=2cm,text width=5em, text badly centered,  minimum height=2em]
\tikzstyle{elp} = [draw,ellipse, node distance=2cm, minimum height=2em]


 

\usepackage{color}

\definecolor{mygreen}{rgb}{0,0.6,0}
\definecolor{mygray}{rgb}{0.5,0.5,0.5}
\definecolor{mymauve}{rgb}{0.58,0,0.82}

 


\begin{document}

\noindent
\begin{tabular*}{\textwidth}{l @{\extracolsep{\fill}} r @{\extracolsep{8pt}} l}
\textbf{\class} & \textbf{Name:} & \makebox[2in]{\hrulefill}\\
%\textbf{\term} &&\\
& & \\
\textbf{\examnum}  \textbf{Date}: \textbf{\examdate} &  \textbf{ID:} & \makebox[2in]{\hrulefill}\\
%\textbf{\examdate} &&\\
%\textbf{Time Limit: \timelimit} & Teaching Assistant & \makebox[2in]{\hrulefill}
\end{tabular*}\\

%\rule[2ex]{\textwidth}{2pt}

This exam contains \numpages\ pages and \numquestions\ questions.  Total of points is \numpoints.

% Rest of introduction. Rest of introduction. Rest of introduction. Rest of introduction. Rest of introduction. Rest of introduction. Rest of introduction. Rest of introduction. 


% \begin{center}
% Grade Table (for teacher use only)\\
% \addpoints
% \gradetable[v][questions]
% \end{center}

\noindent
\rule[2ex]{\textwidth}{2pt}

 
\begin{questions}
		\addpoints
 \question[5] Define \textbf{crosstab} and \textbf{roll-up }operatiosn in OLAP and how can they be used to analyze data across different dimensions?	\fillwithdottedlines{28em} 	 
 
\question[5]    Using the given Sales schema, cross tabulate the total sales amount by Product Category and Region. Use standard SQL for cross tabulation.
\textit{Sales(SaleID, SaleDate, ProductID, ProductCategory, RegionID, SalesAmount)}	\fillwithdottedlines{12em} 	  
	\addpoints
 
 
 	\question [5]  What are the key considerations for designing a scalable and efficient data lake architecture to support big data processing and analytics?  Introduce in one or two lines the different components of a typical data lake architecture.	\fillwithdottedlines{35em} 	  
 	\question[5]   Explain an use case of date lakes.	\fillwithdottedlines{13em} 	
 	
\end{questions}

\end{document}
