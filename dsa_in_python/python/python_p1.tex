\documentclass[11pt]{article}

    \usepackage[breakable]{tcolorbox}
    \usepackage{parskip} % Stop auto-indenting (to mimic markdown behaviour)
    
    \usepackage{iftex}
    \ifPDFTeX
    	\usepackage[T1]{fontenc}
    	\usepackage{mathpazo}
    \else
    	\usepackage{fontspec}
    \fi

    % Basic figure setup, for now with no caption control since it's done
    % automatically by Pandoc (which extracts ![](path) syntax from Markdown).
    \usepackage{graphicx}
    % Maintain compatibility with old templates. Remove in nbconvert 6.0
    \let\Oldincludegraphics\includegraphics
    % Ensure that by default, figures have no caption (until we provide a
    % proper Figure object with a Caption API and a way to capture that
    % in the conversion process - todo).
    \usepackage{caption}
    \DeclareCaptionFormat{nocaption}{}
    \captionsetup{format=nocaption,aboveskip=0pt,belowskip=0pt}

    \usepackage[Export]{adjustbox} % Used to constrain images to a maximum size
    \adjustboxset{max size={0.9\linewidth}{0.9\paperheight}}
    \usepackage{float}
    \floatplacement{figure}{H} % forces figures to be placed at the correct location
    \usepackage{xcolor} % Allow colors to be defined
    \usepackage{enumerate} % Needed for markdown enumerations to work
    \usepackage{geometry} % Used to adjust the document margins
    \usepackage{amsmath} % Equations
    \usepackage{amssymb} % Equations
    \usepackage{textcomp} % defines textquotesingle
    % Hack from http://tex.stackexchange.com/a/47451/13684:
    \AtBeginDocument{%
        \def\PYZsq{\textquotesingle}% Upright quotes in Pygmentized code
    }
    \usepackage{upquote} % Upright quotes for verbatim code
    \usepackage{eurosym} % defines \euro
    \usepackage[mathletters]{ucs} % Extended unicode (utf-8) support
    \usepackage{fancyvrb} % verbatim replacement that allows latex
    \usepackage{grffile} % extends the file name processing of package graphics 
                         % to support a larger range
    \makeatletter % fix for grffile with XeLaTeX
    \def\Gread@@xetex#1{%
      \IfFileExists{"\Gin@base".bb}%
      {\Gread@eps{\Gin@base.bb}}%
      {\Gread@@xetex@aux#1}%
    }
    \makeatother

    % The hyperref package gives us a pdf with properly built
    % internal navigation ('pdf bookmarks' for the table of contents,
    % internal cross-reference links, web links for URLs, etc.)
    \usepackage{hyperref}
    % The default LaTeX title has an obnoxious amount of whitespace. By default,
    % titling removes some of it. It also provides customization options.
    \usepackage{titling}
    \usepackage{longtable} % longtable support required by pandoc >1.10
    \usepackage{booktabs}  % table support for pandoc > 1.12.2
    \usepackage[inline]{enumitem} % IRkernel/repr support (it uses the enumerate* environment)
    \usepackage[normalem]{ulem} % ulem is needed to support strikethroughs (\sout)
                                % normalem makes italics be italics, not underlines
    \usepackage{mathrsfs}
    

    
    % Colors for the hyperref package
    \definecolor{urlcolor}{rgb}{0,.145,.698}
    \definecolor{linkcolor}{rgb}{.71,0.21,0.01}
    \definecolor{citecolor}{rgb}{.12,.54,.11}

    % ANSI colors
    \definecolor{ansi-black}{HTML}{3E424D}
    \definecolor{ansi-black-intense}{HTML}{282C36}
    \definecolor{ansi-red}{HTML}{E75C58}
    \definecolor{ansi-red-intense}{HTML}{B22B31}
    \definecolor{ansi-green}{HTML}{00A250}
    \definecolor{ansi-green-intense}{HTML}{007427}
    \definecolor{ansi-yellow}{HTML}{DDB62B}
    \definecolor{ansi-yellow-intense}{HTML}{B27D12}
    \definecolor{ansi-blue}{HTML}{208FFB}
    \definecolor{ansi-blue-intense}{HTML}{0065CA}
    \definecolor{ansi-magenta}{HTML}{D160C4}
    \definecolor{ansi-magenta-intense}{HTML}{A03196}
    \definecolor{ansi-cyan}{HTML}{60C6C8}
    \definecolor{ansi-cyan-intense}{HTML}{258F8F}
    \definecolor{ansi-white}{HTML}{C5C1B4}
    \definecolor{ansi-white-intense}{HTML}{A1A6B2}
    \definecolor{ansi-default-inverse-fg}{HTML}{FFFFFF}
    \definecolor{ansi-default-inverse-bg}{HTML}{000000}

    % commands and environments needed by pandoc snippets
    % extracted from the output of `pandoc -s`
    \providecommand{\tightlist}{%
      \setlength{\itemsep}{0pt}\setlength{\parskip}{0pt}}
    \DefineVerbatimEnvironment{Highlighting}{Verbatim}{commandchars=\\\{\}}
    % Add ',fontsize=\small' for more characters per line
    \newenvironment{Shaded}{}{}
    \newcommand{\KeywordTok}[1]{\textcolor[rgb]{0.00,0.44,0.13}{\textbf{{#1}}}}
    \newcommand{\DataTypeTok}[1]{\textcolor[rgb]{0.56,0.13,0.00}{{#1}}}
    \newcommand{\DecValTok}[1]{\textcolor[rgb]{0.25,0.63,0.44}{{#1}}}
    \newcommand{\BaseNTok}[1]{\textcolor[rgb]{0.25,0.63,0.44}{{#1}}}
    \newcommand{\FloatTok}[1]{\textcolor[rgb]{0.25,0.63,0.44}{{#1}}}
    \newcommand{\CharTok}[1]{\textcolor[rgb]{0.25,0.44,0.63}{{#1}}}
    \newcommand{\StringTok}[1]{\textcolor[rgb]{0.25,0.44,0.63}{{#1}}}
    \newcommand{\CommentTok}[1]{\textcolor[rgb]{0.38,0.63,0.69}{\textit{{#1}}}}
    \newcommand{\OtherTok}[1]{\textcolor[rgb]{0.00,0.44,0.13}{{#1}}}
    \newcommand{\AlertTok}[1]{\textcolor[rgb]{1.00,0.00,0.00}{\textbf{{#1}}}}
    \newcommand{\FunctionTok}[1]{\textcolor[rgb]{0.02,0.16,0.49}{{#1}}}
    \newcommand{\RegionMarkerTok}[1]{{#1}}
    \newcommand{\ErrorTok}[1]{\textcolor[rgb]{1.00,0.00,0.00}{\textbf{{#1}}}}
    \newcommand{\NormalTok}[1]{{#1}}
    
    % Additional commands for more recent versions of Pandoc
    \newcommand{\ConstantTok}[1]{\textcolor[rgb]{0.53,0.00,0.00}{{#1}}}
    \newcommand{\SpecialCharTok}[1]{\textcolor[rgb]{0.25,0.44,0.63}{{#1}}}
    \newcommand{\VerbatimStringTok}[1]{\textcolor[rgb]{0.25,0.44,0.63}{{#1}}}
    \newcommand{\SpecialStringTok}[1]{\textcolor[rgb]{0.73,0.40,0.53}{{#1}}}
    \newcommand{\ImportTok}[1]{{#1}}
    \newcommand{\DocumentationTok}[1]{\textcolor[rgb]{0.73,0.13,0.13}{\textit{{#1}}}}
    \newcommand{\AnnotationTok}[1]{\textcolor[rgb]{0.38,0.63,0.69}{\textbf{\textit{{#1}}}}}
    \newcommand{\CommentVarTok}[1]{\textcolor[rgb]{0.38,0.63,0.69}{\textbf{\textit{{#1}}}}}
    \newcommand{\VariableTok}[1]{\textcolor[rgb]{0.10,0.09,0.49}{{#1}}}
    \newcommand{\ControlFlowTok}[1]{\textcolor[rgb]{0.00,0.44,0.13}{\textbf{{#1}}}}
    \newcommand{\OperatorTok}[1]{\textcolor[rgb]{0.40,0.40,0.40}{{#1}}}
    \newcommand{\BuiltInTok}[1]{{#1}}
    \newcommand{\ExtensionTok}[1]{{#1}}
    \newcommand{\PreprocessorTok}[1]{\textcolor[rgb]{0.74,0.48,0.00}{{#1}}}
    \newcommand{\AttributeTok}[1]{\textcolor[rgb]{0.49,0.56,0.16}{{#1}}}
    \newcommand{\InformationTok}[1]{\textcolor[rgb]{0.38,0.63,0.69}{\textbf{\textit{{#1}}}}}
    \newcommand{\WarningTok}[1]{\textcolor[rgb]{0.38,0.63,0.69}{\textbf{\textit{{#1}}}}}
    
    
    % Define a nice break command that doesn't care if a line doesn't already
    % exist.
    \def\br{\hspace*{\fill} \\* }
    % Math Jax compatibility definitions
    \def\gt{>}
    \def\lt{<}
    \let\Oldtex\TeX
    \let\Oldlatex\LaTeX
    \renewcommand{\TeX}{\textrm{\Oldtex}}
    \renewcommand{\LaTeX}{\textrm{\Oldlatex}}
    % Document parameters
    % Document title
    \title{python\_p1}
    
    
    
    
    
% Pygments definitions
\makeatletter
\def\PY@reset{\let\PY@it=\relax \let\PY@bf=\relax%
    \let\PY@ul=\relax \let\PY@tc=\relax%
    \let\PY@bc=\relax \let\PY@ff=\relax}
\def\PY@tok#1{\csname PY@tok@#1\endcsname}
\def\PY@toks#1+{\ifx\relax#1\empty\else%
    \PY@tok{#1}\expandafter\PY@toks\fi}
\def\PY@do#1{\PY@bc{\PY@tc{\PY@ul{%
    \PY@it{\PY@bf{\PY@ff{#1}}}}}}}
\def\PY#1#2{\PY@reset\PY@toks#1+\relax+\PY@do{#2}}

\@namedef{PY@tok@w}{\def\PY@tc##1{\textcolor[rgb]{0.73,0.73,0.73}{##1}}}
\@namedef{PY@tok@c}{\let\PY@it=\textit\def\PY@tc##1{\textcolor[rgb]{0.25,0.50,0.50}{##1}}}
\@namedef{PY@tok@cp}{\def\PY@tc##1{\textcolor[rgb]{0.74,0.48,0.00}{##1}}}
\@namedef{PY@tok@k}{\let\PY@bf=\textbf\def\PY@tc##1{\textcolor[rgb]{0.00,0.50,0.00}{##1}}}
\@namedef{PY@tok@kp}{\def\PY@tc##1{\textcolor[rgb]{0.00,0.50,0.00}{##1}}}
\@namedef{PY@tok@kt}{\def\PY@tc##1{\textcolor[rgb]{0.69,0.00,0.25}{##1}}}
\@namedef{PY@tok@o}{\def\PY@tc##1{\textcolor[rgb]{0.40,0.40,0.40}{##1}}}
\@namedef{PY@tok@ow}{\let\PY@bf=\textbf\def\PY@tc##1{\textcolor[rgb]{0.67,0.13,1.00}{##1}}}
\@namedef{PY@tok@nb}{\def\PY@tc##1{\textcolor[rgb]{0.00,0.50,0.00}{##1}}}
\@namedef{PY@tok@nf}{\def\PY@tc##1{\textcolor[rgb]{0.00,0.00,1.00}{##1}}}
\@namedef{PY@tok@nc}{\let\PY@bf=\textbf\def\PY@tc##1{\textcolor[rgb]{0.00,0.00,1.00}{##1}}}
\@namedef{PY@tok@nn}{\let\PY@bf=\textbf\def\PY@tc##1{\textcolor[rgb]{0.00,0.00,1.00}{##1}}}
\@namedef{PY@tok@ne}{\let\PY@bf=\textbf\def\PY@tc##1{\textcolor[rgb]{0.82,0.25,0.23}{##1}}}
\@namedef{PY@tok@nv}{\def\PY@tc##1{\textcolor[rgb]{0.10,0.09,0.49}{##1}}}
\@namedef{PY@tok@no}{\def\PY@tc##1{\textcolor[rgb]{0.53,0.00,0.00}{##1}}}
\@namedef{PY@tok@nl}{\def\PY@tc##1{\textcolor[rgb]{0.63,0.63,0.00}{##1}}}
\@namedef{PY@tok@ni}{\let\PY@bf=\textbf\def\PY@tc##1{\textcolor[rgb]{0.60,0.60,0.60}{##1}}}
\@namedef{PY@tok@na}{\def\PY@tc##1{\textcolor[rgb]{0.49,0.56,0.16}{##1}}}
\@namedef{PY@tok@nt}{\let\PY@bf=\textbf\def\PY@tc##1{\textcolor[rgb]{0.00,0.50,0.00}{##1}}}
\@namedef{PY@tok@nd}{\def\PY@tc##1{\textcolor[rgb]{0.67,0.13,1.00}{##1}}}
\@namedef{PY@tok@s}{\def\PY@tc##1{\textcolor[rgb]{0.73,0.13,0.13}{##1}}}
\@namedef{PY@tok@sd}{\let\PY@it=\textit\def\PY@tc##1{\textcolor[rgb]{0.73,0.13,0.13}{##1}}}
\@namedef{PY@tok@si}{\let\PY@bf=\textbf\def\PY@tc##1{\textcolor[rgb]{0.73,0.40,0.53}{##1}}}
\@namedef{PY@tok@se}{\let\PY@bf=\textbf\def\PY@tc##1{\textcolor[rgb]{0.73,0.40,0.13}{##1}}}
\@namedef{PY@tok@sr}{\def\PY@tc##1{\textcolor[rgb]{0.73,0.40,0.53}{##1}}}
\@namedef{PY@tok@ss}{\def\PY@tc##1{\textcolor[rgb]{0.10,0.09,0.49}{##1}}}
\@namedef{PY@tok@sx}{\def\PY@tc##1{\textcolor[rgb]{0.00,0.50,0.00}{##1}}}
\@namedef{PY@tok@m}{\def\PY@tc##1{\textcolor[rgb]{0.40,0.40,0.40}{##1}}}
\@namedef{PY@tok@gh}{\let\PY@bf=\textbf\def\PY@tc##1{\textcolor[rgb]{0.00,0.00,0.50}{##1}}}
\@namedef{PY@tok@gu}{\let\PY@bf=\textbf\def\PY@tc##1{\textcolor[rgb]{0.50,0.00,0.50}{##1}}}
\@namedef{PY@tok@gd}{\def\PY@tc##1{\textcolor[rgb]{0.63,0.00,0.00}{##1}}}
\@namedef{PY@tok@gi}{\def\PY@tc##1{\textcolor[rgb]{0.00,0.63,0.00}{##1}}}
\@namedef{PY@tok@gr}{\def\PY@tc##1{\textcolor[rgb]{1.00,0.00,0.00}{##1}}}
\@namedef{PY@tok@ge}{\let\PY@it=\textit}
\@namedef{PY@tok@gs}{\let\PY@bf=\textbf}
\@namedef{PY@tok@gp}{\let\PY@bf=\textbf\def\PY@tc##1{\textcolor[rgb]{0.00,0.00,0.50}{##1}}}
\@namedef{PY@tok@go}{\def\PY@tc##1{\textcolor[rgb]{0.53,0.53,0.53}{##1}}}
\@namedef{PY@tok@gt}{\def\PY@tc##1{\textcolor[rgb]{0.00,0.27,0.87}{##1}}}
\@namedef{PY@tok@err}{\def\PY@bc##1{{\setlength{\fboxsep}{\string -\fboxrule}\fcolorbox[rgb]{1.00,0.00,0.00}{1,1,1}{\strut ##1}}}}
\@namedef{PY@tok@kc}{\let\PY@bf=\textbf\def\PY@tc##1{\textcolor[rgb]{0.00,0.50,0.00}{##1}}}
\@namedef{PY@tok@kd}{\let\PY@bf=\textbf\def\PY@tc##1{\textcolor[rgb]{0.00,0.50,0.00}{##1}}}
\@namedef{PY@tok@kn}{\let\PY@bf=\textbf\def\PY@tc##1{\textcolor[rgb]{0.00,0.50,0.00}{##1}}}
\@namedef{PY@tok@kr}{\let\PY@bf=\textbf\def\PY@tc##1{\textcolor[rgb]{0.00,0.50,0.00}{##1}}}
\@namedef{PY@tok@bp}{\def\PY@tc##1{\textcolor[rgb]{0.00,0.50,0.00}{##1}}}
\@namedef{PY@tok@fm}{\def\PY@tc##1{\textcolor[rgb]{0.00,0.00,1.00}{##1}}}
\@namedef{PY@tok@vc}{\def\PY@tc##1{\textcolor[rgb]{0.10,0.09,0.49}{##1}}}
\@namedef{PY@tok@vg}{\def\PY@tc##1{\textcolor[rgb]{0.10,0.09,0.49}{##1}}}
\@namedef{PY@tok@vi}{\def\PY@tc##1{\textcolor[rgb]{0.10,0.09,0.49}{##1}}}
\@namedef{PY@tok@vm}{\def\PY@tc##1{\textcolor[rgb]{0.10,0.09,0.49}{##1}}}
\@namedef{PY@tok@sa}{\def\PY@tc##1{\textcolor[rgb]{0.73,0.13,0.13}{##1}}}
\@namedef{PY@tok@sb}{\def\PY@tc##1{\textcolor[rgb]{0.73,0.13,0.13}{##1}}}
\@namedef{PY@tok@sc}{\def\PY@tc##1{\textcolor[rgb]{0.73,0.13,0.13}{##1}}}
\@namedef{PY@tok@dl}{\def\PY@tc##1{\textcolor[rgb]{0.73,0.13,0.13}{##1}}}
\@namedef{PY@tok@s2}{\def\PY@tc##1{\textcolor[rgb]{0.73,0.13,0.13}{##1}}}
\@namedef{PY@tok@sh}{\def\PY@tc##1{\textcolor[rgb]{0.73,0.13,0.13}{##1}}}
\@namedef{PY@tok@s1}{\def\PY@tc##1{\textcolor[rgb]{0.73,0.13,0.13}{##1}}}
\@namedef{PY@tok@mb}{\def\PY@tc##1{\textcolor[rgb]{0.40,0.40,0.40}{##1}}}
\@namedef{PY@tok@mf}{\def\PY@tc##1{\textcolor[rgb]{0.40,0.40,0.40}{##1}}}
\@namedef{PY@tok@mh}{\def\PY@tc##1{\textcolor[rgb]{0.40,0.40,0.40}{##1}}}
\@namedef{PY@tok@mi}{\def\PY@tc##1{\textcolor[rgb]{0.40,0.40,0.40}{##1}}}
\@namedef{PY@tok@il}{\def\PY@tc##1{\textcolor[rgb]{0.40,0.40,0.40}{##1}}}
\@namedef{PY@tok@mo}{\def\PY@tc##1{\textcolor[rgb]{0.40,0.40,0.40}{##1}}}
\@namedef{PY@tok@ch}{\let\PY@it=\textit\def\PY@tc##1{\textcolor[rgb]{0.25,0.50,0.50}{##1}}}
\@namedef{PY@tok@cm}{\let\PY@it=\textit\def\PY@tc##1{\textcolor[rgb]{0.25,0.50,0.50}{##1}}}
\@namedef{PY@tok@cpf}{\let\PY@it=\textit\def\PY@tc##1{\textcolor[rgb]{0.25,0.50,0.50}{##1}}}
\@namedef{PY@tok@c1}{\let\PY@it=\textit\def\PY@tc##1{\textcolor[rgb]{0.25,0.50,0.50}{##1}}}
\@namedef{PY@tok@cs}{\let\PY@it=\textit\def\PY@tc##1{\textcolor[rgb]{0.25,0.50,0.50}{##1}}}

\def\PYZbs{\char`\\}
\def\PYZus{\char`\_}
\def\PYZob{\char`\{}
\def\PYZcb{\char`\}}
\def\PYZca{\char`\^}
\def\PYZam{\char`\&}
\def\PYZlt{\char`\<}
\def\PYZgt{\char`\>}
\def\PYZsh{\char`\#}
\def\PYZpc{\char`\%}
\def\PYZdl{\char`\$}
\def\PYZhy{\char`\-}
\def\PYZsq{\char`\'}
\def\PYZdq{\char`\"}
\def\PYZti{\char`\~}
% for compatibility with earlier versions
\def\PYZat{@}
\def\PYZlb{[}
\def\PYZrb{]}
\makeatother


    % For linebreaks inside Verbatim environment from package fancyvrb. 
    \makeatletter
        \newbox\Wrappedcontinuationbox 
        \newbox\Wrappedvisiblespacebox 
        \newcommand*\Wrappedvisiblespace {\textcolor{red}{\textvisiblespace}} 
        \newcommand*\Wrappedcontinuationsymbol {\textcolor{red}{\llap{\tiny$\m@th\hookrightarrow$}}} 
        \newcommand*\Wrappedcontinuationindent {3ex } 
        \newcommand*\Wrappedafterbreak {\kern\Wrappedcontinuationindent\copy\Wrappedcontinuationbox} 
        % Take advantage of the already applied Pygments mark-up to insert 
        % potential linebreaks for TeX processing. 
        %        {, <, #, %, $, ' and ": go to next line. 
        %        _, }, ^, &, >, - and ~: stay at end of broken line. 
        % Use of \textquotesingle for straight quote. 
        \newcommand*\Wrappedbreaksatspecials {% 
            \def\PYGZus{\discretionary{\char`\_}{\Wrappedafterbreak}{\char`\_}}% 
            \def\PYGZob{\discretionary{}{\Wrappedafterbreak\char`\{}{\char`\{}}% 
            \def\PYGZcb{\discretionary{\char`\}}{\Wrappedafterbreak}{\char`\}}}% 
            \def\PYGZca{\discretionary{\char`\^}{\Wrappedafterbreak}{\char`\^}}% 
            \def\PYGZam{\discretionary{\char`\&}{\Wrappedafterbreak}{\char`\&}}% 
            \def\PYGZlt{\discretionary{}{\Wrappedafterbreak\char`\<}{\char`\<}}% 
            \def\PYGZgt{\discretionary{\char`\>}{\Wrappedafterbreak}{\char`\>}}% 
            \def\PYGZsh{\discretionary{}{\Wrappedafterbreak\char`\#}{\char`\#}}% 
            \def\PYGZpc{\discretionary{}{\Wrappedafterbreak\char`\%}{\char`\%}}% 
            \def\PYGZdl{\discretionary{}{\Wrappedafterbreak\char`\$}{\char`\$}}% 
            \def\PYGZhy{\discretionary{\char`\-}{\Wrappedafterbreak}{\char`\-}}% 
            \def\PYGZsq{\discretionary{}{\Wrappedafterbreak\textquotesingle}{\textquotesingle}}% 
            \def\PYGZdq{\discretionary{}{\Wrappedafterbreak\char`\"}{\char`\"}}% 
            \def\PYGZti{\discretionary{\char`\~}{\Wrappedafterbreak}{\char`\~}}% 
        } 
        % Some characters . , ; ? ! / are not pygmentized. 
        % This macro makes them "active" and they will insert potential linebreaks 
        \newcommand*\Wrappedbreaksatpunct {% 
            \lccode`\~`\.\lowercase{\def~}{\discretionary{\hbox{\char`\.}}{\Wrappedafterbreak}{\hbox{\char`\.}}}% 
            \lccode`\~`\,\lowercase{\def~}{\discretionary{\hbox{\char`\,}}{\Wrappedafterbreak}{\hbox{\char`\,}}}% 
            \lccode`\~`\;\lowercase{\def~}{\discretionary{\hbox{\char`\;}}{\Wrappedafterbreak}{\hbox{\char`\;}}}% 
            \lccode`\~`\:\lowercase{\def~}{\discretionary{\hbox{\char`\:}}{\Wrappedafterbreak}{\hbox{\char`\:}}}% 
            \lccode`\~`\?\lowercase{\def~}{\discretionary{\hbox{\char`\?}}{\Wrappedafterbreak}{\hbox{\char`\?}}}% 
            \lccode`\~`\!\lowercase{\def~}{\discretionary{\hbox{\char`\!}}{\Wrappedafterbreak}{\hbox{\char`\!}}}% 
            \lccode`\~`\/\lowercase{\def~}{\discretionary{\hbox{\char`\/}}{\Wrappedafterbreak}{\hbox{\char`\/}}}% 
            \catcode`\.\active
            \catcode`\,\active 
            \catcode`\;\active
            \catcode`\:\active
            \catcode`\?\active
            \catcode`\!\active
            \catcode`\/\active 
            \lccode`\~`\~ 	
        }
    \makeatother

    \let\OriginalVerbatim=\Verbatim
    \makeatletter
    \renewcommand{\Verbatim}[1][1]{%
        %\parskip\z@skip
        \sbox\Wrappedcontinuationbox {\Wrappedcontinuationsymbol}%
        \sbox\Wrappedvisiblespacebox {\FV@SetupFont\Wrappedvisiblespace}%
        \def\FancyVerbFormatLine ##1{\hsize\linewidth
            \vtop{\raggedright\hyphenpenalty\z@\exhyphenpenalty\z@
                \doublehyphendemerits\z@\finalhyphendemerits\z@
                \strut ##1\strut}%
        }%
        % If the linebreak is at a space, the latter will be displayed as visible
        % space at end of first line, and a continuation symbol starts next line.
        % Stretch/shrink are however usually zero for typewriter font.
        \def\FV@Space {%
            \nobreak\hskip\z@ plus\fontdimen3\font minus\fontdimen4\font
            \discretionary{\copy\Wrappedvisiblespacebox}{\Wrappedafterbreak}
            {\kern\fontdimen2\font}%
        }%
        
        % Allow breaks at special characters using \PYG... macros.
        \Wrappedbreaksatspecials
        % Breaks at punctuation characters . , ; ? ! and / need catcode=\active 	
        \OriginalVerbatim[#1,codes*=\Wrappedbreaksatpunct]%
    }
    \makeatother

    % Exact colors from NB
    \definecolor{incolor}{HTML}{303F9F}
    \definecolor{outcolor}{HTML}{D84315}
    \definecolor{cellborder}{HTML}{CFCFCF}
    \definecolor{cellbackground}{HTML}{F7F7F7}
    
    % prompt
    \makeatletter
    \newcommand{\boxspacing}{\kern\kvtcb@left@rule\kern\kvtcb@boxsep}
    \makeatother
    \newcommand{\prompt}[4]{
        \ttfamily\llap{{\color{#2}[#3]:\hspace{3pt}#4}}\vspace{-\baselineskip}
    }
    

    
    % Prevent overflowing lines due to hard-to-break entities
    \sloppy 
    % Setup hyperref package
    \hypersetup{
      breaklinks=true,  % so long urls are correctly broken across lines
      colorlinks=true,
      urlcolor=urlcolor,
      linkcolor=linkcolor,
      citecolor=citecolor,
      }
    % Slightly bigger margins than the latex defaults
    
    \geometry{verbose,tmargin=1in,bmargin=1in,lmargin=1in,rmargin=1in}
    
    

\begin{document}
    
    \maketitle
    
    

    
    \includegraphics{figures/python-logo.png}

\hypertarget{python-features}{%
\subsection{Python features}\label{python-features}}

\begin{itemize}
\tightlist
\item
  Easy to use and learn
\item
  Readability
\item
  Interpreted language
\item
  Portability
\end{itemize}

    \hypertarget{python2python3}{%
\subsection{Python2/Python3}\label{python2python3}}

\begin{itemize}
\tightlist
\item
  Two main branches:
\item
  python2: it is the legacy, surely maintained, but will not undergo
  updates / evolution
\item
  python3: it is the future of python, still less used compared to
  python2.
\item
  We aim to use python3
\end{itemize}

    \hypertarget{what-is-jupyter-formerly-ipython-notebook}{%
\subsection{What is jupyter (formerly ipython)
notebook?}\label{what-is-jupyter-formerly-ipython-notebook}}

\begin{itemize}
\tightlist
\item
  this document is a jupyter notebook
\item
  flexible tool to create readable documents that embed:
\item
  code
\item
  images
\item
  comments
\item
  formulas
\item
  plots
\item
  jupyter supports notebooks form many different languages (including
  python, julia and R)
\item
  each document runs a computational engine that executes the code
  written in the cells
\end{itemize}

    \hypertarget{python-interpreter}{%
\subsection{Python interpreter}\label{python-interpreter}}

\begin{itemize}
\tightlist
\item
  Compile source to bytecode then execute it on a virtual machine
\item
  Survives any execution error, even in case of syntax errors
\item
  in python also indentation is syntax
\item
  expression is code that returns an object
\item
  if no error, the prompt (``\textgreater\textgreater\textgreater{}'')
  automatically:
\item
  prints the result on screen
\item
  assigns the result to ``\_''
\end{itemize}

    \begin{tcolorbox}[breakable, size=fbox, boxrule=1pt, pad at break*=1mm,colback=cellbackground, colframe=cellborder]
\prompt{In}{incolor}{1}{\boxspacing}
\begin{Verbatim}[commandchars=\\\{\}]
\PY{c+c1}{\PYZsh{} expression that returns something}
\PY{l+m+mi}{12345}
\end{Verbatim}
\end{tcolorbox}

            \begin{tcolorbox}[breakable, size=fbox, boxrule=.5pt, pad at break*=1mm, opacityfill=0]
\prompt{Out}{outcolor}{1}{\boxspacing}
\begin{Verbatim}[commandchars=\\\{\}]
12345
\end{Verbatim}
\end{tcolorbox}
        
    \begin{tcolorbox}[breakable, size=fbox, boxrule=1pt, pad at break*=1mm,colback=cellbackground, colframe=cellborder]
\prompt{In}{incolor}{2}{\boxspacing}
\begin{Verbatim}[commandchars=\\\{\}]
\PY{c+c1}{\PYZsh{} that something was assigned to \PYZdq{}\PYZus{}\PYZdq{}}
\PY{n}{\PYZus{}}
\end{Verbatim}
\end{tcolorbox}

            \begin{tcolorbox}[breakable, size=fbox, boxrule=.5pt, pad at break*=1mm, opacityfill=0]
\prompt{Out}{outcolor}{2}{\boxspacing}
\begin{Verbatim}[commandchars=\\\{\}]
12345
\end{Verbatim}
\end{tcolorbox}
        
    \begin{tcolorbox}[breakable, size=fbox, boxrule=1pt, pad at break*=1mm,colback=cellbackground, colframe=cellborder]
\prompt{In}{incolor}{3}{\boxspacing}
\begin{Verbatim}[commandchars=\\\{\}]
\PY{c+c1}{\PYZsh{} Error example: division by 0}
\PY{l+m+mi}{1}\PY{o}{/}\PY{l+m+mi}{0}
\end{Verbatim}
\end{tcolorbox}

    \begin{Verbatim}[commandchars=\\\{\}]

        ---------------------------------------------------------------------------

        ZeroDivisionError                         Traceback (most recent call last)

        /var/folders/1\_/2rbdt5292zdb57b821k90d5r0000gn/T/ipykernel\_56789/1872088744.py in <module>
          1 \# Error example: division by 0
    ----> 2 1/0
    

        ZeroDivisionError: division by zero

    \end{Verbatim}

    \hypertarget{print}{%
\subsection{print()}\label{print}}

\begin{itemize}
\tightlist
\item
  prints arguments on standard output
\item
  can receive an arbitrary number of arguments
\item
  ends with ``\n'' unless differently specified with ``end'' key
\item
  --
\end{itemize}

\hypertarget{type}{%
\subsection{type()}\label{type}}

\begin{itemize}
\tightlist
\item
  returns the type of the argument
\item
  --
\end{itemize}

\hypertarget{help}{%
\subsection{help()}\label{help}}

\begin{itemize}
\tightlist
\item
  The Python interpreter comes with a help() function
\item
  call ``help(something)'' to learn about something (if something is
  defined
\item
  type ``help()'' to start the help prompt (``help\textgreater{}''), to
  exit execute quit
\end{itemize}

    \begin{tcolorbox}[breakable, size=fbox, boxrule=1pt, pad at break*=1mm,colback=cellbackground, colframe=cellborder]
\prompt{In}{incolor}{4}{\boxspacing}
\begin{Verbatim}[commandchars=\\\{\}]
\PY{n+nb}{print}\PY{p}{(}\PY{l+s+s2}{\PYZdq{}}\PY{l+s+s2}{print}\PY{l+s+s2}{\PYZdq{}}\PY{p}{)} 
\PY{n+nb}{print}\PY{p}{(}\PY{l+s+s2}{\PYZdq{}}\PY{l+s+s2}{ends}\PY{l+s+s2}{\PYZdq{}}\PY{p}{)}
\PY{n+nb}{print}\PY{p}{(}\PY{l+s+s2}{\PYZdq{}}\PY{l+s+s2}{with}\PY{l+s+s2}{\PYZdq{}}\PY{p}{)}
\PY{n+nb}{print}\PY{p}{(}\PY{l+s+s2}{\PYZdq{}}\PY{l+s+s2}{newline}\PY{l+s+s2}{\PYZdq{}}\PY{p}{)} 

\PY{n+nb}{print}\PY{p}{(}\PY{l+s+s2}{\PYZdq{}}\PY{l+s+se}{\PYZbs{}n}\PY{l+s+s2}{unless}\PY{l+s+s2}{\PYZdq{}}\PY{p}{,} \PY{n}{end}\PY{o}{=}\PY{l+s+s2}{\PYZdq{}}\PY{l+s+s2}{ }\PY{l+s+s2}{\PYZdq{}}\PY{p}{)}   \PY{c+c1}{\PYZsh{} ends with a space instead of newline}
\PY{n+nb}{print}\PY{p}{(}\PY{l+s+s2}{\PYZdq{}}\PY{l+s+s2}{differently specified}\PY{l+s+s2}{\PYZdq{}}\PY{p}{)}

\PY{n+nb}{print}\PY{p}{(}\PY{l+s+s2}{\PYZdq{}}\PY{l+s+se}{\PYZbs{}n}\PY{l+s+s2}{print}\PY{l+s+s2}{\PYZdq{}}\PY{p}{,}\PY{l+s+s2}{\PYZdq{}}\PY{l+s+s2}{can}\PY{l+s+s2}{\PYZdq{}}\PY{p}{,}\PY{l+s+s2}{\PYZdq{}}\PY{l+s+s2}{accept}\PY{l+s+s2}{\PYZdq{}}\PY{p}{,}\PY{l+s+s2}{\PYZdq{}}\PY{l+s+s2}{multiple}\PY{l+s+s2}{\PYZdq{}}\PY{p}{,}\PY{l+s+s2}{\PYZdq{}}\PY{l+s+s2}{arguments}\PY{l+s+s2}{\PYZdq{}}\PY{p}{)}   \PY{c+c1}{\PYZsh{} sep with a separator instead of a space }
\PY{n+nb}{print}\PY{p}{(}\PY{l+s+s2}{\PYZdq{}}\PY{l+s+se}{\PYZbs{}n}\PY{l+s+s2}{print}\PY{l+s+s2}{\PYZdq{}}\PY{p}{,}\PY{l+s+s2}{\PYZdq{}}\PY{l+s+s2}{can}\PY{l+s+s2}{\PYZdq{}}\PY{p}{,}\PY{l+s+s2}{\PYZdq{}}\PY{l+s+s2}{accept}\PY{l+s+s2}{\PYZdq{}}\PY{p}{,}\PY{l+s+s2}{\PYZdq{}}\PY{l+s+s2}{multiple}\PY{l+s+s2}{\PYZdq{}}\PY{p}{,}\PY{l+s+s2}{\PYZdq{}}\PY{l+s+s2}{arguments}\PY{l+s+s2}{\PYZdq{}}\PY{p}{,} \PY{n}{sep}\PY{o}{=}\PY{l+s+s2}{\PYZdq{}}\PY{l+s+s2}{\PYZbs{}}\PY{l+s+s2}{\PYZus{}/}\PY{l+s+s2}{\PYZdq{}}\PY{p}{)}  
\end{Verbatim}
\end{tcolorbox}

    \begin{Verbatim}[commandchars=\\\{\}]
print
ends
with
newline

unless differently specified

print can accept multiple arguments

print\textbackslash{}\_/can\textbackslash{}\_/accept\textbackslash{}\_/multiple\textbackslash{}\_/arguments
    \end{Verbatim}

    \begin{tcolorbox}[breakable, size=fbox, boxrule=1pt, pad at break*=1mm,colback=cellbackground, colframe=cellborder]
\prompt{In}{incolor}{5}{\boxspacing}
\begin{Verbatim}[commandchars=\\\{\}]
\PY{n+nb}{type}\PY{p}{(}\PY{n+nb}{print}\PY{p}{)}
\end{Verbatim}
\end{tcolorbox}

            \begin{tcolorbox}[breakable, size=fbox, boxrule=.5pt, pad at break*=1mm, opacityfill=0]
\prompt{Out}{outcolor}{5}{\boxspacing}
\begin{Verbatim}[commandchars=\\\{\}]
builtin\_function\_or\_method
\end{Verbatim}
\end{tcolorbox}
        
    \begin{tcolorbox}[breakable, size=fbox, boxrule=1pt, pad at break*=1mm,colback=cellbackground, colframe=cellborder]
\prompt{In}{incolor}{6}{\boxspacing}
\begin{Verbatim}[commandchars=\\\{\}]
\PY{n}{help}\PY{p}{(}\PY{n+nb}{print}\PY{p}{)}
\end{Verbatim}
\end{tcolorbox}

    \begin{Verbatim}[commandchars=\\\{\}]
Help on built-in function print in module builtins:

print({\ldots})
    print(value, {\ldots}, sep=' ', end='\textbackslash{}n', file=sys.stdout, flush=False)

    Prints the values to a stream, or to sys.stdout by default.
    Optional keyword arguments:
    file:  a file-like object (stream); defaults to the current sys.stdout.
    sep:   string inserted between values, default a space.
    end:   string appended after the last value, default a newline.
    flush: whether to forcibly flush the stream.

    \end{Verbatim}

    \hypertarget{variables-name}{%
\subsection{Variables' name}\label{variables-name}}

\begin{itemize}
\tightlist
\item
  Cannot start with a digit
\item
  nor contain hyphens (minus signs)
\item
  nor periods (dot signs)
\item
  nor blank spaces
\item
  descriptive names should be preferred
\item
  short names should be preferred
\end{itemize}

    \hypertarget{python3-keywords-list}{%
\subsection{Python3 keywords list}\label{python3-keywords-list}}

\begin{itemize}
\tightlist
\item
  keywords are reserved words that already means something
\item
  reserved means these words cannot be Identifiers
\end{itemize}

\begin{longtable}[]{@{}llllllll@{}}
\toprule
\endhead
False & None & True & and & as & assert & break & \\
class & continue & def & del & elif & else & except & \\
finally & for & from & global & if & import & in & \\
is & lambda & nonlocal & not & or & pass & raise & \\
return & try & while & with & yield & & & \\
\bottomrule
\end{longtable}

    \hypertarget{assignment-operator}{%
\subsection{Assignment operator}\label{assignment-operator}}

\begin{itemize}
\tightlist
\item
  the equal sign is the assignment operator, which means\ldots{}
\item
  \ldots{}``evaluate the operand/expression on the right side of the
  equal sign and assign that to the operand on the left (the
  `lvalue').''
\item
  python provides cascaded assignments\\
\item
  var1 = var2 = var3 = value
\item
  python provides augmented assignment operators
\item
  standard math operations: +=,-=,*=,/=
\item
  modulo,remainder,exponentiation: \%=,//=,**=
\item
  bitwise math operations:
  \&=,\textbar=,\^{}=,\textgreater\textgreater=,\textless\textless=
\end{itemize}

    \begin{tcolorbox}[breakable, size=fbox, boxrule=1pt, pad at break*=1mm,colback=cellbackground, colframe=cellborder]
\prompt{In}{incolor}{7}{\boxspacing}
\begin{Verbatim}[commandchars=\\\{\}]
\PY{c+c1}{\PYZsh{} cascade assignments}
\PY{n}{x} \PY{o}{=} \PY{n}{y} \PY{o}{=} \PY{n}{z} \PY{o}{=} \PY{l+m+mi}{10}  
\PY{c+c1}{\PYZsh{} augmented assignments}
\PY{n}{x}\PY{o}{/}\PY{o}{=}\PY{l+m+mi}{2}
\PY{n}{y}\PY{o}{+}\PY{o}{=}\PY{l+m+mi}{1}
\PY{n+nb}{print}\PY{p}{(}\PY{l+s+s2}{\PYZdq{}}\PY{l+s+s2}{x =}\PY{l+s+s2}{\PYZdq{}}\PY{p}{,}\PY{n}{x}\PY{p}{,}\PY{n+nb}{type}\PY{p}{(}\PY{n}{x}\PY{p}{)}\PY{p}{)}
\PY{n+nb}{print}\PY{p}{(}\PY{l+s+s2}{\PYZdq{}}\PY{l+s+s2}{y =}\PY{l+s+s2}{\PYZdq{}}\PY{p}{,}\PY{n}{y}\PY{p}{,}\PY{n+nb}{type}\PY{p}{(}\PY{n}{x}\PY{p}{)}\PY{p}{)}
\PY{n+nb}{print}\PY{p}{(}\PY{l+s+s2}{\PYZdq{}}\PY{l+s+s2}{z =}\PY{l+s+s2}{\PYZdq{}}\PY{p}{,}\PY{n}{z}\PY{p}{,}\PY{n+nb}{type}\PY{p}{(}\PY{n}{x}\PY{p}{)}\PY{p}{)}
\end{Verbatim}
\end{tcolorbox}

    \begin{Verbatim}[commandchars=\\\{\}]
x = 5.0 <class 'float'>
y = 11 <class 'float'>
z = 10 <class 'float'>
    \end{Verbatim}

    \hypertarget{list-of-python-base-literals-value-of-a-specific-type}{%
\subsection{List of python (base) literals (value of a specific
type)}\label{list-of-python-base-literals-value-of-a-specific-type}}

\begin{itemize}
\tightlist
\item
  strings (collection of characters enclosed in quotes ``\,'' or '\,')
\item
  '\,'\,' or ``\,``\,'' starts (ends) a multiline string (triple-quoted
  string)
\item
  integer numeric literals
\item
  float numeric literals
\item
  complex numeric literals (`j'/`J' yields an imaginary number)
\item
  boolean literals which are ``True'' or ``False'' (both keywords)
\item
  None
\end{itemize}

    \begin{tcolorbox}[breakable, size=fbox, boxrule=1pt, pad at break*=1mm,colback=cellbackground, colframe=cellborder]
\prompt{In}{incolor}{8}{\boxspacing}
\begin{Verbatim}[commandchars=\\\{\}]
\PY{n+nb}{print}\PY{p}{(}\PY{l+s+s2}{\PYZdq{}}\PY{l+s+se}{\PYZbs{}\PYZdq{}}\PY{l+s+s2}{A}\PY{l+s+se}{\PYZbs{}\PYZdq{}}\PY{l+s+s2}{\PYZdq{}}\PY{p}{,} \PY{l+s+s2}{\PYZdq{}}\PY{l+s+s2}{is a}\PY{l+s+s2}{\PYZdq{}}\PY{p}{,} \PY{n+nb}{type}\PY{p}{(}\PY{l+s+s2}{\PYZdq{}}\PY{l+s+s2}{A}\PY{l+s+s2}{\PYZdq{}}\PY{p}{)}\PY{p}{)}
\PY{n+nb}{print}\PY{p}{(}\PY{l+s+s2}{\PYZdq{}}\PY{l+s+se}{\PYZbs{}\PYZsq{}}\PY{l+s+s2}{A}\PY{l+s+se}{\PYZbs{}\PYZsq{}}\PY{l+s+s2}{\PYZdq{}}\PY{p}{,} \PY{l+s+s2}{\PYZdq{}}\PY{l+s+s2}{is also a}\PY{l+s+s2}{\PYZdq{}}\PY{p}{,} \PY{n+nb}{type}\PY{p}{(}\PY{l+s+s2}{\PYZdq{}}\PY{l+s+s2}{A}\PY{l+s+s2}{\PYZdq{}}\PY{p}{)}\PY{p}{)}
\PY{n+nb}{print}\PY{p}{(}\PY{l+s+s2}{\PYZdq{}}\PY{l+s+s2}{in other words, }\PY{l+s+se}{\PYZbs{}\PYZdq{}}\PY{l+s+s2}{A}\PY{l+s+se}{\PYZbs{}\PYZdq{}}\PY{l+s+s2}{ is }\PY{l+s+s2}{\PYZsq{}}\PY{l+s+s2}{A}\PY{l+s+s2}{\PYZsq{}}\PY{l+s+s2}{ \PYZhy{} \PYZgt{}}\PY{l+s+s2}{\PYZdq{}}\PY{p}{,}\PY{l+s+s2}{\PYZdq{}}\PY{l+s+s2}{A}\PY{l+s+s2}{\PYZdq{}} \PY{o+ow}{is} \PY{l+s+s1}{\PYZsq{}}\PY{l+s+s1}{A}\PY{l+s+s1}{\PYZsq{}}\PY{p}{)}
\PY{n+nb}{print}\PY{p}{(}\PY{l+s+s2}{\PYZdq{}}\PY{l+s+se}{\PYZbs{}\PYZdq{}}\PY{l+s+se}{\PYZbs{}\PYZdq{}}\PY{l+s+se}{\PYZbs{}\PYZdq{}}\PY{l+s+se}{\PYZbs{}n}\PY{l+s+s2}{multi}\PY{l+s+se}{\PYZbs{}n}\PY{l+s+s2}{line}\PY{l+s+se}{\PYZbs{}n}\PY{l+s+s2}{string}\PY{l+s+se}{\PYZbs{}n}\PY{l+s+se}{\PYZbs{}\PYZdq{}}\PY{l+s+se}{\PYZbs{}\PYZdq{}}\PY{l+s+se}{\PYZbs{}\PYZdq{}}\PY{l+s+s2}{\PYZdq{}}\PY{p}{,} \PY{l+s+s2}{\PYZdq{}}\PY{l+s+s2}{is also a}\PY{l+s+s2}{\PYZdq{}}\PY{p}{,} \PY{n+nb}{type}\PY{p}{(}\PY{l+s+s2}{\PYZdq{}\PYZdq{}\PYZdq{}}\PY{l+s+se}{\PYZbs{}n}\PY{l+s+s2}{multi}\PY{l+s+se}{\PYZbs{}n}\PY{l+s+s2}{line}\PY{l+s+se}{\PYZbs{}n}\PY{l+s+s2}{string}\PY{l+s+se}{\PYZbs{}n}\PY{l+s+s2}{\PYZdq{}\PYZdq{}\PYZdq{}}\PY{p}{)}\PY{p}{)}
\PY{n+nb}{print}\PY{p}{(}\PY{l+m+mi}{1}\PY{p}{,}\PY{l+s+s2}{\PYZdq{}}\PY{l+s+s2}{is a}\PY{l+s+s2}{\PYZdq{}}\PY{p}{,} \PY{n+nb}{type}\PY{p}{(}\PY{l+m+mi}{1}\PY{p}{)}\PY{p}{)}
\PY{n+nb}{print}\PY{p}{(}\PY{l+m+mf}{1.}\PY{p}{,}\PY{l+s+s2}{\PYZdq{}}\PY{l+s+s2}{is a}\PY{l+s+s2}{\PYZdq{}}\PY{p}{,} \PY{n+nb}{type}\PY{p}{(}\PY{l+m+mf}{1.}\PY{p}{)}\PY{p}{)}
\PY{n+nb}{print}\PY{p}{(}\PY{l+m+mf}{1.}\PY{o}{+}\PY{l+m+mi}{1}\PY{n}{j}\PY{p}{,}\PY{l+s+s2}{\PYZdq{}}\PY{l+s+s2}{is a}\PY{l+s+s2}{\PYZdq{}}\PY{p}{,}\PY{n+nb}{type}\PY{p}{(}\PY{l+m+mf}{1.}\PY{o}{+}\PY{l+m+mi}{1}\PY{n}{j}\PY{p}{)}\PY{p}{)}
\PY{n+nb}{print}\PY{p}{(}\PY{l+s+s2}{\PYZdq{}}\PY{l+s+s2}{True}\PY{l+s+s2}{\PYZdq{}}\PY{p}{,}\PY{l+s+s2}{\PYZdq{}}\PY{l+s+s2}{is a}\PY{l+s+s2}{\PYZdq{}}\PY{p}{,}\PY{n+nb}{type}\PY{p}{(}\PY{k+kc}{True}\PY{p}{)}\PY{p}{)}
\PY{n+nb}{print}\PY{p}{(}\PY{k+kc}{None}\PY{p}{,}\PY{l+s+s2}{\PYZdq{}}\PY{l+s+s2}{is a}\PY{l+s+s2}{\PYZdq{}}\PY{p}{,}\PY{n+nb}{type}\PY{p}{(}\PY{k+kc}{None}\PY{p}{)}\PY{p}{)}
\end{Verbatim}
\end{tcolorbox}

    \begin{Verbatim}[commandchars=\\\{\}]
"A" is a <class 'str'>
'A' is also a <class 'str'>
in other words, "A" is 'A' - > True
"""
multi
line
string
""" is also a <class 'str'>
1 is a <class 'int'>
1.0 is a <class 'float'>
(1+1j) is a <class 'complex'>
True is a <class 'bool'>
None is a <class 'NoneType'>
    \end{Verbatim}

    \begin{Verbatim}[commandchars=\\\{\}]
<>:3: SyntaxWarning: "is" with a literal. Did you mean "=="?
<>:3: SyntaxWarning: "is" with a literal. Did you mean "=="?
/var/folders/1\_/2rbdt5292zdb57b821k90d5r0000gn/T/ipykernel\_56789/2405636746.py:3
: SyntaxWarning: "is" with a literal. Did you mean "=="?
  print("in other words, \textbackslash{}"A\textbackslash{}" is 'A' - >","A" is 'A')
    \end{Verbatim}

    \begin{tcolorbox}[breakable, size=fbox, boxrule=1pt, pad at break*=1mm,colback=cellbackground, colframe=cellborder]
\prompt{In}{incolor}{9}{\boxspacing}
\begin{Verbatim}[commandchars=\\\{\}]
\PY{n+nb}{print}\PY{p}{(}\PY{l+s+s2}{\PYZdq{}}\PY{l+s+se}{\PYZbs{}n}\PY{l+s+s2}{you should use }\PY{l+s+se}{\PYZbs{}\PYZbs{}}\PY{l+s+s2}{ to print special characters }\PY{l+s+se}{\PYZbs{}\PYZdq{}}\PY{l+s+s2}{ or }\PY{l+s+se}{\PYZbs{}\PYZsq{}}\PY{l+s+s2}{\PYZdq{}}\PY{p}{)}
\PY{n+nb}{print}\PY{p}{(}\PY{l+s+s2}{\PYZdq{}}\PY{l+s+se}{\PYZbs{}n}\PY{l+s+s2}{strings defined by }\PY{l+s+se}{\PYZbs{}\PYZdq{}}\PY{l+s+s2}{ understands special character }\PY{l+s+s2}{\PYZsq{}}\PY{l+s+s2}{ without }\PY{l+s+se}{\PYZbs{}\PYZbs{}}\PY{l+s+s2}{ }\PY{l+s+s2}{\PYZdq{}}\PY{p}{)}
\PY{n+nb}{print}\PY{p}{(}\PY{l+s+s1}{\PYZsq{}}\PY{l+s+s1}{strings defined by }\PY{l+s+se}{\PYZbs{}\PYZsq{}}\PY{l+s+s1}{ understands special character }\PY{l+s+s1}{\PYZdq{}}\PY{l+s+s1}{ without }\PY{l+s+se}{\PYZbs{}\PYZbs{}}\PY{l+s+s1}{\PYZsq{}} \PY{p}{)}
\end{Verbatim}
\end{tcolorbox}

    \begin{Verbatim}[commandchars=\\\{\}]

you should use \textbackslash{} to print special characters " or '

strings defined by " understands special character ' without \textbackslash{}
strings defined by ' understands special character " without \textbackslash{}
    \end{Verbatim}

    \hypertarget{comments-and-multiline-strings-in-python}{%
\subsection{Comments and multiline strings in
python}\label{comments-and-multiline-strings-in-python}}

\begin{itemize}
\tightlist
\item
  \# starts an inline comment
\item
  triple-quoted strings can be used for documentation (returned by
  help() function)
\item
  we'll see this later
\item
  triple-quoted strings can be used as multiline comments
\end{itemize}

    \begin{tcolorbox}[breakable, size=fbox, boxrule=1pt, pad at break*=1mm,colback=cellbackground, colframe=cellborder]
\prompt{In}{incolor}{10}{\boxspacing}
\begin{Verbatim}[commandchars=\\\{\}]
\PY{c+c1}{\PYZsh{} to comment this line}

\PY{n}{s} \PY{o}{=} \PY{l+s+s1}{\PYZsq{}\PYZsq{}\PYZsq{}}\PY{l+s+s1}{triple\PYZhy{}quoted }
\PY{l+s+s1}{strings}
\PY{l+s+s1}{are   }
\PY{l+s+s1}{multiline}
\PY{l+s+s1}{\PYZsq{}\PYZsq{}\PYZsq{}}
\PY{l+s+sd}{\PYZsq{}\PYZsq{}\PYZsq{}}
\PY{l+s+sd}{triple\PYZhy{}quoted strings that are not docstrings (first thing in a class/function/module) are ignored }
\PY{l+s+sd}{and can be used }
\PY{l+s+sd}{as multiline comments}
\PY{l+s+sd}{\PYZsq{}\PYZsq{}\PYZsq{}}

\PY{n+nb}{print}\PY{p}{(}\PY{n}{s}\PY{p}{)} 
\end{Verbatim}
\end{tcolorbox}

    \begin{Verbatim}[commandchars=\\\{\}]
triple-quoted
strings
are
multiline

    \end{Verbatim}

    \hypertarget{python-is-strongly-dinamically-typed}{%
\subsection{Python is strongly dinamically
typed}\label{python-is-strongly-dinamically-typed}}

\begin{itemize}
\tightlist
\item
  strong means that every change requires an explicit conversion
\item
  dynamic typing means that values have a type, not variables
\item
  i.e., variables can name any object
\item
  only keywords cannot be reassigned
\item
  also functions such as print() and type() use names that are not
  keywords
\item
  they could potentially reassigned to new objects
\end{itemize}

    \begin{tcolorbox}[breakable, size=fbox, boxrule=1pt, pad at break*=1mm,colback=cellbackground, colframe=cellborder]
\prompt{In}{incolor}{11}{\boxspacing}
\begin{Verbatim}[commandchars=\\\{\}]
\PY{n}{myVar} \PY{o}{=} \PY{l+m+mi}{1}        \PY{c+c1}{\PYZsh{} class \PYZsq{}int\PYZsq{}}
\PY{n}{myVar} \PY{o}{=} \PY{l+s+s2}{\PYZdq{}}\PY{l+s+s2}{myVar}\PY{l+s+s2}{\PYZdq{}}  \PY{c+c1}{\PYZsh{} class \PYZsq{}str\PYZsq{}}
\PY{n}{myVar} \PY{o}{=} \PY{k+kc}{None}     \PY{c+c1}{\PYZsh{} class \PYZsq{}NoneType\PYZsq{}}
\end{Verbatim}
\end{tcolorbox}

    \begin{tcolorbox}[breakable, size=fbox, boxrule=1pt, pad at break*=1mm,colback=cellbackground, colframe=cellborder]
\prompt{In}{incolor}{12}{\boxspacing}
\begin{Verbatim}[commandchars=\\\{\}]
\PY{n+nb}{print}\PY{p}{(}\PY{l+s+s2}{\PYZdq{}}\PY{l+s+se}{\PYZbs{}n}\PY{l+s+s2}{\PYZsh{}reassign objects to print and type }\PY{l+s+se}{\PYZbs{}n}\PY{l+s+s2}{print=10 }\PY{l+s+se}{\PYZbs{}n}\PY{l+s+s2}{type=\PYZhy{}1 }\PY{l+s+se}{\PYZbs{}n}\PY{l+s+s2}{print,type}\PY{l+s+s2}{\PYZdq{}}\PY{p}{)}
\PY{n+nb}{print} \PY{o}{=} \PY{l+m+mi}{10}
\PY{n+nb}{type} \PY{o}{=} \PY{o}{\PYZhy{}}\PY{l+m+mi}{1}
\PY{n+nb}{print}\PY{p}{,} \PY{n+nb}{type}
\end{Verbatim}
\end{tcolorbox}

    \begin{Verbatim}[commandchars=\\\{\}]

\#reassign objects to print and type
print=10
type=-1
print,type
    \end{Verbatim}

            \begin{tcolorbox}[breakable, size=fbox, boxrule=.5pt, pad at break*=1mm, opacityfill=0]
\prompt{Out}{outcolor}{12}{\boxspacing}
\begin{Verbatim}[commandchars=\\\{\}]
(10, -1)
\end{Verbatim}
\end{tcolorbox}
        
    \begin{tcolorbox}[breakable, size=fbox, boxrule=1pt, pad at break*=1mm,colback=cellbackground, colframe=cellborder]
\prompt{In}{incolor}{13}{\boxspacing}
\begin{Verbatim}[commandchars=\\\{\}]
\PY{n+nb}{print}\PY{p}{(}\PY{l+m+mi}{1}\PY{p}{)}
\end{Verbatim}
\end{tcolorbox}

    \begin{Verbatim}[commandchars=\\\{\}]

        ---------------------------------------------------------------------------

        TypeError                                 Traceback (most recent call last)

        /var/folders/1\_/2rbdt5292zdb57b821k90d5r0000gn/T/ipykernel\_56789/3773950048.py in <module>
    ----> 1 print(1)
    

        TypeError: 'int' object is not callable

    \end{Verbatim}

    \begin{tcolorbox}[breakable, size=fbox, boxrule=1pt, pad at break*=1mm,colback=cellbackground, colframe=cellborder]
\prompt{In}{incolor}{14}{\boxspacing}
\begin{Verbatim}[commandchars=\\\{\}]
\PY{c+c1}{\PYZsh{} force reset of all user defined names, included print and type}
\PY{o}{\PYZpc{}}\PY{k}{reset} \PYZhy{}f 
\PY{c+c1}{\PYZsh{} now print and type are back to work}
\PY{n+nb}{print}\PY{p}{(}\PY{l+s+s2}{\PYZdq{}}\PY{l+s+s2}{back to work}\PY{l+s+s2}{\PYZdq{}}\PY{p}{,} \PY{n+nb}{type}\PY{p}{(}\PY{l+s+s2}{\PYZdq{}}\PY{l+s+s2}{back to work}\PY{l+s+s2}{\PYZdq{}}\PY{p}{)}\PY{p}{)}
\end{Verbatim}
\end{tcolorbox}

    \begin{Verbatim}[commandchars=\\\{\}]
back to work <class 'str'>
    \end{Verbatim}

    \hypertarget{type-conversion}{%
\subsection{Type conversion}\label{type-conversion}}

\begin{itemize}
\tightlist
\item
  Generally, conversions MUST be explicit because Python is strongly
  dinamically typed
\item
  However, Python does some automatic conversions
\item
  between numeric types (where it makes sense)
\item
  empty container types as well as zero are considered False in a
  boolean context
\end{itemize}

\hypertarget{math-operators}{%
\subsection{Math operators}\label{math-operators}}

\begin{itemize}
\tightlist
\item
  python provides standard arithmetic opearators +,-,*,/
\item
  exponentiation operator **
\item
  ** is right associative
\item
  floor division operator //
\item
  a//b returns the greatest whole number that does not exceed a/b
\item
  modulo (reminder) \%
\item
  complements // by telling what remains from the floor division
\item
  parentheses can change the order of operations
\end{itemize}

    \begin{tcolorbox}[breakable, size=fbox, boxrule=1pt, pad at break*=1mm,colback=cellbackground, colframe=cellborder]
\prompt{In}{incolor}{3}{\boxspacing}
\begin{Verbatim}[commandchars=\\\{\}]
\PY{n+nb}{print}\PY{p}{(}\PY{l+s+s2}{\PYZdq{}}\PY{l+s+s2}{implicit conversion between numeric literals is allowed, if that makes sense}\PY{l+s+s2}{\PYZdq{}}\PY{p}{)}
\PY{n+nb}{print}\PY{p}{(}\PY{l+s+s2}{\PYZdq{}}\PY{l+s+s2}{(12.+1j)+3 \PYZhy{}\PYZgt{} }\PY{l+s+s2}{\PYZdq{}}\PY{p}{,}\PY{p}{(}\PY{l+m+mf}{12.}\PY{o}{+}\PY{l+m+mi}{1}\PY{n}{j}\PY{p}{)}\PY{o}{+}\PY{l+m+mi}{3}\PY{p}{)}
\PY{n+nb}{print}\PY{p}{(}\PY{l+s+s2}{\PYZdq{}}\PY{l+s+se}{\PYZbs{}n}\PY{l+s+s2}{implicit conversion from string to a numeric literal returs an error}\PY{l+s+s2}{\PYZdq{}}\PY{p}{)}
\PY{n+nb}{print}\PY{p}{(}\PY{l+s+s2}{\PYZdq{}}\PY{l+s+s2}{\PYZsq{}}\PY{l+s+s2}{12}\PY{l+s+s2}{\PYZsq{}}\PY{l+s+s2}{+3 }\PY{l+s+se}{\PYZbs{}n}\PY{l+s+s2}{ \PYZhy{}\PYZgt{} returns}\PY{l+s+s2}{\PYZdq{}}\PY{p}{)}
\PY{l+s+s1}{\PYZsq{}}\PY{l+s+s1}{12}\PY{l+s+s1}{\PYZsq{}}\PY{o}{+}\PY{l+m+mi}{3}
\end{Verbatim}
\end{tcolorbox}

    \begin{Verbatim}[commandchars=\\\{\}]
implicit conversion between numeric literals is allowed, if that makes sense
(12.+1j)+3 ->  (15+1j)

implicit conversion from string to a numeric literal returs an error
'12'+3
 -> returns
    \end{Verbatim}

    \begin{Verbatim}[commandchars=\\\{\}]

        ---------------------------------------------------------------------------

        TypeError                                 Traceback (most recent call last)

        <ipython-input-3-795ef93fcb9b> in <module>()
          3 print("\textbackslash{}nimplicit conversion from string to a numeric literal returs an error")
          4 print("'12'+3 \textbackslash{}n -> returns")
    ----> 5 '12'+3
    

        TypeError: Can't convert 'int' object to str implicitly

    \end{Verbatim}

    \begin{tcolorbox}[breakable, size=fbox, boxrule=1pt, pad at break*=1mm,colback=cellbackground, colframe=cellborder]
\prompt{In}{incolor}{4}{\boxspacing}
\begin{Verbatim}[commandchars=\\\{\}]
\PY{n+nb}{print}\PY{p}{(}\PY{l+s+s2}{\PYZdq{}}\PY{l+s+se}{\PYZbs{}n}\PY{l+s+s2}{\PYZsh{}Operations between numbers}\PY{l+s+s2}{\PYZdq{}}\PY{p}{)}
\PY{n+nb}{print}\PY{p}{(}\PY{l+s+s2}{\PYZdq{}}\PY{l+s+se}{\PYZbs{}n}\PY{l+s+s2}{\PYZsh{} summation with float \PYZhy{}\PYZgt{} float}\PY{l+s+s2}{\PYZdq{}}\PY{p}{,}\PY{l+s+s2}{\PYZdq{}}\PY{l+s+se}{\PYZbs{}n}\PY{l+s+s2}{type(1+1.)}\PY{l+s+s2}{\PYZdq{}}\PY{p}{)}
\PY{n+nb}{print}\PY{p}{(}\PY{n+nb}{type}\PY{p}{(}\PY{l+m+mi}{1}\PY{o}{+}\PY{l+m+mf}{1.}\PY{p}{)}\PY{p}{)}     
\PY{n+nb}{print}\PY{p}{(}\PY{l+s+s2}{\PYZdq{}}\PY{l+s+se}{\PYZbs{}n}\PY{l+s+s2}{\PYZsh{} division with integers \PYZhy{}\PYZgt{} float}\PY{l+s+s2}{\PYZdq{}}\PY{p}{,}\PY{l+s+s2}{\PYZdq{}}\PY{l+s+se}{\PYZbs{}n}\PY{l+s+s2}{type(3/2)}\PY{l+s+s2}{\PYZdq{}}\PY{p}{)}
\PY{n+nb}{print}\PY{p}{(}\PY{n+nb}{type}\PY{p}{(}\PY{l+m+mi}{3}\PY{o}{/}\PY{l+m+mi}{2}\PY{p}{)}\PY{p}{)}      
\PY{n+nb}{print}\PY{p}{(}\PY{l+s+s2}{\PYZdq{}}\PY{l+s+se}{\PYZbs{}n}\PY{l+s+s2}{\PYZsh{} operations between parentheses are executed earlier }\PY{l+s+s2}{\PYZdq{}}\PY{p}{,}\PY{l+s+s2}{\PYZdq{}}\PY{l+s+se}{\PYZbs{}n}\PY{l+s+s2}{4/(2+2)}\PY{l+s+s2}{\PYZdq{}}\PY{p}{)}
\PY{n+nb}{print}\PY{p}{(}\PY{l+m+mi}{4}\PY{o}{/}\PY{p}{(}\PY{l+m+mi}{2}\PY{o}{+}\PY{l+m+mi}{2}\PY{p}{)}\PY{p}{)}        
\PY{n+nb}{print}\PY{p}{(}\PY{l+s+s2}{\PYZdq{}}\PY{l+s+se}{\PYZbs{}n}\PY{l+s+s2}{\PYZsh{} exponentiation (default right associative)}\PY{l+s+s2}{\PYZdq{}}\PY{p}{,}\PY{l+s+s2}{\PYZdq{}}\PY{l+s+se}{\PYZbs{}n}\PY{l+s+s2}{2**3}\PY{l+s+s2}{\PYZdq{}}\PY{p}{,}\PY{l+s+s2}{\PYZdq{}}\PY{l+s+s2}{, 2**3**2}\PY{l+s+s2}{\PYZdq{}}\PY{p}{,}\PY{l+s+s2}{\PYZdq{}}\PY{l+s+s2}{, 2**(3**2)}\PY{l+s+s2}{\PYZdq{}}\PY{p}{,}\PY{l+s+s2}{\PYZdq{}}\PY{l+s+s2}{, (2**3)**2}\PY{l+s+s2}{\PYZdq{}}\PY{p}{)}
\PY{n+nb}{print}\PY{p}{(}\PY{l+m+mi}{2}\PY{o}{*}\PY{o}{*}\PY{l+m+mi}{3}\PY{p}{,} \PY{l+s+s2}{\PYZdq{}}\PY{l+s+s2}{,}\PY{l+s+s2}{\PYZdq{}}\PY{p}{,} \PY{l+m+mi}{2}\PY{o}{*}\PY{o}{*}\PY{l+m+mi}{3}\PY{o}{*}\PY{o}{*}\PY{l+m+mi}{2}\PY{p}{,} \PY{l+s+s2}{\PYZdq{}}\PY{l+s+s2}{,}\PY{l+s+s2}{\PYZdq{}}\PY{p}{,} \PY{l+m+mi}{2}\PY{o}{*}\PY{o}{*}\PY{p}{(}\PY{l+m+mi}{3}\PY{o}{*}\PY{o}{*}\PY{l+m+mi}{2}\PY{p}{)}\PY{p}{,} \PY{l+s+s2}{\PYZdq{}}\PY{l+s+s2}{,}\PY{l+s+s2}{\PYZdq{}}\PY{p}{,} \PY{p}{(}\PY{l+m+mi}{2}\PY{o}{*}\PY{o}{*}\PY{l+m+mi}{3}\PY{p}{)}\PY{o}{*}\PY{o}{*}\PY{l+m+mi}{2}\PY{p}{)}
\PY{n+nb}{print}\PY{p}{(}\PY{l+s+s2}{\PYZdq{}}\PY{l+s+se}{\PYZbs{}n}\PY{l+s+s2}{\PYZsh{} floor division }\PY{l+s+s2}{\PYZdq{}}\PY{p}{,}\PY{l+s+s2}{\PYZdq{}}\PY{l+s+se}{\PYZbs{}n}\PY{l+s+s2}{9.5/2}\PY{l+s+s2}{\PYZdq{}}\PY{p}{,}\PY{l+s+s2}{\PYZdq{}}\PY{l+s+s2}{, 9.5//2}\PY{l+s+s2}{\PYZdq{}}\PY{p}{,}\PY{l+s+s2}{\PYZdq{}}\PY{l+s+s2}{, 9.1//2}\PY{l+s+s2}{\PYZdq{}}\PY{p}{,}\PY{l+s+s2}{\PYZdq{}}\PY{l+s+s2}{, 9.9//2}\PY{l+s+s2}{\PYZdq{}}\PY{p}{)}
\PY{n+nb}{print}\PY{p}{(}\PY{l+m+mf}{9.5}\PY{o}{/}\PY{l+m+mi}{2}\PY{p}{,} \PY{l+s+s2}{\PYZdq{}}\PY{l+s+s2}{,}\PY{l+s+s2}{\PYZdq{}}\PY{p}{,} \PY{l+m+mf}{9.5}\PY{o}{/}\PY{o}{/}\PY{l+m+mi}{2}\PY{p}{,} \PY{l+s+s2}{\PYZdq{}}\PY{l+s+s2}{,}\PY{l+s+s2}{\PYZdq{}}\PY{p}{,} \PY{l+m+mf}{9.1}\PY{o}{/}\PY{o}{/}\PY{l+m+mi}{2}\PY{p}{,} \PY{l+s+s2}{\PYZdq{}}\PY{l+s+s2}{,}\PY{l+s+s2}{\PYZdq{}}\PY{p}{,} \PY{l+m+mf}{9.9}\PY{o}{/}\PY{o}{/}\PY{l+m+mi}{2}\PY{p}{)}
\PY{n+nb}{print}\PY{p}{(}\PY{l+s+s2}{\PYZdq{}}\PY{l+s+se}{\PYZbs{}n}\PY{l+s+s2}{\PYZsh{} modulo }\PY{l+s+s2}{\PYZdq{}}\PY{p}{,}\PY{l+s+s2}{\PYZdq{}}\PY{l+s+s2}{, 9.5}\PY{l+s+s2}{\PYZpc{}}\PY{l+s+s2}{2}\PY{l+s+s2}{\PYZdq{}}\PY{p}{,}\PY{l+s+s2}{\PYZdq{}}\PY{l+s+s2}{, 9.1}\PY{l+s+s2}{\PYZpc{}}\PY{l+s+s2}{2}\PY{l+s+s2}{\PYZdq{}}\PY{p}{,}\PY{l+s+s2}{\PYZdq{}}\PY{l+s+s2}{, 9.9}\PY{l+s+s2}{\PYZpc{}}\PY{l+s+s2}{2}\PY{l+s+s2}{\PYZdq{}}\PY{p}{)}
\PY{n+nb}{print}\PY{p}{(}\PY{l+m+mf}{9.5}\PY{o}{\PYZpc{}}\PY{k}{2}, \PYZdq{},\PYZdq{}, 9.1\PYZpc{}2, \PYZdq{},\PYZdq{}, 9.9\PYZpc{}2)
\end{Verbatim}
\end{tcolorbox}

    \begin{Verbatim}[commandchars=\\\{\}]

\#Operations between numbers

\# summation with float -> float
type(1+1.)
<class 'float'>

\# division with integers -> float
type(3/2)
<class 'float'>

\# operations between parentheses are executed earlier
4/(2+2)
1.0

\# exponentiation (default right associative)
2**3 , 2**3**2 , 2**(3**2) , (2**3)**2
8 , 512 , 512 , 64

\# floor division
9.5/2 , 9.5//2 , 9.1//2 , 9.9//2
4.75 , 4.0 , 4.0 , 4.0

\# modulo  , 9.5\%2 , 9.1\%2 , 9.9\%2
1.5 , 1.0999999999999996 , 1.9000000000000004
    \end{Verbatim}

    \hypertarget{explicit-type-conversion}{%
\subsection{Explicit type conversion}\label{explicit-type-conversion}}

\begin{itemize}
\tightlist
\item
  int(x): x converted to integer
\item
  bin(x): returns a string with the binary representation
\item
  float(x): x converted to float
\item
  complex(re,im): real part re, imaginary part im (im default is zero)
\item
  str(x): x converted to string
\end{itemize}

    \hypertarget{boolean-operations}{%
\subsection{Boolean operations}\label{boolean-operations}}

\begin{itemize}
\tightlist
\item
  standard boolean operations: and, or, not
\item
  x or y (if x is false, then y, else x)
\item
  x and y (if x is false, then x, else y)
\item
  not x (if x is false, then True, else False)
\item
  standard comparison operations:
  \textless,\textless=,\textgreater,\textgreater=,==,!=
\item
  (strictly) greater/less (or equal) than, equal, not equal
\item
  object identity ``is''
\item
  negated object identity ``is not''
\end{itemize}

    \begin{tcolorbox}[breakable, size=fbox, boxrule=1pt, pad at break*=1mm,colback=cellbackground, colframe=cellborder]
\prompt{In}{incolor}{ }{\boxspacing}
\begin{Verbatim}[commandchars=\\\{\}]
\PY{n}{x} \PY{o}{=} \PY{k+kc}{True}
\PY{n}{y} \PY{o}{=} \PY{k+kc}{False}

\PY{n+nb}{print}\PY{p}{(}\PY{l+s+s2}{\PYZdq{}}\PY{l+s+se}{\PYZbs{}n}\PY{l+s+s2}{x=True}\PY{l+s+s2}{\PYZdq{}}\PY{p}{,}\PY{l+s+s2}{\PYZdq{}}\PY{l+s+s2}{y=False}\PY{l+s+s2}{\PYZdq{}}\PY{p}{)}
\PY{n+nb}{print} \PY{p}{(}\PY{l+s+s2}{\PYZdq{}}\PY{l+s+se}{\PYZbs{}n}\PY{l+s+s2}{x or y =}\PY{l+s+s2}{\PYZdq{}}\PY{p}{,}\PY{n}{x} \PY{o+ow}{or} \PY{n}{y}\PY{p}{,}\PY{l+s+s2}{\PYZdq{}}\PY{l+s+se}{\PYZbs{}n}\PY{l+s+s2}{x and y =}\PY{l+s+s2}{\PYZdq{}}\PY{p}{,}\PY{n}{x} \PY{o+ow}{and} \PY{n}{y}\PY{p}{,}\PY{l+s+s2}{\PYZdq{}}\PY{l+s+se}{\PYZbs{}n}\PY{l+s+s2}{not x =}\PY{l+s+s2}{\PYZdq{}}\PY{p}{,} \PY{o+ow}{not} \PY{n}{x}\PY{p}{)}

\PY{n+nb}{print}\PY{p}{(}\PY{l+s+s2}{\PYZdq{}}\PY{l+s+se}{\PYZbs{}n}\PY{l+s+s2}{0\PYZgt{}0 =}\PY{l+s+s2}{\PYZdq{}}\PY{p}{,}\PY{l+m+mi}{0}\PY{o}{\PYZgt{}}\PY{l+m+mi}{0}\PY{p}{,} \PY{l+s+s2}{\PYZdq{}}\PY{l+s+se}{\PYZbs{}n}\PY{l+s+s2}{0\PYZgt{}=0 =}\PY{l+s+s2}{\PYZdq{}}\PY{p}{,}\PY{l+m+mi}{0}\PY{o}{\PYZgt{}}\PY{o}{=}\PY{l+m+mi}{0}\PY{p}{,}\PY{l+s+s2}{\PYZdq{}}\PY{l+s+se}{\PYZbs{}n}\PY{l+s+s2}{0\PYZlt{}0 =}\PY{l+s+s2}{\PYZdq{}}\PY{p}{,}\PY{l+m+mi}{0}\PY{o}{\PYZgt{}}\PY{l+m+mi}{0}\PY{p}{,} \PY{l+s+s2}{\PYZdq{}}\PY{l+s+se}{\PYZbs{}n}\PY{l+s+s2}{0\PYZlt{}=0 =}\PY{l+s+s2}{\PYZdq{}}\PY{p}{,}\PY{l+m+mi}{0}\PY{o}{\PYZgt{}}\PY{o}{=}\PY{l+m+mi}{0}\PY{p}{,} \PY{l+s+s2}{\PYZdq{}}\PY{l+s+se}{\PYZbs{}n}\PY{l+s+s2}{0==0 =}\PY{l+s+s2}{\PYZdq{}}\PY{p}{,}\PY{l+m+mi}{0}\PY{o}{==}\PY{l+m+mi}{0}\PY{p}{,} \PY{l+s+s2}{\PYZdq{}}\PY{l+s+se}{\PYZbs{}n}\PY{l+s+s2}{0!=0 =}\PY{l+s+s2}{\PYZdq{}}\PY{p}{,}\PY{l+m+mi}{0}\PY{o}{!=}\PY{l+m+mi}{0}\PY{p}{)}

\PY{n+nb}{print}\PY{p}{(}\PY{l+s+s2}{\PYZdq{}}\PY{l+s+se}{\PYZbs{}n}\PY{l+s+s2}{type(5.) is float =}\PY{l+s+s2}{\PYZdq{}}\PY{p}{,}\PY{n+nb}{type}\PY{p}{(}\PY{l+m+mf}{5.}\PY{p}{)} \PY{o+ow}{is} \PY{n+nb}{float}\PY{p}{,} \PY{l+s+s2}{\PYZdq{}}\PY{l+s+se}{\PYZbs{}n}\PY{l+s+s2}{type(5.) is not float =}\PY{l+s+s2}{\PYZdq{}}\PY{p}{,}\PY{n+nb}{type}\PY{p}{(}\PY{l+m+mf}{5.}\PY{p}{)} \PY{o+ow}{is} \PY{o+ow}{not} \PY{n+nb}{float}\PY{p}{)}
\end{Verbatim}
\end{tcolorbox}

    \hypertarget{standard-numeric-operations}{%
\subsection{Standard numeric
operations}\label{standard-numeric-operations}}

\begin{itemize}
\tightlist
\item
  abs(x): absolute value or magnitude of x
\item
  c.conjugate(): conjugate of the complex number c
\item
  divmod(x,y): the pair (x // y, x \% y)
\item
  pow(x,y): x to the power y, same as x ** y
\item
  round(x,n): rounds x to a given precision in n decimal digits
\item
  default n=0
\item
  n can be negative
\end{itemize}

    \begin{tcolorbox}[breakable, size=fbox, boxrule=1pt, pad at break*=1mm,colback=cellbackground, colframe=cellborder]
\prompt{In}{incolor}{ }{\boxspacing}
\begin{Verbatim}[commandchars=\\\{\}]
\PY{n+nb}{print}\PY{p}{(}\PY{l+s+s2}{\PYZdq{}}\PY{l+s+se}{\PYZbs{}n}\PY{l+s+s2}{abs(\PYZhy{}4.5) \PYZhy{}\PYZgt{}}\PY{l+s+s2}{\PYZdq{}}\PY{p}{,}\PY{n+nb}{abs}\PY{p}{(}\PY{o}{\PYZhy{}}\PY{l+m+mf}{4.5}\PY{p}{)}\PY{p}{,}\PY{n+nb}{type}\PY{p}{(}\PY{n+nb}{abs}\PY{p}{(}\PY{o}{\PYZhy{}}\PY{l+m+mf}{4.5}\PY{p}{)}\PY{p}{)}\PY{p}{)}
\PY{n+nb}{print}\PY{p}{(}\PY{l+s+s2}{\PYZdq{}}\PY{l+s+s2}{complex(\PYZhy{}4+1j).conjugate() \PYZhy{}\PYZgt{}}\PY{l+s+s2}{\PYZdq{}}\PY{p}{,}\PY{n+nb}{complex}\PY{p}{(}\PY{o}{\PYZhy{}}\PY{l+m+mi}{4}\PY{o}{+}\PY{l+m+mi}{1}\PY{n}{j}\PY{p}{)}\PY{o}{.}\PY{n}{conjugate}\PY{p}{(}\PY{p}{)}\PY{p}{)}
\PY{n+nb}{print}\PY{p}{(}\PY{l+s+s2}{\PYZdq{}}\PY{l+s+s2}{divmod(10,3) \PYZhy{}\PYZgt{} }\PY{l+s+s2}{\PYZdq{}}\PY{p}{,}\PY{n+nb}{divmod}\PY{p}{(}\PY{l+m+mi}{10}\PY{p}{,}\PY{l+m+mi}{3}\PY{p}{)}\PY{p}{)}
\PY{n+nb}{print}\PY{p}{(}\PY{l+s+s2}{\PYZdq{}}\PY{l+s+s2}{ \PYZhy{}\PYZgt{} (10//3,10}\PY{l+s+s2}{\PYZpc{}}\PY{l+s+s2}{3) \PYZhy{}\PYZgt{} }\PY{l+s+s2}{\PYZdq{}}\PY{p}{,}\PY{p}{(}\PY{l+m+mi}{10}\PY{o}{/}\PY{o}{/}\PY{l+m+mi}{3}\PY{p}{,}\PY{l+m+mi}{10}\PY{o}{\PYZpc{}}\PY{k}{3}))
\PY{n+nb}{print}\PY{p}{(}\PY{l+s+s2}{\PYZdq{}}\PY{l+s+s2}{pow(10,3) \PYZhy{}\PYZgt{} }\PY{l+s+s2}{\PYZdq{}}\PY{p}{,}\PY{n+nb}{pow}\PY{p}{(}\PY{l+m+mi}{10}\PY{p}{,}\PY{l+m+mi}{3}\PY{p}{)}\PY{p}{)}
\PY{n+nb}{print}\PY{p}{(}\PY{l+s+s2}{\PYZdq{}}\PY{l+s+s2}{ \PYZhy{}\PYZgt{} (10**3) \PYZhy{}\PYZgt{} }\PY{l+s+s2}{\PYZdq{}}\PY{p}{,}\PY{p}{(}\PY{l+m+mi}{10}\PY{o}{*}\PY{o}{*}\PY{l+m+mi}{3}\PY{p}{)}\PY{p}{)}
\end{Verbatim}
\end{tcolorbox}

    \begin{tcolorbox}[breakable, size=fbox, boxrule=1pt, pad at break*=1mm,colback=cellbackground, colframe=cellborder]
\prompt{In}{incolor}{7}{\boxspacing}
\begin{Verbatim}[commandchars=\\\{\}]
\PY{n+nb}{print}\PY{p}{(}\PY{l+s+s1}{\PYZsq{}}\PY{l+s+s1}{round(1234.1234,4) \PYZhy{}\PYZgt{}}\PY{l+s+s1}{\PYZsq{}}\PY{p}{,}\PY{n+nb}{round}\PY{p}{(}\PY{l+m+mf}{1234.1234}\PY{p}{,}\PY{l+m+mi}{4}\PY{p}{)}\PY{p}{)}
\PY{n+nb}{print}\PY{p}{(}\PY{l+s+s1}{\PYZsq{}}\PY{l+s+s1}{round(1234.1234,3) \PYZhy{}\PYZgt{}}\PY{l+s+s1}{\PYZsq{}}\PY{p}{,}\PY{n+nb}{round}\PY{p}{(}\PY{l+m+mf}{1234.1234}\PY{p}{,}\PY{l+m+mi}{3}\PY{p}{)}\PY{p}{)}
\PY{n+nb}{print}\PY{p}{(}\PY{l+s+s1}{\PYZsq{}}\PY{l+s+s1}{round(1234.1234,2) \PYZhy{}\PYZgt{}}\PY{l+s+s1}{\PYZsq{}}\PY{p}{,}\PY{n+nb}{round}\PY{p}{(}\PY{l+m+mf}{1234.1234}\PY{p}{,}\PY{l+m+mi}{2}\PY{p}{)}\PY{p}{)}
\PY{n+nb}{print}\PY{p}{(}\PY{l+s+s1}{\PYZsq{}}\PY{l+s+s1}{round(1234.1234,1) \PYZhy{}\PYZgt{}}\PY{l+s+s1}{\PYZsq{}}\PY{p}{,}\PY{n+nb}{round}\PY{p}{(}\PY{l+m+mf}{1234.1234}\PY{p}{,}\PY{l+m+mi}{1}\PY{p}{)}\PY{p}{)}
\PY{n+nb}{print}\PY{p}{(}\PY{l+s+s1}{\PYZsq{}}\PY{l+s+s1}{round(1234.1234,0) \PYZhy{}\PYZgt{}}\PY{l+s+s1}{\PYZsq{}}\PY{p}{,}\PY{n+nb}{round}\PY{p}{(}\PY{l+m+mf}{1234.1234}\PY{p}{,}\PY{l+m+mi}{0}\PY{p}{)}\PY{p}{)}
\PY{n+nb}{print}\PY{p}{(}\PY{l+s+s1}{\PYZsq{}}\PY{l+s+s1}{round(1234.1234,\PYZhy{}1) \PYZhy{}\PYZgt{}}\PY{l+s+s1}{\PYZsq{}}\PY{p}{,}\PY{n+nb}{round}\PY{p}{(}\PY{l+m+mf}{1234.1234}\PY{p}{,}\PY{o}{\PYZhy{}}\PY{l+m+mi}{1}\PY{p}{)}\PY{p}{)}
\PY{n+nb}{print}\PY{p}{(}\PY{l+s+s1}{\PYZsq{}}\PY{l+s+s1}{round(1234.1234,\PYZhy{}2) \PYZhy{}\PYZgt{}}\PY{l+s+s1}{\PYZsq{}}\PY{p}{,}\PY{n+nb}{round}\PY{p}{(}\PY{l+m+mf}{1234.1234}\PY{p}{,}\PY{o}{\PYZhy{}}\PY{l+m+mi}{2}\PY{p}{)}\PY{p}{)}
\PY{n+nb}{print}\PY{p}{(}\PY{l+s+s1}{\PYZsq{}}\PY{l+s+s1}{round(1234.1234,\PYZhy{}3) \PYZhy{}\PYZgt{}}\PY{l+s+s1}{\PYZsq{}}\PY{p}{,}\PY{n+nb}{round}\PY{p}{(}\PY{l+m+mf}{1234.1234}\PY{p}{,}\PY{o}{\PYZhy{}}\PY{l+m+mi}{3}\PY{p}{)}\PY{p}{)}
\PY{n+nb}{print}\PY{p}{(}\PY{l+s+s1}{\PYZsq{}}\PY{l+s+s1}{round(1234.1234,\PYZhy{}4) \PYZhy{}\PYZgt{}}\PY{l+s+s1}{\PYZsq{}}\PY{p}{,}\PY{n+nb}{round}\PY{p}{(}\PY{l+m+mf}{1234.1234}\PY{p}{,}\PY{o}{\PYZhy{}}\PY{l+m+mi}{4}\PY{p}{)}\PY{p}{)}
\end{Verbatim}
\end{tcolorbox}

    \begin{Verbatim}[commandchars=\\\{\}]
round(1234.1234,4) -> 1234.1234
round(1234.1234,3) -> 1234.123
round(1234.1234,2) -> 1234.12
round(1234.1234,1) -> 1234.1
round(1234.1234,0) -> 1234.0
round(1234.1234,-1) -> 1230.0
round(1234.1234,-2) -> 1200.0
round(1234.1234,-3) -> 1000.0
round(1234.1234,-4) -> 0.0
    \end{Verbatim}

    \hypertarget{input-data}{%
\subsection{Input data}\label{input-data}}

\begin{itemize}
\tightlist
\item
  input(): forward row input to frontends (as a string)
\item
  eval(): evaluate the expression contained in a string
\end{itemize}

    \begin{tcolorbox}[breakable, size=fbox, boxrule=1pt, pad at break*=1mm,colback=cellbackground, colframe=cellborder]
\prompt{In}{incolor}{ }{\boxspacing}
\begin{Verbatim}[commandchars=\\\{\}]
\PY{n}{name} \PY{o}{=} \PY{n+nb}{input}\PY{p}{(}\PY{l+s+s2}{\PYZdq{}}\PY{l+s+s2}{name: }\PY{l+s+s2}{\PYZdq{}}\PY{p}{)}
\PY{n}{age} \PY{o}{=} \PY{n+nb}{input}\PY{p}{(}\PY{l+s+s2}{\PYZdq{}}\PY{l+s+s2}{age: }\PY{l+s+s2}{\PYZdq{}}\PY{p}{)}
\PY{n+nb}{print}\PY{p}{(}\PY{n}{name}\PY{p}{,}\PY{l+s+s2}{\PYZdq{}}\PY{l+s+s2}{is}\PY{l+s+s2}{\PYZdq{}}\PY{p}{,}\PY{n}{age}\PY{p}{,}\PY{l+s+s2}{\PYZdq{}}\PY{l+s+s2}{years old}\PY{l+s+s2}{\PYZdq{}}\PY{p}{)}
\end{Verbatim}
\end{tcolorbox}

    \begin{tcolorbox}[breakable, size=fbox, boxrule=1pt, pad at break*=1mm,colback=cellbackground, colframe=cellborder]
\prompt{In}{incolor}{ }{\boxspacing}
\begin{Verbatim}[commandchars=\\\{\}]
\PY{n}{thisVal} \PY{o}{=} \PY{l+s+s2}{\PYZdq{}}\PY{l+s+se}{\PYZbs{}n}\PY{l+s+s2}{this expression just prints a string}\PY{l+s+se}{\PYZbs{}n}\PY{l+s+s2}{\PYZdq{}}
\PY{n}{expression} \PY{o}{=} \PY{n+nb}{input}\PY{p}{(}\PY{l+s+s2}{\PYZdq{}}\PY{l+s+s2}{expression (e.g., print(thisVal)): }\PY{l+s+s2}{\PYZdq{}}\PY{p}{)}
\PY{n+nb}{eval}\PY{p}{(}\PY{n}{expression}\PY{p}{)}
\end{Verbatim}
\end{tcolorbox}

    \hypertarget{function-definition-in-python}{%
\subsection{Function definition in
python}\label{function-definition-in-python}}

\begin{itemize}
\tightlist
\item
  def {[}return{]} keywords
\item
  non-void functions should be called for the returned value
\item
  void functions should be called to obtain side effects (e.g., print()
  function)
\item
  parentheses follow function's name and contain comma separated
  parameters
\item
  mandatory even if no parameter
\item
  return keywords embeds multiple outputs in a tuple
\item
  functions can embed docstrings for documentation
\item
  scope of a function
\end{itemize}

\begin{Shaded}
\begin{Highlighting}[]
\KeywordTok{def} \OperatorTok{\textless{}}\NormalTok{function\_name}\OperatorTok{\textgreater{}}\NormalTok{(}\OperatorTok{\textless{}}\NormalTok{parameter\_list}\OperatorTok{\textgreater{}}\NormalTok{):}
    \CommentTok{\textquotesingle{}\textquotesingle{}\textquotesingle{}docstring\textquotesingle{}\textquotesingle{}\textquotesingle{}}
    \OperatorTok{\textless{}}\NormalTok{body}\OperatorTok{\textgreater{}}
    \ControlFlowTok{return} \OperatorTok{\textless{}}\NormalTok{result}\OperatorTok{\textgreater{}}
\end{Highlighting}
\end{Shaded}

    \begin{tcolorbox}[breakable, size=fbox, boxrule=1pt, pad at break*=1mm,colback=cellbackground, colframe=cellborder]
\prompt{In}{incolor}{147}{\boxspacing}
\begin{Verbatim}[commandchars=\\\{\}]
\PY{k}{def} \PY{n+nf}{EUR\PYZus{}to\PYZus{}USD}\PY{p}{(}\PY{n}{eur}\PY{p}{)}\PY{p}{:}
    \PY{l+s+sd}{\PYZsq{}\PYZsq{}\PYZsq{}}
\PY{l+s+sd}{    calculate conversion from EUR to USD}
\PY{l+s+sd}{    \PYZsq{}\PYZsq{}\PYZsq{}}
    \PY{n}{usd\PYZus{}for\PYZus{}1eur} \PY{o}{=} \PY{l+m+mf}{1.17}
    \PY{k}{return} \PY{n}{eur}\PY{o}{*}\PY{n}{usd\PYZus{}for\PYZus{}1eur}

\PY{n+nb}{print}\PY{p}{(}\PY{l+s+s2}{\PYZdq{}}\PY{l+s+s2}{10 EUR means}\PY{l+s+s2}{\PYZdq{}}\PY{p}{,}\PY{n}{EUR\PYZus{}to\PYZus{}USD}\PY{p}{(}\PY{l+m+mi}{10}\PY{p}{)}\PY{p}{,}\PY{l+s+s2}{\PYZdq{}}\PY{l+s+s2}{USD}\PY{l+s+s2}{\PYZdq{}}\PY{p}{)}
\PY{n+nb}{print}\PY{p}{(}\PY{l+s+s2}{\PYZdq{}}\PY{l+s+se}{\PYZbs{}n}\PY{l+s+s2}{\PYZhy{}\PYZhy{}\PYZhy{}\PYZhy{}\PYZhy{}\PYZhy{}\PYZhy{}\PYZhy{}\PYZhy{}\PYZhy{}\PYZhy{}\PYZhy{}\PYZhy{}\PYZhy{}\PYZhy{}}\PY{l+s+se}{\PYZbs{}n}\PY{l+s+s2}{\PYZdq{}}\PY{p}{)}
\PY{n}{help}\PY{p}{(}\PY{n}{EUR\PYZus{}to\PYZus{}USD}\PY{p}{)}
\end{Verbatim}
\end{tcolorbox}

    \begin{Verbatim}[commandchars=\\\{\}]
10 EUR means 11.7 USD

---------------

Help on function EUR\_to\_USD in module \_\_main\_\_:

EUR\_to\_USD(eur)
    calculate conversion from EUR to USD

    \end{Verbatim}

    \begin{tcolorbox}[breakable, size=fbox, boxrule=1pt, pad at break*=1mm,colback=cellbackground, colframe=cellborder]
\prompt{In}{incolor}{151}{\boxspacing}
\begin{Verbatim}[commandchars=\\\{\}]
\PY{k}{def} \PY{n+nf}{EUR\PYZus{}to\PYZus{}many}\PY{p}{(}\PY{n}{eur}\PY{p}{)}\PY{p}{:}
    \PY{l+s+sd}{\PYZsq{}\PYZsq{}\PYZsq{}}
\PY{l+s+sd}{    calculate conversion from EUR to different currencies}
\PY{l+s+sd}{    \PYZsq{}\PYZsq{}\PYZsq{}}
    \PY{n}{usd\PYZus{}for\PYZus{}1eur} \PY{o}{=} \PY{l+m+mf}{1.17}
    \PY{n}{yen\PYZus{}for\PYZus{}1eur} \PY{o}{=} \PY{l+m+mf}{0.0075}
    \PY{n}{gbp\PYZus{}for\PYZus{}1eur} \PY{o}{=} \PY{l+m+mf}{1.13} 
    
    \PY{k}{return} \PY{p}{(}\PY{n}{eur}\PY{o}{*}\PY{n}{usd\PYZus{}for\PYZus{}1eur}\PY{p}{,}\PYZbs{}
            \PY{n}{eur}\PY{o}{*}\PY{n}{yen\PYZus{}for\PYZus{}1eur}\PY{p}{,}\PYZbs{}
            \PY{n}{eur}\PY{o}{*}\PY{n}{gbp\PYZus{}for\PYZus{}1eur}\PY{p}{)}

\PY{n}{currencies} \PY{o}{=} \PY{n}{EUR\PYZus{}to\PYZus{}many}\PY{p}{(}\PY{l+m+mi}{10}\PY{p}{)} \PY{c+c1}{\PYZsh{} tuple}
\PY{n}{USD}\PY{p}{,}\PY{n}{YEN}\PY{p}{,}\PY{n}{GBP} \PY{o}{=} \PY{n}{currencies}     \PY{c+c1}{\PYZsh{} unpack tuple}
\PY{n+nb}{print}\PY{p}{(}\PY{l+s+s2}{\PYZdq{}}\PY{l+s+s2}{10 EUR means}\PY{l+s+s2}{\PYZdq{}}\PY{p}{,}\PY{n}{USD}\PY{p}{,}\PY{l+s+s2}{\PYZdq{}}\PY{l+s+s2}{USD}\PY{l+s+se}{\PYZbs{}n}\PY{l+s+se}{\PYZbs{}t}\PY{l+s+s2}{\PYZdq{}}\PY{p}{,}\PYZbs{}
                \PY{l+s+s2}{\PYZdq{}}\PY{l+s+s2}{or}\PY{l+s+s2}{\PYZdq{}}\PY{p}{,}\PY{n}{YEN}\PY{p}{,}\PY{l+s+s2}{\PYZdq{}}\PY{l+s+s2}{YEN}\PY{l+s+se}{\PYZbs{}n}\PY{l+s+se}{\PYZbs{}t}\PY{l+s+s2}{\PYZdq{}}\PY{p}{,}\PYZbs{}
                \PY{l+s+s2}{\PYZdq{}}\PY{l+s+s2}{or}\PY{l+s+s2}{\PYZdq{}}\PY{p}{,}\PY{n}{GBP}\PY{p}{,}\PY{l+s+s2}{\PYZdq{}}\PY{l+s+s2}{GBP}\PY{l+s+s2}{\PYZdq{}}\PY{p}{)}
\PY{n+nb}{print}\PY{p}{(}\PY{l+s+s2}{\PYZdq{}}\PY{l+s+s2}{EUR\PYZus{}to\PYZus{}many returns}\PY{l+s+s2}{\PYZdq{}}\PY{p}{,}\PY{n+nb}{type}\PY{p}{(}\PY{n}{currencies}\PY{p}{)}\PY{p}{)}

\PY{n+nb}{print}\PY{p}{(}\PY{l+s+s2}{\PYZdq{}}\PY{l+s+se}{\PYZbs{}n}\PY{l+s+s2}{\PYZhy{}\PYZhy{}\PYZhy{}\PYZhy{}\PYZhy{}\PYZhy{}\PYZhy{}\PYZhy{}\PYZhy{}\PYZhy{}\PYZhy{}\PYZhy{}\PYZhy{}\PYZhy{}\PYZhy{}}\PY{l+s+se}{\PYZbs{}n}\PY{l+s+s2}{\PYZdq{}}\PY{p}{)}
\PY{n}{help}\PY{p}{(}\PY{n}{EUR\PYZus{}to\PYZus{}many}\PY{p}{)} \PY{c+c1}{\PYZsh{} docstring}
\end{Verbatim}
\end{tcolorbox}

    \begin{Verbatim}[commandchars=\\\{\}]
10 EUR means 11.7 USD
         or 0.075 YEN
         or 11.299999999999999 GBP
EUR\_to\_many returns <class 'tuple'>

---------------

Help on function EUR\_to\_many in module \_\_main\_\_:

EUR\_to\_many(eur)
    calculate conversion from EUR to different currencies

    \end{Verbatim}

    \hypertarget{keyword-and-positional-arguments}{%
\subsection{Keyword and positional
arguments}\label{keyword-and-positional-arguments}}

\begin{itemize}
\tightlist
\item
  default argument values can be passed, even unordered, using keyword
  arguments (named arguments)
\item
  keyword arguments are passed as a dictionary \{key:value,\ldots\}
\item
  not keyword arguments are called positional arguments
\item
  positional arguments should be starred if after a keyword argument
\item
  *expression must evaluate an iterable
\end{itemize}

    \begin{tcolorbox}[breakable, size=fbox, boxrule=1pt, pad at break*=1mm,colback=cellbackground, colframe=cellborder]
\prompt{In}{incolor}{6}{\boxspacing}
\begin{Verbatim}[commandchars=\\\{\}]
\PY{k}{def} \PY{n+nf}{foo}\PY{p}{(}\PY{n}{par1}\PY{o}{=}\PY{k+kc}{None}\PY{p}{,}\PY{n}{par2}\PY{o}{=}\PY{k+kc}{None}\PY{p}{)}\PY{p}{:}
    \PY{n}{res} \PY{o}{=} \PY{k+kc}{None}
    \PY{k}{if} \PY{n}{par1} \PY{o+ow}{is} \PY{l+m+mi}{1}\PY{p}{:}
        \PY{n}{res} \PY{o}{=} \PY{n}{par1}
    \PY{k}{elif} \PY{n}{par2} \PY{o+ow}{is} \PY{l+m+mi}{2}\PY{p}{:}
        \PY{n}{res} \PY{o}{=} \PY{n}{par2}
    \PY{k}{if} \PY{n}{par1} \PY{o+ow}{is} \PY{l+m+mi}{1} \PY{o+ow}{and} \PY{n}{par2} \PY{o+ow}{is} \PY{l+m+mi}{2}\PY{p}{:}
        \PY{n}{res} \PY{o}{=} \PY{l+s+s2}{\PYZdq{}}\PY{l+s+s2}{what a surprise!}\PY{l+s+s2}{\PYZdq{}}
    \PY{k}{return} \PY{n}{res}
\end{Verbatim}
\end{tcolorbox}

    \begin{tcolorbox}[breakable, size=fbox, boxrule=1pt, pad at break*=1mm,colback=cellbackground, colframe=cellborder]
\prompt{In}{incolor}{14}{\boxspacing}
\begin{Verbatim}[commandchars=\\\{\}]
\PY{n}{a} \PY{o}{=} \PY{n}{foo}\PY{p}{(}\PY{n}{par1}\PY{o}{=}\PY{l+m+mi}{1}\PY{p}{)}
\PY{n}{b} \PY{o}{=} \PY{n}{foo}\PY{p}{(}\PY{l+m+mi}{1}\PY{p}{)}
\PY{n+nb}{print}\PY{p}{(}\PY{n}{a}\PY{p}{,}\PY{n}{b}\PY{p}{)}
\end{Verbatim}
\end{tcolorbox}

    \begin{Verbatim}[commandchars=\\\{\}]
1 1
    \end{Verbatim}

    \begin{tcolorbox}[breakable, size=fbox, boxrule=1pt, pad at break*=1mm,colback=cellbackground, colframe=cellborder]
\prompt{In}{incolor}{12}{\boxspacing}
\begin{Verbatim}[commandchars=\\\{\}]
\PY{n}{a} \PY{o}{=} \PY{n}{foo}\PY{p}{(}\PY{n}{par2}\PY{o}{=}\PY{l+m+mi}{2}\PY{p}{)}
\PY{n}{b} \PY{o}{=} \PY{n}{foo}\PY{p}{(}\PY{l+m+mi}{2}\PY{p}{)}
\PY{n+nb}{print}\PY{p}{(}\PY{n}{a}\PY{p}{,}\PY{n}{b}\PY{p}{)}
\end{Verbatim}
\end{tcolorbox}

    \begin{Verbatim}[commandchars=\\\{\}]
2 None
    \end{Verbatim}

    \begin{tcolorbox}[breakable, size=fbox, boxrule=1pt, pad at break*=1mm,colback=cellbackground, colframe=cellborder]
\prompt{In}{incolor}{15}{\boxspacing}
\begin{Verbatim}[commandchars=\\\{\}]
\PY{c+c1}{\PYZsh{} order of keyword args is not important}
\PY{n}{a} \PY{o}{=} \PY{n}{foo}\PY{p}{(}\PY{n}{par1}\PY{o}{=}\PY{l+m+mi}{1}\PY{p}{,}\PY{n}{par2}\PY{o}{=}\PY{l+m+mi}{2}\PY{p}{)}
\PY{n}{b} \PY{o}{=} \PY{n}{foo}\PY{p}{(}\PY{n}{par2}\PY{o}{=}\PY{l+m+mi}{2}\PY{p}{,}\PY{n}{par1}\PY{o}{=}\PY{l+m+mi}{1}\PY{p}{)}
\PY{n+nb}{print}\PY{p}{(}\PY{n}{a}\PY{p}{,}\PY{n}{b}\PY{p}{)}
\end{Verbatim}
\end{tcolorbox}

    \begin{Verbatim}[commandchars=\\\{\}]
what a surprise! what a surprise!
    \end{Verbatim}

    \hypertarget{function-annotations}{%
\subsection{Function annotations}\label{function-annotations}}

\begin{itemize}
\tightlist
\item
  Optional metadata information about the types used by user-defined
  functions
\item
  parameter and return annotations can integrate the docstring
\item
  annotations are stored in the \_\emph{annotations\_} attribute of the
  function as a dictionary
\item
  annotations have no effect on any other part of the function
\end{itemize}

    \begin{tcolorbox}[breakable, size=fbox, boxrule=1pt, pad at break*=1mm,colback=cellbackground, colframe=cellborder]
\prompt{In}{incolor}{30}{\boxspacing}
\begin{Verbatim}[commandchars=\\\{\}]
\PY{k}{def} \PY{n+nf}{division}\PY{p}{(}\PY{n}{a}\PY{p}{:} \PY{l+s+s1}{\PYZsq{}}\PY{l+s+s1}{dividend}\PY{l+s+s1}{\PYZsq{}}\PY{p}{,} \PY{n}{b}\PY{p}{:} \PY{l+s+s1}{\PYZsq{}}\PY{l+s+s1}{divisor (never 0)}\PY{l+s+s1}{\PYZsq{}}\PY{p}{)} \PY{o}{\PYZhy{}}\PY{o}{\PYZgt{}} \PY{l+s+s1}{\PYZsq{}}\PY{l+s+s1}{the result of a/b}\PY{l+s+s1}{\PYZsq{}}\PY{p}{:} 
    \PY{l+s+sd}{\PYZdq{}\PYZdq{}\PYZdq{}Divide a by b\PYZdq{}\PYZdq{}\PYZdq{}} \PY{c+c1}{\PYZsh{} docstring}
    \PY{k}{return} \PY{n}{a} \PY{o}{/} \PY{n}{b} 

\PY{n+nb}{print}\PY{p}{(}\PY{n}{division}\PY{p}{(}\PY{l+m+mi}{1}\PY{p}{,}\PY{l+m+mi}{1}\PY{p}{)}\PY{p}{)}
\PY{n+nb}{print}\PY{p}{(}\PY{n}{division}\PY{o}{.}\PY{n+nv+vm}{\PYZus{}\PYZus{}annotations\PYZus{}\PYZus{}}\PY{p}{)}


\PY{n+nb}{print}\PY{p}{(}\PY{l+s+s2}{\PYZdq{}}\PY{l+s+se}{\PYZbs{}n}\PY{l+s+se}{\PYZbs{}n}\PY{l+s+s2}{\PYZdq{}}\PY{p}{)}
\PY{n}{help}\PY{p}{(}\PY{n}{division}\PY{p}{)}
\end{Verbatim}
\end{tcolorbox}

    \begin{Verbatim}[commandchars=\\\{\}]
1.0
\{'b': 'divisor (never 0)', 'return': 'the result of a/b', 'a': 'dividend'\}



Help on function division in module \_\_main\_\_:

division(a:'dividend', b:'divisor (never 0)') -> 'the result of a/b'
    Divide a by b

    \end{Verbatim}

    \hypertarget{common-compound-data-structure-literals}{%
\subsection{Common compound data structure
literals}\label{common-compound-data-structure-literals}}

\begin{itemize}
\item
  sequence types
\item
  tuples: (1,2,3)
\item
  lists: {[}1,2,3{]}
\item
  range() - explained with loop statements
\item
  set types
\item
  sets: \{1,2,3\}
\item
  mapping type
\item
  dictionaries: \{1:`one',2:`two',3:`three'\}
\end{itemize}

    \hypertarget{tuples}{%
\subsection{Tuples}\label{tuples}}

\begin{itemize}
\tightlist
\item
  tuples store a collection of ordered objects
\item
  tuples are meant to store data and are immutable (do not support item
  assignment)
\item
  no add, remove or or replace on the fly
\item
  immutability is a feature, not a restriction
\item
  tuples can be used to assign simultaneously multiple objects
\end{itemize}

    \begin{tcolorbox}[breakable, size=fbox, boxrule=1pt, pad at break*=1mm,colback=cellbackground, colframe=cellborder]
\prompt{In}{incolor}{8}{\boxspacing}
\begin{Verbatim}[commandchars=\\\{\}]
\PY{n}{t} \PY{o}{=} \PY{p}{(}\PY{l+m+mi}{1}\PY{p}{,}\PY{l+m+mf}{1.}\PY{p}{,}\PY{l+s+s2}{\PYZdq{}}\PY{l+s+s2}{str}\PY{l+s+s2}{\PYZdq{}}\PY{p}{,}\PY{p}{(}\PY{l+s+s2}{\PYZdq{}}\PY{l+s+s2}{another}\PY{l+s+s2}{\PYZdq{}}\PY{p}{,}\PY{l+s+s2}{\PYZdq{}}\PY{l+s+s2}{tuple}\PY{l+s+s2}{\PYZdq{}}\PY{p}{)}\PY{p}{)}  \PY{c+c1}{\PYZsh{} tuple packing}
\PY{n}{v1}\PY{p}{,}\PY{n}{v2}\PY{p}{,}\PY{n}{v3}\PY{p}{,}\PY{n}{v4} \PY{o}{=} \PY{n}{t}  \PY{c+c1}{\PYZsh{} tuple un\PYZhy{}packing}

\PY{n+nb}{print}\PY{p}{(}\PY{l+s+s2}{\PYZdq{}}\PY{l+s+se}{\PYZbs{}n}\PY{l+s+s2}{simultaneous assignment}\PY{l+s+s2}{\PYZdq{}}\PY{p}{)}
\PY{n+nb}{print}\PY{p}{(}\PY{l+s+s2}{\PYZdq{}}\PY{l+s+s2}{t = (1,1.,}\PY{l+s+se}{\PYZbs{}\PYZdq{}}\PY{l+s+s2}{str}\PY{l+s+se}{\PYZbs{}\PYZdq{}}\PY{l+s+s2}{,(}\PY{l+s+se}{\PYZbs{}\PYZdq{}}\PY{l+s+s2}{another}\PY{l+s+se}{\PYZbs{}\PYZdq{}}\PY{l+s+s2}{,}\PY{l+s+se}{\PYZbs{}\PYZdq{}}\PY{l+s+s2}{tuple}\PY{l+s+se}{\PYZbs{}\PYZdq{}}\PY{l+s+s2}{))}\PY{l+s+s2}{\PYZdq{}}\PY{p}{,}\PY{n+nb}{type}\PY{p}{(}\PY{n}{t}\PY{p}{)}\PY{p}{)}
\PY{n+nb}{print}\PY{p}{(}\PY{l+s+s2}{\PYZdq{}}\PY{l+s+se}{\PYZbs{}n}\PY{l+s+s2}{simultaneous assignment}\PY{l+s+s2}{\PYZdq{}}\PY{p}{)}
\PY{n+nb}{print}\PY{p}{(}\PY{l+s+s2}{\PYZdq{}}\PY{l+s+s2}{v1,v2,v3,v4 = t}\PY{l+s+s2}{\PYZdq{}}\PY{p}{)}
\PY{n+nb}{print}\PY{p}{(}\PY{l+s+s2}{\PYZdq{}}\PY{l+s+s2}{\PYZhy{}\PYZhy{}\PYZgt{}}\PY{l+s+se}{\PYZbs{}n}\PY{l+s+s2}{v1 =}\PY{l+s+s2}{\PYZdq{}}\PY{p}{,}\PY{n}{v1}\PY{p}{,}\PY{n+nb}{type}\PY{p}{(}\PY{n}{v1}\PY{p}{)}\PY{p}{,}\PY{l+s+s2}{\PYZdq{}}\PY{l+s+se}{\PYZbs{}n}\PY{l+s+s2}{v2 =}\PY{l+s+s2}{\PYZdq{}}\PY{p}{,}\PY{n}{v2}\PY{p}{,}\PY{n+nb}{type}\PY{p}{(}\PY{n}{v2}\PY{p}{)}\PY{p}{,}\PY{l+s+s2}{\PYZdq{}}\PY{l+s+se}{\PYZbs{}n}\PY{l+s+s2}{v3 =}\PY{l+s+s2}{\PYZdq{}}\PY{p}{,}\PY{n}{v3}\PY{p}{,}\PY{n+nb}{type}\PY{p}{(}\PY{n}{v3}\PY{p}{)}\PY{p}{,}\PY{l+s+s2}{\PYZdq{}}\PY{l+s+se}{\PYZbs{}n}\PY{l+s+s2}{v4 =}\PY{l+s+s2}{\PYZdq{}}\PY{p}{,}\PY{n}{v4}\PY{p}{,}\PY{n+nb}{type}\PY{p}{(}\PY{n}{v4}\PY{p}{)}\PY{p}{)}
\end{Verbatim}
\end{tcolorbox}

    \begin{Verbatim}[commandchars=\\\{\}]

simultaneous assignment
t = (1,1.,"str",("another","tuple")) <class 'tuple'>

simultaneous assignment
v1,v2,v3,v4 = t
-->
v1 = 1 <class 'int'>
v2 = 1.0 <class 'float'>
v3 = str <class 'str'>
v4 = ('another', 'tuple') <class 'tuple'>
    \end{Verbatim}

    \begin{tcolorbox}[breakable, size=fbox, boxrule=1pt, pad at break*=1mm,colback=cellbackground, colframe=cellborder]
\prompt{In}{incolor}{9}{\boxspacing}
\begin{Verbatim}[commandchars=\\\{\}]
\PY{n+nb}{print}\PY{p}{(}\PY{l+s+s2}{\PYZdq{}}\PY{l+s+s2}{tuples are immutable}\PY{l+s+s2}{\PYZdq{}}\PY{p}{)}
\PY{n}{t}\PY{p}{[}\PY{l+m+mi}{0}\PY{p}{]} \PY{o}{=} \PY{l+m+mi}{2}
\end{Verbatim}
\end{tcolorbox}

    \begin{Verbatim}[commandchars=\\\{\}]
tuples are immutable
    \end{Verbatim}

    \begin{Verbatim}[commandchars=\\\{\}]

        ---------------------------------------------------------------------------

        TypeError                                 Traceback (most recent call last)

        <ipython-input-9-3004b1da0aeb> in <module>()
          1 print("tuples are immutable")
    ----> 2 t[0] = 2
    

        TypeError: 'tuple' object does not support item assignment

    \end{Verbatim}

    \hypertarget{how-to-swap-the-contents-of-two-variables-in-a-pythonic-way}{%
\subsection{How to swap the contents of two variables in a pythonic
way?}\label{how-to-swap-the-contents-of-two-variables-in-a-pythonic-way}}

\hypertarget{hint-one-line}{%
\subsubsection{Hint: one line}\label{hint-one-line}}

    \begin{tcolorbox}[breakable, size=fbox, boxrule=1pt, pad at break*=1mm,colback=cellbackground, colframe=cellborder]
\prompt{In}{incolor}{10}{\boxspacing}
\begin{Verbatim}[commandchars=\\\{\}]
\PY{n}{a} \PY{o}{=} \PY{l+m+mi}{3}
\PY{n}{b} \PY{o}{=} \PY{l+m+mi}{7}
\PY{l+s+sd}{\PYZsq{}\PYZsq{}\PYZsq{}}
\PY{l+s+sd}{swap the content of a and b such that a contains 7 and b contains 3}
\PY{l+s+sd}{\PYZsq{}\PYZsq{}\PYZsq{}}

\PY{n+nb}{print} \PY{p}{(}\PY{n}{a}\PY{p}{,}\PY{n}{b}\PY{p}{)}
\end{Verbatim}
\end{tcolorbox}

    \begin{Verbatim}[commandchars=\\\{\}]
3 7
    \end{Verbatim}

    \hypertarget{lists}{%
\subsection{Lists}\label{lists}}

\begin{itemize}
\tightlist
\item
  a list collects comma separated values between squared brackets
\item
  len(list) or list.\_\emph{len\_}() gives the list length
\item
  lists are mutable
\item
  a list can indexed and sliced (using ``:'')
\item
  a list can contain any data type
\item
  the same list can contain different data types (e.g., lists can nest
  lists)
\item
  support concatenation
\end{itemize}

    \begin{tcolorbox}[breakable, size=fbox, boxrule=1pt, pad at break*=1mm,colback=cellbackground, colframe=cellborder]
\prompt{In}{incolor}{17}{\boxspacing}
\begin{Verbatim}[commandchars=\\\{\}]
\PY{n}{squares} \PY{o}{=} \PY{p}{[}\PY{l+m+mi}{1}\PY{p}{,} \PY{l+m+mi}{4}\PY{p}{,} \PY{l+m+mi}{9}\PY{p}{,} \PY{l+m+mi}{16}\PY{p}{,} \PY{l+m+mi}{24}\PY{p}{]} \PY{c+c1}{\PYZsh{} 24 should be 25 }
\PY{n+nb}{print}\PY{p}{(}\PY{l+s+s2}{\PYZdq{}}\PY{l+s+s2}{squares \PYZhy{}\PYZgt{}}\PY{l+s+s2}{\PYZdq{}}\PY{p}{,} \PY{n}{squares}\PY{p}{)} 
\PY{n+nb}{print}\PY{p}{(}\PY{l+s+s2}{\PYZdq{}}\PY{l+s+s2}{length of square is}\PY{l+s+s2}{\PYZdq{}}\PY{p}{,} \PY{n}{squares}\PY{o}{.}\PY{n+nf+fm}{\PYZus{}\PYZus{}len\PYZus{}\PYZus{}}\PY{p}{(}\PY{p}{)}\PY{p}{)}
\PY{n+nb}{print}\PY{p}{(}\PY{l+s+s2}{\PYZdq{}}\PY{l+s+se}{\PYZbs{}n}\PY{l+s+s2}{\PYZsh{} indexing returns the item}\PY{l+s+s2}{\PYZdq{}}\PY{p}{)}
\PY{n+nb}{print}\PY{p}{(}\PY{l+s+s2}{\PYZdq{}}\PY{l+s+s2}{squares[0] \PYZhy{}\PYZgt{}}\PY{l+s+s2}{\PYZdq{}}\PY{p}{,}\PY{n}{squares}\PY{p}{[}\PY{l+m+mi}{0}\PY{p}{]}\PY{p}{,}\PY{n+nb}{type}\PY{p}{(}\PY{n}{squares}\PY{p}{[}\PY{l+m+mi}{0}\PY{p}{]}\PY{p}{)}\PY{p}{)} 
\PY{n+nb}{print}\PY{p}{(}\PY{l+s+s2}{\PYZdq{}}\PY{l+s+s2}{squares[\PYZhy{}1] \PYZhy{}\PYZgt{} }\PY{l+s+s2}{\PYZdq{}}\PY{p}{,}\PY{n}{squares}\PY{p}{[}\PY{o}{\PYZhy{}}\PY{l+m+mi}{1}\PY{p}{]}\PY{p}{,}\PY{n+nb}{type}\PY{p}{(}\PY{n}{squares}\PY{p}{[}\PY{o}{\PYZhy{}}\PY{l+m+mi}{1}\PY{p}{]}\PY{p}{)}\PY{p}{)}   

\PY{n+nb}{print}\PY{p}{(}\PY{l+s+s2}{\PYZdq{}}\PY{l+s+se}{\PYZbs{}n}\PY{l+s+s2}{\PYZsh{} slicing returns a list of items}\PY{l+s+s2}{\PYZdq{}}\PY{p}{)}
\PY{n+nb}{print}\PY{p}{(}\PY{l+s+s2}{\PYZdq{}}\PY{l+s+s2}{squares[2:] \PYZhy{}\PYZgt{}}\PY{l+s+s2}{\PYZdq{}}\PY{p}{,}\PY{n}{squares}\PY{p}{[}\PY{l+m+mi}{2}\PY{p}{:}\PY{p}{]}\PY{p}{,}\PY{n+nb}{type}\PY{p}{(}\PY{n}{squares}\PY{p}{[}\PY{l+m+mi}{2}\PY{p}{:}\PY{p}{]}\PY{p}{)}\PY{p}{)} \PY{c+c1}{\PYZsh{} list of idx 2,3,...,len(list)}
\PY{n+nb}{print}\PY{p}{(}\PY{l+s+s2}{\PYZdq{}}\PY{l+s+s2}{squares[:3] \PYZhy{}\PYZgt{}}\PY{l+s+s2}{\PYZdq{}}\PY{p}{,}\PY{n}{squares}\PY{p}{[}\PY{p}{:}\PY{l+m+mi}{3}\PY{p}{]}\PY{p}{,}\PY{n+nb}{type}\PY{p}{(}\PY{n}{squares}\PY{p}{[}\PY{p}{:}\PY{l+m+mi}{3}\PY{p}{]}\PY{p}{)}\PY{p}{)} \PY{c+c1}{\PYZsh{} list of idx 0,1,2}
\PY{n+nb}{print}\PY{p}{(}\PY{l+s+s2}{\PYZdq{}}\PY{l+s+s2}{squares[0:1] \PYZhy{}\PYZgt{}}\PY{l+s+s2}{\PYZdq{}}\PY{p}{,}\PY{n}{squares}\PY{p}{[}\PY{l+m+mi}{0}\PY{p}{:}\PY{l+m+mi}{1}\PY{p}{]}\PY{p}{,}\PY{n+nb}{type}\PY{p}{(}\PY{n}{squares}\PY{p}{[}\PY{l+m+mi}{0}\PY{p}{:}\PY{l+m+mi}{1}\PY{p}{]}\PY{p}{)}\PY{p}{)}  
\PY{n+nb}{print}\PY{p}{(}\PY{l+s+s2}{\PYZdq{}}\PY{l+s+s2}{squares[\PYZhy{}3:] \PYZhy{}\PYZgt{} }\PY{l+s+s2}{\PYZdq{}}\PY{p}{,}\PY{n}{squares}\PY{p}{[}\PY{o}{\PYZhy{}}\PY{l+m+mi}{3}\PY{p}{:}\PY{p}{]}\PY{p}{,}\PY{n+nb}{type}\PY{p}{(}\PY{n}{squares}\PY{p}{[}\PY{o}{\PYZhy{}}\PY{l+m+mi}{3}\PY{p}{:}\PY{p}{]}\PY{p}{)}\PY{p}{)} \PY{c+c1}{\PYZsh{} list of last three elements}
\end{Verbatim}
\end{tcolorbox}

    \begin{Verbatim}[commandchars=\\\{\}]
squares -> [1, 4, 9, 16, 24]
length of square is 5

\# indexing returns the item
squares[0] -> 1 <class 'int'>
squares[-1] ->  24 <class 'int'>

\# slicing returns a list of items
squares[2:] -> [9, 16, 24] <class 'list'>
squares[:3] -> [1, 4, 9] <class 'list'>
squares[0:1] -> [1] <class 'list'>
squares[-3:] ->  [9, 16, 24] <class 'list'>
    \end{Verbatim}

    \begin{tcolorbox}[breakable, size=fbox, boxrule=1pt, pad at break*=1mm,colback=cellbackground, colframe=cellborder]
\prompt{In}{incolor}{19}{\boxspacing}
\begin{Verbatim}[commandchars=\\\{\}]
\PY{n}{squares} \PY{o}{=} \PY{p}{[}\PY{l+m+mi}{1}\PY{p}{,} \PY{l+m+mi}{4}\PY{p}{,} \PY{l+m+mi}{9}\PY{p}{,} \PY{l+m+mi}{16}\PY{p}{,} \PY{l+m+mi}{24}\PY{p}{]} \PY{c+c1}{\PYZsh{} 24 should be 25 }
\PY{n+nb}{print}\PY{p}{(}\PY{l+s+s2}{\PYZdq{}}\PY{l+s+se}{\PYZbs{}n}\PY{l+s+s2}{\PYZsh{} new values can be assigned both using indexing and slicing}\PY{l+s+s2}{\PYZdq{}}\PY{p}{)}
\PY{n+nb}{print}\PY{p}{(}\PY{l+s+s2}{\PYZdq{}}\PY{l+s+s2}{squares \PYZhy{}\PYZgt{}}\PY{l+s+s2}{\PYZdq{}}\PY{p}{,}\PY{n}{squares}\PY{p}{)}
\PY{n}{squares}\PY{p}{[}\PY{l+m+mi}{4}\PY{p}{]} \PY{o}{=} \PY{l+m+mi}{25} 
\PY{n+nb}{print}\PY{p}{(}\PY{l+s+s2}{\PYZdq{}}\PY{l+s+s2}{squares[4] = 25 }\PY{l+s+s2}{\PYZdq{}}\PY{p}{)}
\PY{n+nb}{print}\PY{p}{(}\PY{l+s+s2}{\PYZdq{}}\PY{l+s+s2}{squares \PYZhy{}\PYZgt{}}\PY{l+s+s2}{\PYZdq{}}\PY{p}{,}\PY{n}{squares}\PY{p}{)}
\PY{n+nb}{print}\PY{p}{(}\PY{l+s+s2}{\PYZdq{}}\PY{l+s+s2}{squares[0:2] = [\PYZhy{}1,\PYZhy{}1]}\PY{l+s+s2}{\PYZdq{}}\PY{p}{)}
\PY{n}{squares}\PY{p}{[}\PY{l+m+mi}{0}\PY{p}{:}\PY{l+m+mi}{2}\PY{p}{]} \PY{o}{=} \PY{p}{[}\PY{o}{\PYZhy{}}\PY{l+m+mi}{1}\PY{p}{,}\PY{o}{\PYZhy{}}\PY{l+m+mi}{1}\PY{p}{]} 
\PY{n+nb}{print}\PY{p}{(}\PY{l+s+s2}{\PYZdq{}}\PY{l+s+s2}{squares \PYZhy{}\PYZgt{}}\PY{l+s+s2}{\PYZdq{}}\PY{p}{,}\PY{n}{squares}\PY{p}{)}

\PY{n+nb}{print}\PY{p}{(}\PY{l+s+s2}{\PYZdq{}}\PY{l+s+se}{\PYZbs{}n}\PY{l+s+s2}{\PYZsh{} NB: assignment does not need consistency between lengths of slices}\PY{l+s+s2}{\PYZdq{}}\PY{p}{)}
\PY{n+nb}{print}\PY{p}{(}\PY{l+s+s2}{\PYZdq{}}\PY{l+s+s2}{squares[0:1] = [}\PY{l+s+s2}{\PYZsq{}}\PY{l+s+s2}{what?}\PY{l+s+s2}{\PYZsq{}}\PY{l+s+s2}{,squares[0]]}\PY{l+s+s2}{\PYZdq{}}\PY{p}{)}
\PY{n}{squares}\PY{p}{[}\PY{l+m+mi}{0}\PY{p}{:}\PY{l+m+mi}{1}\PY{p}{]} \PY{o}{=} \PY{p}{[}\PY{l+s+s1}{\PYZsq{}}\PY{l+s+s1}{what?}\PY{l+s+s1}{\PYZsq{}}\PY{p}{,}\PY{n}{squares}\PY{p}{[}\PY{l+m+mi}{0}\PY{p}{]}\PY{p}{]}
\PY{n+nb}{print}\PY{p}{(}\PY{l+s+s2}{\PYZdq{}}\PY{l+s+s2}{squares \PYZhy{}\PYZgt{}}\PY{l+s+s2}{\PYZdq{}}\PY{p}{,} \PY{n}{squares}\PY{p}{)}
\PY{n+nb}{print}\PY{p}{(}\PY{l+s+s2}{\PYZdq{}}\PY{l+s+s2}{squares[0:3] \PYZhy{}\PYZgt{}}\PY{l+s+s2}{\PYZdq{}}\PY{p}{,} \PY{n}{squares}\PY{p}{[}\PY{l+m+mi}{0}\PY{p}{:}\PY{l+m+mi}{3}\PY{p}{]}\PY{p}{)}
\PY{n+nb}{print}\PY{p}{(}\PY{l+s+s2}{\PYZdq{}}\PY{l+s+s2}{squares[0:3] = [1,4]}\PY{l+s+s2}{\PYZdq{}}\PY{p}{)}
\PY{n}{squares}\PY{p}{[}\PY{l+m+mi}{0}\PY{p}{:}\PY{l+m+mi}{3}\PY{p}{]} \PY{o}{=} \PY{p}{[}\PY{l+m+mi}{1}\PY{p}{,}\PY{l+m+mi}{4}\PY{p}{]}
\PY{n+nb}{print}\PY{p}{(}\PY{l+s+s2}{\PYZdq{}}\PY{l+s+s2}{squares \PYZhy{}\PYZgt{}}\PY{l+s+s2}{\PYZdq{}}\PY{p}{,} \PY{n}{squares}\PY{p}{)}
\end{Verbatim}
\end{tcolorbox}

    \begin{Verbatim}[commandchars=\\\{\}]

\# new values can be assigned both using indexing and slicing
squares -> [1, 4, 9, 16, 24]
squares[4] = 25
squares -> [1, 4, 9, 16, 25]
squares[0:2] = [-1,-1]
squares -> [-1, -1, 9, 16, 25]

\# NB: assignment does not need consistency between lengths of slices
squares[0:1] = ['what?',squares[0]]
squares -> ['what?', -1, -1, 9, 16, 25]
squares[0:3] -> ['what?', -1, -1]
squares[0:3] = [1,4]
squares -> [1, 4, 9, 16, 25]
    \end{Verbatim}

    \begin{tcolorbox}[breakable, size=fbox, boxrule=1pt, pad at break*=1mm,colback=cellbackground, colframe=cellborder]
\prompt{In}{incolor}{20}{\boxspacing}
\begin{Verbatim}[commandchars=\\\{\}]
\PY{n}{squares} \PY{o}{=} \PY{p}{[}\PY{l+m+mi}{1}\PY{p}{,} \PY{l+m+mi}{4}\PY{p}{,} \PY{l+m+mi}{9}\PY{p}{,} \PY{l+m+mi}{16}\PY{p}{,} \PY{l+m+mi}{25}\PY{p}{]}
\PY{n}{other\PYZus{}squares} \PY{o}{=} \PY{p}{[}\PY{l+m+mi}{36}\PY{p}{,}\PY{l+m+mi}{49}\PY{p}{,}\PY{l+m+mi}{64}\PY{p}{]}
\PY{n+nb}{print}\PY{p}{(}\PY{l+s+s2}{\PYZdq{}}\PY{l+s+se}{\PYZbs{}n}\PY{l+s+s2}{\PYZsh{} lists can be concatenated}\PY{l+s+s2}{\PYZdq{}}\PY{p}{)}
\PY{n+nb}{print}\PY{p}{(}\PY{l+s+s2}{\PYZdq{}}\PY{l+s+s2}{squares \PYZhy{}\PYZgt{} }\PY{l+s+s2}{\PYZdq{}}\PY{p}{,}\PY{n}{squares}\PY{p}{)}
\PY{n+nb}{print}\PY{p}{(}\PY{l+s+s2}{\PYZdq{}}\PY{l+s+s2}{other\PYZus{}squares \PYZhy{}\PYZgt{} }\PY{l+s+s2}{\PYZdq{}}\PY{p}{,}\PY{n}{other\PYZus{}squares}\PY{p}{)}
\PY{n+nb}{print}\PY{p}{(}\PY{l+s+s2}{\PYZdq{}}\PY{l+s+s2}{squares+other\PYZus{}squares \PYZhy{}\PYZgt{} }\PY{l+s+s2}{\PYZdq{}}\PY{p}{,}\PY{n}{squares}\PY{o}{+}\PY{n}{other\PYZus{}squares}\PY{p}{)}

\PY{n+nb}{print}\PY{p}{(}\PY{l+s+s2}{\PYZdq{}}\PY{l+s+se}{\PYZbs{}n}\PY{l+s+s2}{\PYZsh{} new values can be appended to list}\PY{l+s+s2}{\PYZdq{}}\PY{p}{)}
\PY{n+nb}{print}\PY{p}{(}\PY{l+s+s2}{\PYZdq{}}\PY{l+s+s2}{squares \PYZhy{}\PYZgt{} }\PY{l+s+s2}{\PYZdq{}}\PY{p}{,}\PY{n}{squares}\PY{p}{)}
\PY{n+nb}{print}\PY{p}{(}\PY{l+s+s2}{\PYZdq{}}\PY{l+s+s2}{squares.append(36)}\PY{l+s+s2}{\PYZdq{}}\PY{p}{)}
\PY{n}{squares}\PY{o}{.}\PY{n}{append}\PY{p}{(}\PY{l+m+mi}{36}\PY{p}{)}
\PY{n+nb}{print}\PY{p}{(}\PY{l+s+s2}{\PYZdq{}}\PY{l+s+s2}{squares \PYZhy{}\PYZgt{} }\PY{l+s+s2}{\PYZdq{}}\PY{p}{,}\PY{n}{squares}\PY{p}{)}
\end{Verbatim}
\end{tcolorbox}

    \begin{Verbatim}[commandchars=\\\{\}]

\# lists can be concatenated
squares ->  [1, 4, 9, 16, 25]
other\_squares ->  [36, 49, 64]
squares+other\_squares ->  [1, 4, 9, 16, 25, 36, 49, 64]

\# new values can be appended to list
squares ->  [1, 4, 9, 16, 25]
squares.append(36)
squares ->  [1, 4, 9, 16, 25, 36]
    \end{Verbatim}

    \begin{tcolorbox}[breakable, size=fbox, boxrule=1pt, pad at break*=1mm,colback=cellbackground, colframe=cellborder]
\prompt{In}{incolor}{21}{\boxspacing}
\begin{Verbatim}[commandchars=\\\{\}]
\PY{n+nb}{print}\PY{p}{(}\PY{l+s+s2}{\PYZdq{}}\PY{l+s+se}{\PYZbs{}n}\PY{l+s+s2}{\PYZsh{} values can be removed just by slicing}\PY{l+s+s2}{\PYZdq{}}\PY{p}{)}
\PY{n+nb}{print}\PY{p}{(}\PY{l+s+s2}{\PYZdq{}}\PY{l+s+s2}{squares[5:6] = []}\PY{l+s+s2}{\PYZdq{}}\PY{p}{)}
\PY{n}{squares}\PY{p}{[}\PY{o}{\PYZhy{}}\PY{l+m+mi}{3}\PY{p}{:}\PY{p}{]} \PY{o}{=} \PY{p}{[}\PY{p}{]}
\PY{n+nb}{print}\PY{p}{(}\PY{l+s+s2}{\PYZdq{}}\PY{l+s+s2}{squares \PYZhy{}\PYZgt{} }\PY{l+s+s2}{\PYZdq{}}\PY{p}{,}\PY{n}{squares}\PY{p}{)}

\PY{n+nb}{print}\PY{p}{(}\PY{l+s+s2}{\PYZdq{}}\PY{l+s+se}{\PYZbs{}n}\PY{l+s+s2}{\PYZsh{} clear a list replacing with empty list}\PY{l+s+s2}{\PYZdq{}}\PY{p}{)}
\PY{n+nb}{print}\PY{p}{(}\PY{l+s+s2}{\PYZdq{}}\PY{l+s+s2}{squares[:] = []}\PY{l+s+s2}{\PYZdq{}}\PY{p}{)}
\PY{n}{squares}\PY{p}{[}\PY{p}{:}\PY{p}{]} \PY{o}{=} \PY{p}{[}\PY{p}{]}
\PY{n+nb}{print}\PY{p}{(}\PY{l+s+s2}{\PYZdq{}}\PY{l+s+s2}{squares \PYZhy{}\PYZgt{} }\PY{l+s+s2}{\PYZdq{}}\PY{p}{,}\PY{n}{squares}\PY{p}{)}
\end{Verbatim}
\end{tcolorbox}

    \begin{Verbatim}[commandchars=\\\{\}]

\# values can be removed just by slicing
squares[5:6] = []
squares ->  [1, 4, 9]

\# clear a list replacing with empty list
squares[:] = []
squares ->  []
    \end{Verbatim}

    \begin{tcolorbox}[breakable, size=fbox, boxrule=1pt, pad at break*=1mm,colback=cellbackground, colframe=cellborder]
\prompt{In}{incolor}{22}{\boxspacing}
\begin{Verbatim}[commandchars=\\\{\}]
\PY{n+nb}{print}\PY{p}{(}\PY{l+s+s2}{\PYZdq{}}\PY{l+s+se}{\PYZbs{}n}\PY{l+s+s2}{\PYZsh{} nested list}\PY{l+s+s2}{\PYZdq{}}\PY{p}{)}
\PY{n+nb}{print}\PY{p}{(}\PY{l+s+s2}{\PYZdq{}}\PY{l+s+s2}{l = [[}\PY{l+s+s2}{\PYZsq{}}\PY{l+s+s2}{a}\PY{l+s+s2}{\PYZsq{}}\PY{l+s+s2}{,}\PY{l+s+s2}{\PYZsq{}}\PY{l+s+s2}{b}\PY{l+s+s2}{\PYZsq{}}\PY{l+s+s2}{],[1,2]]}\PY{l+s+s2}{\PYZdq{}}\PY{p}{)}
\PY{n}{l} \PY{o}{=} \PY{p}{[}\PY{p}{[}\PY{l+s+s1}{\PYZsq{}}\PY{l+s+s1}{a}\PY{l+s+s1}{\PYZsq{}}\PY{p}{,}\PY{l+s+s1}{\PYZsq{}}\PY{l+s+s1}{b}\PY{l+s+s1}{\PYZsq{}}\PY{p}{]}\PY{p}{,}\PY{p}{[}\PY{l+m+mi}{1}\PY{p}{,}\PY{l+m+mi}{2}\PY{p}{]}\PY{p}{]}
\PY{n+nb}{print}\PY{p}{(}\PY{l+s+s2}{\PYZdq{}}\PY{l+s+s2}{l[0][1] \PYZhy{}\PYZgt{}}\PY{l+s+s2}{\PYZdq{}}\PY{p}{,}\PY{n}{l}\PY{p}{[}\PY{l+m+mi}{0}\PY{p}{]}\PY{p}{[}\PY{l+m+mi}{1}\PY{p}{]}\PY{p}{)}
\PY{n+nb}{print}\PY{p}{(}\PY{l+s+s2}{\PYZdq{}}\PY{l+s+s2}{l[1][0] \PYZhy{}\PYZgt{}}\PY{l+s+s2}{\PYZdq{}}\PY{p}{,}\PY{n}{l}\PY{p}{[}\PY{l+m+mi}{1}\PY{p}{]}\PY{p}{[}\PY{l+m+mi}{0}\PY{p}{]}\PY{p}{)}
\end{Verbatim}
\end{tcolorbox}

    \begin{Verbatim}[commandchars=\\\{\}]

\# nested list
l = [['a','b'],[1,2]]
l[0][1] -> b
l[1][0] -> 1
    \end{Verbatim}

    \hypertarget{shallow-copy-vs-deep-copy}{%
\subsection{Shallow copy vs Deep copy}\label{shallow-copy-vs-deep-copy}}

\begin{itemize}
\item
  shallow copy: the target is bind to the assigned object (i.e., they
  have the same reference)
\item
  deep copy: the target contains the same values of the assigned object
  (i.e., they do not have the same reference)
\item
  python copies the reference
\item
  there is a difference for compound objects (i.e., objects that contain
  other objects, like nested lists)
\item
  module copy contains the implementation for deep copy
\item
  copy.copy(x): returns a shallow copy of x
\item
  copy.deepcopy(x): returns a deep copy of x
\end{itemize}

    \begin{tcolorbox}[breakable, size=fbox, boxrule=1pt, pad at break*=1mm,colback=cellbackground, colframe=cellborder]
\prompt{In}{incolor}{23}{\boxspacing}
\begin{Verbatim}[commandchars=\\\{\}]
\PY{o}{\PYZpc{}}\PY{k}{reset} \PYZhy{}f
\PY{k+kn}{import} \PY{n+nn}{copy}

\PY{n}{l\PYZus{}0} \PY{o}{=} \PY{p}{[}\PY{l+m+mi}{1}\PY{p}{,}\PY{l+m+mi}{2}\PY{p}{]}
\PY{n}{l\PYZus{}1} \PY{o}{=} \PY{p}{[}\PY{n}{l\PYZus{}0}\PY{p}{,}\PY{l+s+s1}{\PYZsq{}}\PY{l+s+s1}{3}\PY{l+s+s1}{\PYZsq{}}\PY{p}{]}          \PY{c+c1}{\PYZsh{} shallow copy}
\PY{n}{l\PYZus{}2} \PY{o}{=} \PY{n}{copy}\PY{o}{.}\PY{n}{copy}\PY{p}{(}\PY{n}{l\PYZus{}1}\PY{p}{)}     \PY{c+c1}{\PYZsh{} shallow copy}
\PY{n}{l\PYZus{}3} \PY{o}{=} \PY{n}{copy}\PY{o}{.}\PY{n}{deepcopy}\PY{p}{(}\PY{n}{l\PYZus{}1}\PY{p}{)} \PY{c+c1}{\PYZsh{} deep copy}

\PY{n}{l\PYZus{}0}\PY{p}{[}\PY{l+m+mi}{0}\PY{p}{]} \PY{o}{=} \PY{l+s+s1}{\PYZsq{}}\PY{l+s+s1}{changes!}\PY{l+s+s1}{\PYZsq{}}

\PY{n+nb}{print}\PY{p}{(}\PY{n}{l\PYZus{}1}\PY{p}{)}
\PY{n+nb}{print}\PY{p}{(}\PY{n}{l\PYZus{}2}\PY{p}{,} \PY{l+s+s2}{\PYZdq{}}\PY{l+s+s2}{\PYZlt{}\PYZhy{} shallow copy}\PY{l+s+s2}{\PYZdq{}}\PY{p}{)}
\PY{n+nb}{print}\PY{p}{(}\PY{n}{l\PYZus{}3}\PY{p}{,} \PY{l+s+s2}{\PYZdq{}}\PY{l+s+s2}{\PYZlt{}\PYZhy{} deep copy}\PY{l+s+s2}{\PYZdq{}}\PY{p}{)}
\end{Verbatim}
\end{tcolorbox}

    \begin{Verbatim}[commandchars=\\\{\}]
[['changes!', 2], '3']
[['changes!', 2], '3'] <- shallow copy
[[1, 2], '3'] <- deep copy
    \end{Verbatim}

    \hypertarget{more-on-lists}{%
\subsection{More on lists}\label{more-on-lists}}

\begin{itemize}
\tightlist
\item
  list.append(x): appends x to the end of the list
\item
  list.extend(iterable): appends all items from iterable
\item
  same of a{[}len(a):{]} = iterable
\item
  list.insert(i,x): inserts the value x at a given i-th position
\item
  list.remove(x): removes the first item whose value is x
\item
  list.pop({[}i{]}): returns the i-th item and removes it from the list
\item
  list.clear(x): clear the list
\item
  same of del a{[}:{]}
\item
  list.index(x{[}, start{[}, end{]}{]}): returns index of first item
  whose value is x
\item
  if none raises a ValueError
\item
  start/end limit the search in a slice
\item
  list.count(x): returns the number of times x appears in the list
\item
  list.sort(key=None,reverse=False): sort items, see sorted()
\item
  list.reverse(): reverse the elements in place\\
\item
  list.copy(): returns a shallow copy
\end{itemize}

    \begin{tcolorbox}[breakable, size=fbox, boxrule=1pt, pad at break*=1mm,colback=cellbackground, colframe=cellborder]
\prompt{In}{incolor}{56}{\boxspacing}
\begin{Verbatim}[commandchars=\\\{\}]
\PY{o}{\PYZpc{}}\PY{k}{reset} \PYZhy{}f
\PY{c+c1}{\PYZsh{} append, extend}
\PY{n}{l} \PY{o}{=} \PY{p}{[}\PY{p}{]} 
\PY{n}{l}\PY{o}{.}\PY{n}{append}\PY{p}{(}\PY{l+m+mi}{1}\PY{p}{)}
\PY{n}{l}\PY{o}{.}\PY{n}{append}\PY{p}{(}\PY{l+m+mi}{2}\PY{p}{)}

\PY{n}{l2} \PY{o}{=} \PY{p}{[}\PY{l+m+mi}{0}\PY{p}{]} 
\PY{n}{l2}\PY{o}{.}\PY{n}{extend}\PY{p}{(}\PY{n}{l}\PY{p}{)}

\PY{n+nb}{print}\PY{p}{(}\PY{l+s+s1}{\PYZsq{}}\PY{l+s+s1}{l  \PYZhy{}\PYZgt{}}\PY{l+s+s1}{\PYZsq{}}\PY{p}{,}\PY{n}{l}\PY{p}{)}
\PY{n+nb}{print}\PY{p}{(}\PY{l+s+s1}{\PYZsq{}}\PY{l+s+s1}{l2 \PYZhy{}\PYZgt{}}\PY{l+s+s1}{\PYZsq{}}\PY{p}{,}\PY{n}{l2}\PY{p}{)}
\end{Verbatim}
\end{tcolorbox}

    \begin{Verbatim}[commandchars=\\\{\}]
l  -> [1, 2]
l2 -> [0, 1, 2]
    \end{Verbatim}

    \begin{tcolorbox}[breakable, size=fbox, boxrule=1pt, pad at break*=1mm,colback=cellbackground, colframe=cellborder]
\prompt{In}{incolor}{104}{\boxspacing}
\begin{Verbatim}[commandchars=\\\{\}]
\PY{o}{\PYZpc{}}\PY{k}{reset} \PYZhy{}f
\PY{c+c1}{\PYZsh{} insert, remove, pop, clear}
\PY{n}{l} \PY{o}{=} \PY{p}{[}\PY{l+m+mi}{0}\PY{p}{,}\PY{l+m+mi}{1}\PY{p}{,}\PY{l+m+mi}{2}\PY{p}{]} 
\PY{n}{l}\PY{o}{.}\PY{n}{insert}\PY{p}{(}\PY{l+m+mi}{2}\PY{p}{,}\PY{l+m+mf}{1.5}\PY{p}{)}
\PY{n}{l}\PY{o}{.}\PY{n}{insert}\PY{p}{(}\PY{l+m+mi}{4}\PY{p}{,}\PY{l+m+mf}{2.5}\PY{p}{)}
\PY{n}{l}\PY{o}{.}\PY{n}{insert}\PY{p}{(}\PY{l+m+mi}{4}\PY{p}{,}\PY{l+m+mf}{3.5}\PY{p}{)}
\PY{n+nb}{print}\PY{p}{(}\PY{l+s+s1}{\PYZsq{}}\PY{l+s+s1}{l  \PYZhy{}\PYZgt{}}\PY{l+s+s1}{\PYZsq{}}\PY{p}{,}\PY{n}{l}\PY{p}{)}
\PY{n}{l}\PY{o}{.}\PY{n}{remove}\PY{p}{(}\PY{l+m+mf}{2.5}\PY{p}{)}
\PY{n+nb}{print}\PY{p}{(}\PY{l+s+s1}{\PYZsq{}}\PY{l+s+s1}{l.remove(2.5)  \PYZhy{}\PYZgt{}}\PY{l+s+se}{\PYZbs{}n}\PY{l+s+se}{\PYZbs{}t}\PY{l+s+s1}{ l =}\PY{l+s+s1}{\PYZsq{}}\PY{p}{,}\PY{n}{l}\PY{p}{)}
\PY{n}{el} \PY{o}{=} \PY{n}{l}\PY{o}{.}\PY{n}{pop}\PY{p}{(}\PY{p}{)}
\PY{n+nb}{print}\PY{p}{(}\PY{l+s+s1}{\PYZsq{}}\PY{l+s+s1}{el = l.pop() \PYZhy{}\PYZgt{}}\PY{l+s+se}{\PYZbs{}n}\PY{l+s+se}{\PYZbs{}t}\PY{l+s+s1}{l =}\PY{l+s+s1}{\PYZsq{}}\PY{p}{,}\PY{n}{l}\PY{p}{,} \PY{l+s+s1}{\PYZsq{}}\PY{l+s+se}{\PYZbs{}n}\PY{l+s+se}{\PYZbs{}t}\PY{l+s+s1}{el =}\PY{l+s+s1}{\PYZsq{}}\PY{p}{,}\PY{n}{el}\PY{p}{)}
\PY{n}{l}\PY{o}{.}\PY{n}{clear}\PY{p}{(}\PY{p}{)}
\PY{n+nb}{print}\PY{p}{(}\PY{l+s+s1}{\PYZsq{}}\PY{l+s+s1}{l.clear() \PYZhy{}\PYZgt{}}\PY{l+s+se}{\PYZbs{}n}\PY{l+s+se}{\PYZbs{}t}\PY{l+s+s1}{ l =}\PY{l+s+s1}{\PYZsq{}}\PY{p}{,}\PY{n}{l}\PY{p}{)}
\end{Verbatim}
\end{tcolorbox}

    \begin{Verbatim}[commandchars=\\\{\}]
l  -> [0, 1, 1.5, 2, 3.5, 2.5]
l.remove(2.5)  ->
         l = [0, 1, 1.5, 2, 3.5]
el = l.pop() ->
        l = [0, 1, 1.5, 2]
        el = 3.5
l.clear() ->
         l = []
    \end{Verbatim}

    \begin{tcolorbox}[breakable, size=fbox, boxrule=1pt, pad at break*=1mm,colback=cellbackground, colframe=cellborder]
\prompt{In}{incolor}{29}{\boxspacing}
\begin{Verbatim}[commandchars=\\\{\}]
\PY{o}{\PYZpc{}}\PY{k}{reset} \PYZhy{}f
\PY{c+c1}{\PYZsh{} sort,count,reverse,index}
\PY{n}{l} \PY{o}{=} \PY{p}{[}\PY{l+m+mi}{0}\PY{p}{,}\PY{l+m+mi}{1}\PY{p}{,}\PY{l+m+mi}{0}\PY{p}{,}\PY{l+m+mi}{1}\PY{p}{,}\PY{l+m+mi}{1}\PY{p}{,}\PY{l+m+mi}{2}\PY{p}{,}\PY{l+m+mi}{2}\PY{p}{,}\PY{l+m+mi}{2}\PY{p}{,}\PY{l+m+mi}{3}\PY{p}{,}\PY{l+m+mi}{1}\PY{p}{,}\PY{l+m+mi}{1}\PY{p}{,}\PY{l+m+mi}{2}\PY{p}{,}\PY{l+m+mi}{2}\PY{p}{,}\PY{l+m+mi}{2}\PY{p}{,}\PY{l+m+mi}{2}\PY{p}{]} 
\PY{n+nb}{print}\PY{p}{(}\PY{l+s+s1}{\PYZsq{}}\PY{l+s+se}{\PYZbs{}n}\PY{l+s+s1}{l \PYZhy{}\PYZgt{}}\PY{l+s+s1}{\PYZsq{}}\PY{p}{,}\PY{n}{l}\PY{p}{)}
\PY{n}{l}\PY{o}{.}\PY{n}{sort}\PY{p}{(}\PY{p}{)}
\PY{n+nb}{print}\PY{p}{(}\PY{l+s+s1}{\PYZsq{}}\PY{l+s+se}{\PYZbs{}n}\PY{l+s+s1}{l.sort()}\PY{l+s+se}{\PYZbs{}n}\PY{l+s+s1}{l \PYZhy{}\PYZgt{}}\PY{l+s+s1}{\PYZsq{}}\PY{p}{,}\PY{n}{l}\PY{p}{)}
\PY{n+nb}{print}\PY{p}{(}\PY{l+s+s1}{\PYZsq{}}\PY{l+s+se}{\PYZbs{}n}\PY{l+s+s1}{l.count(2) is}\PY{l+s+s1}{\PYZsq{}}\PY{p}{,}\PY{n}{l}\PY{o}{.}\PY{n}{count}\PY{p}{(}\PY{l+m+mi}{2}\PY{p}{)}\PY{p}{)}
\PY{n}{l}\PY{o}{.}\PY{n}{reverse}\PY{p}{(}\PY{p}{)}
\PY{n+nb}{print}\PY{p}{(}\PY{l+s+s1}{\PYZsq{}}\PY{l+s+se}{\PYZbs{}n}\PY{l+s+s1}{l.reverse()}\PY{l+s+se}{\PYZbs{}n}\PY{l+s+s1}{l \PYZhy{}\PYZgt{}}\PY{l+s+s1}{\PYZsq{}}\PY{p}{,}\PY{n}{l}\PY{p}{)}
\PY{n+nb}{print}\PY{p}{(}\PY{l+s+s1}{\PYZsq{}}\PY{l+s+se}{\PYZbs{}n}\PY{l+s+s1}{l.index(0)}\PY{l+s+se}{\PYZbs{}n}\PY{l+s+s1}{first zero at idx}\PY{l+s+s1}{\PYZsq{}}\PY{p}{,}\PY{n}{l}\PY{o}{.}\PY{n}{index}\PY{p}{(}\PY{l+m+mi}{0}\PY{p}{)}\PY{p}{)}
\PY{n+nb}{print}\PY{p}{(}\PY{l+s+s1}{\PYZsq{}}\PY{l+s+s1}{l[12:] \PYZhy{}\PYZgt{}}\PY{l+s+s1}{\PYZsq{}}\PY{p}{,}\PY{n}{l}\PY{p}{[}\PY{l+m+mi}{12}\PY{p}{:}\PY{p}{]}\PY{p}{)}
\end{Verbatim}
\end{tcolorbox}

    \begin{Verbatim}[commandchars=\\\{\}]

l -> [0, 1, 0, 1, 1, 2, 2, 2, 3, 1, 1, 2, 2, 2, 2]

l.sort()
l -> [0, 0, 1, 1, 1, 1, 1, 2, 2, 2, 2, 2, 2, 2, 3]

l.count(2) is 7

l.reverse()
l -> [3, 2, 2, 2, 2, 2, 2, 2, 1, 1, 1, 1, 1, 0, 0]

l.index(0)
first zero at idx 13
l[12:] -> [1, 0, 0]
    \end{Verbatim}

    \hypertarget{hashable}{%
\subsection{Hashable}\label{hashable}}

\begin{itemize}
\tightlist
\item
  objects are hashable if has a hash value that never changes
\item
  they need a \_\_hash\_\_() method
\item
  hashable objects must be compared to other objects
\item
  they need a \_\_eq\_\_() method
\item
  objects which compare equal must have the same hash value
\item
  immutable built-in objects are hashable
\item
  mutable containers are not hashable
\item
  instances of user-defined classes are hashable by default
\item
  their hash value is derived by id and they all compare unequal (except
  with themselves)
\end{itemize}

    \begin{tcolorbox}[breakable, size=fbox, boxrule=1pt, pad at break*=1mm,colback=cellbackground, colframe=cellborder]
\prompt{In}{incolor}{30}{\boxspacing}
\begin{Verbatim}[commandchars=\\\{\}]
\PY{o}{\PYZpc{}}\PY{k}{reset} \PYZhy{}f
\PY{n}{a} \PY{o}{=} \PY{l+m+mi}{5} \PY{c+c1}{\PYZsh{} int is immutable type}
\PY{n}{b} \PY{o}{=} \PY{l+m+mi}{5}
\PY{n+nb}{print}\PY{p}{(}\PY{l+s+s1}{\PYZsq{}}\PY{l+s+s1}{id(a) == id(b) \PYZhy{}\PYZgt{}}\PY{l+s+s1}{\PYZsq{}}\PY{p}{,}\PY{n+nb}{id}\PY{p}{(}\PY{n}{a}\PY{p}{)} \PY{o}{==} \PY{n+nb}{id}\PY{p}{(}\PY{n}{b}\PY{p}{)}\PY{p}{,}\PY{l+s+s1}{\PYZsq{}}\PY{l+s+s1}{==}\PY{l+s+s1}{\PYZsq{}}\PY{p}{,}\PY{n+nb}{id}\PY{p}{(}\PY{n}{a}\PY{p}{)}\PY{p}{)}

\PY{n}{c} \PY{o}{=} \PY{p}{[}\PY{l+m+mi}{5}\PY{p}{]} \PY{c+c1}{\PYZsh{} list is mutable type}
\PY{n}{d} \PY{o}{=} \PY{p}{[}\PY{l+m+mi}{5}\PY{p}{]}
\PY{n+nb}{print}\PY{p}{(}\PY{l+s+s1}{\PYZsq{}}\PY{l+s+s1}{id(c) == id(d) \PYZhy{}\PYZgt{}}\PY{l+s+s1}{\PYZsq{}}\PY{p}{,}\PY{n+nb}{id}\PY{p}{(}\PY{n}{c}\PY{p}{)} \PY{o}{==} \PY{n+nb}{id}\PY{p}{(}\PY{n}{d}\PY{p}{)}\PY{p}{)}
\PY{n+nb}{print}\PY{p}{(}\PY{l+s+s1}{\PYZsq{}}\PY{l+s+s1}{id(c) ==}\PY{l+s+s1}{\PYZsq{}}\PY{p}{,}\PY{n+nb}{id}\PY{p}{(}\PY{n}{c}\PY{p}{)}\PY{p}{,}\PY{l+s+s1}{\PYZsq{}}\PY{l+s+s1}{; id(d) ==}\PY{l+s+s1}{\PYZsq{}}\PY{p}{,}\PY{n+nb}{id}\PY{p}{(}\PY{n}{d}\PY{p}{)}\PY{p}{)}
\end{Verbatim}
\end{tcolorbox}

    \begin{Verbatim}[commandchars=\\\{\}]
id(a) == id(b) -> True == 139915757249664
id(c) == id(d) -> False
id(c) == 139915540396616 ; id(d) == 139915540395272
    \end{Verbatim}

    \hypertarget{set-types}{%
\subsection{Set types}\label{set-types}}

\begin{itemize}
\tightlist
\item
  curly braces or set() function create sets
\item
  set() with no arguments creates a set
\item
  \{\} with no arguments creates a dictionary - see later
\item
  sets are unordered collections with no duplicate elements
\item
  sets are mutable objects
\item
  in keyword is suitable to check the membership of a value in the list
\item
  elements in set must be hashable object
\item
  means they have a hash value which never changes during their lifetime
\item
  hashability makes objects usable as a dictionary key and a set member
\item
  sets theirselves are NOT hashable
\item
  cannot be used as dictionary keys
\item
  sets of sets are not allowed
\end{itemize}

    \hypertarget{set-operations}{%
\subsection{Set operations}\label{set-operations}}

\begin{itemize}
\tightlist
\item
  a.difference(b) or a-b: elements in set a that are not in set b
\item
  a.union(b) a\textbar b: union of elements in set a and in set b
\item
  a.intersection(b) a\&b: intersection of elements in set a and in set b
\item
  a.symmetric\_difference(b) a\^{}b: elements in set a or set b but not
  in both
\item
  equals union-intersection ((a\textbar b)-(a\&b))
\item
  a.issubset(b) a\textless=b: test whether every element of a is in b
\item
  a.issuperset(b) a\textgreater=b: test whether every element of b is in
  a
\end{itemize}

    \begin{tcolorbox}[breakable, size=fbox, boxrule=1pt, pad at break*=1mm,colback=cellbackground, colframe=cellborder]
\prompt{In}{incolor}{31}{\boxspacing}
\begin{Verbatim}[commandchars=\\\{\}]
\PY{n}{set\PYZus{}1} \PY{o}{=} \PY{n+nb}{set}\PY{p}{(}\PY{p}{[}\PY{l+m+mi}{1}\PY{p}{,}\PY{l+m+mi}{2}\PY{p}{,}\PY{l+m+mi}{1}\PY{p}{,}\PY{l+m+mi}{2}\PY{p}{,}\PY{l+m+mi}{3}\PY{p}{]}\PY{p}{)} 
\PY{n}{set\PYZus{}2} \PY{o}{=} \PY{n+nb}{set}\PY{p}{(}\PY{p}{[}\PY{l+m+mi}{4}\PY{p}{,}\PY{l+m+mi}{5}\PY{p}{,}\PY{l+m+mi}{6}\PY{p}{,}\PY{l+m+mi}{1}\PY{p}{,}\PY{l+m+mi}{7}\PY{p}{]}\PY{p}{)} 

\PY{n+nb}{print}\PY{p}{(}\PY{l+s+s2}{\PYZdq{}}\PY{l+s+s2}{set\PYZus{}1 \PYZhy{}\PYZgt{}}\PY{l+s+s2}{\PYZdq{}}\PY{p}{,}\PY{l+s+s2}{\PYZdq{}}\PY{l+s+s2}{\PYZob{}}\PY{l+s+s2}{1,2,1,2,3\PYZcb{} \PYZsh{} 1 and 2 are repeated}\PY{l+s+s2}{\PYZdq{}}\PY{p}{)}
\PY{n+nb}{print}\PY{p}{(}\PY{l+s+s2}{\PYZdq{}}\PY{l+s+s2}{set\PYZus{}2 \PYZhy{}\PYZgt{}}\PY{l+s+s2}{\PYZdq{}}\PY{p}{,}\PY{l+s+s2}{\PYZdq{}}\PY{l+s+s2}{\PYZob{}}\PY{l+s+s2}{4,5,6,1,7\PYZcb{} \PYZsh{} 1 is in both }\PY{l+s+s2}{\PYZdq{}}\PY{p}{)}
\PY{n+nb}{print}\PY{p}{(}\PY{l+s+s2}{\PYZdq{}}\PY{l+s+se}{\PYZbs{}n}\PY{l+s+s2}{\PYZsh{} use of }\PY{l+s+s2}{\PYZsq{}}\PY{l+s+s2}{in}\PY{l+s+s2}{\PYZsq{}}\PY{l+s+s2}{ keyword}\PY{l+s+s2}{\PYZdq{}}\PY{p}{)}
\PY{n+nb}{print}\PY{p}{(}\PY{l+s+s2}{\PYZdq{}}\PY{l+s+s2}{3 in set\PYZus{}1 \PYZhy{}\PYZgt{}}\PY{l+s+s2}{\PYZdq{}}\PY{p}{,}\PY{l+m+mi}{3} \PY{o+ow}{in} \PY{n}{set\PYZus{}1}\PY{p}{)}
\PY{n+nb}{print}\PY{p}{(}\PY{l+s+s2}{\PYZdq{}}\PY{l+s+se}{\PYZbs{}n}\PY{l+s+s2}{\PYZsh{}set difference: elements in one but not in the other}\PY{l+s+s2}{\PYZdq{}}\PY{p}{)}
\PY{n+nb}{print}\PY{p}{(}\PY{l+s+s2}{\PYZdq{}}\PY{l+s+s2}{set\PYZus{}1\PYZhy{}set\PYZus{}2 \PYZhy{}\PYZgt{}}\PY{l+s+s2}{\PYZdq{}}\PY{p}{,}\PY{n}{set\PYZus{}1}\PY{o}{\PYZhy{}}\PY{n}{set\PYZus{}2}\PY{p}{)}
\PY{n+nb}{print}\PY{p}{(}\PY{l+s+s2}{\PYZdq{}}\PY{l+s+s2}{set\PYZus{}2\PYZhy{}set\PYZus{}1 \PYZhy{}\PYZgt{}}\PY{l+s+s2}{\PYZdq{}}\PY{p}{,}\PY{n}{set\PYZus{}2}\PY{o}{\PYZhy{}}\PY{n}{set\PYZus{}1}\PY{p}{)}
\PY{n+nb}{print}\PY{p}{(}\PY{l+s+s2}{\PYZdq{}}\PY{l+s+se}{\PYZbs{}n}\PY{l+s+s2}{\PYZsh{}set union and intersection}\PY{l+s+s2}{\PYZdq{}}\PY{p}{)}
\PY{n+nb}{print}\PY{p}{(}\PY{l+s+s2}{\PYZdq{}}\PY{l+s+s2}{set\PYZus{}1|set\PYZus{}2 \PYZhy{}\PYZgt{}}\PY{l+s+s2}{\PYZdq{}}\PY{p}{,}\PY{l+s+s2}{\PYZdq{}}\PY{l+s+s2}{set\PYZus{}2|set\PYZus{}1 \PYZhy{}\PYZgt{}}\PY{l+s+s2}{\PYZdq{}}\PY{p}{,}\PY{n}{set\PYZus{}1}\PY{o}{|}\PY{n}{set\PYZus{}2}\PY{p}{)}
\PY{n+nb}{print}\PY{p}{(}\PY{l+s+s2}{\PYZdq{}}\PY{l+s+s2}{set\PYZus{}1\PYZam{}set\PYZus{}2 \PYZhy{}\PYZgt{}}\PY{l+s+s2}{\PYZdq{}}\PY{p}{,}\PY{l+s+s2}{\PYZdq{}}\PY{l+s+s2}{set\PYZus{}2\PYZam{}set\PYZus{}1 \PYZhy{}\PYZgt{}}\PY{l+s+s2}{\PYZdq{}}\PY{p}{,}\PY{n}{set\PYZus{}1}\PY{o}{\PYZam{}}\PY{n}{set\PYZus{}2}\PY{p}{)}
\PY{n+nb}{print}\PY{p}{(}\PY{l+s+s2}{\PYZdq{}}\PY{l+s+se}{\PYZbs{}n}\PY{l+s+s2}{\PYZsh{}set symmetric difference == set difference between set union and set intersection}\PY{l+s+s2}{\PYZdq{}}\PY{p}{)}
\PY{n+nb}{print}\PY{p}{(}\PY{l+s+s2}{\PYZdq{}}\PY{l+s+s2}{set\PYZus{}1\PYZca{}set\PYZus{}2 \PYZhy{}\PYZgt{}}\PY{l+s+s2}{\PYZdq{}}\PY{p}{,}\PY{l+s+s2}{\PYZdq{}}\PY{l+s+s2}{set\PYZus{}2\PYZca{}set\PYZus{}1  \PYZhy{}\PYZgt{}}\PY{l+s+s2}{\PYZdq{}}\PY{p}{,}\PY{n}{set\PYZus{}1}\PY{o}{\PYZca{}}\PY{n}{set\PYZus{}2}\PY{p}{)}
\PY{n+nb}{print}\PY{p}{(}\PY{l+s+s2}{\PYZdq{}}\PY{l+s+s2}{(set\PYZus{}1|set\PYZus{}2)\PYZhy{}(set\PYZus{}1\PYZam{}set\PYZus{}2) \PYZhy{}\PYZgt{}}\PY{l+s+s2}{\PYZdq{}}\PY{p}{,}\PY{p}{(}\PY{n}{set\PYZus{}1}\PY{o}{|}\PY{n}{set\PYZus{}2}\PY{p}{)}\PY{o}{\PYZhy{}}\PY{p}{(}\PY{n}{set\PYZus{}1}\PY{o}{\PYZam{}}\PY{n}{set\PYZus{}2}\PY{p}{)}\PY{p}{)}
\PY{n+nb}{print}\PY{p}{(}\PY{l+s+s2}{\PYZdq{}}\PY{l+s+se}{\PYZbs{}n}\PY{l+s+s2}{\PYZsh{}super/subset test}\PY{l+s+s2}{\PYZdq{}}\PY{p}{)}
\PY{n+nb}{print}\PY{p}{(}\PY{l+s+s2}{\PYZdq{}}\PY{l+s+s2}{set\PYZus{}1\PYZgt{}=set\PYZus{}2 \PYZhy{}\PYZgt{}}\PY{l+s+s2}{\PYZdq{}}\PY{p}{,}\PY{n}{set\PYZus{}1}\PY{o}{\PYZgt{}}\PY{o}{=}\PY{n}{set\PYZus{}2}\PY{p}{)}
\PY{n+nb}{print}\PY{p}{(}\PY{l+s+s2}{\PYZdq{}}\PY{l+s+s2}{set\PYZus{}1\PYZlt{}=set\PYZus{}2 \PYZhy{}\PYZgt{}}\PY{l+s+s2}{\PYZdq{}}\PY{p}{,}\PY{n}{set\PYZus{}1}\PY{o}{\PYZlt{}}\PY{o}{=}\PY{n}{set\PYZus{}2}\PY{p}{)}
\PY{n+nb}{print}\PY{p}{(}\PY{l+s+s2}{\PYZdq{}}\PY{l+s+s2}{\PYZob{}}\PY{l+s+s2}{1,2,3\PYZcb{}\PYZlt{}=}\PY{l+s+s2}{\PYZob{}}\PY{l+s+s2}{1,2,3,4,5,6,7\PYZcb{} \PYZhy{}\PYZgt{}}\PY{l+s+s2}{\PYZdq{}}\PY{p}{,}\PY{p}{\PYZob{}}\PY{l+m+mi}{1}\PY{p}{,}\PY{l+m+mi}{2}\PY{p}{,}\PY{l+m+mi}{3}\PY{p}{\PYZcb{}}\PY{o}{\PYZlt{}}\PY{o}{=}\PY{p}{\PYZob{}}\PY{l+m+mi}{1}\PY{p}{,}\PY{l+m+mi}{2}\PY{p}{,}\PY{l+m+mi}{3}\PY{p}{,}\PY{l+m+mi}{4}\PY{p}{,}\PY{l+m+mi}{5}\PY{p}{,}\PY{l+m+mi}{6}\PY{p}{,}\PY{l+m+mi}{7}\PY{p}{\PYZcb{}}\PY{p}{)}
\PY{n+nb}{print}\PY{p}{(}\PY{l+s+s2}{\PYZdq{}}\PY{l+s+s2}{\PYZob{}}\PY{l+s+s2}{1,2,3,4,5,6,7\PYZcb{}\PYZgt{}=}\PY{l+s+s2}{\PYZob{}}\PY{l+s+s2}{1,2,3\PYZcb{} \PYZhy{}\PYZgt{}}\PY{l+s+s2}{\PYZdq{}}\PY{p}{,}\PY{p}{\PYZob{}}\PY{l+m+mi}{1}\PY{p}{,}\PY{l+m+mi}{2}\PY{p}{,}\PY{l+m+mi}{3}\PY{p}{,}\PY{l+m+mi}{4}\PY{p}{,}\PY{l+m+mi}{5}\PY{p}{,}\PY{l+m+mi}{6}\PY{p}{,}\PY{l+m+mi}{7}\PY{p}{\PYZcb{}}\PY{o}{\PYZgt{}}\PY{o}{=}\PY{p}{\PYZob{}}\PY{l+m+mi}{1}\PY{p}{,}\PY{l+m+mi}{2}\PY{p}{,}\PY{l+m+mi}{3}\PY{p}{\PYZcb{}}\PY{p}{)}
\end{Verbatim}
\end{tcolorbox}

    \begin{Verbatim}[commandchars=\\\{\}]
set\_1 -> \{1,2,1,2,3\} \# 1 and 2 are repeated
set\_2 -> \{4,5,6,1,7\} \# 1 is in both

\# use of 'in' keyword
3 in set\_1 -> True

\#set difference: elements in one but not in the other
set\_1-set\_2 -> \{2, 3\}
set\_2-set\_1 -> \{4, 5, 6, 7\}

\#set union and intersection
set\_1|set\_2 -> set\_2|set\_1 -> \{1, 2, 3, 4, 5, 6, 7\}
set\_1\&set\_2 -> set\_2\&set\_1 -> \{1\}

\#set symmetric difference == set difference between set union and set
intersection
set\_1\^{}set\_2 -> set\_2\^{}set\_1  -> \{2, 3, 4, 5, 6, 7\}
(set\_1|set\_2)-(set\_1\&set\_2) -> \{2, 3, 4, 5, 6, 7\}

\#super/subset test
set\_1>=set\_2 -> False
set\_1<=set\_2 -> False
\{1,2,3\}<=\{1,2,3,4,5,6,7\} -> True
\{1,2,3,4,5,6,7\}>=\{1,2,3\} -> True
    \end{Verbatim}

    \begin{tcolorbox}[breakable, size=fbox, boxrule=1pt, pad at break*=1mm,colback=cellbackground, colframe=cellborder]
\prompt{In}{incolor}{32}{\boxspacing}
\begin{Verbatim}[commandchars=\\\{\}]
\PY{n+nb}{print}\PY{p}{(}\PY{l+s+s2}{\PYZdq{}}\PY{l+s+s2}{\PYZsh{} sets cannot contain unhashable types}\PY{l+s+s2}{\PYZdq{}}\PY{p}{)}
\PY{n}{set\PYZus{}1} \PY{o}{=} \PY{n+nb}{set}\PY{p}{(}\PY{p}{[}\PY{n+nb}{set}\PY{p}{(}\PY{p}{[}\PY{l+s+s1}{\PYZsq{}}\PY{l+s+s1}{a}\PY{l+s+s1}{\PYZsq{}}\PY{p}{]}\PY{p}{)}\PY{p}{]}\PY{p}{)}
\end{Verbatim}
\end{tcolorbox}

    \begin{Verbatim}[commandchars=\\\{\}]
\# sets cannot contain unhashable types
    \end{Verbatim}

    \begin{Verbatim}[commandchars=\\\{\}]

        ---------------------------------------------------------------------------

        TypeError                                 Traceback (most recent call last)

        <ipython-input-32-2a5c1e0ad27a> in <module>()
          1 print("\# sets cannot contain unhashable types")
    ----> 2 set\_1 = set([set(['a'])])
    

        TypeError: unhashable type: 'set'

    \end{Verbatim}

    \hypertarget{dictionaries}{%
\subsection{Dictionaries}\label{dictionaries}}

\begin{itemize}
\tightlist
\item
  it is mapping type (also called ``associative memories'' or
  ``associative arrays'')
\item
  dict() builds dictionaries from sequences of key-value pairs
\item
  unlike sequence types, indexed by numbers, dictionaries are indexed by
  keys
\item
  only hashable types can be keys
\item
  dict are unordered sets of key:value pairs
\item
  module collections offers a class of ordered dictionaries (i.e.,
  OrderedDict, which remembers the order in which its contents are
  added)
\item
  ordered dictionaries support reversed()
\item
  {[}not{]} in keywords check if a key is {[}not{]} in dictionaries
\item
  del keyword is needed to delete a key:value pair
\end{itemize}

    \begin{tcolorbox}[breakable, size=fbox, boxrule=1pt, pad at break*=1mm,colback=cellbackground, colframe=cellborder]
\prompt{In}{incolor}{33}{\boxspacing}
\begin{Verbatim}[commandchars=\\\{\}]
\PY{o}{\PYZpc{}}\PY{k}{reset} \PYZhy{}f

\PY{c+c1}{\PYZsh{} using \PYZob{}\PYZcb{}}
\PY{n}{address\PYZus{}book}\PY{o}{=}\PY{p}{\PYZob{}}\PY{l+s+s1}{\PYZsq{}}\PY{l+s+s1}{Jacopo}\PY{l+s+s1}{\PYZsq{}}\PY{p}{:}\PY{l+m+mi}{1}\PY{p}{,}\PY{l+s+s1}{\PYZsq{}}\PY{l+s+s1}{Alberto}\PY{l+s+s1}{\PYZsq{}}\PY{p}{:}\PY{l+m+mi}{2}\PY{p}{\PYZcb{}}
\PY{n+nb}{print}\PY{p}{(}\PY{l+s+s2}{\PYZdq{}}\PY{l+s+se}{\PYZbs{}n}\PY{l+s+s2}{address\PYZus{}book \PYZhy{}\PYZgt{}}\PY{l+s+s2}{\PYZdq{}}\PY{p}{,}\PYZbs{}
      \PY{n}{address\PYZus{}book}\PY{p}{,}\PY{l+s+s2}{\PYZdq{}}\PY{l+s+se}{\PYZbs{}n}\PY{l+s+s2}{ \PYZhy{}\PYZgt{}}\PY{l+s+s2}{\PYZdq{}}\PY{p}{,}\PYZbs{}
      \PY{n+nb}{type}\PY{p}{(}\PY{n}{address\PYZus{}book}\PY{p}{)}\PY{p}{)}
\PY{n+nb}{print}\PY{p}{(}\PY{l+s+s2}{\PYZdq{}}\PY{l+s+s2}{\PYZsq{}}\PY{l+s+s2}{Luca}\PY{l+s+s2}{\PYZsq{}}\PY{l+s+s2}{ not in address\PYZus{}book}\PY{l+s+s2}{\PYZsq{}}\PY{l+s+s2}{ \PYZhy{}\PYZgt{}}\PY{l+s+s2}{\PYZdq{}}\PY{p}{,}\PYZbs{}
         \PY{l+s+s1}{\PYZsq{}}\PY{l+s+s1}{Luca}\PY{l+s+s1}{\PYZsq{}} \PY{o+ow}{not} \PY{o+ow}{in} \PY{n}{address\PYZus{}book}\PY{p}{)}


\PY{c+c1}{\PYZsh{} using dict(list\PYZus{}of\PYZus{}tuples)}
\PY{n}{mensa\PYZus{}dishes}\PY{o}{=}\PY{n+nb}{dict}\PY{p}{(}\PY{p}{[}\PY{p}{(}\PY{l+s+s1}{\PYZsq{}}\PY{l+s+s1}{first}\PY{l+s+s1}{\PYZsq{}}\PY{p}{,}\PY{l+s+s1}{\PYZsq{}}\PY{l+s+s1}{pasta}\PY{l+s+s1}{\PYZsq{}}\PY{p}{)}\PY{p}{,}\PYZbs{}
                   \PY{p}{(}\PY{l+s+s1}{\PYZsq{}}\PY{l+s+s1}{second}\PY{l+s+s1}{\PYZsq{}}\PY{p}{,}\PY{l+s+s1}{\PYZsq{}}\PY{l+s+s1}{salade}\PY{l+s+s1}{\PYZsq{}}\PY{p}{)}\PY{p}{,}\PYZbs{}
                   \PY{p}{(}\PY{l+s+s1}{\PYZsq{}}\PY{l+s+s1}{dessert}\PY{l+s+s1}{\PYZsq{}}\PY{p}{,}\PY{l+s+s1}{\PYZsq{}}\PY{l+s+s1}{apple}\PY{l+s+s1}{\PYZsq{}}\PY{p}{)}\PY{p}{]}\PY{p}{)}
\PY{n+nb}{print}\PY{p}{(}\PY{l+s+s2}{\PYZdq{}}\PY{l+s+se}{\PYZbs{}n}\PY{l+s+s2}{mensa\PYZus{}dishes \PYZhy{}\PYZgt{}}\PY{l+s+s2}{\PYZdq{}}\PY{p}{,}\PYZbs{}
      \PY{n}{mensa\PYZus{}dishes}\PY{p}{,}\PY{l+s+s2}{\PYZdq{}}\PY{l+s+se}{\PYZbs{}n}\PY{l+s+s2}{ \PYZhy{}\PYZgt{}}\PY{l+s+s2}{\PYZdq{}}\PY{p}{,}\PYZbs{}
      \PY{n+nb}{type}\PY{p}{(}\PY{n}{address\PYZus{}book}\PY{p}{)}\PY{p}{)}
\PY{n+nb}{print}\PY{p}{(}\PY{l+s+s2}{\PYZdq{}}\PY{l+s+s2}{\PYZsq{}}\PY{l+s+s2}{first}\PY{l+s+s2}{\PYZsq{}}\PY{l+s+s2}{ in address\PYZus{}book}\PY{l+s+s2}{\PYZsq{}}\PY{l+s+s2}{ \PYZhy{}\PYZgt{}}\PY{l+s+s2}{\PYZdq{}}\PY{p}{,}\PYZbs{}
         \PY{l+s+s1}{\PYZsq{}}\PY{l+s+s1}{first}\PY{l+s+s1}{\PYZsq{}} \PY{o+ow}{in} \PY{n}{mensa\PYZus{}dishes}\PY{p}{)}
\end{Verbatim}
\end{tcolorbox}

    \begin{Verbatim}[commandchars=\\\{\}]

address\_book -> \{'Alberto': 2, 'Jacopo': 1\}
 -> <class 'dict'>
'Luca' not in address\_book' -> True

mensa\_dishes -> \{'first': 'pasta', 'second': 'salade', 'dessert': 'apple'\}
 -> <class 'dict'>
'first' in address\_book' -> True
    \end{Verbatim}

    \begin{tcolorbox}[breakable, size=fbox, boxrule=1pt, pad at break*=1mm,colback=cellbackground, colframe=cellborder]
\prompt{In}{incolor}{35}{\boxspacing}
\begin{Verbatim}[commandchars=\\\{\}]
\PY{o}{\PYZpc{}}\PY{k}{reset} \PYZhy{}f
\PY{n}{address\PYZus{}book}\PY{o}{=}\PY{p}{\PYZob{}}\PY{l+s+s1}{\PYZsq{}}\PY{l+s+s1}{Jacopo}\PY{l+s+s1}{\PYZsq{}}\PY{p}{:}\PY{l+m+mi}{1}\PY{p}{,}\PY{l+s+s1}{\PYZsq{}}\PY{l+s+s1}{Alberto}\PY{l+s+s1}{\PYZsq{}}\PY{p}{:}\PY{l+m+mi}{2}\PY{p}{\PYZcb{}}
\PY{n+nb}{print}\PY{p}{(}\PY{l+s+s2}{\PYZdq{}}\PY{l+s+s2}{address\PYZus{}book =}\PY{l+s+s2}{\PYZdq{}}\PY{p}{,}\PY{n}{address\PYZus{}book}\PY{p}{)}

\PY{n+nb}{print}\PY{p}{(}\PY{l+s+s2}{\PYZdq{}}\PY{l+s+se}{\PYZbs{}n}\PY{l+s+s2}{\PYZsh{} delete key}\PY{l+s+s2}{\PYZdq{}}\PY{p}{)}
\PY{n+nb}{print}\PY{p}{(}\PY{l+s+s2}{\PYZdq{}}\PY{l+s+s2}{\PYZsq{}}\PY{l+s+s2}{Jacopo}\PY{l+s+s2}{\PYZsq{}}\PY{l+s+s2}{ in address\PYZus{}book \PYZhy{}\PYZgt{}}\PY{l+s+s2}{\PYZdq{}}\PY{p}{,}\PY{l+s+s1}{\PYZsq{}}\PY{l+s+s1}{Jacopo}\PY{l+s+s1}{\PYZsq{}} \PY{o+ow}{in} \PY{n}{address\PYZus{}book}\PY{p}{)}
\PY{n+nb}{print}\PY{p}{(}\PY{l+s+s2}{\PYZdq{}}\PY{l+s+s2}{del address\PYZus{}book[}\PY{l+s+s2}{\PYZsq{}}\PY{l+s+s2}{Jacopo}\PY{l+s+s2}{\PYZsq{}}\PY{l+s+s2}{]}\PY{l+s+s2}{\PYZdq{}}\PY{p}{)}
\PY{k}{del} \PY{n}{address\PYZus{}book}\PY{p}{[}\PY{l+s+s1}{\PYZsq{}}\PY{l+s+s1}{Jacopo}\PY{l+s+s1}{\PYZsq{}}\PY{p}{]}
\PY{n+nb}{print}\PY{p}{(}\PY{l+s+s2}{\PYZdq{}}\PY{l+s+s2}{address\PYZus{}book \PYZhy{}\PYZgt{}}\PY{l+s+s2}{\PYZdq{}}\PY{p}{,}\PY{n}{address\PYZus{}book}\PY{p}{)}
\PY{n+nb}{print}\PY{p}{(}\PY{l+s+s2}{\PYZdq{}}\PY{l+s+s2}{\PYZsq{}}\PY{l+s+s2}{Jacopo}\PY{l+s+s2}{\PYZsq{}}\PY{l+s+s2}{ in address\PYZus{}book \PYZhy{}\PYZgt{}}\PY{l+s+s2}{\PYZdq{}}\PY{p}{,}\PY{l+s+s1}{\PYZsq{}}\PY{l+s+s1}{Jacopo}\PY{l+s+s1}{\PYZsq{}} \PY{o+ow}{in} \PY{n}{address\PYZus{}book}\PY{p}{)}

\PY{n+nb}{print}\PY{p}{(}\PY{l+s+s2}{\PYZdq{}}\PY{l+s+se}{\PYZbs{}n}\PY{l+s+s2}{\PYZsh{} add new key}\PY{l+s+s2}{\PYZdq{}}\PY{p}{)}
\PY{n+nb}{print}\PY{p}{(}\PY{l+s+s2}{\PYZdq{}}\PY{l+s+s2}{address\PYZus{}book[}\PY{l+s+s2}{\PYZsq{}}\PY{l+s+s2}{Luca}\PY{l+s+s2}{\PYZsq{}}\PY{l+s+s2}{] = 3}\PY{l+s+s2}{\PYZdq{}}\PY{p}{)}
\PY{n}{address\PYZus{}book}\PY{p}{[}\PY{l+s+s1}{\PYZsq{}}\PY{l+s+s1}{Luca}\PY{l+s+s1}{\PYZsq{}}\PY{p}{]} \PY{o}{=} \PY{l+m+mi}{3}
\PY{n+nb}{print}\PY{p}{(}\PY{l+s+s2}{\PYZdq{}}\PY{l+s+s2}{address\PYZus{}book \PYZhy{}\PYZgt{}}\PY{l+s+s2}{\PYZdq{}}\PY{p}{,}\PY{n}{address\PYZus{}book}\PY{p}{)}
\end{Verbatim}
\end{tcolorbox}

    \begin{Verbatim}[commandchars=\\\{\}]
address\_book = \{'Alberto': 2, 'Jacopo': 1\}

\# delete key
'Jacopo' in address\_book -> True
del address\_book['Jacopo']
address\_book -> \{'Alberto': 2\}
'Jacopo' in address\_book -> False

\# add new key
address\_book['Luca'] = 3
address\_book -> \{'Alberto': 2, 'Luca': 3\}
    \end{Verbatim}

    \hypertarget{looping-dictionarys-elements}{%
\subsection{Looping dictionary's
elements}\label{looping-dictionarys-elements}}

\begin{itemize}
\tightlist
\item
  it is possible looping dictionary's elements
\item
  dict.keys(): retreives list of keys
\item
  dict.values(): retreives list of values
\item
  dict.items(): retreives list of (key,value) pairs
\item
  list() or sorted() built-in functions list values (arbitrary or sorted
  order)
\item
  to sort keys by value
\item
  sorted(dict,key=dict.get)
\item
  use lambda functions
\end{itemize}

    \begin{tcolorbox}[breakable, size=fbox, boxrule=1pt, pad at break*=1mm,colback=cellbackground, colframe=cellborder]
\prompt{In}{incolor}{36}{\boxspacing}
\begin{Verbatim}[commandchars=\\\{\}]
\PY{o}{\PYZpc{}}\PY{k}{reset} \PYZhy{}f
\PY{n}{week} \PY{o}{=} \PY{p}{\PYZob{}}\PY{l+s+s1}{\PYZsq{}}\PY{l+s+s1}{mon}\PY{l+s+s1}{\PYZsq{}}\PY{p}{:}\PY{l+m+mi}{1}\PY{p}{,}\PY{l+s+s1}{\PYZsq{}}\PY{l+s+s1}{tue}\PY{l+s+s1}{\PYZsq{}}\PY{p}{:}\PY{l+m+mi}{2}\PY{p}{,}\PY{l+s+s1}{\PYZsq{}}\PY{l+s+s1}{wed}\PY{l+s+s1}{\PYZsq{}}\PY{p}{:}\PY{l+m+mi}{3}\PY{p}{,}\PY{l+s+s1}{\PYZsq{}}\PY{l+s+s1}{thu}\PY{l+s+s1}{\PYZsq{}}\PY{p}{:}\PY{l+m+mi}{4}\PY{p}{,}\PY{l+s+s1}{\PYZsq{}}\PY{l+s+s1}{fri}\PY{l+s+s1}{\PYZsq{}}\PY{p}{:}\PY{l+m+mi}{5}\PY{p}{,}\PY{l+s+s1}{\PYZsq{}}\PY{l+s+s1}{sat}\PY{l+s+s1}{\PYZsq{}}\PY{p}{:}\PY{l+m+mi}{6}\PY{p}{,}\PY{l+s+s1}{\PYZsq{}}\PY{l+s+s1}{sun}\PY{l+s+s1}{\PYZsq{}}\PY{p}{:}\PY{l+m+mi}{7}\PY{p}{\PYZcb{}}
\PY{n+nb}{print}\PY{p}{(}\PY{n+nb}{list}\PY{p}{(}\PY{n}{week}\PY{p}{)}\PY{p}{)}
\PY{n+nb}{print}\PY{p}{(}\PY{n+nb}{sorted}\PY{p}{(}\PY{n}{week}\PY{p}{)}\PY{p}{)}
\PY{n+nb}{print}\PY{p}{(}\PY{n+nb}{sorted}\PY{p}{(}\PY{n}{week}\PY{p}{,} \PY{n}{key}\PY{o}{=}\PY{n}{week}\PY{o}{.}\PY{n}{get}\PY{p}{)}\PY{p}{)}
\end{Verbatim}
\end{tcolorbox}

    \begin{Verbatim}[commandchars=\\\{\}]
['tue', 'sun', 'wed', 'sat', 'thu', 'mon', 'fri']
['fri', 'mon', 'sat', 'sun', 'thu', 'tue', 'wed']
['mon', 'tue', 'wed', 'thu', 'fri', 'sat', 'sun']
    \end{Verbatim}

    \hypertarget{other-dictionarys-methods}{%
\subsection{Other dictionary's
methods}\label{other-dictionarys-methods}}

\begin{itemize}
\tightlist
\item
  dict.copy(): shallow copy of of dict
\item
  dict.clear(): removes all key/value pairs in dict
\item
  dict.update({[}key:value{]}): overwrites key/value pairs from another
  dict
\item
  dict.get(k{[},x{]}): retreives dict{[}k{]} if k in dict else x\\
\item
  dict.setdefault(k{[},x{]}): if k in keys, then get(), otherwise set
  the key
\item
  dict.pop(k{[},x{]}): get(k{[},x{]}), then remove key k
\end{itemize}

    \begin{tcolorbox}[breakable, size=fbox, boxrule=1pt, pad at break*=1mm,colback=cellbackground, colframe=cellborder]
\prompt{In}{incolor}{105}{\boxspacing}
\begin{Verbatim}[commandchars=\\\{\}]
\PY{o}{\PYZpc{}}\PY{k}{reset} \PYZhy{}f
\PY{n}{week} \PY{o}{=} \PY{p}{\PYZob{}}\PY{l+s+s1}{\PYZsq{}}\PY{l+s+s1}{mon}\PY{l+s+s1}{\PYZsq{}}\PY{p}{:}\PY{l+m+mi}{1}\PY{p}{,}\PY{l+s+s1}{\PYZsq{}}\PY{l+s+s1}{tue}\PY{l+s+s1}{\PYZsq{}}\PY{p}{:}\PY{l+m+mi}{2}\PY{p}{,}\PY{l+s+s1}{\PYZsq{}}\PY{l+s+s1}{wed}\PY{l+s+s1}{\PYZsq{}}\PY{p}{:}\PY{l+m+mi}{3}\PY{p}{,}\PY{l+s+s1}{\PYZsq{}}\PY{l+s+s1}{thu}\PY{l+s+s1}{\PYZsq{}}\PY{p}{:}\PY{l+m+mi}{4}\PY{p}{,}\PY{l+s+s1}{\PYZsq{}}\PY{l+s+s1}{fri}\PY{l+s+s1}{\PYZsq{}}\PY{p}{:}\PY{l+m+mi}{5}\PY{p}{,}\PY{l+s+s1}{\PYZsq{}}\PY{l+s+s1}{sat}\PY{l+s+s1}{\PYZsq{}}\PY{p}{:}\PY{l+m+mi}{6}\PY{p}{,}\PY{l+s+s1}{\PYZsq{}}\PY{l+s+s1}{sun}\PY{l+s+s1}{\PYZsq{}}\PY{p}{:}\PY{l+m+mi}{7}\PY{p}{\PYZcb{}}

\PY{n+nb}{print}\PY{p}{(}\PY{l+s+s2}{\PYZdq{}}\PY{l+s+se}{\PYZbs{}n}\PY{l+s+s2}{\PYZsh{}update value for a key}\PY{l+s+s2}{\PYZdq{}}\PY{p}{)}
\PY{n+nb}{print}\PY{p}{(}\PY{l+s+s2}{\PYZdq{}}\PY{l+s+s2}{week.get(}\PY{l+s+s2}{\PYZsq{}}\PY{l+s+s2}{\PYZus{}8thday}\PY{l+s+s2}{\PYZsq{}}\PY{l+s+s2}{,8) \PYZhy{}\PYZgt{}}\PY{l+s+s2}{\PYZdq{}}\PY{p}{,}\PYZbs{}
      \PY{n}{week}\PY{o}{.}\PY{n}{get}\PY{p}{(}\PY{l+s+s1}{\PYZsq{}}\PY{l+s+s1}{\PYZus{}8thday}\PY{l+s+s1}{\PYZsq{}}\PY{p}{,}\PY{l+m+mi}{8}\PY{p}{)}\PY{p}{)} \PY{c+c1}{\PYZsh{} week[\PYZsq{}\PYZus{}8thday\PYZsq{}] does not exist}

\PY{n}{week}\PY{o}{.}\PY{n}{setdefault}\PY{p}{(}\PY{l+s+s1}{\PYZsq{}}\PY{l+s+s1}{\PYZus{}8thday}\PY{l+s+s1}{\PYZsq{}}\PY{p}{,}\PY{l+m+mi}{8}\PY{p}{)}
\PY{n+nb}{print}\PY{p}{(}\PY{l+s+s2}{\PYZdq{}}\PY{l+s+se}{\PYZbs{}n}\PY{l+s+s2}{week.setdefault(}\PY{l+s+s2}{\PYZsq{}}\PY{l+s+s2}{\PYZus{}8thday}\PY{l+s+s2}{\PYZsq{}}\PY{l+s+s2}{,8)}\PY{l+s+s2}{\PYZdq{}}\PY{p}{)} 
\PY{n+nb}{print}\PY{p}{(}\PY{l+s+s2}{\PYZdq{}}\PY{l+s+s2}{week}\PY{l+s+se}{\PYZbs{}n}\PY{l+s+s2}{ \PYZhy{}\PYZgt{}}\PY{l+s+s2}{\PYZdq{}}\PY{p}{,}\PY{n}{week}\PY{p}{)}

\PY{n+nb}{print}\PY{p}{(}\PY{l+s+s2}{\PYZdq{}}\PY{l+s+se}{\PYZbs{}n}\PY{l+s+s2}{\PYZsh{}update value for a given key}\PY{l+s+s2}{\PYZdq{}}\PY{p}{)}
\PY{n+nb}{print}\PY{p}{(}\PY{l+s+s2}{\PYZdq{}}\PY{l+s+s2}{week[}\PY{l+s+s2}{\PYZsq{}}\PY{l+s+s2}{\PYZus{}8thday}\PY{l+s+s2}{\PYZsq{}}\PY{l+s+s2}{] \PYZhy{}\PYZgt{}}\PY{l+s+s2}{\PYZdq{}}\PY{p}{,}\PY{n}{week}\PY{p}{[}\PY{l+s+s1}{\PYZsq{}}\PY{l+s+s1}{\PYZus{}8thday}\PY{l+s+s1}{\PYZsq{}}\PY{p}{]}\PY{p}{)}
\PY{n}{week}\PY{o}{.}\PY{n}{update}\PY{p}{(}\PY{p}{\PYZob{}}\PY{l+s+s1}{\PYZsq{}}\PY{l+s+s1}{\PYZus{}8thday}\PY{l+s+s1}{\PYZsq{}}\PY{p}{:}\PY{k+kc}{None}\PY{p}{\PYZcb{}}\PY{p}{)}
\PY{n+nb}{print}\PY{p}{(}\PY{l+s+s2}{\PYZdq{}}\PY{l+s+s2}{week.update(}\PY{l+s+s2}{\PYZob{}}\PY{l+s+s2}{\PYZsq{}}\PY{l+s+s2}{\PYZus{}8thday}\PY{l+s+s2}{\PYZsq{}}\PY{l+s+s2}{:None\PYZcb{})}\PY{l+s+se}{\PYZbs{}n}\PY{l+s+s2}{week[}\PY{l+s+s2}{\PYZsq{}}\PY{l+s+s2}{\PYZus{}8thday}\PY{l+s+s2}{\PYZsq{}}\PY{l+s+s2}{] \PYZhy{}\PYZgt{}}\PY{l+s+s2}{\PYZdq{}}\PY{p}{,}\PY{n}{week}\PY{p}{[}\PY{l+s+s1}{\PYZsq{}}\PY{l+s+s1}{\PYZus{}8thday}\PY{l+s+s1}{\PYZsq{}}\PY{p}{]}\PY{p}{)}
\end{Verbatim}
\end{tcolorbox}

    \begin{Verbatim}[commandchars=\\\{\}]

\#update value for a key
week.get('\_8thday',8) -> 8

week.setdefault('\_8thday',8)
week
 -> \{'\_8thday': 8, 'thu': 4, 'sat': 6, 'tue': 2, 'mon': 1, 'sun': 7, 'wed': 3,
'fri': 5\}

\#update value for a given key
week['\_8thday'] -> 8
week.update(\{'\_8thday':None\})
week['\_8thday'] -> None
    \end{Verbatim}

    \begin{tcolorbox}[breakable, size=fbox, boxrule=1pt, pad at break*=1mm,colback=cellbackground, colframe=cellborder]
\prompt{In}{incolor}{37}{\boxspacing}
\begin{Verbatim}[commandchars=\\\{\}]
\PY{o}{\PYZpc{}}\PY{k}{reset} \PYZhy{}f
\PY{n}{week} \PY{o}{=} \PY{p}{\PYZob{}}\PY{l+s+s1}{\PYZsq{}}\PY{l+s+s1}{mon}\PY{l+s+s1}{\PYZsq{}}\PY{p}{:}\PY{l+m+mi}{1}\PY{p}{,}\PY{l+s+s1}{\PYZsq{}}\PY{l+s+s1}{tue}\PY{l+s+s1}{\PYZsq{}}\PY{p}{:}\PY{l+m+mi}{2}\PY{p}{,}\PY{l+s+s1}{\PYZsq{}}\PY{l+s+s1}{wed}\PY{l+s+s1}{\PYZsq{}}\PY{p}{:}\PY{l+m+mi}{3}\PY{p}{,}\PY{l+s+s1}{\PYZsq{}}\PY{l+s+s1}{thu}\PY{l+s+s1}{\PYZsq{}}\PY{p}{:}\PY{l+m+mi}{4}\PY{p}{,}\PY{l+s+s1}{\PYZsq{}}\PY{l+s+s1}{fri}\PY{l+s+s1}{\PYZsq{}}\PY{p}{:}\PY{l+m+mi}{5}\PY{p}{,}\PY{l+s+s1}{\PYZsq{}}\PY{l+s+s1}{sat}\PY{l+s+s1}{\PYZsq{}}\PY{p}{:}\PY{l+m+mi}{6}\PY{p}{,}\PY{l+s+s1}{\PYZsq{}}\PY{l+s+s1}{sun}\PY{l+s+s1}{\PYZsq{}}\PY{p}{:}\PY{l+m+mi}{7}\PY{p}{\PYZcb{}}

\PY{n+nb}{print}\PY{p}{(}\PY{l+s+s2}{\PYZdq{}}\PY{l+s+se}{\PYZbs{}n}\PY{l+s+s2}{week}\PY{l+s+se}{\PYZbs{}n}\PY{l+s+s2}{ \PYZhy{}\PYZgt{}}\PY{l+s+s2}{\PYZdq{}}\PY{p}{,}\PY{n}{week}\PY{p}{)}
\PY{n}{\PYZus{}9thday} \PY{o}{=} \PY{n}{week}\PY{o}{.}\PY{n}{pop}\PY{p}{(}\PY{l+s+s1}{\PYZsq{}}\PY{l+s+s1}{\PYZus{}9thday}\PY{l+s+s1}{\PYZsq{}}\PY{p}{,}\PY{l+s+s1}{\PYZsq{}}\PY{l+s+s1}{8\PYZus{}days\PYZus{}already\PYZus{}too\PYZus{}much}\PY{l+s+s1}{\PYZsq{}}\PY{p}{)}
\PY{n+nb}{print}\PY{p}{(}\PY{l+s+s2}{\PYZdq{}}\PY{l+s+s2}{\PYZus{}9thday = week.pop(}\PY{l+s+s2}{\PYZsq{}}\PY{l+s+s2}{\PYZus{}9thday}\PY{l+s+s2}{\PYZsq{}}\PY{l+s+s2}{,}\PY{l+s+s2}{\PYZsq{}}\PY{l+s+s2}{8\PYZus{}days\PYZus{}already\PYZus{}too\PYZus{}much}\PY{l+s+s2}{\PYZsq{}}\PY{l+s+s2}{)}\PY{l+s+s2}{\PYZdq{}}\PY{o}{+}\PYZbs{}
      \PY{l+s+s2}{\PYZdq{}}\PY{l+s+se}{\PYZbs{}n}\PY{l+s+s2}{\PYZus{}9thday \PYZhy{}\PYZgt{}}\PY{l+s+s2}{\PYZdq{}}\PY{p}{,}\PY{n}{\PYZus{}9thday}\PY{p}{)}

\PY{n+nb}{print}\PY{p}{(}\PY{l+s+s2}{\PYZdq{}}\PY{l+s+se}{\PYZbs{}n}\PY{l+s+s2}{week}\PY{l+s+se}{\PYZbs{}n}\PY{l+s+s2}{ \PYZhy{}\PYZgt{}}\PY{l+s+s2}{\PYZdq{}}\PY{p}{,}\PY{n}{week}\PY{p}{)}
\PY{n}{\PYZus{}8thday} \PY{o}{=} \PY{n}{week}\PY{o}{.}\PY{n}{pop}\PY{p}{(}\PY{l+s+s1}{\PYZsq{}}\PY{l+s+s1}{\PYZus{}8thday}\PY{l+s+s1}{\PYZsq{}}\PY{p}{,}\PY{l+s+s1}{\PYZsq{}}\PY{l+s+s1}{8\PYZus{}days\PYZus{}already\PYZus{}too\PYZus{}much}\PY{l+s+s1}{\PYZsq{}}\PY{p}{)}
\PY{n+nb}{print}\PY{p}{(}\PY{l+s+s2}{\PYZdq{}}\PY{l+s+s2}{\PYZus{}8thday = week.pop(}\PY{l+s+s2}{\PYZsq{}}\PY{l+s+s2}{\PYZus{}8thday}\PY{l+s+s2}{\PYZsq{}}\PY{l+s+s2}{,}\PY{l+s+s2}{\PYZsq{}}\PY{l+s+s2}{8\PYZus{}days\PYZus{}already\PYZus{}too\PYZus{}much}\PY{l+s+s2}{\PYZsq{}}\PY{l+s+s2}{)}\PY{l+s+s2}{\PYZdq{}}\PY{o}{+}\PYZbs{}
      \PY{l+s+s2}{\PYZdq{}}\PY{l+s+se}{\PYZbs{}n}\PY{l+s+s2}{\PYZus{}8thday \PYZhy{}\PYZgt{}}\PY{l+s+s2}{\PYZdq{}}\PY{p}{,}\PY{n}{\PYZus{}8thday}\PY{p}{)}
\PY{n+nb}{print}\PY{p}{(}\PY{l+s+s2}{\PYZdq{}}\PY{l+s+s2}{week}\PY{l+s+se}{\PYZbs{}n}\PY{l+s+s2}{ \PYZhy{}\PYZgt{}}\PY{l+s+s2}{\PYZdq{}}\PY{p}{,}\PY{n}{week}\PY{p}{)}
\end{Verbatim}
\end{tcolorbox}

    \begin{Verbatim}[commandchars=\\\{\}]

week
 -> \{'tue': 2, 'sun': 7, 'wed': 3, 'sat': 6, 'thu': 4, 'mon': 1, 'fri': 5\}
\_9thday = week.pop('\_9thday','8\_days\_already\_too\_much')
\_9thday -> 8\_days\_already\_too\_much

week
 -> \{'tue': 2, 'sun': 7, 'wed': 3, 'sat': 6, 'thu': 4, 'mon': 1, 'fri': 5\}
\_8thday = week.pop('\_8thday','8\_days\_already\_too\_much')
\_8thday -> 8\_days\_already\_too\_much
week
 -> \{'tue': 2, 'sun': 7, 'wed': 3, 'sat': 6, 'thu': 4, 'mon': 1, 'fri': 5\}
    \end{Verbatim}

    \hypertarget{control-flow-tools}{%
\subsection{Control flow tools:}\label{control-flow-tools}}

\begin{itemize}
\tightlist
\item
  conditional statements
\item
  if, elif, else, keywords
\item
  loop statements
\item
  while, for, break, continue, else keywords
\item
  range() function
\item
  reminder statement
\item
  pass keyword
\item
  context manager
\item
  with, as keywords
\item
  open() function
\item
  handling exceptions statement
\item
  try, except, finally keywords
\end{itemize}

    \hypertarget{conditional-statements}{%
\subsection{Conditional statements}\label{conditional-statements}}

\begin{itemize}
\tightlist
\item
  conditional statements
\item
  if, elif, else, keywords
\end{itemize}

\begin{Shaded}
\begin{Highlighting}[]
\ControlFlowTok{if} \OperatorTok{\textless{}}\NormalTok{condition}\OperatorTok{\textgreater{}}\NormalTok{:}
        \OperatorTok{\textless{}}\NormalTok{body\_1}\OperatorTok{\textgreater{}}
\ControlFlowTok{elif} \OperatorTok{\textless{}}\NormalTok{another\_condition}\OperatorTok{\textgreater{}}\NormalTok{:}
        \OperatorTok{\textless{}}\NormalTok{body\_2}\OperatorTok{\textgreater{}}
\ControlFlowTok{else}\NormalTok{:}
        \OperatorTok{\textless{}}\NormalTok{body\_default}\OperatorTok{\textgreater{}}    
\end{Highlighting}
\end{Shaded}

\begin{itemize}
\tightlist
\item
  in, not keywords
\end{itemize}

\begin{Shaded}
\begin{Highlighting}[]
\CommentTok{\# is there an item equal to value?}
\ControlFlowTok{if} \OperatorTok{\textless{}}\NormalTok{value}\OperatorTok{\textgreater{}} \KeywordTok{in} \OperatorTok{\textless{}}\NormalTok{group\_of\_items}\OperatorTok{\textgreater{}} 
\CommentTok{\# none items are equal to value?}
\ControlFlowTok{if} \OperatorTok{\textless{}}\NormalTok{value}\OperatorTok{\textgreater{}} \KeywordTok{not} \KeywordTok{in} \OperatorTok{\textless{}}\NormalTok{group\_of\_items}\OperatorTok{\textgreater{}} 
\end{Highlighting}
\end{Shaded}

    \begin{tcolorbox}[breakable, size=fbox, boxrule=1pt, pad at break*=1mm,colback=cellbackground, colframe=cellborder]
\prompt{In}{incolor}{1}{\boxspacing}
\begin{Verbatim}[commandchars=\\\{\}]
\PY{k}{if} \PY{k+kc}{True}\PY{p}{:} \PY{n+nb}{print}\PY{p}{(}\PY{l+s+s2}{\PYZdq{}}\PY{l+s+s2}{if can be inline}\PY{l+s+s2}{\PYZdq{}}\PY{p}{)}
\end{Verbatim}
\end{tcolorbox}

    \begin{Verbatim}[commandchars=\\\{\}]
if can be inline
    \end{Verbatim}

    \begin{tcolorbox}[breakable, size=fbox, boxrule=1pt, pad at break*=1mm,colback=cellbackground, colframe=cellborder]
\prompt{In}{incolor}{38}{\boxspacing}
\begin{Verbatim}[commandchars=\\\{\}]
\PY{k}{if} \PY{l+m+mi}{5}\PY{o}{\PYZgt{}}\PY{l+m+mi}{4}\PY{p}{:}
    \PY{n+nb}{print}\PY{p}{(}\PY{l+s+s2}{\PYZdq{}}\PY{l+s+s2}{5 is bigger than 4}\PY{l+s+s2}{\PYZdq{}}\PY{p}{)}
\PY{k}{if} \PY{l+m+mi}{4}\PY{o}{\PYZlt{}}\PY{l+m+mi}{5}\PY{p}{:}
    \PY{n+nb}{print}\PY{p}{(}\PY{l+s+s2}{\PYZdq{}}\PY{l+s+s2}{because 4 is smaller than 5}\PY{l+s+s2}{\PYZdq{}}\PY{p}{)}
\PY{k}{else}\PY{p}{:}
    \PY{n+nb}{print}\PY{p}{(}\PY{l+s+s2}{\PYZdq{}}\PY{l+s+s2}{I would be confused otherwise}\PY{l+s+s2}{\PYZdq{}}\PY{p}{)}
\end{Verbatim}
\end{tcolorbox}

    \begin{Verbatim}[commandchars=\\\{\}]
5 is bigger than 4
because 4 is smaller than 5
    \end{Verbatim}

    \begin{tcolorbox}[breakable, size=fbox, boxrule=1pt, pad at break*=1mm,colback=cellbackground, colframe=cellborder]
\prompt{In}{incolor}{41}{\boxspacing}
\begin{Verbatim}[commandchars=\\\{\}]
\PY{k}{if} \PY{l+m+mi}{0}\PY{p}{:}
    \PY{n+nb}{print}\PY{p}{(}\PY{l+s+s2}{\PYZdq{}}\PY{l+s+s2}{this should not be printed}\PY{l+s+s2}{\PYZdq{}}\PY{p}{)}
\PY{k}{elif} \PY{k+kc}{False}\PY{p}{:} 
    \PY{n+nb}{print}\PY{p}{(}\PY{l+s+s2}{\PYZdq{}}\PY{l+s+s2}{this should not be printed}\PY{l+s+s2}{\PYZdq{}}\PY{p}{)}    
\PY{k}{elif} \PY{k+kc}{None}\PY{p}{:}    
    \PY{n+nb}{print}\PY{p}{(}\PY{l+s+s2}{\PYZdq{}}\PY{l+s+s2}{this should not be printed}\PY{l+s+s2}{\PYZdq{}}\PY{p}{)}
\PY{k}{elif} \PY{p}{(}\PY{o}{\PYZhy{}}\PY{l+m+mf}{1.}\PY{o}{+}\PY{l+m+mi}{2}\PY{n}{j}\PY{p}{)}\PY{p}{:} 
    \PY{n+nb}{print}\PY{p}{(}\PY{l+s+s2}{\PYZdq{}}\PY{l+s+s2}{0 and None means False any other value is True }\PY{l+s+s2}{\PYZdq{}}\PY{p}{)}
\end{Verbatim}
\end{tcolorbox}

    \begin{Verbatim}[commandchars=\\\{\}]
0 and None means False any other value is True
    \end{Verbatim}

    \begin{tcolorbox}[breakable, size=fbox, boxrule=1pt, pad at break*=1mm,colback=cellbackground, colframe=cellborder]
\prompt{In}{incolor}{42}{\boxspacing}
\begin{Verbatim}[commandchars=\\\{\}]
\PY{o}{\PYZpc{}}\PY{k}{reset} \PYZhy{}f
\PY{n}{t} \PY{o}{=} \PY{p}{(}\PY{l+m+mi}{3}\PY{p}{,}\PY{l+m+mi}{4}\PY{p}{,}\PY{l+m+mi}{5}\PY{p}{)}       \PY{c+c1}{\PYZsh{} tuple}
\PY{n}{l} \PY{o}{=} \PY{p}{[}\PY{l+m+mi}{3}\PY{p}{,}\PY{l+m+mi}{4}\PY{p}{,}\PY{l+m+mi}{5}\PY{p}{]}       \PY{c+c1}{\PYZsh{} list}
\PY{n}{s} \PY{o}{=} \PY{p}{\PYZob{}}\PY{l+m+mi}{3}\PY{p}{,}\PY{l+m+mi}{4}\PY{p}{,}\PY{l+m+mi}{5}\PY{p}{\PYZcb{}}       \PY{c+c1}{\PYZsh{} set}
\PY{n}{d} \PY{o}{=} \PY{p}{\PYZob{}}\PY{l+m+mi}{3}\PY{p}{:}\PY{l+m+mi}{1}\PY{p}{,}\PY{l+m+mi}{4}\PY{p}{:}\PY{l+m+mi}{2}\PY{p}{,}\PY{l+m+mi}{5}\PY{p}{:}\PY{l+m+mi}{3}\PY{p}{\PYZcb{}} \PY{c+c1}{\PYZsh{} dictionary}
\PY{n}{v} \PY{o}{=} \PY{l+m+mi}{1}
\PY{n+nb}{print}\PY{p}{(}\PY{l+s+s2}{\PYZdq{}}\PY{l+s+s2}{\PYZsq{}}\PY{l+s+s2}{in}\PY{l+s+s2}{\PYZsq{}}\PY{l+s+s2}{/}\PY{l+s+s2}{\PYZsq{}}\PY{l+s+s2}{not in}\PY{l+s+s2}{\PYZsq{}}\PY{l+s+s2}{ work on any iterable}\PY{l+s+s2}{\PYZdq{}}\PY{p}{)}
\PY{k}{if} \PY{n}{v} \PY{o+ow}{not} \PY{o+ow}{in} \PY{n}{t}\PY{p}{:}
    \PY{n+nb}{print}\PY{p}{(}\PY{l+s+s2}{\PYZdq{}}\PY{l+s+s2}{if v not in t:     \PYZhy{}\PYZgt{} True if none is equal to v}\PY{l+s+s2}{\PYZdq{}}\PY{p}{)}
\PY{k}{if} \PY{n}{v} \PY{o+ow}{not} \PY{o+ow}{in} \PY{n}{l}\PY{p}{:}
    \PY{n+nb}{print}\PY{p}{(}\PY{l+s+s2}{\PYZdq{}}\PY{l+s+s2}{if v not in l:     \PYZhy{}\PYZgt{} True if none is equal to v}\PY{l+s+s2}{\PYZdq{}}\PY{p}{)}   
\PY{k}{if} \PY{n}{v} \PY{o+ow}{not} \PY{o+ow}{in} \PY{n}{s}\PY{p}{:}
    \PY{n+nb}{print}\PY{p}{(}\PY{l+s+s2}{\PYZdq{}}\PY{l+s+s2}{if v not in s:     \PYZhy{}\PYZgt{} True if none is equal to v}\PY{l+s+s2}{\PYZdq{}}\PY{p}{)}    
\PY{k}{if} \PY{n}{v} \PY{o+ow}{not} \PY{o+ow}{in} \PY{n}{d}\PY{p}{:}
    \PY{n+nb}{print}\PY{p}{(}\PY{l+s+s2}{\PYZdq{}}\PY{l+s+s2}{if v not in d:     \PYZhy{}\PYZgt{} True if none is equal to v}\PY{l+s+s2}{\PYZdq{}}\PY{p}{)}      
\end{Verbatim}
\end{tcolorbox}

    \begin{Verbatim}[commandchars=\\\{\}]
'in'/'not in' work on any iterable
if v not in t:     -> True if none is equal to v
if v not in l:     -> True if none is equal to v
if v not in s:     -> True if none is equal to v
if v not in d:     -> True if none is equal to v
    \end{Verbatim}

    \hypertarget{loops}{%
\subsection{Loops}\label{loops}}

\begin{itemize}
\tightlist
\item
  loop statements
\item
  while, for, break, continue keywords
\item
  range({[}start,{]}stop{[},step{]}) is very handy in loops
\item
  returns an object that produces a sequence of integers from start
  (inclusive, default 0) to stop (exclusive) by step (default 1)
\item
  start, stop and step must be integers
\item
  pass statement is a reminder and can be used anywhere
\end{itemize}

\begin{Shaded}
\begin{Highlighting}[]
\ControlFlowTok{for} \OperatorTok{\textless{}}\NormalTok{var}\OperatorTok{\textgreater{}} \KeywordTok{in} \OperatorTok{\textless{}}\NormalTok{set\_of\_values}\OperatorTok{\textgreater{}}\NormalTok{: }
    \OperatorTok{\textless{}}\NormalTok{body}\OperatorTok{\textgreater{}}
    \ControlFlowTok{if} \OperatorTok{\textless{}}\NormalTok{condition}\OperatorTok{\textgreater{}}\NormalTok{:}
        \ControlFlowTok{continue}
    \OperatorTok{\textless{}}\NormalTok{something\_that\_runs\_if\_condition\_not\_True}\OperatorTok{\textgreater{}}        
\end{Highlighting}
\end{Shaded}

\begin{Shaded}
\begin{Highlighting}[]
\ControlFlowTok{while} \OperatorTok{\textless{}}\NormalTok{true\_condition}\OperatorTok{\textgreater{}}\NormalTok{:}
    \OperatorTok{\textless{}}\NormalTok{body}\OperatorTok{\textgreater{}}
    \ControlFlowTok{if} \OperatorTok{\textless{}}\NormalTok{another\_condition}\OperatorTok{\textgreater{}}\NormalTok{:}
        \ControlFlowTok{break}    
\end{Highlighting}
\end{Shaded}

    \begin{tcolorbox}[breakable, size=fbox, boxrule=1pt, pad at break*=1mm,colback=cellbackground, colframe=cellborder]
\prompt{In}{incolor}{43}{\boxspacing}
\begin{Verbatim}[commandchars=\\\{\}]
\PY{n+nb}{print}\PY{p}{(}\PY{l+s+s2}{\PYZdq{}}\PY{l+s+s2}{range(2) is}\PY{l+s+s2}{\PYZdq{}}\PY{p}{,}\PY{n+nb}{type}\PY{p}{(}\PY{n+nb}{range}\PY{p}{(}\PY{l+m+mi}{2}\PY{p}{)}\PY{p}{)}\PY{p}{,}\PY{l+s+s2}{\PYZdq{}}\PY{l+s+se}{\PYZbs{}n}\PY{l+s+s2}{\PYZdq{}}\PY{p}{)}
\PY{n+nb}{print}\PY{p}{(}\PY{l+s+s2}{\PYZdq{}}\PY{l+s+se}{\PYZbs{}n}\PY{l+s+s2}{for loop example}\PY{l+s+s2}{\PYZdq{}}\PY{p}{)}
\PY{k}{for} \PY{n}{i} \PY{o+ow}{in} \PY{n+nb}{range}\PY{p}{(}\PY{l+m+mi}{2}\PY{p}{)}\PY{p}{:}
    \PY{n+nb}{print}\PY{p}{(}\PY{n}{i}\PY{p}{)}
    
\PY{n+nb}{print}\PY{p}{(}\PY{l+s+s2}{\PYZdq{}}\PY{l+s+se}{\PYZbs{}n}\PY{l+s+s2}{same behavior using while loop}\PY{l+s+s2}{\PYZdq{}}\PY{p}{)}
\PY{n}{i} \PY{o}{=} \PY{l+m+mi}{0}    
\PY{k}{while} \PY{n}{i} \PY{o+ow}{in} \PY{n+nb}{range}\PY{p}{(}\PY{l+m+mi}{2}\PY{p}{)}\PY{p}{:} 
    \PY{n+nb}{print}\PY{p}{(}\PY{n}{i}\PY{p}{)}
    \PY{n}{i}\PY{o}{+}\PY{o}{=}\PY{l+m+mi}{1}
\end{Verbatim}
\end{tcolorbox}

    \begin{Verbatim}[commandchars=\\\{\}]
range(2) is <class 'range'>


for loop example
0
1

same behavior using while loop
0
1
    \end{Verbatim}

    \begin{tcolorbox}[breakable, size=fbox, boxrule=1pt, pad at break*=1mm,colback=cellbackground, colframe=cellborder]
\prompt{In}{incolor}{115}{\boxspacing}
\begin{Verbatim}[commandchars=\\\{\}]
\PY{k}{for} \PY{n}{i} \PY{o+ow}{in} \PY{n+nb}{range}\PY{p}{(}\PY{l+m+mi}{5}\PY{p}{)}\PY{p}{:}
    \PY{k}{if} \PY{n}{i} \PY{o+ow}{is} \PY{l+m+mi}{0}\PY{p}{:} 
        \PY{k}{continue} \PY{c+c1}{\PYZsh{} 0 is neither odd nor even}
    \PY{k}{if} \PY{n}{i}\PY{o}{\PYZpc{}}\PY{k}{2} is 0:
        \PY{n+nb}{print}\PY{p}{(}\PY{n}{i}\PY{p}{,}\PY{l+s+s2}{\PYZdq{}}\PY{l+s+s2}{is even}\PY{l+s+s2}{\PYZdq{}}\PY{p}{)}
        \PY{k}{continue} \PY{c+c1}{\PYZsh{} an even number is not odd}
    \PY{n+nb}{print}\PY{p}{(}\PY{n}{i}\PY{p}{,}\PY{l+s+s2}{\PYZdq{}}\PY{l+s+s2}{is odd}\PY{l+s+s2}{\PYZdq{}}\PY{p}{)} \PY{c+c1}{\PYZsh{} if i is even, this expression is not run}
\end{Verbatim}
\end{tcolorbox}

    \begin{Verbatim}[commandchars=\\\{\}]
1 is odd
2 is even
3 is odd
4 is even
    \end{Verbatim}

    \hypertarget{the-pass-statement}{%
\subsection{The pass statement}\label{the-pass-statement}}

\begin{itemize}
\tightlist
\item
  does nothing
\item
  helps avoiding unexpected EOF
\item
  useful to include conditions that have not been coded yet
\end{itemize}

    \begin{tcolorbox}[breakable, size=fbox, boxrule=1pt, pad at break*=1mm,colback=cellbackground, colframe=cellborder]
\prompt{In}{incolor}{ }{\boxspacing}
\begin{Verbatim}[commandchars=\\\{\}]
\PY{k}{if} \PY{k+kc}{True}\PY{p}{:}
    \PY{k}{pass}
\PY{n+nb}{print}\PY{p}{(}\PY{l+s+s1}{\PYZsq{}}\PY{l+s+s1}{no syntax error}\PY{l+s+s1}{\PYZsq{}}\PY{p}{)}
\end{Verbatim}
\end{tcolorbox}

    \hypertarget{iterables}{%
\subsection{Iterables}\label{iterables}}

\begin{itemize}
\tightlist
\item
  objects capable of returning its members one at a time
\item
  e.g., list, str, tuple, dict are all iterables
\end{itemize}

\hypertarget{looping-techniques-on-iterables}{%
\subsection{Looping techniques on
iterables}\label{looping-techniques-on-iterables}}

\begin{itemize}
\tightlist
\item
  reversed(iterable): returnes a reversed iterator (iterates from last
  to first)
\item
  zip(*iterables): returns an iterator of tuples that aggregates
  elements from each iterable
\item
  if *iterables have different size, zip aggregates up to the shortest
  length
\item
  enumerate(iterable): returns an iterator of tuples that retreives both
  position index and item\\
\item
  sorted(iterable{[},key{]}{[},reverse{]}): returns the {[}reverse{]}
  sorted sequence as a key for sort comparison
\item
  dictionary items() method: retreives key and value zipped
\end{itemize}

    \begin{tcolorbox}[breakable, size=fbox, boxrule=1pt, pad at break*=1mm,colback=cellbackground, colframe=cellborder]
\prompt{In}{incolor}{124}{\boxspacing}
\begin{Verbatim}[commandchars=\\\{\}]
\PY{o}{\PYZpc{}}\PY{k}{reset} \PYZhy{}f

\PY{n}{myList1} \PY{o}{=} \PY{p}{[}\PY{l+s+s1}{\PYZsq{}}\PY{l+s+s1}{1st\PYZus{}el}\PY{l+s+s1}{\PYZsq{}}\PY{p}{,}\PY{l+s+s1}{\PYZsq{}}\PY{l+s+s1}{2nd\PYZus{}el}\PY{l+s+s1}{\PYZsq{}}\PY{p}{,}\PY{l+s+s1}{\PYZsq{}}\PY{l+s+s1}{3rd\PYZus{}el}\PY{l+s+s1}{\PYZsq{}}\PY{p}{]} 
\PY{n}{myList2} \PY{o}{=} \PY{p}{[}\PY{l+m+mi}{111}\PY{p}{,}\PY{l+m+mi}{222}\PY{p}{,}\PY{l+m+mi}{333}\PY{p}{]}

\PY{c+c1}{\PYZsh{} myList1}
\PY{k}{for} \PY{n}{i} \PY{o+ow}{in} \PY{n}{myList1}\PY{p}{:}
    \PY{n+nb}{print}\PY{p}{(}\PY{n}{i}\PY{p}{,}\PY{n}{end}\PY{o}{=}\PY{l+s+s2}{\PYZdq{}}\PY{l+s+s2}{, }\PY{l+s+s2}{\PYZdq{}}\PY{p}{)}
\PY{n+nb}{print}\PY{p}{(}\PY{p}{)}    

\PY{c+c1}{\PYZsh{} reversed myList1}
\PY{k}{for} \PY{n}{i} \PY{o+ow}{in} \PY{n+nb}{reversed}\PY{p}{(}\PY{n}{myList1}\PY{p}{)}\PY{p}{:}
    \PY{n+nb}{print}\PY{p}{(}\PY{n}{i}\PY{p}{,}\PY{n}{end}\PY{o}{=}\PY{l+s+s2}{\PYZdq{}}\PY{l+s+s2}{, }\PY{l+s+s2}{\PYZdq{}}\PY{p}{)}  
\PY{n+nb}{print}\PY{p}{(}\PY{l+s+s1}{\PYZsq{}}\PY{l+s+se}{\PYZbs{}n}\PY{l+s+s1}{\PYZsq{}}\PY{p}{)}        

\PY{c+c1}{\PYZsh{} zipped lists (with same size)}
\PY{k}{for} \PY{n}{i}\PY{p}{,}\PY{n}{j} \PY{o+ow}{in} \PY{n+nb}{zip}\PY{p}{(}\PY{n}{myList1}\PY{p}{,}\PY{n}{myList2}\PY{p}{)}\PY{p}{:}
    \PY{n+nb}{print}\PY{p}{(}\PY{l+s+s2}{\PYZdq{}}\PY{l+s+s2}{[}\PY{l+s+s2}{\PYZdq{}}\PY{p}{,}\PY{n}{i}\PY{p}{,}\PY{l+s+s2}{\PYZdq{}}\PY{l+s+s2}{,}\PY{l+s+s2}{\PYZdq{}}\PY{p}{,}\PY{n}{j}\PY{p}{,}\PY{l+s+s2}{\PYZdq{}}\PY{l+s+s2}{]}\PY{l+s+s2}{\PYZdq{}}\PY{p}{,} \PY{n}{end}\PY{o}{=}\PY{l+s+s2}{\PYZdq{}}\PY{l+s+s2}{, }\PY{l+s+s2}{\PYZdq{}}\PY{p}{)}
\PY{n+nb}{print}\PY{p}{(}\PY{p}{)}

\PY{c+c1}{\PYZsh{} zipped lists (with different size)}
\PY{n}{myList2}\PY{o}{.}\PY{n}{pop}\PY{p}{(}\PY{p}{)}
\PY{k}{for} \PY{n}{i}\PY{p}{,}\PY{n}{j} \PY{o+ow}{in} \PY{n+nb}{zip}\PY{p}{(}\PY{n}{myList1}\PY{p}{,}\PY{n}{myList2}\PY{p}{)}\PY{p}{:}
    \PY{n+nb}{print}\PY{p}{(}\PY{l+s+s2}{\PYZdq{}}\PY{l+s+s2}{[}\PY{l+s+s2}{\PYZdq{}}\PY{p}{,}\PY{n}{i}\PY{p}{,}\PY{l+s+s2}{\PYZdq{}}\PY{l+s+s2}{,}\PY{l+s+s2}{\PYZdq{}}\PY{p}{,}\PY{n}{j}\PY{p}{,}\PY{l+s+s2}{\PYZdq{}}\PY{l+s+s2}{]}\PY{l+s+s2}{\PYZdq{}}\PY{p}{,} \PY{n}{end}\PY{o}{=}\PY{l+s+s2}{\PYZdq{}}\PY{l+s+s2}{, }\PY{l+s+s2}{\PYZdq{}}\PY{p}{)}
\end{Verbatim}
\end{tcolorbox}

    \begin{Verbatim}[commandchars=\\\{\}]
1st\_el, 2nd\_el, 3rd\_el,
3rd\_el, 2nd\_el, 1st\_el,

[ 1st\_el , 111 ], [ 2nd\_el , 222 ], [ 3rd\_el , 333 ],
[ 1st\_el , 111 ], [ 2nd\_el , 222 ],
    \end{Verbatim}

    \begin{tcolorbox}[breakable, size=fbox, boxrule=1pt, pad at break*=1mm,colback=cellbackground, colframe=cellborder]
\prompt{In}{incolor}{128}{\boxspacing}
\begin{Verbatim}[commandchars=\\\{\}]
\PY{o}{\PYZpc{}}\PY{k}{reset} \PYZhy{}f

\PY{n}{myList1} \PY{o}{=} \PY{p}{[}\PY{l+s+s1}{\PYZsq{}}\PY{l+s+s1}{1st\PYZus{}el}\PY{l+s+s1}{\PYZsq{}}\PY{p}{,}\PY{l+s+s1}{\PYZsq{}}\PY{l+s+s1}{2nd\PYZus{}el}\PY{l+s+s1}{\PYZsq{}}\PY{p}{,}\PY{l+s+s1}{\PYZsq{}}\PY{l+s+s1}{3rd\PYZus{}el}\PY{l+s+s1}{\PYZsq{}}\PY{p}{]} 
\PY{k}{for} \PY{n}{i}\PY{p}{,}\PY{n}{j} \PY{o+ow}{in} \PY{n+nb}{enumerate}\PY{p}{(}\PY{n}{myList1}\PY{p}{)}\PY{p}{:}
    \PY{n+nb}{print}\PY{p}{(}\PY{l+s+s2}{\PYZdq{}}\PY{l+s+s2}{[}\PY{l+s+s2}{\PYZdq{}}\PY{p}{,}\PY{n}{i}\PY{p}{,}\PY{l+s+s2}{\PYZdq{}}\PY{l+s+s2}{,}\PY{l+s+s2}{\PYZdq{}}\PY{p}{,}\PY{n}{j}\PY{p}{,}\PY{l+s+s2}{\PYZdq{}}\PY{l+s+s2}{]}\PY{l+s+s2}{\PYZdq{}}\PY{p}{,} \PY{n}{end}\PY{o}{=}\PY{l+s+s2}{\PYZdq{}}\PY{l+s+s2}{, }\PY{l+s+s2}{\PYZdq{}}\PY{p}{)}   
\PY{n+nb}{print}\PY{p}{(}\PY{l+s+s1}{\PYZsq{}}\PY{l+s+se}{\PYZbs{}n}\PY{l+s+s1}{\PYZsq{}}\PY{p}{)}   

\PY{n}{unorderedList} \PY{o}{=} \PY{p}{[}\PY{l+s+s1}{\PYZsq{}}\PY{l+s+s1}{2nd\PYZus{}el}\PY{l+s+s1}{\PYZsq{}}\PY{p}{,}\PY{l+s+s1}{\PYZsq{}}\PY{l+s+s1}{1st\PYZus{}el}\PY{l+s+s1}{\PYZsq{}}\PY{p}{,}\PY{l+s+s1}{\PYZsq{}}\PY{l+s+s1}{3rd\PYZus{}el}\PY{l+s+s1}{\PYZsq{}}\PY{p}{]} 
\PY{n+nb}{print}\PY{p}{(}\PY{n}{unorderedList}\PY{p}{)}
\PY{k}{for} \PY{n}{i} \PY{o+ow}{in} \PY{n+nb}{sorted}\PY{p}{(}\PY{n}{myList1}\PY{p}{)}\PY{p}{:}
    \PY{n+nb}{print}\PY{p}{(}\PY{n}{i}\PY{p}{,}\PY{n}{end}\PY{o}{=}\PY{l+s+s2}{\PYZdq{}}\PY{l+s+s2}{, }\PY{l+s+s2}{\PYZdq{}}\PY{p}{)}
\end{Verbatim}
\end{tcolorbox}

    \begin{Verbatim}[commandchars=\\\{\}]
[ 0 , 1st\_el ], [ 1 , 2nd\_el ], [ 2 , 3rd\_el ],

['2nd\_el', '1st\_el', '3rd\_el']
1st\_el, 2nd\_el, 3rd\_el,
    \end{Verbatim}

    \hypertarget{context-manager}{%
\subsection{Context manager}\label{context-manager}}

\begin{itemize}
\tightlist
\item
  with {[}as{]} creates a context manager
\end{itemize}

\begin{Shaded}
\begin{Highlighting}[]
\ControlFlowTok{with} \OperatorTok{\textless{}}\NormalTok{expression}\OperatorTok{\textgreater{}}\NormalTok{ [}\ImportTok{as} \OperatorTok{\textless{}}\NormalTok{variable}\OperatorTok{\textgreater{}}\NormalTok{]:}
    \OperatorTok{\textless{}}\ControlFlowTok{with}\OperatorTok{{-}}\NormalTok{body}\OperatorTok{{-}}\NormalTok{block}\OperatorTok{\textgreater{}}    
\end{Highlighting}
\end{Shaded}

\begin{itemize}
\tightlist
\item
  the expression (e.g., open()) must be an object that defines
  \_\emph{enter\_} and \_\emph{exit\_} methods
\item
  open() opens a file and returns a stream (read/write)
\end{itemize}

    \begin{tcolorbox}[breakable, size=fbox, boxrule=1pt, pad at break*=1mm,colback=cellbackground, colframe=cellborder]
\prompt{In}{incolor}{45}{\boxspacing}
\begin{Verbatim}[commandchars=\\\{\}]
\PY{k}{with} \PY{n+nb}{open}\PY{p}{(}\PY{l+s+s1}{\PYZsq{}}\PY{l+s+s1}{writeInIt.txt}\PY{l+s+s1}{\PYZsq{}}\PY{p}{,}\PY{l+s+s1}{\PYZsq{}}\PY{l+s+s1}{w}\PY{l+s+s1}{\PYZsq{}}\PY{p}{)} \PY{k}{as} \PY{n}{f}\PY{p}{:}
    \PY{n}{f}\PY{o}{.}\PY{n}{write}\PY{p}{(}\PY{l+s+s1}{\PYZsq{}}\PY{l+s+s1}{this sentence}\PY{l+s+s1}{\PYZsq{}}\PY{p}{)}
\end{Verbatim}
\end{tcolorbox}

    \begin{tcolorbox}[breakable, size=fbox, boxrule=1pt, pad at break*=1mm,colback=cellbackground, colframe=cellborder]
\prompt{In}{incolor}{46}{\boxspacing}
\begin{Verbatim}[commandchars=\\\{\}]
\PY{k}{with} \PY{n+nb}{open}\PY{p}{(}\PY{l+s+s1}{\PYZsq{}}\PY{l+s+s1}{writeInIt.txt}\PY{l+s+s1}{\PYZsq{}}\PY{p}{,} \PY{l+s+s1}{\PYZsq{}}\PY{l+s+s1}{r}\PY{l+s+s1}{\PYZsq{}}\PY{p}{)} \PY{k}{as} \PY{n}{f}\PY{p}{:}
     \PY{n+nb}{print}\PY{p}{(}\PY{n}{f}\PY{o}{.}\PY{n}{readline}\PY{p}{(}\PY{p}{)}\PY{p}{)} \PY{c+c1}{\PYZsh{} read first line of the file README.md }
\end{Verbatim}
\end{tcolorbox}

    \begin{Verbatim}[commandchars=\\\{\}]
this sentence
    \end{Verbatim}

    \hypertarget{optional}{%
\paragraph{optional}\label{optional}}

\hypertarget{handling-exceptions}{%
\subsection{Handling exceptions}\label{handling-exceptions}}

\begin{itemize}
\tightlist
\item
  handling exceptions using try, except, finally keywords
\item
  finally is executed no matters what
\item
  finally is good to release external resources
\item
  common exceptions are ZeroDivisionError, NameError, TypeError,
  ValueError, SyntaxError
\item
  exceptions of type AssertionError: assert keyword
\item
  exceptions of type Exception: raise keyword
\end{itemize}

\begin{Shaded}
\begin{Highlighting}[]
\ControlFlowTok{try}\NormalTok{: }
    \OperatorTok{\textless{}}\ControlFlowTok{try}\OperatorTok{{-}}\NormalTok{body}\OperatorTok{{-}}\NormalTok{block}\OperatorTok{\textgreater{}}    
\ControlFlowTok{except} \OperatorTok{\textless{}}\PreprocessorTok{Exception}\OperatorTok{{-}}\NormalTok{name}\OperatorTok{\textgreater{}} \ImportTok{as} \OperatorTok{\textless{}}\NormalTok{alias}\OperatorTok{\textgreater{}}\NormalTok{: }
    \OperatorTok{\textless{}}\ControlFlowTok{except}\OperatorTok{{-}}\NormalTok{body}\OperatorTok{{-}}\NormalTok{block}\OperatorTok{\textgreater{}}        
\end{Highlighting}
\end{Shaded}

    \begin{tcolorbox}[breakable, size=fbox, boxrule=1pt, pad at break*=1mm,colback=cellbackground, colframe=cellborder]
\prompt{In}{incolor}{133}{\boxspacing}
\begin{Verbatim}[commandchars=\\\{\}]
\PY{n+nb}{print}\PY{p}{(}\PY{l+s+s2}{\PYZdq{}}\PY{l+s+s2}{\PYZsh{} let}\PY{l+s+s2}{\PYZsq{}}\PY{l+s+s2}{s catch some common exceptions}\PY{l+s+se}{\PYZbs{}n}\PY{l+s+s2}{\PYZdq{}}\PY{p}{)}
\PY{n}{to\PYZus{}execute} \PY{o}{=} \PY{p}{[}\PY{l+s+s1}{\PYZsq{}}\PY{l+s+s1}{1/0}\PY{l+s+s1}{\PYZsq{}}\PY{p}{,}\PYZbs{}
              \PY{l+s+s1}{\PYZsq{}}\PY{l+s+s1}{4+unknown\PYZus{}var}\PY{l+s+s1}{\PYZsq{}}\PY{p}{,}\PYZbs{}
              \PY{l+s+s1}{\PYZsq{}}\PY{l+s+s1}{\PYZdq{}}\PY{l+s+s1}{2}\PY{l+s+s1}{\PYZdq{}}\PY{l+s+s1}{+2}\PY{l+s+s1}{\PYZsq{}}\PY{p}{,}\PYZbs{}
              \PY{l+s+s2}{\PYZdq{}}\PY{l+s+s2}{int(}\PY{l+s+s2}{\PYZsq{}}\PY{l+s+s2}{s}\PY{l+s+s2}{\PYZsq{}}\PY{l+s+s2}{)}\PY{l+s+s2}{\PYZdq{}}\PY{p}{,}
              \PY{l+s+s2}{\PYZdq{}}\PY{l+s+s2}{print 1}\PY{l+s+s2}{\PYZdq{}}\PY{p}{]}
\PY{k}{for} \PY{n}{i} \PY{o+ow}{in} \PY{n}{to\PYZus{}execute}\PY{p}{:}
    \PY{k}{try}\PY{p}{:}
        \PY{n+nb}{print}\PY{p}{(}\PY{n}{i}\PY{p}{)}
        \PY{n+nb}{eval}\PY{p}{(}\PY{n}{i}\PY{p}{)}
        \PY{n+nb}{print}\PY{p}{(}\PY{l+s+s2}{\PYZdq{}}\PY{l+s+s2}{I am not going to be printed}\PY{l+s+s2}{\PYZdq{}}\PY{p}{)}
    \PY{k}{except} \PY{p}{(}\PY{n+ne}{ZeroDivisionError}\PY{p}{,}\PY{n+ne}{NameError}\PY{p}{,}\PY{n+ne}{TypeError}\PY{p}{,}\PY{n+ne}{ValueError}\PY{p}{)} \PY{k}{as} \PY{n}{err}\PY{p}{:}
        \PY{n+nb}{print}\PY{p}{(}\PY{n}{err}\PY{o}{.}\PY{n+nv+vm}{\PYZus{}\PYZus{}class\PYZus{}\PYZus{}}\PY{p}{,}\PY{l+s+s2}{\PYZdq{}}\PY{l+s+s2}{:}\PY{l+s+s2}{\PYZdq{}}\PY{p}{,}\PY{n}{err}\PY{p}{)}
    \PY{k}{except} \PY{n+ne}{SyntaxError} \PY{k}{as} \PY{n}{err}\PY{p}{:} 
        \PY{n+nb}{print}\PY{p}{(}\PY{l+s+s2}{\PYZdq{}}\PY{l+s+s2}{\PYZsq{}}\PY{l+s+s2}{print 1}\PY{l+s+s2}{\PYZsq{}}\PY{l+s+s2}{ was not caught because it causes SyntaxError}\PY{l+s+s2}{\PYZdq{}}\PY{p}{)}
        \PY{n+nb}{print}\PY{p}{(}\PY{n}{err}\PY{o}{.}\PY{n+nv+vm}{\PYZus{}\PYZus{}class\PYZus{}\PYZus{}}\PY{p}{,}\PY{l+s+s2}{\PYZdq{}}\PY{l+s+s2}{:}\PY{l+s+s2}{\PYZdq{}}\PY{p}{,}\PY{n}{err}\PY{p}{)}
    \PY{k}{finally}\PY{p}{:} \PY{c+c1}{\PYZsh{} should be at the end of try statement}
             \PY{c+c1}{\PYZsh{} useful to make sure all resources are released}
             \PY{c+c1}{\PYZsh{} even if an exception occurs}
             \PY{c+c1}{\PYZsh{} even if no exception was caught}
        \PY{n+nb}{print}\PY{p}{(}\PY{l+s+s2}{\PYZdq{}}\PY{l+s+se}{\PYZbs{}t}\PY{l+s+s2}{\PYZhy{}\PYZhy{}\PYZhy{}last but not least, finally is executed\PYZhy{}\PYZhy{}\PYZhy{}}\PY{l+s+s2}{\PYZdq{}}\PY{p}{)}
\end{Verbatim}
\end{tcolorbox}

    \begin{Verbatim}[commandchars=\\\{\}]
\# let's catch some common exceptions

1/0
<class 'ZeroDivisionError'> : division by zero
        ---last but not least, finally is executed---
4+unknown\_var
<class 'NameError'> : name 'unknown\_var' is not defined
        ---last but not least, finally is executed---
"2"+2
<class 'TypeError'> : Can't convert 'int' object to str implicitly
        ---last but not least, finally is executed---
int('s')
<class 'ValueError'> : invalid literal for int() with base 10: 's'
        ---last but not least, finally is executed---
print 1
'print 1' was not caught because it causes SyntaxError
<class 'SyntaxError'> : Missing parentheses in call to 'print' (<string>, line
1)
        ---last but not least, finally is executed---
    \end{Verbatim}

    \begin{tcolorbox}[breakable, size=fbox, boxrule=1pt, pad at break*=1mm,colback=cellbackground, colframe=cellborder]
\prompt{In}{incolor}{134}{\boxspacing}
\begin{Verbatim}[commandchars=\\\{\}]
\PY{n}{x} \PY{o}{=} \PY{o}{\PYZhy{}}\PY{l+m+mi}{1}
\PY{k}{try}\PY{p}{:}
    \PY{k}{assert} \PY{n}{x}\PY{o}{\PYZgt{}}\PY{o}{=}\PY{l+m+mi}{0}\PY{p}{,} \PY{l+s+s1}{\PYZsq{}}\PY{l+s+s1}{x is negative}\PY{l+s+s1}{\PYZsq{}}
\PY{k}{except} \PY{n+ne}{AssertionError} \PY{k}{as} \PY{n}{err}\PY{p}{:} 
    \PY{n+nb}{print}\PY{p}{(}\PY{n}{err}\PY{p}{)}
\end{Verbatim}
\end{tcolorbox}

    \begin{Verbatim}[commandchars=\\\{\}]
x is negative
    \end{Verbatim}

    \begin{tcolorbox}[breakable, size=fbox, boxrule=1pt, pad at break*=1mm,colback=cellbackground, colframe=cellborder]
\prompt{In}{incolor}{144}{\boxspacing}
\begin{Verbatim}[commandchars=\\\{\}]
\PY{k}{try}\PY{p}{:} 
    \PY{k}{raise} \PY{n+ne}{Exception}\PY{p}{(}\PY{l+m+mi}{1}\PY{p}{,}\PY{p}{[}\PY{l+m+mi}{2}\PY{p}{]}\PY{p}{,}\PY{p}{\PYZob{}}\PY{l+s+s1}{\PYZsq{}}\PY{l+s+s1}{3}\PY{l+s+s1}{\PYZsq{}}\PY{p}{:}\PY{l+m+mi}{3}\PY{p}{\PYZcb{}}\PY{p}{)} \PY{c+c1}{\PYZsh{} an exception can be raised}
                                   \PY{c+c1}{\PYZsh{} with any argument}
\PY{k}{except} \PY{n+ne}{Exception} \PY{k}{as} \PY{n}{err}\PY{p}{:}
    \PY{n+nb}{print}\PY{p}{(}\PY{n+nb}{type}\PY{p}{(}\PY{n}{err}\PY{p}{)}\PY{p}{)}
    \PY{n+nb}{print}\PY{p}{(}\PY{n}{err}\PY{o}{.}\PY{n}{args}\PY{p}{)}
    \PY{n+nb}{print}\PY{p}{(}\PY{n}{err}\PY{p}{)}      \PY{c+c1}{\PYZsh{} print its arguments}
    \PY{n}{a}\PY{p}{,}\PY{n}{b}\PY{p}{,}\PY{n}{c} \PY{o}{=} \PY{n}{err}\PY{o}{.}\PY{n}{args}    
    \PY{n+nb}{print}\PY{p}{(}\PY{n}{a}\PY{p}{,}\PY{n}{b}\PY{p}{,}\PY{n}{c}\PY{p}{)}
    \PY{k}{try}\PY{p}{:} 
        \PY{n}{a}\PY{p}{,}\PY{n}{b}\PY{p}{,}\PY{n}{c} \PY{o}{=} \PY{n}{err}
    \PY{k}{except} \PY{n+ne}{TypeError} \PY{k}{as} \PY{n}{err}\PY{p}{:} 
        \PY{n+nb}{print}\PY{p}{(}\PY{n}{err}\PY{p}{)}
\end{Verbatim}
\end{tcolorbox}

    \begin{Verbatim}[commandchars=\\\{\}]
<class 'Exception'>
(1, [2], \{'3': 3\})
(1, [2], \{'3': 3\})
1 [2] \{'3': 3\}
'Exception' object is not iterable
    \end{Verbatim}

    \hypertarget{exercises}{%
\subsection{Exercises}\label{exercises}}

\begin{itemize}
\tightlist
\item
  is\_palindrome(string): True if s is palindrome
\item
  histogram\_letters(string): return dict of \{``letter'': counts\}
\item
  get\_most\_frequent(list): return tuple (element,counts)
\item
  which\_duplicates(list): return dict \{element:{[}occurrences{]}\} of
  duplicates
\item
  compute\_factorial(a)
\item
  is\_prime(a): True if a is prime, otherwise False
\item
  is\_divisor(a,b): True if b is divisor of a, otherwise False
\item
  which\_divisors(a): list of divisors of a
\item
  which\_prime\_divisors(a): list of prime divisors of a
\end{itemize}


    % Add a bibliography block to the postdoc
    
    
    
\end{document}
