\documentclass[11pt]{article}

    \usepackage[breakable]{tcolorbox}
    \usepackage{parskip} % Stop auto-indenting (to mimic markdown behaviour)
    
    \usepackage{iftex}
    \ifPDFTeX
    	\usepackage[T1]{fontenc}
    	\usepackage{mathpazo}
    \else
    	\usepackage{fontspec}
    \fi

    % Basic figure setup, for now with no caption control since it's done
    % automatically by Pandoc (which extracts ![](path) syntax from Markdown).
    \usepackage{graphicx}
    % Maintain compatibility with old templates. Remove in nbconvert 6.0
    \let\Oldincludegraphics\includegraphics
    % Ensure that by default, figures have no caption (until we provide a
    % proper Figure object with a Caption API and a way to capture that
    % in the conversion process - todo).
    \usepackage{caption}
    \DeclareCaptionFormat{nocaption}{}
    \captionsetup{format=nocaption,aboveskip=0pt,belowskip=0pt}

    \usepackage[Export]{adjustbox} % Used to constrain images to a maximum size
    \adjustboxset{max size={0.9\linewidth}{0.9\paperheight}}
    \usepackage{float}
    \floatplacement{figure}{H} % forces figures to be placed at the correct location
    \usepackage{xcolor} % Allow colors to be defined
    \usepackage{enumerate} % Needed for markdown enumerations to work
    \usepackage{geometry} % Used to adjust the document margins
    \usepackage{amsmath} % Equations
    \usepackage{amssymb} % Equations
    \usepackage{textcomp} % defines textquotesingle
    % Hack from http://tex.stackexchange.com/a/47451/13684:
    \AtBeginDocument{%
        \def\PYZsq{\textquotesingle}% Upright quotes in Pygmentized code
    }
    \usepackage{upquote} % Upright quotes for verbatim code
    \usepackage{eurosym} % defines \euro
    \usepackage[mathletters]{ucs} % Extended unicode (utf-8) support
    \usepackage{fancyvrb} % verbatim replacement that allows latex
    \usepackage{grffile} % extends the file name processing of package graphics 
                         % to support a larger range
    \makeatletter % fix for grffile with XeLaTeX
    \def\Gread@@xetex#1{%
      \IfFileExists{"\Gin@base".bb}%
      {\Gread@eps{\Gin@base.bb}}%
      {\Gread@@xetex@aux#1}%
    }
    \makeatother

    % The hyperref package gives us a pdf with properly built
    % internal navigation ('pdf bookmarks' for the table of contents,
    % internal cross-reference links, web links for URLs, etc.)
    \usepackage{hyperref}
    % The default LaTeX title has an obnoxious amount of whitespace. By default,
    % titling removes some of it. It also provides customization options.
    \usepackage{titling}
    \usepackage{longtable} % longtable support required by pandoc >1.10
    \usepackage{booktabs}  % table support for pandoc > 1.12.2
    \usepackage[inline]{enumitem} % IRkernel/repr support (it uses the enumerate* environment)
    \usepackage[normalem]{ulem} % ulem is needed to support strikethroughs (\sout)
                                % normalem makes italics be italics, not underlines
    \usepackage{mathrsfs}
    

    
    % Colors for the hyperref package
    \definecolor{urlcolor}{rgb}{0,.145,.698}
    \definecolor{linkcolor}{rgb}{.71,0.21,0.01}
    \definecolor{citecolor}{rgb}{.12,.54,.11}

    % ANSI colors
    \definecolor{ansi-black}{HTML}{3E424D}
    \definecolor{ansi-black-intense}{HTML}{282C36}
    \definecolor{ansi-red}{HTML}{E75C58}
    \definecolor{ansi-red-intense}{HTML}{B22B31}
    \definecolor{ansi-green}{HTML}{00A250}
    \definecolor{ansi-green-intense}{HTML}{007427}
    \definecolor{ansi-yellow}{HTML}{DDB62B}
    \definecolor{ansi-yellow-intense}{HTML}{B27D12}
    \definecolor{ansi-blue}{HTML}{208FFB}
    \definecolor{ansi-blue-intense}{HTML}{0065CA}
    \definecolor{ansi-magenta}{HTML}{D160C4}
    \definecolor{ansi-magenta-intense}{HTML}{A03196}
    \definecolor{ansi-cyan}{HTML}{60C6C8}
    \definecolor{ansi-cyan-intense}{HTML}{258F8F}
    \definecolor{ansi-white}{HTML}{C5C1B4}
    \definecolor{ansi-white-intense}{HTML}{A1A6B2}
    \definecolor{ansi-default-inverse-fg}{HTML}{FFFFFF}
    \definecolor{ansi-default-inverse-bg}{HTML}{000000}

    % commands and environments needed by pandoc snippets
    % extracted from the output of `pandoc -s`
    \providecommand{\tightlist}{%
      \setlength{\itemsep}{0pt}\setlength{\parskip}{0pt}}
    \DefineVerbatimEnvironment{Highlighting}{Verbatim}{commandchars=\\\{\}}
    % Add ',fontsize=\small' for more characters per line
    \newenvironment{Shaded}{}{}
    \newcommand{\KeywordTok}[1]{\textcolor[rgb]{0.00,0.44,0.13}{\textbf{{#1}}}}
    \newcommand{\DataTypeTok}[1]{\textcolor[rgb]{0.56,0.13,0.00}{{#1}}}
    \newcommand{\DecValTok}[1]{\textcolor[rgb]{0.25,0.63,0.44}{{#1}}}
    \newcommand{\BaseNTok}[1]{\textcolor[rgb]{0.25,0.63,0.44}{{#1}}}
    \newcommand{\FloatTok}[1]{\textcolor[rgb]{0.25,0.63,0.44}{{#1}}}
    \newcommand{\CharTok}[1]{\textcolor[rgb]{0.25,0.44,0.63}{{#1}}}
    \newcommand{\StringTok}[1]{\textcolor[rgb]{0.25,0.44,0.63}{{#1}}}
    \newcommand{\CommentTok}[1]{\textcolor[rgb]{0.38,0.63,0.69}{\textit{{#1}}}}
    \newcommand{\OtherTok}[1]{\textcolor[rgb]{0.00,0.44,0.13}{{#1}}}
    \newcommand{\AlertTok}[1]{\textcolor[rgb]{1.00,0.00,0.00}{\textbf{{#1}}}}
    \newcommand{\FunctionTok}[1]{\textcolor[rgb]{0.02,0.16,0.49}{{#1}}}
    \newcommand{\RegionMarkerTok}[1]{{#1}}
    \newcommand{\ErrorTok}[1]{\textcolor[rgb]{1.00,0.00,0.00}{\textbf{{#1}}}}
    \newcommand{\NormalTok}[1]{{#1}}
    
    % Additional commands for more recent versions of Pandoc
    \newcommand{\ConstantTok}[1]{\textcolor[rgb]{0.53,0.00,0.00}{{#1}}}
    \newcommand{\SpecialCharTok}[1]{\textcolor[rgb]{0.25,0.44,0.63}{{#1}}}
    \newcommand{\VerbatimStringTok}[1]{\textcolor[rgb]{0.25,0.44,0.63}{{#1}}}
    \newcommand{\SpecialStringTok}[1]{\textcolor[rgb]{0.73,0.40,0.53}{{#1}}}
    \newcommand{\ImportTok}[1]{{#1}}
    \newcommand{\DocumentationTok}[1]{\textcolor[rgb]{0.73,0.13,0.13}{\textit{{#1}}}}
    \newcommand{\AnnotationTok}[1]{\textcolor[rgb]{0.38,0.63,0.69}{\textbf{\textit{{#1}}}}}
    \newcommand{\CommentVarTok}[1]{\textcolor[rgb]{0.38,0.63,0.69}{\textbf{\textit{{#1}}}}}
    \newcommand{\VariableTok}[1]{\textcolor[rgb]{0.10,0.09,0.49}{{#1}}}
    \newcommand{\ControlFlowTok}[1]{\textcolor[rgb]{0.00,0.44,0.13}{\textbf{{#1}}}}
    \newcommand{\OperatorTok}[1]{\textcolor[rgb]{0.40,0.40,0.40}{{#1}}}
    \newcommand{\BuiltInTok}[1]{{#1}}
    \newcommand{\ExtensionTok}[1]{{#1}}
    \newcommand{\PreprocessorTok}[1]{\textcolor[rgb]{0.74,0.48,0.00}{{#1}}}
    \newcommand{\AttributeTok}[1]{\textcolor[rgb]{0.49,0.56,0.16}{{#1}}}
    \newcommand{\InformationTok}[1]{\textcolor[rgb]{0.38,0.63,0.69}{\textbf{\textit{{#1}}}}}
    \newcommand{\WarningTok}[1]{\textcolor[rgb]{0.38,0.63,0.69}{\textbf{\textit{{#1}}}}}
    
    
    % Define a nice break command that doesn't care if a line doesn't already
    % exist.
    \def\br{\hspace*{\fill} \\* }
    % Math Jax compatibility definitions
    \def\gt{>}
    \def\lt{<}
    \let\Oldtex\TeX
    \let\Oldlatex\LaTeX
    \renewcommand{\TeX}{\textrm{\Oldtex}}
    \renewcommand{\LaTeX}{\textrm{\Oldlatex}}
    % Document parameters
    % Document title
    \title{python\_list}
    
    
    
    
    
% Pygments definitions
\makeatletter
\def\PY@reset{\let\PY@it=\relax \let\PY@bf=\relax%
    \let\PY@ul=\relax \let\PY@tc=\relax%
    \let\PY@bc=\relax \let\PY@ff=\relax}
\def\PY@tok#1{\csname PY@tok@#1\endcsname}
\def\PY@toks#1+{\ifx\relax#1\empty\else%
    \PY@tok{#1}\expandafter\PY@toks\fi}
\def\PY@do#1{\PY@bc{\PY@tc{\PY@ul{%
    \PY@it{\PY@bf{\PY@ff{#1}}}}}}}
\def\PY#1#2{\PY@reset\PY@toks#1+\relax+\PY@do{#2}}

\@namedef{PY@tok@w}{\def\PY@tc##1{\textcolor[rgb]{0.73,0.73,0.73}{##1}}}
\@namedef{PY@tok@c}{\let\PY@it=\textit\def\PY@tc##1{\textcolor[rgb]{0.25,0.50,0.50}{##1}}}
\@namedef{PY@tok@cp}{\def\PY@tc##1{\textcolor[rgb]{0.74,0.48,0.00}{##1}}}
\@namedef{PY@tok@k}{\let\PY@bf=\textbf\def\PY@tc##1{\textcolor[rgb]{0.00,0.50,0.00}{##1}}}
\@namedef{PY@tok@kp}{\def\PY@tc##1{\textcolor[rgb]{0.00,0.50,0.00}{##1}}}
\@namedef{PY@tok@kt}{\def\PY@tc##1{\textcolor[rgb]{0.69,0.00,0.25}{##1}}}
\@namedef{PY@tok@o}{\def\PY@tc##1{\textcolor[rgb]{0.40,0.40,0.40}{##1}}}
\@namedef{PY@tok@ow}{\let\PY@bf=\textbf\def\PY@tc##1{\textcolor[rgb]{0.67,0.13,1.00}{##1}}}
\@namedef{PY@tok@nb}{\def\PY@tc##1{\textcolor[rgb]{0.00,0.50,0.00}{##1}}}
\@namedef{PY@tok@nf}{\def\PY@tc##1{\textcolor[rgb]{0.00,0.00,1.00}{##1}}}
\@namedef{PY@tok@nc}{\let\PY@bf=\textbf\def\PY@tc##1{\textcolor[rgb]{0.00,0.00,1.00}{##1}}}
\@namedef{PY@tok@nn}{\let\PY@bf=\textbf\def\PY@tc##1{\textcolor[rgb]{0.00,0.00,1.00}{##1}}}
\@namedef{PY@tok@ne}{\let\PY@bf=\textbf\def\PY@tc##1{\textcolor[rgb]{0.82,0.25,0.23}{##1}}}
\@namedef{PY@tok@nv}{\def\PY@tc##1{\textcolor[rgb]{0.10,0.09,0.49}{##1}}}
\@namedef{PY@tok@no}{\def\PY@tc##1{\textcolor[rgb]{0.53,0.00,0.00}{##1}}}
\@namedef{PY@tok@nl}{\def\PY@tc##1{\textcolor[rgb]{0.63,0.63,0.00}{##1}}}
\@namedef{PY@tok@ni}{\let\PY@bf=\textbf\def\PY@tc##1{\textcolor[rgb]{0.60,0.60,0.60}{##1}}}
\@namedef{PY@tok@na}{\def\PY@tc##1{\textcolor[rgb]{0.49,0.56,0.16}{##1}}}
\@namedef{PY@tok@nt}{\let\PY@bf=\textbf\def\PY@tc##1{\textcolor[rgb]{0.00,0.50,0.00}{##1}}}
\@namedef{PY@tok@nd}{\def\PY@tc##1{\textcolor[rgb]{0.67,0.13,1.00}{##1}}}
\@namedef{PY@tok@s}{\def\PY@tc##1{\textcolor[rgb]{0.73,0.13,0.13}{##1}}}
\@namedef{PY@tok@sd}{\let\PY@it=\textit\def\PY@tc##1{\textcolor[rgb]{0.73,0.13,0.13}{##1}}}
\@namedef{PY@tok@si}{\let\PY@bf=\textbf\def\PY@tc##1{\textcolor[rgb]{0.73,0.40,0.53}{##1}}}
\@namedef{PY@tok@se}{\let\PY@bf=\textbf\def\PY@tc##1{\textcolor[rgb]{0.73,0.40,0.13}{##1}}}
\@namedef{PY@tok@sr}{\def\PY@tc##1{\textcolor[rgb]{0.73,0.40,0.53}{##1}}}
\@namedef{PY@tok@ss}{\def\PY@tc##1{\textcolor[rgb]{0.10,0.09,0.49}{##1}}}
\@namedef{PY@tok@sx}{\def\PY@tc##1{\textcolor[rgb]{0.00,0.50,0.00}{##1}}}
\@namedef{PY@tok@m}{\def\PY@tc##1{\textcolor[rgb]{0.40,0.40,0.40}{##1}}}
\@namedef{PY@tok@gh}{\let\PY@bf=\textbf\def\PY@tc##1{\textcolor[rgb]{0.00,0.00,0.50}{##1}}}
\@namedef{PY@tok@gu}{\let\PY@bf=\textbf\def\PY@tc##1{\textcolor[rgb]{0.50,0.00,0.50}{##1}}}
\@namedef{PY@tok@gd}{\def\PY@tc##1{\textcolor[rgb]{0.63,0.00,0.00}{##1}}}
\@namedef{PY@tok@gi}{\def\PY@tc##1{\textcolor[rgb]{0.00,0.63,0.00}{##1}}}
\@namedef{PY@tok@gr}{\def\PY@tc##1{\textcolor[rgb]{1.00,0.00,0.00}{##1}}}
\@namedef{PY@tok@ge}{\let\PY@it=\textit}
\@namedef{PY@tok@gs}{\let\PY@bf=\textbf}
\@namedef{PY@tok@gp}{\let\PY@bf=\textbf\def\PY@tc##1{\textcolor[rgb]{0.00,0.00,0.50}{##1}}}
\@namedef{PY@tok@go}{\def\PY@tc##1{\textcolor[rgb]{0.53,0.53,0.53}{##1}}}
\@namedef{PY@tok@gt}{\def\PY@tc##1{\textcolor[rgb]{0.00,0.27,0.87}{##1}}}
\@namedef{PY@tok@err}{\def\PY@bc##1{{\setlength{\fboxsep}{\string -\fboxrule}\fcolorbox[rgb]{1.00,0.00,0.00}{1,1,1}{\strut ##1}}}}
\@namedef{PY@tok@kc}{\let\PY@bf=\textbf\def\PY@tc##1{\textcolor[rgb]{0.00,0.50,0.00}{##1}}}
\@namedef{PY@tok@kd}{\let\PY@bf=\textbf\def\PY@tc##1{\textcolor[rgb]{0.00,0.50,0.00}{##1}}}
\@namedef{PY@tok@kn}{\let\PY@bf=\textbf\def\PY@tc##1{\textcolor[rgb]{0.00,0.50,0.00}{##1}}}
\@namedef{PY@tok@kr}{\let\PY@bf=\textbf\def\PY@tc##1{\textcolor[rgb]{0.00,0.50,0.00}{##1}}}
\@namedef{PY@tok@bp}{\def\PY@tc##1{\textcolor[rgb]{0.00,0.50,0.00}{##1}}}
\@namedef{PY@tok@fm}{\def\PY@tc##1{\textcolor[rgb]{0.00,0.00,1.00}{##1}}}
\@namedef{PY@tok@vc}{\def\PY@tc##1{\textcolor[rgb]{0.10,0.09,0.49}{##1}}}
\@namedef{PY@tok@vg}{\def\PY@tc##1{\textcolor[rgb]{0.10,0.09,0.49}{##1}}}
\@namedef{PY@tok@vi}{\def\PY@tc##1{\textcolor[rgb]{0.10,0.09,0.49}{##1}}}
\@namedef{PY@tok@vm}{\def\PY@tc##1{\textcolor[rgb]{0.10,0.09,0.49}{##1}}}
\@namedef{PY@tok@sa}{\def\PY@tc##1{\textcolor[rgb]{0.73,0.13,0.13}{##1}}}
\@namedef{PY@tok@sb}{\def\PY@tc##1{\textcolor[rgb]{0.73,0.13,0.13}{##1}}}
\@namedef{PY@tok@sc}{\def\PY@tc##1{\textcolor[rgb]{0.73,0.13,0.13}{##1}}}
\@namedef{PY@tok@dl}{\def\PY@tc##1{\textcolor[rgb]{0.73,0.13,0.13}{##1}}}
\@namedef{PY@tok@s2}{\def\PY@tc##1{\textcolor[rgb]{0.73,0.13,0.13}{##1}}}
\@namedef{PY@tok@sh}{\def\PY@tc##1{\textcolor[rgb]{0.73,0.13,0.13}{##1}}}
\@namedef{PY@tok@s1}{\def\PY@tc##1{\textcolor[rgb]{0.73,0.13,0.13}{##1}}}
\@namedef{PY@tok@mb}{\def\PY@tc##1{\textcolor[rgb]{0.40,0.40,0.40}{##1}}}
\@namedef{PY@tok@mf}{\def\PY@tc##1{\textcolor[rgb]{0.40,0.40,0.40}{##1}}}
\@namedef{PY@tok@mh}{\def\PY@tc##1{\textcolor[rgb]{0.40,0.40,0.40}{##1}}}
\@namedef{PY@tok@mi}{\def\PY@tc##1{\textcolor[rgb]{0.40,0.40,0.40}{##1}}}
\@namedef{PY@tok@il}{\def\PY@tc##1{\textcolor[rgb]{0.40,0.40,0.40}{##1}}}
\@namedef{PY@tok@mo}{\def\PY@tc##1{\textcolor[rgb]{0.40,0.40,0.40}{##1}}}
\@namedef{PY@tok@ch}{\let\PY@it=\textit\def\PY@tc##1{\textcolor[rgb]{0.25,0.50,0.50}{##1}}}
\@namedef{PY@tok@cm}{\let\PY@it=\textit\def\PY@tc##1{\textcolor[rgb]{0.25,0.50,0.50}{##1}}}
\@namedef{PY@tok@cpf}{\let\PY@it=\textit\def\PY@tc##1{\textcolor[rgb]{0.25,0.50,0.50}{##1}}}
\@namedef{PY@tok@c1}{\let\PY@it=\textit\def\PY@tc##1{\textcolor[rgb]{0.25,0.50,0.50}{##1}}}
\@namedef{PY@tok@cs}{\let\PY@it=\textit\def\PY@tc##1{\textcolor[rgb]{0.25,0.50,0.50}{##1}}}

\def\PYZbs{\char`\\}
\def\PYZus{\char`\_}
\def\PYZob{\char`\{}
\def\PYZcb{\char`\}}
\def\PYZca{\char`\^}
\def\PYZam{\char`\&}
\def\PYZlt{\char`\<}
\def\PYZgt{\char`\>}
\def\PYZsh{\char`\#}
\def\PYZpc{\char`\%}
\def\PYZdl{\char`\$}
\def\PYZhy{\char`\-}
\def\PYZsq{\char`\'}
\def\PYZdq{\char`\"}
\def\PYZti{\char`\~}
% for compatibility with earlier versions
\def\PYZat{@}
\def\PYZlb{[}
\def\PYZrb{]}
\makeatother


    % For linebreaks inside Verbatim environment from package fancyvrb. 
    \makeatletter
        \newbox\Wrappedcontinuationbox 
        \newbox\Wrappedvisiblespacebox 
        \newcommand*\Wrappedvisiblespace {\textcolor{red}{\textvisiblespace}} 
        \newcommand*\Wrappedcontinuationsymbol {\textcolor{red}{\llap{\tiny$\m@th\hookrightarrow$}}} 
        \newcommand*\Wrappedcontinuationindent {3ex } 
        \newcommand*\Wrappedafterbreak {\kern\Wrappedcontinuationindent\copy\Wrappedcontinuationbox} 
        % Take advantage of the already applied Pygments mark-up to insert 
        % potential linebreaks for TeX processing. 
        %        {, <, #, %, $, ' and ": go to next line. 
        %        _, }, ^, &, >, - and ~: stay at end of broken line. 
        % Use of \textquotesingle for straight quote. 
        \newcommand*\Wrappedbreaksatspecials {% 
            \def\PYGZus{\discretionary{\char`\_}{\Wrappedafterbreak}{\char`\_}}% 
            \def\PYGZob{\discretionary{}{\Wrappedafterbreak\char`\{}{\char`\{}}% 
            \def\PYGZcb{\discretionary{\char`\}}{\Wrappedafterbreak}{\char`\}}}% 
            \def\PYGZca{\discretionary{\char`\^}{\Wrappedafterbreak}{\char`\^}}% 
            \def\PYGZam{\discretionary{\char`\&}{\Wrappedafterbreak}{\char`\&}}% 
            \def\PYGZlt{\discretionary{}{\Wrappedafterbreak\char`\<}{\char`\<}}% 
            \def\PYGZgt{\discretionary{\char`\>}{\Wrappedafterbreak}{\char`\>}}% 
            \def\PYGZsh{\discretionary{}{\Wrappedafterbreak\char`\#}{\char`\#}}% 
            \def\PYGZpc{\discretionary{}{\Wrappedafterbreak\char`\%}{\char`\%}}% 
            \def\PYGZdl{\discretionary{}{\Wrappedafterbreak\char`\$}{\char`\$}}% 
            \def\PYGZhy{\discretionary{\char`\-}{\Wrappedafterbreak}{\char`\-}}% 
            \def\PYGZsq{\discretionary{}{\Wrappedafterbreak\textquotesingle}{\textquotesingle}}% 
            \def\PYGZdq{\discretionary{}{\Wrappedafterbreak\char`\"}{\char`\"}}% 
            \def\PYGZti{\discretionary{\char`\~}{\Wrappedafterbreak}{\char`\~}}% 
        } 
        % Some characters . , ; ? ! / are not pygmentized. 
        % This macro makes them "active" and they will insert potential linebreaks 
        \newcommand*\Wrappedbreaksatpunct {% 
            \lccode`\~`\.\lowercase{\def~}{\discretionary{\hbox{\char`\.}}{\Wrappedafterbreak}{\hbox{\char`\.}}}% 
            \lccode`\~`\,\lowercase{\def~}{\discretionary{\hbox{\char`\,}}{\Wrappedafterbreak}{\hbox{\char`\,}}}% 
            \lccode`\~`\;\lowercase{\def~}{\discretionary{\hbox{\char`\;}}{\Wrappedafterbreak}{\hbox{\char`\;}}}% 
            \lccode`\~`\:\lowercase{\def~}{\discretionary{\hbox{\char`\:}}{\Wrappedafterbreak}{\hbox{\char`\:}}}% 
            \lccode`\~`\?\lowercase{\def~}{\discretionary{\hbox{\char`\?}}{\Wrappedafterbreak}{\hbox{\char`\?}}}% 
            \lccode`\~`\!\lowercase{\def~}{\discretionary{\hbox{\char`\!}}{\Wrappedafterbreak}{\hbox{\char`\!}}}% 
            \lccode`\~`\/\lowercase{\def~}{\discretionary{\hbox{\char`\/}}{\Wrappedafterbreak}{\hbox{\char`\/}}}% 
            \catcode`\.\active
            \catcode`\,\active 
            \catcode`\;\active
            \catcode`\:\active
            \catcode`\?\active
            \catcode`\!\active
            \catcode`\/\active 
            \lccode`\~`\~ 	
        }
    \makeatother

    \let\OriginalVerbatim=\Verbatim
    \makeatletter
    \renewcommand{\Verbatim}[1][1]{%
        %\parskip\z@skip
        \sbox\Wrappedcontinuationbox {\Wrappedcontinuationsymbol}%
        \sbox\Wrappedvisiblespacebox {\FV@SetupFont\Wrappedvisiblespace}%
        \def\FancyVerbFormatLine ##1{\hsize\linewidth
            \vtop{\raggedright\hyphenpenalty\z@\exhyphenpenalty\z@
                \doublehyphendemerits\z@\finalhyphendemerits\z@
                \strut ##1\strut}%
        }%
        % If the linebreak is at a space, the latter will be displayed as visible
        % space at end of first line, and a continuation symbol starts next line.
        % Stretch/shrink are however usually zero for typewriter font.
        \def\FV@Space {%
            \nobreak\hskip\z@ plus\fontdimen3\font minus\fontdimen4\font
            \discretionary{\copy\Wrappedvisiblespacebox}{\Wrappedafterbreak}
            {\kern\fontdimen2\font}%
        }%
        
        % Allow breaks at special characters using \PYG... macros.
        \Wrappedbreaksatspecials
        % Breaks at punctuation characters . , ; ? ! and / need catcode=\active 	
        \OriginalVerbatim[#1,codes*=\Wrappedbreaksatpunct]%
    }
    \makeatother

    % Exact colors from NB
    \definecolor{incolor}{HTML}{303F9F}
    \definecolor{outcolor}{HTML}{D84315}
    \definecolor{cellborder}{HTML}{CFCFCF}
    \definecolor{cellbackground}{HTML}{F7F7F7}
    
    % prompt
    \makeatletter
    \newcommand{\boxspacing}{\kern\kvtcb@left@rule\kern\kvtcb@boxsep}
    \makeatother
    \newcommand{\prompt}[4]{
        \ttfamily\llap{{\color{#2}[#3]:\hspace{3pt}#4}}\vspace{-\baselineskip}
    }
    

    
    % Prevent overflowing lines due to hard-to-break entities
    \sloppy 
    % Setup hyperref package
    \hypersetup{
      breaklinks=true,  % so long urls are correctly broken across lines
      colorlinks=true,
      urlcolor=urlcolor,
      linkcolor=linkcolor,
      citecolor=citecolor,
      }
    % Slightly bigger margins than the latex defaults
    
    \geometry{verbose,tmargin=1in,bmargin=1in,lmargin=1in,rmargin=1in}
    
    

\begin{document}
    
    \maketitle
    
    

    
    \hypertarget{lists-and-tuples}{%
\section{Lists and Tuples}\label{lists-and-tuples}}

In this notebook, you will learn to store more than one valuable in a
single variable. This by itself is one of the most powerful ideas in
programming, and it introduces a number of other central concepts such
as loops. If this section ends up making sense to you, you will be able
to start writing some interesting programs, and you can be more
confident that you will be able to develop overall competence as a
programmer.

    \href{http://nbviewer.ipython.org/urls/raw.github.com/ehmatthes/intro_programming/master/notebooks/var_string_num.ipynb}{Previous:
Variables, Strings, and Numbers} \textbar{}
\href{http://nbviewer.ipython.org/urls/raw.github.com/ehmatthes/intro_programming/master/notebooks/index.ipynb}{Home}
\textbar{}
\href{http://nbviewer.ipython.org/urls/raw.github.com/ehmatthes/intro_programming/master/notebooks/introducing_functions.ipynb}{Next:
Introducing Functions}

    \hypertarget{contents}{%
\section{Contents}\label{contents}}

\begin{itemize}
\tightlist
\item
  Section \ref{lists}

  \begin{itemize}
  \tightlist
  \item
    Section \ref{introducing-lists}

    \begin{itemize}
    \tightlist
    \item
      Section \ref{example}
    \item
      Section \ref{naming-and-defining-a-list}
    \item
      Section \ref{accessing-one-item-in-a-list}
    \item
      Section \ref{exercises-lists}
    \end{itemize}
  \item
    Section \ref{lists-and-looping}

    \begin{itemize}
    \tightlist
    \item
      Section \ref{accessing-all-elements-in-a-list}
    \item
      Section \ref{enumerating-a-list}
    \item
      Section \ref{exercises-loops}
    \end{itemize}
  \item
    Section \ref{common-list-operations}

    \begin{itemize}
    \tightlist
    \item
      Section \ref{modifying-elements-in-a-list}
    \item
      Section \ref{finding-an-element-in-a-list}
    \item
      Section \ref{testing-whether-an-element-is-in-a-list}
    \item
      Section \ref{adding-items-to-a-list}
    \item
      Section \ref{creating-an-empty-list}
    \item
      Section \ref{sorting-a-list}
    \item
      Section \ref{finding-the-length-of-a-list}
    \item
      Section \ref{exercises-operations}
    \end{itemize}
  \item
    Section \ref{removing-items-from-a-list}

    \begin{itemize}
    \tightlist
    \item
      Section \ref{removing-items-by-position}
    \item
      Section \ref{removing-items-by-value}
    \item
      Section \ref{popping-items}
    \item
      Section \ref{exercises-removing}
    \end{itemize}
  \item
    Section \ref{want-to-see-what-functions-are}
  \item
    Section \ref{slicing-a-list}

    \begin{itemize}
    \tightlist
    \item
      Section \ref{copying-a-list}
    \item
      Section \ref{exercises_slicing}
    \end{itemize}
  \item
    Section \ref{numerical-lists}

    \begin{itemize}
    \tightlist
    \item
      Section \ref{the-range-function}
    \item
      Section \ref{min_max_sum}
    \item
      Section \ref{exercises_numerical}
    \end{itemize}
  \item
    Section \ref{list-comprehensions}

    \begin{itemize}
    \tightlist
    \item
      Section \ref{numerical-comprehensions}
    \item
      Section \ref{non-numerical-comprehensions}
    \item
      Section \ref{exercises_comprehensions}
    \end{itemize}
  \item
    Section \ref{strings-as-lists}

    \begin{itemize}
    \tightlist
    \item
      Section \ref{strings-as-a-list-of-characters}
    \item
      Section \ref{slicing-strings}
    \item
      Section \ref{finding-substrings}
    \item
      Section \ref{replacing-substrings}
    \item
      Section \ref{counting-substrings}
    \item
      Section \ref{splitting-strings}
    \item
      Section \ref{other-string-methods}
    \item
      Section \ref{exercises-strings-as-lists}
    \item
      Section \ref{challenges-strings-as-lists}
    \end{itemize}
  \item
    Section \ref{tuples}

    \begin{itemize}
    \tightlist
    \item
      Section \ref{defining-tuples-and-accessing-elements}
    \item
      Section \ref{using-tuples-to-make-strings}
    \item
      Section \ref{exercises_tuples}
    \end{itemize}
  \item
    Section \ref{coging-style-pep-8}

    \begin{itemize}
    \tightlist
    \item
      Section \ref{why-have-style-conventions}
    \item
      Section \ref{what-is-a-pep}
    \item
      Section \ref{basic-python-style-guidelines}
    \item
      Section \ref{exercises-pep8}
    \end{itemize}
  \item
    Section \ref{overall-challenges}
  \end{itemize}
\end{itemize}

    \hypertarget{lists}{%
\section{Lists}\label{lists}}

    \hypertarget{introducing-lists}{%
\section{Introducing Lists}\label{introducing-lists}}

\hypertarget{example}{%
\subsection{Example}\label{example}}

A list is a collection of items, that is stored in a variable. The items
should be related in some way, but there are no restrictions on what can
be stored in a list. Here is a simple example of a list, and how we can
quickly access each item in the list.

    \begin{tcolorbox}[breakable, size=fbox, boxrule=1pt, pad at break*=1mm,colback=cellbackground, colframe=cellborder]
\prompt{In}{incolor}{53}{\boxspacing}
\begin{Verbatim}[commandchars=\\\{\}]
 \PY{n}{animals} \PY{o}{=} \PY{p}{[}\PY{l+s+s2}{\PYZdq{}}\PY{l+s+s2}{cat}\PY{l+s+s2}{\PYZdq{}}\PY{p}{,} \PY{l+s+s1}{\PYZsq{}}\PY{l+s+s1}{dog}\PY{l+s+s1}{\PYZsq{}}\PY{p}{,} \PY{l+s+s1}{\PYZsq{}}\PY{l+s+s1}{tiger}\PY{l+s+s1}{\PYZsq{}}\PY{p}{,} \PY{l+s+s1}{\PYZsq{}}\PY{l+s+s1}{deer}\PY{l+s+s1}{\PYZsq{}}\PY{p}{]}

 \PY{k}{for} \PY{n}{animal} \PY{o+ow}{in} \PY{n}{animals}\PY{p}{:}
    \PY{n+nb}{print}\PY{p}{(}\PY{l+s+sa}{f}\PY{l+s+s1}{\PYZsq{}}\PY{l+s+s1}{Hello }\PY{l+s+si}{\PYZob{}}\PY{n}{animal}\PY{l+s+si}{\PYZcb{}}\PY{l+s+s1}{!}\PY{l+s+s1}{\PYZsq{}}\PY{p}{)}

 \PY{n+nb}{print}\PY{p}{(}\PY{l+s+s2}{\PYZdq{}}\PY{l+s+se}{\PYZbs{}n}\PY{l+s+se}{\PYZbs{}n}\PY{l+s+se}{\PYZbs{}n}\PY{l+s+s2}{\PYZdq{}}\PY{p}{)}
 \PY{k}{for} \PY{n}{i} \PY{o+ow}{in} \PY{n+nb}{range}\PY{p}{(}\PY{n+nb}{len}\PY{p}{(}\PY{n}{animals}\PY{p}{)}\PY{p}{)}\PY{p}{:}
    \PY{n+nb}{print}\PY{p}{(}\PY{l+s+sa}{f}\PY{l+s+s1}{\PYZsq{}}\PY{l+s+s1}{Hello }\PY{l+s+si}{\PYZob{}}\PY{n}{animals}\PY{p}{[}\PY{n}{i}\PY{p}{]}\PY{l+s+si}{\PYZcb{}}\PY{l+s+s1}{!}\PY{l+s+s1}{\PYZsq{}}\PY{p}{)}
\end{Verbatim}
\end{tcolorbox}

    \begin{Verbatim}[commandchars=\\\{\}]
Hello cat!
Hello dog!
Hello tiger!
Hello deer!




Hello cat!
Hello dog!
Hello tiger!
Hello deer!
    \end{Verbatim}

    \hypertarget{naming-and-defining-a-list}{%
\subsection{Naming and defining a
list}\label{naming-and-defining-a-list}}

Since lists are collection of objects, it is good practice to give them
a plural name. If each item in your list is a car, call the list `cars'.
If each item is a dog, call your list `dogs'. This gives you a
straightforward way to refer to the entire list (`dogs'), and to a
single item in the list (`dog').

In Python, square brackets designate a list. To define a list, you give
the name of the list, the equals sign, and the values you want to
include in your list within square brackets.

    \begin{tcolorbox}[breakable, size=fbox, boxrule=1pt, pad at break*=1mm,colback=cellbackground, colframe=cellborder]
\prompt{In}{incolor}{2}{\boxspacing}
\begin{Verbatim}[commandchars=\\\{\}]
\PY{n}{dogs} \PY{o}{=} \PY{p}{[}\PY{l+s+s1}{\PYZsq{}}\PY{l+s+s1}{border collie}\PY{l+s+s1}{\PYZsq{}}\PY{p}{,} \PY{l+s+s1}{\PYZsq{}}\PY{l+s+s1}{australian cattle dog}\PY{l+s+s1}{\PYZsq{}}\PY{p}{,} \PY{l+s+s1}{\PYZsq{}}\PY{l+s+s1}{labrador retriever}\PY{l+s+s1}{\PYZsq{}}\PY{p}{]}
\end{Verbatim}
\end{tcolorbox}

    \hypertarget{accessing-one-item-in-a-list}{%
\subsection{Accessing one item in a
list}\label{accessing-one-item-in-a-list}}

Items in a list are identified by their position in the list, starting
with zero. This will almost certainly trip you up at some point.
Programmers even joke about how often we all make ``off-by-one'' errors,
so don't feel bad when you make this kind of error.

To access the first element in a list, you give the name of the list,
followed by a zero in parentheses.

    \begin{tcolorbox}[breakable, size=fbox, boxrule=1pt, pad at break*=1mm,colback=cellbackground, colframe=cellborder]
\prompt{In}{incolor}{3}{\boxspacing}
\begin{Verbatim}[commandchars=\\\{\}]
\PY{n}{dogs} \PY{o}{=} \PY{p}{[}\PY{l+s+s1}{\PYZsq{}}\PY{l+s+s1}{border collie}\PY{l+s+s1}{\PYZsq{}}\PY{p}{,} \PY{l+s+s1}{\PYZsq{}}\PY{l+s+s1}{australian cattle dog}\PY{l+s+s1}{\PYZsq{}}\PY{p}{,} \PY{l+s+s1}{\PYZsq{}}\PY{l+s+s1}{labrador retriever}\PY{l+s+s1}{\PYZsq{}}\PY{p}{]}

\PY{n}{dog} \PY{o}{=} \PY{n}{dogs}\PY{p}{[}\PY{l+m+mi}{0}\PY{p}{]}
\PY{n+nb}{print}\PY{p}{(}\PY{n}{dog}\PY{o}{.}\PY{n}{title}\PY{p}{(}\PY{p}{)}\PY{p}{)}
\end{Verbatim}
\end{tcolorbox}

    \begin{Verbatim}[commandchars=\\\{\}]
Border Collie
    \end{Verbatim}

    The number in parentheses is called the \textbf{index} of the item.
Because lists start at zero, the index of an item is always one less
than its position in the list. So to get the second item in the list, we
need to use an index of 1.

    \begin{tcolorbox}[breakable, size=fbox, boxrule=1pt, pad at break*=1mm,colback=cellbackground, colframe=cellborder]
\prompt{In}{incolor}{4}{\boxspacing}
\begin{Verbatim}[commandchars=\\\{\}]
\PY{c+c1}{\PYZsh{}\PYZsh{}\PYZsh{}highlight=[4]}
\PY{n}{dogs} \PY{o}{=} \PY{p}{[}\PY{l+s+s1}{\PYZsq{}}\PY{l+s+s1}{border collie}\PY{l+s+s1}{\PYZsq{}}\PY{p}{,} \PY{l+s+s1}{\PYZsq{}}\PY{l+s+s1}{australian cattle dog}\PY{l+s+s1}{\PYZsq{}}\PY{p}{,} \PY{l+s+s1}{\PYZsq{}}\PY{l+s+s1}{labrador retriever}\PY{l+s+s1}{\PYZsq{}}\PY{p}{]}

\PY{n}{dog} \PY{o}{=} \PY{n}{dogs}\PY{p}{[}\PY{l+m+mi}{1}\PY{p}{]}
\PY{n+nb}{print}\PY{p}{(}\PY{n}{dog}\PY{o}{.}\PY{n}{title}\PY{p}{(}\PY{p}{)}\PY{p}{)}
\end{Verbatim}
\end{tcolorbox}

    \begin{Verbatim}[commandchars=\\\{\}]
Australian Cattle Dog
    \end{Verbatim}

    \hypertarget{accessing-the-last-items-in-a-list}{%
\subsubsection{Accessing the last items in a
list}\label{accessing-the-last-items-in-a-list}}

You can probably see that to get the last item in this list, we would
use an index of 2. This works, but it would only work because our list
has exactly three items. To get the last item in a list, no matter how
long the list is, you can use an index of -1.

    \begin{tcolorbox}[breakable, size=fbox, boxrule=1pt, pad at break*=1mm,colback=cellbackground, colframe=cellborder]
\prompt{In}{incolor}{54}{\boxspacing}
\begin{Verbatim}[commandchars=\\\{\}]
\PY{c+c1}{\PYZsh{}\PYZsh{}\PYZsh{}highlight=[4]}
\PY{n}{dogs} \PY{o}{=} \PY{p}{[}\PY{l+s+s1}{\PYZsq{}}\PY{l+s+s1}{border collie}\PY{l+s+s1}{\PYZsq{}}\PY{p}{,} \PY{l+s+s1}{\PYZsq{}}\PY{l+s+s1}{australian cattle dog}\PY{l+s+s1}{\PYZsq{}}\PY{p}{,} \PY{l+s+s1}{\PYZsq{}}\PY{l+s+s1}{labrador retriever}\PY{l+s+s1}{\PYZsq{}}\PY{p}{]}

\PY{n+nb}{print}\PY{p}{(}\PY{n}{dogs}\PY{p}{[}\PY{o}{\PYZhy{}}\PY{l+m+mi}{1}\PY{p}{]}\PY{p}{)}
\end{Verbatim}
\end{tcolorbox}

    \begin{Verbatim}[commandchars=\\\{\}]
labrador retriever
    \end{Verbatim}

    This syntax also works for the second to last item, the third to last,
and so forth.

    \begin{tcolorbox}[breakable, size=fbox, boxrule=1pt, pad at break*=1mm,colback=cellbackground, colframe=cellborder]
\prompt{In}{incolor}{6}{\boxspacing}
\begin{Verbatim}[commandchars=\\\{\}]
\PY{c+c1}{\PYZsh{}\PYZsh{}\PYZsh{}highlight=[4]}
\PY{n}{dogs} \PY{o}{=} \PY{p}{[}\PY{l+s+s1}{\PYZsq{}}\PY{l+s+s1}{border collie}\PY{l+s+s1}{\PYZsq{}}\PY{p}{,} \PY{l+s+s1}{\PYZsq{}}\PY{l+s+s1}{australian cattle dog}\PY{l+s+s1}{\PYZsq{}}\PY{p}{,} \PY{l+s+s1}{\PYZsq{}}\PY{l+s+s1}{labrador retriever}\PY{l+s+s1}{\PYZsq{}}\PY{p}{]}

\PY{n}{dog} \PY{o}{=} \PY{n}{dogs}\PY{p}{[}\PY{o}{\PYZhy{}}\PY{l+m+mi}{2}\PY{p}{]}
\PY{n+nb}{print}\PY{p}{(}\PY{n}{dog}\PY{o}{.}\PY{n}{title}\PY{p}{(}\PY{p}{)}\PY{p}{)}
\end{Verbatim}
\end{tcolorbox}

    \begin{Verbatim}[commandchars=\\\{\}]
Australian Cattle Dog
    \end{Verbatim}

    You can't use a negative number larger than the length of the list,
however.

    \begin{tcolorbox}[breakable, size=fbox, boxrule=1pt, pad at break*=1mm,colback=cellbackground, colframe=cellborder]
\prompt{In}{incolor}{55}{\boxspacing}
\begin{Verbatim}[commandchars=\\\{\}]
\PY{c+c1}{\PYZsh{}\PYZsh{}\PYZsh{}highlight=[4]}
\PY{n}{dogs} \PY{o}{=} \PY{p}{[}\PY{l+s+s1}{\PYZsq{}}\PY{l+s+s1}{border collie}\PY{l+s+s1}{\PYZsq{}}\PY{p}{,} \PY{l+s+s1}{\PYZsq{}}\PY{l+s+s1}{australian cattle dog}\PY{l+s+s1}{\PYZsq{}}\PY{p}{,} \PY{l+s+s1}{\PYZsq{}}\PY{l+s+s1}{labrador retriever}\PY{l+s+s1}{\PYZsq{}}\PY{p}{]}

\PY{n}{dog} \PY{o}{=} \PY{n}{dogs}\PY{p}{[}\PY{o}{\PYZhy{}}\PY{l+m+mi}{3}\PY{p}{]}
\PY{n+nb}{print}\PY{p}{(}\PY{n}{dog}\PY{o}{.}\PY{n}{title}\PY{p}{(}\PY{p}{)}\PY{p}{)}
\end{Verbatim}
\end{tcolorbox}

    \begin{Verbatim}[commandchars=\\\{\}]
Border Collie
    \end{Verbatim}

    \protect\hyperlink{}{top}

    Exercises --- \#\#\#\# First List - Store the values `python', `c', and
`java' in a list. Print each of these values out, using their position
in the list.

\hypertarget{first-neat-list}{%
\paragraph{First Neat List}\label{first-neat-list}}

\begin{itemize}
\tightlist
\item
  Store the values `python', `c', and `java' in a list. Print a
  statement about each of these values, using their position in the
  list.
\item
  Your statement could simply be, `A nice programming language is
  \emph{value}.'
\end{itemize}

\hypertarget{your-first-list}{%
\paragraph{Your First List}\label{your-first-list}}

\begin{itemize}
\tightlist
\item
  Think of something you can store in a list. Make a list with three or
  four items, and then print a message that includes at least one item
  from your list. Your sentence could be as simple as, ``One item in my
  list is a \_\_\_\_.''
\end{itemize}

    \begin{tcolorbox}[breakable, size=fbox, boxrule=1pt, pad at break*=1mm,colback=cellbackground, colframe=cellborder]
\prompt{In}{incolor}{59}{\boxspacing}
\begin{Verbatim}[commandchars=\\\{\}]
\PY{n}{items} \PY{o}{=} \PY{p}{[}\PY{l+s+s1}{\PYZsq{}}\PY{l+s+s1}{python}\PY{l+s+s1}{\PYZsq{}}\PY{p}{,} \PY{l+s+s1}{\PYZsq{}}\PY{l+s+s1}{c}\PY{l+s+s1}{\PYZsq{}}\PY{p}{,} \PY{l+s+s1}{\PYZsq{}}\PY{l+s+s1}{java}\PY{l+s+s1}{\PYZsq{}}\PY{p}{]}

\PY{k}{for} \PY{n}{index}\PY{p}{,} \PY{n}{item} \PY{o+ow}{in} \PY{n+nb}{enumerate}\PY{p}{(}\PY{n}{items}\PY{p}{)}\PY{p}{:}
    \PY{n+nb}{print}\PY{p}{(}\PY{l+s+sa}{f}\PY{l+s+s1}{\PYZsq{}}\PY{l+s+si}{\PYZob{}}\PY{n}{item}\PY{l+s+si}{\PYZcb{}}\PY{l+s+s1}{ at }\PY{l+s+si}{\PYZob{}}\PY{n}{index}\PY{l+s+si}{\PYZcb{}}\PY{l+s+s1}{th position}\PY{l+s+s1}{\PYZsq{}}\PY{p}{)}
\end{Verbatim}
\end{tcolorbox}

    \begin{Verbatim}[commandchars=\\\{\}]
python at 0th position
c at 1th position
java at 2th position
    \end{Verbatim}

    \protect\hyperlink{}{top}

    \hypertarget{lists-and-looping}{%
\section{Lists and Looping}\label{lists-and-looping}}

    \hypertarget{accessing-all-elements-in-a-list}{%
\subsection{Accessing all elements in a
list}\label{accessing-all-elements-in-a-list}}

This is one of the most important concepts related to lists. You can
have a list with a million items in it, and in three lines of code you
can write a sentence for each of those million items. If you want to
understand lists, and become a competent programmer, make sure you take
the time to understand this section.

We use a loop to access all the elements in a list. A loop is a block of
code that repeats itself until it runs out of items to work with, or
until a certain condition is met. In this case, our loop will run once
for every item in our list. With a list that is three items long, our
loop will run three times.

Let's take a look at how we access all the items in a list, and then try
to understand how it works.

    \begin{tcolorbox}[breakable, size=fbox, boxrule=1pt, pad at break*=1mm,colback=cellbackground, colframe=cellborder]
\prompt{In}{incolor}{ }{\boxspacing}
\begin{Verbatim}[commandchars=\\\{\}]
\PY{n}{dogs} \PY{o}{=} \PY{p}{[}\PY{l+s+s1}{\PYZsq{}}\PY{l+s+s1}{border collie}\PY{l+s+s1}{\PYZsq{}}\PY{p}{,} \PY{l+s+s1}{\PYZsq{}}\PY{l+s+s1}{australian cattle dog}\PY{l+s+s1}{\PYZsq{}}\PY{p}{,} \PY{l+s+s1}{\PYZsq{}}\PY{l+s+s1}{labrador retriever}\PY{l+s+s1}{\PYZsq{}}\PY{p}{]}

\PY{k}{for} \PY{n}{dog} \PY{o+ow}{in} \PY{n}{dogs}\PY{p}{:}
    \PY{n+nb}{print}\PY{p}{(}\PY{n}{dog}\PY{p}{)}
\end{Verbatim}
\end{tcolorbox}

    We have already seen how to create a list, so we are really just trying
to understand how the last two lines work. These last two lines make up
a loop, and the language here can help us see what is happening:

\begin{verbatim}
for dog in dogs:
\end{verbatim}

\begin{itemize}
\tightlist
\item
  The keyword ``for'' tells Python to get ready to use a loop.
\item
  The variable ``dog'', with no ``s'' on it, is a temporary placeholder
  variable. This is the variable that Python will place each item in the
  list into, one at a time.
\item
  The first time through the loop, the value of ``dog'' will be `border
  collie'.
\item
  The second time through the loop, the value of ``dog'' will be
  `australian cattle dog'.
\item
  The third time through, ``dog'' will be `labrador retriever'.
\item
  After this, there are no more items in the list, and the loop will
  end.
\end{itemize}

The site pythontutor.com allows you to run Python code one line at a
time. As you run the code, there is also a visualization on the screen
that shows you how the variable ``dog'' holds different values as the
loop progresses. There is also an arrow that moves around your code,
showing you how some lines are run just once, while other lines are run
multiple tiimes. If you would like to see this in action, click the
Forward button and watch the visualization, and the output as it is
printed to the screen. Tools like this are incredibly valuable for
seeing what Python is doing with your code.

\hypertarget{doing-more-with-each-item}{%
\subsubsection{Doing more with each
item}\label{doing-more-with-each-item}}

We can do whatever we want with the value of ``dog'' inside the loop. In
this case, we just print the name of the dog.

\begin{verbatim}
print(dog)
\end{verbatim}

We are not limited to just printing the word dog. We can do whatever we
want with this value, and this action will be carried out for every item
in the list. Let's say something about each dog in our list.

    \begin{tcolorbox}[breakable, size=fbox, boxrule=1pt, pad at break*=1mm,colback=cellbackground, colframe=cellborder]
\prompt{In}{incolor}{ }{\boxspacing}
\begin{Verbatim}[commandchars=\\\{\}]
\PY{c+c1}{\PYZsh{}\PYZsh{}\PYZsh{}highlight=[5]}
\PY{n}{dogs} \PY{o}{=} \PY{p}{[}\PY{l+s+s1}{\PYZsq{}}\PY{l+s+s1}{border collie}\PY{l+s+s1}{\PYZsq{}}\PY{p}{,} \PY{l+s+s1}{\PYZsq{}}\PY{l+s+s1}{australian cattle dog}\PY{l+s+s1}{\PYZsq{}}\PY{p}{,} \PY{l+s+s1}{\PYZsq{}}\PY{l+s+s1}{labrador retriever}\PY{l+s+s1}{\PYZsq{}}\PY{p}{]}

\PY{k}{for} \PY{n}{dog} \PY{o+ow}{in} \PY{n}{dogs}\PY{p}{:}
    \PY{n+nb}{print}\PY{p}{(}\PY{l+s+s1}{\PYZsq{}}\PY{l+s+s1}{I like }\PY{l+s+s1}{\PYZsq{}} \PY{o}{+} \PY{n}{dog} \PY{o}{+} \PY{l+s+s1}{\PYZsq{}}\PY{l+s+s1}{s.}\PY{l+s+s1}{\PYZsq{}}\PY{p}{)}
\end{Verbatim}
\end{tcolorbox}

    Visualize this on pythontutor.

\hypertarget{inside-and-outside-the-loop}{%
\subsubsection{Inside and outside the
loop}\label{inside-and-outside-the-loop}}

Python uses indentation to decide what is inside the loop and what is
outside the loop. Code that is inside the loop will be run for every
item in the list. Code that is not indented, which comes after the loop,
will be run once just like regular code.

    \begin{tcolorbox}[breakable, size=fbox, boxrule=1pt, pad at break*=1mm,colback=cellbackground, colframe=cellborder]
\prompt{In}{incolor}{ }{\boxspacing}
\begin{Verbatim}[commandchars=\\\{\}]
\PY{c+c1}{\PYZsh{}\PYZsh{}\PYZsh{}highlight=[6,7,8]}
\PY{n}{dogs} \PY{o}{=} \PY{p}{[}\PY{l+s+s1}{\PYZsq{}}\PY{l+s+s1}{border collie}\PY{l+s+s1}{\PYZsq{}}\PY{p}{,} \PY{l+s+s1}{\PYZsq{}}\PY{l+s+s1}{australian cattle dog}\PY{l+s+s1}{\PYZsq{}}\PY{p}{,} \PY{l+s+s1}{\PYZsq{}}\PY{l+s+s1}{labrador retriever}\PY{l+s+s1}{\PYZsq{}}\PY{p}{]}

\PY{k}{for} \PY{n}{dog} \PY{o+ow}{in} \PY{n}{dogs}\PY{p}{:}
    \PY{n+nb}{print}\PY{p}{(}\PY{l+s+s1}{\PYZsq{}}\PY{l+s+s1}{I like }\PY{l+s+s1}{\PYZsq{}} \PY{o}{+} \PY{n}{dog} \PY{o}{+} \PY{l+s+s1}{\PYZsq{}}\PY{l+s+s1}{s.}\PY{l+s+s1}{\PYZsq{}}\PY{p}{)}
    \PY{n+nb}{print}\PY{p}{(}\PY{l+s+s1}{\PYZsq{}}\PY{l+s+s1}{No, I really really like }\PY{l+s+s1}{\PYZsq{}} \PY{o}{+} \PY{n}{dog} \PY{o}{+}\PY{l+s+s1}{\PYZsq{}}\PY{l+s+s1}{s!}\PY{l+s+se}{\PYZbs{}n}\PY{l+s+s1}{\PYZsq{}}\PY{p}{)}
    
\PY{n+nb}{print}\PY{p}{(}\PY{l+s+s2}{\PYZdq{}}\PY{l+s+se}{\PYZbs{}n}\PY{l+s+s2}{That}\PY{l+s+s2}{\PYZsq{}}\PY{l+s+s2}{s just how I feel about dogs.}\PY{l+s+s2}{\PYZdq{}}\PY{p}{)}
\end{Verbatim}
\end{tcolorbox}

    Notice that the last line only runs once, after the loop is completed.
Also notice the use of newlines (``\n'') to make the output easier to
read. Run this code on pythontutor.

    \protect\hyperlink{}{top}

    \hypertarget{enumerating-a-list}{%
\subsection{Enumerating a list}\label{enumerating-a-list}}

When you are looping through a list, you may want to know the index of
the current item. You could always use the \emph{list.index(value)}
syntax, but there is a simpler way. The \emph{enumerate()} function
tracks the index of each item for you, as it loops through the list:

    \begin{tcolorbox}[breakable, size=fbox, boxrule=1pt, pad at break*=1mm,colback=cellbackground, colframe=cellborder]
\prompt{In}{incolor}{60}{\boxspacing}
\begin{Verbatim}[commandchars=\\\{\}]
\PY{n}{dogs} \PY{o}{=} \PY{p}{[}\PY{l+s+s1}{\PYZsq{}}\PY{l+s+s1}{border collie}\PY{l+s+s1}{\PYZsq{}}\PY{p}{,} \PY{l+s+s1}{\PYZsq{}}\PY{l+s+s1}{australian cattle dog}\PY{l+s+s1}{\PYZsq{}}\PY{p}{,} \PY{l+s+s1}{\PYZsq{}}\PY{l+s+s1}{labrador retriever}\PY{l+s+s1}{\PYZsq{}}\PY{p}{]}

\PY{n+nb}{print}\PY{p}{(}\PY{l+s+s2}{\PYZdq{}}\PY{l+s+s2}{Results for the dog show are as follows:}\PY{l+s+se}{\PYZbs{}n}\PY{l+s+s2}{\PYZdq{}}\PY{p}{)}
\PY{k}{for} \PY{n}{index}\PY{p}{,} \PY{n}{dog} \PY{o+ow}{in} \PY{n+nb}{enumerate}\PY{p}{(}\PY{n}{dogs}\PY{p}{)}\PY{p}{:}
    \PY{n}{place} \PY{o}{=} \PY{n+nb}{str}\PY{p}{(}\PY{n}{index}\PY{p}{)}
    \PY{n+nb}{print}\PY{p}{(}\PY{l+s+s2}{\PYZdq{}}\PY{l+s+s2}{Place: }\PY{l+s+s2}{\PYZdq{}} \PY{o}{+} \PY{n}{place} \PY{o}{+} \PY{l+s+s2}{\PYZdq{}}\PY{l+s+s2}{ Dog: }\PY{l+s+s2}{\PYZdq{}} \PY{o}{+} \PY{n}{dog}\PY{o}{.}\PY{n}{title}\PY{p}{(}\PY{p}{)}\PY{p}{)}
\end{Verbatim}
\end{tcolorbox}

    \begin{Verbatim}[commandchars=\\\{\}]
Results for the dog show are as follows:

Place: 0 Dog: Border Collie
Place: 1 Dog: Australian Cattle Dog
Place: 2 Dog: Labrador Retriever
    \end{Verbatim}

    To enumerate a list, you need to add an \emph{index} variable to hold
the current index. So instead of

\begin{verbatim}
for dog in dogs:
\end{verbatim}

You have

\begin{verbatim}
for index, dog in enumerate(dogs)
\end{verbatim}

The value in the variable \emph{index} is always an integer. If you want
to print it in a string, you have to turn the integer into a string:

\begin{verbatim}
str(index)
\end{verbatim}

The index always starts at 0, so in this example the value of
\emph{place} should actually be the current index, plus one:

    \begin{tcolorbox}[breakable, size=fbox, boxrule=1pt, pad at break*=1mm,colback=cellbackground, colframe=cellborder]
\prompt{In}{incolor}{ }{\boxspacing}
\begin{Verbatim}[commandchars=\\\{\}]
\PY{c+c1}{\PYZsh{}\PYZsh{}\PYZsh{}highlight=[6]}
\PY{n}{dogs} \PY{o}{=} \PY{p}{[}\PY{l+s+s1}{\PYZsq{}}\PY{l+s+s1}{border collie}\PY{l+s+s1}{\PYZsq{}}\PY{p}{,} \PY{l+s+s1}{\PYZsq{}}\PY{l+s+s1}{australian cattle dog}\PY{l+s+s1}{\PYZsq{}}\PY{p}{,} \PY{l+s+s1}{\PYZsq{}}\PY{l+s+s1}{labrador retriever}\PY{l+s+s1}{\PYZsq{}}\PY{p}{]}

\PY{n+nb}{print}\PY{p}{(}\PY{l+s+s2}{\PYZdq{}}\PY{l+s+s2}{Results for the dog show are as follows:}\PY{l+s+se}{\PYZbs{}n}\PY{l+s+s2}{\PYZdq{}}\PY{p}{)}
\PY{k}{for} \PY{n}{index}\PY{p}{,} \PY{n}{dog} \PY{o+ow}{in} \PY{n+nb}{enumerate}\PY{p}{(}\PY{n}{dogs}\PY{p}{)}\PY{p}{:}
    \PY{n}{place} \PY{o}{=} \PY{n+nb}{str}\PY{p}{(}\PY{n}{index} \PY{o}{+} \PY{l+m+mi}{1}\PY{p}{)}
    \PY{n+nb}{print}\PY{p}{(}\PY{l+s+s2}{\PYZdq{}}\PY{l+s+s2}{Place: }\PY{l+s+s2}{\PYZdq{}} \PY{o}{+} \PY{n}{place} \PY{o}{+} \PY{l+s+s2}{\PYZdq{}}\PY{l+s+s2}{ Dog: }\PY{l+s+s2}{\PYZdq{}} \PY{o}{+} \PY{n}{dog}\PY{o}{.}\PY{n}{title}\PY{p}{(}\PY{p}{)}\PY{p}{)}
\end{Verbatim}
\end{tcolorbox}

    \hypertarget{a-common-looping-error}{%
\subsubsection{A common looping error}\label{a-common-looping-error}}

One common looping error occurs when instead of using the single
variable \emph{dog} inside the loop, we accidentally use the variable
that holds the entire list:

    \begin{tcolorbox}[breakable, size=fbox, boxrule=1pt, pad at break*=1mm,colback=cellbackground, colframe=cellborder]
\prompt{In}{incolor}{61}{\boxspacing}
\begin{Verbatim}[commandchars=\\\{\}]
\PY{c+c1}{\PYZsh{}\PYZsh{}\PYZsh{}highlight=[5]}
\PY{n}{dogs} \PY{o}{=} \PY{p}{[}\PY{l+s+s1}{\PYZsq{}}\PY{l+s+s1}{border collie}\PY{l+s+s1}{\PYZsq{}}\PY{p}{,} \PY{l+s+s1}{\PYZsq{}}\PY{l+s+s1}{australian cattle dog}\PY{l+s+s1}{\PYZsq{}}\PY{p}{,} \PY{l+s+s1}{\PYZsq{}}\PY{l+s+s1}{labrador retriever}\PY{l+s+s1}{\PYZsq{}}\PY{p}{]}

\PY{k}{for} \PY{n}{dog} \PY{o+ow}{in} \PY{n}{dogs}\PY{p}{:}
    \PY{n+nb}{print}\PY{p}{(}\PY{n}{dogs}\PY{p}{)}
\end{Verbatim}
\end{tcolorbox}

    \begin{Verbatim}[commandchars=\\\{\}]
['border collie', 'australian cattle dog', 'labrador retriever']
['border collie', 'australian cattle dog', 'labrador retriever']
['border collie', 'australian cattle dog', 'labrador retriever']
    \end{Verbatim}

    In this example, instead of printing each dog in the list, we print the
entire list every time we go through the loop. Python puts each
individual item in the list into the variable \emph{dog}, but we never
use that variable. Sometimes you will just get an error if you try to do
this:

    \begin{tcolorbox}[breakable, size=fbox, boxrule=1pt, pad at break*=1mm,colback=cellbackground, colframe=cellborder]
\prompt{In}{incolor}{62}{\boxspacing}
\begin{Verbatim}[commandchars=\\\{\}]
\PY{c+c1}{\PYZsh{}\PYZsh{}\PYZsh{}highlight=[5]}
\PY{n}{dogs} \PY{o}{=} \PY{p}{[}\PY{l+s+s1}{\PYZsq{}}\PY{l+s+s1}{border collie}\PY{l+s+s1}{\PYZsq{}}\PY{p}{,} \PY{l+s+s1}{\PYZsq{}}\PY{l+s+s1}{australian cattle dog}\PY{l+s+s1}{\PYZsq{}}\PY{p}{,} \PY{l+s+s1}{\PYZsq{}}\PY{l+s+s1}{labrador retriever}\PY{l+s+s1}{\PYZsq{}}\PY{p}{]}

\PY{k}{for} \PY{n}{dog} \PY{o+ow}{in} \PY{n}{dogs}\PY{p}{:}
    \PY{n+nb}{print}\PY{p}{(}\PY{l+s+s1}{\PYZsq{}}\PY{l+s+s1}{I like }\PY{l+s+s1}{\PYZsq{}} \PY{o}{+} \PY{n}{dogs} \PY{o}{+} \PY{l+s+s1}{\PYZsq{}}\PY{l+s+s1}{s.}\PY{l+s+s1}{\PYZsq{}}\PY{p}{)}
\end{Verbatim}
\end{tcolorbox}

    \begin{Verbatim}[commandchars=\\\{\}]

        ---------------------------------------------------------------------------

        TypeError                                 Traceback (most recent call last)

        /var/folders/1\_/2rbdt5292zdb57b821k90d5r0000gn/T/ipykernel\_56860/2180598526.py in <module>
          3 
          4 for dog in dogs:
    ----> 5     print('I like ' + dogs + 's.')
    

        TypeError: can only concatenate str (not "list") to str

    \end{Verbatim}

    Exercises --- \#\#\#\# First List - Loop - Repeat \emph{First List}, but
this time use a loop to print out each value in the list.

\hypertarget{first-neat-list---loop}{%
\paragraph{First Neat List - Loop}\label{first-neat-list---loop}}

\begin{itemize}
\tightlist
\item
  Repeat \emph{First Neat List}, but this time use a loop to print out
  your statements. Make sure you are writing the same sentence for all
  values in your list. Loops are not effective when you are trying to
  generate different output for each value in your list.
\end{itemize}

\hypertarget{your-first-list---loop}{%
\paragraph{Your First List - Loop}\label{your-first-list---loop}}

\begin{itemize}
\tightlist
\item
  Repeat \emph{Your First List}, but this time use a loop to print out
  your message for each item in your list. Again, if you came up with
  different messages for each value in your list, decide on one message
  to repeat for each value in your list.
\end{itemize}

    \protect\hyperlink{}{top}

    \hypertarget{common-list-operations}{%
\section{Common List Operations}\label{common-list-operations}}

    \hypertarget{modifying-elements-in-a-list}{%
\subsection{Modifying elements in a
list}\label{modifying-elements-in-a-list}}

You can change the value of any element in a list if you know the
position of that item.

    \begin{tcolorbox}[breakable, size=fbox, boxrule=1pt, pad at break*=1mm,colback=cellbackground, colframe=cellborder]
\prompt{In}{incolor}{ }{\boxspacing}
\begin{Verbatim}[commandchars=\\\{\}]
\PY{n}{dogs} \PY{o}{=} \PY{p}{[}\PY{l+s+s1}{\PYZsq{}}\PY{l+s+s1}{border collie}\PY{l+s+s1}{\PYZsq{}}\PY{p}{,} \PY{l+s+s1}{\PYZsq{}}\PY{l+s+s1}{australian cattle dog}\PY{l+s+s1}{\PYZsq{}}\PY{p}{,} \PY{l+s+s1}{\PYZsq{}}\PY{l+s+s1}{labrador retriever}\PY{l+s+s1}{\PYZsq{}}\PY{p}{]}

\PY{n}{dogs}\PY{p}{[}\PY{l+m+mi}{0}\PY{p}{]} \PY{o}{=} \PY{l+s+s1}{\PYZsq{}}\PY{l+s+s1}{australian shepherd}\PY{l+s+s1}{\PYZsq{}}
\PY{n+nb}{print}\PY{p}{(}\PY{n}{dogs}\PY{p}{)}
\end{Verbatim}
\end{tcolorbox}

    \hypertarget{finding-an-element-in-a-list}{%
\subsection{Finding an element in a
list}\label{finding-an-element-in-a-list}}

If you want to find out the position of an element in a list, you can
use the index() function.

    \begin{tcolorbox}[breakable, size=fbox, boxrule=1pt, pad at break*=1mm,colback=cellbackground, colframe=cellborder]
\prompt{In}{incolor}{63}{\boxspacing}
\begin{Verbatim}[commandchars=\\\{\}]
\PY{n}{dogs} \PY{o}{=} \PY{p}{[}\PY{l+s+s1}{\PYZsq{}}\PY{l+s+s1}{border collie}\PY{l+s+s1}{\PYZsq{}}\PY{p}{,} \PY{l+s+s1}{\PYZsq{}}\PY{l+s+s1}{australian cattle dog}\PY{l+s+s1}{\PYZsq{}}\PY{p}{,} \PY{l+s+s1}{\PYZsq{}}\PY{l+s+s1}{labrador retriever}\PY{l+s+s1}{\PYZsq{}}\PY{p}{]}

\PY{n+nb}{print}\PY{p}{(}\PY{n}{dogs}\PY{o}{.}\PY{n}{index}\PY{p}{(}\PY{l+s+s1}{\PYZsq{}}\PY{l+s+s1}{australian cattle dog}\PY{l+s+s1}{\PYZsq{}}\PY{p}{)}\PY{p}{)}
\end{Verbatim}
\end{tcolorbox}

    \begin{Verbatim}[commandchars=\\\{\}]
1
    \end{Verbatim}

    This method returns a ValueError if the requested item is not in the
list.

    \begin{tcolorbox}[breakable, size=fbox, boxrule=1pt, pad at break*=1mm,colback=cellbackground, colframe=cellborder]
\prompt{In}{incolor}{64}{\boxspacing}
\begin{Verbatim}[commandchars=\\\{\}]
\PY{c+c1}{\PYZsh{}\PYZsh{}\PYZsh{}highlight=[4]}
\PY{n}{dogs} \PY{o}{=} \PY{p}{[}\PY{l+s+s1}{\PYZsq{}}\PY{l+s+s1}{border collie}\PY{l+s+s1}{\PYZsq{}}\PY{p}{,} \PY{l+s+s1}{\PYZsq{}}\PY{l+s+s1}{australian cattle dog}\PY{l+s+s1}{\PYZsq{}}\PY{p}{,} \PY{l+s+s1}{\PYZsq{}}\PY{l+s+s1}{labrador retriever}\PY{l+s+s1}{\PYZsq{}}\PY{p}{]}

\PY{n+nb}{print}\PY{p}{(}\PY{n}{dogs}\PY{o}{.}\PY{n}{index}\PY{p}{(}\PY{l+s+s1}{\PYZsq{}}\PY{l+s+s1}{poodle}\PY{l+s+s1}{\PYZsq{}}\PY{p}{)}\PY{p}{)}
\end{Verbatim}
\end{tcolorbox}

    \begin{Verbatim}[commandchars=\\\{\}]

        ---------------------------------------------------------------------------

        ValueError                                Traceback (most recent call last)

        /var/folders/1\_/2rbdt5292zdb57b821k90d5r0000gn/T/ipykernel\_56860/1758207819.py in <module>
          2 dogs = ['border collie', 'australian cattle dog', 'labrador retriever']
          3 
    ----> 4 print(dogs.index('poodle'))
    

        ValueError: 'poodle' is not in list

    \end{Verbatim}

    \hypertarget{testing-whether-an-item-is-in-a-list}{%
\subsection{Testing whether an item is in a
list}\label{testing-whether-an-item-is-in-a-list}}

You can test whether an item is in a list using the ``in'' keyword. This
will become more useful after learning how to use if-else statements.

    \begin{tcolorbox}[breakable, size=fbox, boxrule=1pt, pad at break*=1mm,colback=cellbackground, colframe=cellborder]
\prompt{In}{incolor}{65}{\boxspacing}
\begin{Verbatim}[commandchars=\\\{\}]
\PY{n}{dogs} \PY{o}{=} \PY{p}{[}\PY{l+s+s1}{\PYZsq{}}\PY{l+s+s1}{border collie}\PY{l+s+s1}{\PYZsq{}}\PY{p}{,} \PY{l+s+s1}{\PYZsq{}}\PY{l+s+s1}{australian cattle dog}\PY{l+s+s1}{\PYZsq{}}\PY{p}{,} \PY{l+s+s1}{\PYZsq{}}\PY{l+s+s1}{labrador retriever}\PY{l+s+s1}{\PYZsq{}}\PY{p}{]}

\PY{n+nb}{print}\PY{p}{(}\PY{l+s+s1}{\PYZsq{}}\PY{l+s+s1}{australian cattle dog}\PY{l+s+s1}{\PYZsq{}} \PY{o+ow}{in} \PY{n}{dogs}\PY{p}{)}
\PY{n+nb}{print}\PY{p}{(}\PY{l+s+s1}{\PYZsq{}}\PY{l+s+s1}{poodle}\PY{l+s+s1}{\PYZsq{}} \PY{o+ow}{in} \PY{n}{dogs}\PY{p}{)}
\end{Verbatim}
\end{tcolorbox}

    \begin{Verbatim}[commandchars=\\\{\}]
True
False
    \end{Verbatim}

    \hypertarget{adding-items-to-a-list}{%
\subsection{Adding items to a list}\label{adding-items-to-a-list}}

\hypertarget{appending-items-to-the-end-of-a-list}{%
\subsubsection{Appending items to the end of a
list}\label{appending-items-to-the-end-of-a-list}}

We can add an item to a list using the append() method. This method adds
the new item to the end of the list.

    \begin{tcolorbox}[breakable, size=fbox, boxrule=1pt, pad at break*=1mm,colback=cellbackground, colframe=cellborder]
\prompt{In}{incolor}{ }{\boxspacing}
\begin{Verbatim}[commandchars=\\\{\}]
\PY{n}{dogs} \PY{o}{=} \PY{p}{[}\PY{l+s+s1}{\PYZsq{}}\PY{l+s+s1}{border collie}\PY{l+s+s1}{\PYZsq{}}\PY{p}{,} \PY{l+s+s1}{\PYZsq{}}\PY{l+s+s1}{australian cattle dog}\PY{l+s+s1}{\PYZsq{}}\PY{p}{,} \PY{l+s+s1}{\PYZsq{}}\PY{l+s+s1}{labrador retriever}\PY{l+s+s1}{\PYZsq{}}\PY{p}{]}
\PY{n}{dogs}\PY{o}{.}\PY{n}{append}\PY{p}{(}\PY{l+s+s1}{\PYZsq{}}\PY{l+s+s1}{poodle}\PY{l+s+s1}{\PYZsq{}}\PY{p}{)}

\PY{k}{for} \PY{n}{dog} \PY{o+ow}{in} \PY{n}{dogs}\PY{p}{:}
    \PY{n+nb}{print}\PY{p}{(}\PY{n}{dog}\PY{o}{.}\PY{n}{title}\PY{p}{(}\PY{p}{)} \PY{o}{+} \PY{l+s+s2}{\PYZdq{}}\PY{l+s+s2}{s are cool.}\PY{l+s+s2}{\PYZdq{}}\PY{p}{)}
\end{Verbatim}
\end{tcolorbox}

    \hypertarget{inserting-items-into-a-list}{%
\subsubsection{Inserting items into a
list}\label{inserting-items-into-a-list}}

We can also insert items anywhere we want in a list, using the
\textbf{insert()} function. We specify the position we want the item to
have, and everything from that point on is shifted one position to the
right. In other words, the index of every item after the new item is
increased by one.

    \begin{tcolorbox}[breakable, size=fbox, boxrule=1pt, pad at break*=1mm,colback=cellbackground, colframe=cellborder]
\prompt{In}{incolor}{ }{\boxspacing}
\begin{Verbatim}[commandchars=\\\{\}]
\PY{n}{dogs} \PY{o}{=} \PY{p}{[}\PY{l+s+s1}{\PYZsq{}}\PY{l+s+s1}{border collie}\PY{l+s+s1}{\PYZsq{}}\PY{p}{,} \PY{l+s+s1}{\PYZsq{}}\PY{l+s+s1}{australian cattle dog}\PY{l+s+s1}{\PYZsq{}}\PY{p}{,} \PY{l+s+s1}{\PYZsq{}}\PY{l+s+s1}{labrador retriever}\PY{l+s+s1}{\PYZsq{}}\PY{p}{]}
\PY{n}{dogs}\PY{o}{.}\PY{n}{insert}\PY{p}{(}\PY{l+m+mi}{1}\PY{p}{,} \PY{l+s+s1}{\PYZsq{}}\PY{l+s+s1}{poodle}\PY{l+s+s1}{\PYZsq{}}\PY{p}{)}

\PY{n+nb}{print}\PY{p}{(}\PY{n}{dogs}\PY{p}{)}
\end{Verbatim}
\end{tcolorbox}

    Note that you have to give the position of the new item first, and then
the value of the new item. If you do it in the reverse order, you will
get an error.

    \hypertarget{creating-an-empty-list}{%
\subsection{Creating an empty list}\label{creating-an-empty-list}}

Now that we know how to add items to a list after it is created, we can
use lists more dynamically. We are no longer stuck defining our entire
list at once.

A common approach with lists is to define an empty list, and then let
your program add items to the list as necessary. This approach works,
for example, when starting to build an interactive web site. Your list
of users might start out empty, and then as people register for the site
it will grow. This is a simplified approach to how web sites actually
work, but the idea is realistic.

Here is a brief example of how to start with an empty list, start to
fill it up, and work with the items in the list. The only new thing here
is the way we define an empty list, which is just an empty set of square
brackets.

    \begin{tcolorbox}[breakable, size=fbox, boxrule=1pt, pad at break*=1mm,colback=cellbackground, colframe=cellborder]
\prompt{In}{incolor}{ }{\boxspacing}
\begin{Verbatim}[commandchars=\\\{\}]
\PY{c+c1}{\PYZsh{} Create an empty list to hold our users.}
\PY{n}{usernames} \PY{o}{=} \PY{p}{[}\PY{p}{]}

\PY{c+c1}{\PYZsh{} Add some users.}
\PY{n}{usernames}\PY{o}{.}\PY{n}{append}\PY{p}{(}\PY{l+s+s1}{\PYZsq{}}\PY{l+s+s1}{bernice}\PY{l+s+s1}{\PYZsq{}}\PY{p}{)}
\PY{n}{usernames}\PY{o}{.}\PY{n}{append}\PY{p}{(}\PY{l+s+s1}{\PYZsq{}}\PY{l+s+s1}{cody}\PY{l+s+s1}{\PYZsq{}}\PY{p}{)}
\PY{n}{usernames}\PY{o}{.}\PY{n}{append}\PY{p}{(}\PY{l+s+s1}{\PYZsq{}}\PY{l+s+s1}{aaron}\PY{l+s+s1}{\PYZsq{}}\PY{p}{)}

\PY{c+c1}{\PYZsh{} Greet all of our users.}
\PY{k}{for} \PY{n}{username} \PY{o+ow}{in} \PY{n}{usernames}\PY{p}{:}
    \PY{n+nb}{print}\PY{p}{(}\PY{l+s+s2}{\PYZdq{}}\PY{l+s+s2}{Welcome, }\PY{l+s+s2}{\PYZdq{}} \PY{o}{+} \PY{n}{username}\PY{o}{.}\PY{n}{title}\PY{p}{(}\PY{p}{)} \PY{o}{+} \PY{l+s+s1}{\PYZsq{}}\PY{l+s+s1}{!}\PY{l+s+s1}{\PYZsq{}}\PY{p}{)}
\end{Verbatim}
\end{tcolorbox}

    If we don't change the order in our list, we can use the list to figure
out who our oldest and newest users are.

    \begin{tcolorbox}[breakable, size=fbox, boxrule=1pt, pad at break*=1mm,colback=cellbackground, colframe=cellborder]
\prompt{In}{incolor}{ }{\boxspacing}
\begin{Verbatim}[commandchars=\\\{\}]
\PY{c+c1}{\PYZsh{}\PYZsh{}\PYZsh{}highlight=[10,11,12]}
\PY{c+c1}{\PYZsh{} Create an empty list to hold our users.}
\PY{n}{usernames} \PY{o}{=} \PY{p}{[}\PY{p}{]}

\PY{c+c1}{\PYZsh{} Add some users.}
\PY{n}{usernames}\PY{o}{.}\PY{n}{append}\PY{p}{(}\PY{l+s+s1}{\PYZsq{}}\PY{l+s+s1}{bernice}\PY{l+s+s1}{\PYZsq{}}\PY{p}{)}
\PY{n}{usernames}\PY{o}{.}\PY{n}{append}\PY{p}{(}\PY{l+s+s1}{\PYZsq{}}\PY{l+s+s1}{cody}\PY{l+s+s1}{\PYZsq{}}\PY{p}{)}
\PY{n}{usernames}\PY{o}{.}\PY{n}{append}\PY{p}{(}\PY{l+s+s1}{\PYZsq{}}\PY{l+s+s1}{aaron}\PY{l+s+s1}{\PYZsq{}}\PY{p}{)}

\PY{c+c1}{\PYZsh{} Greet all of our users.}
\PY{k}{for} \PY{n}{username} \PY{o+ow}{in} \PY{n}{usernames}\PY{p}{:}
    \PY{n+nb}{print}\PY{p}{(}\PY{l+s+s2}{\PYZdq{}}\PY{l+s+s2}{Welcome, }\PY{l+s+s2}{\PYZdq{}} \PY{o}{+} \PY{n}{username}\PY{o}{.}\PY{n}{title}\PY{p}{(}\PY{p}{)} \PY{o}{+} \PY{l+s+s1}{\PYZsq{}}\PY{l+s+s1}{!}\PY{l+s+s1}{\PYZsq{}}\PY{p}{)}

\PY{c+c1}{\PYZsh{} Recognize our first user, and welcome our newest user.}
\PY{n+nb}{print}\PY{p}{(}\PY{l+s+s2}{\PYZdq{}}\PY{l+s+se}{\PYZbs{}n}\PY{l+s+s2}{Thank you for being our very first user, }\PY{l+s+s2}{\PYZdq{}} \PY{o}{+} \PY{n}{usernames}\PY{p}{[}\PY{l+m+mi}{0}\PY{p}{]}\PY{o}{.}\PY{n}{title}\PY{p}{(}\PY{p}{)} \PY{o}{+} \PY{l+s+s1}{\PYZsq{}}\PY{l+s+s1}{!}\PY{l+s+s1}{\PYZsq{}}\PY{p}{)}
\PY{n+nb}{print}\PY{p}{(}\PY{l+s+s2}{\PYZdq{}}\PY{l+s+s2}{And a warm welcome to our newest user, }\PY{l+s+s2}{\PYZdq{}} \PY{o}{+} \PY{n}{usernames}\PY{p}{[}\PY{o}{\PYZhy{}}\PY{l+m+mi}{1}\PY{p}{]}\PY{o}{.}\PY{n}{title}\PY{p}{(}\PY{p}{)} \PY{o}{+} \PY{l+s+s1}{\PYZsq{}}\PY{l+s+s1}{!}\PY{l+s+s1}{\PYZsq{}}\PY{p}{)}
\end{Verbatim}
\end{tcolorbox}

    Note that the code welcoming our newest user will always work, because
we have used the index -1. If we had used the index 2 we would always
get the third user, even as our list of users grows and grows.

    \hypertarget{sorting-a-list}{%
\subsection{Sorting a List}\label{sorting-a-list}}

We can sort a list alphabetically, in either order.

    \begin{tcolorbox}[breakable, size=fbox, boxrule=1pt, pad at break*=1mm,colback=cellbackground, colframe=cellborder]
\prompt{In}{incolor}{ }{\boxspacing}
\begin{Verbatim}[commandchars=\\\{\}]
\PY{n}{students} \PY{o}{=} \PY{p}{[}\PY{l+s+s1}{\PYZsq{}}\PY{l+s+s1}{bernice}\PY{l+s+s1}{\PYZsq{}}\PY{p}{,} \PY{l+s+s1}{\PYZsq{}}\PY{l+s+s1}{aaron}\PY{l+s+s1}{\PYZsq{}}\PY{p}{,} \PY{l+s+s1}{\PYZsq{}}\PY{l+s+s1}{cody}\PY{l+s+s1}{\PYZsq{}}\PY{p}{]}

\PY{c+c1}{\PYZsh{} Put students in alphabetical order.}
\PY{n}{students}\PY{o}{.}\PY{n}{sort}\PY{p}{(}\PY{p}{)}

\PY{c+c1}{\PYZsh{} Display the list in its current order.}
\PY{n+nb}{print}\PY{p}{(}\PY{l+s+s2}{\PYZdq{}}\PY{l+s+s2}{Our students are currently in alphabetical order.}\PY{l+s+s2}{\PYZdq{}}\PY{p}{)}
\PY{k}{for} \PY{n}{student} \PY{o+ow}{in} \PY{n}{students}\PY{p}{:}
    \PY{n+nb}{print}\PY{p}{(}\PY{n}{student}\PY{o}{.}\PY{n}{title}\PY{p}{(}\PY{p}{)}\PY{p}{)}

\PY{c+c1}{\PYZsh{}Put students in reverse alphabetical order.}
\PY{n}{students}\PY{o}{.}\PY{n}{sort}\PY{p}{(}\PY{n}{reverse}\PY{o}{=}\PY{k+kc}{True}\PY{p}{)}

\PY{c+c1}{\PYZsh{} Display the list in its current order.}
\PY{n+nb}{print}\PY{p}{(}\PY{l+s+s2}{\PYZdq{}}\PY{l+s+se}{\PYZbs{}n}\PY{l+s+s2}{Our students are now in reverse alphabetical order.}\PY{l+s+s2}{\PYZdq{}}\PY{p}{)}
\PY{k}{for} \PY{n}{student} \PY{o+ow}{in} \PY{n}{students}\PY{p}{:}
    \PY{n+nb}{print}\PY{p}{(}\PY{n}{student}\PY{o}{.}\PY{n}{title}\PY{p}{(}\PY{p}{)}\PY{p}{)}
\end{Verbatim}
\end{tcolorbox}

    \hypertarget{sorted-vs.-sort}{%
\subsubsection{\texorpdfstring{\emph{sorted()}
vs.~\emph{sort()}}{sorted() vs.~sort()}}\label{sorted-vs.-sort}}

Whenever you consider sorting a list, keep in mind that you can not
recover the original order. If you want to display a list in sorted
order, but preserve the original order, you can use the \emph{sorted()}
function. The \emph{sorted()} function also accepts the optional
\emph{reverse=True} argument.

    \begin{tcolorbox}[breakable, size=fbox, boxrule=1pt, pad at break*=1mm,colback=cellbackground, colframe=cellborder]
\prompt{In}{incolor}{ }{\boxspacing}
\begin{Verbatim}[commandchars=\\\{\}]
\PY{n}{students} \PY{o}{=} \PY{p}{[}\PY{l+s+s1}{\PYZsq{}}\PY{l+s+s1}{bernice}\PY{l+s+s1}{\PYZsq{}}\PY{p}{,} \PY{l+s+s1}{\PYZsq{}}\PY{l+s+s1}{aaron}\PY{l+s+s1}{\PYZsq{}}\PY{p}{,} \PY{l+s+s1}{\PYZsq{}}\PY{l+s+s1}{cody}\PY{l+s+s1}{\PYZsq{}}\PY{p}{]}

\PY{c+c1}{\PYZsh{} Display students in alphabetical order, but keep the original order.}
\PY{n+nb}{print}\PY{p}{(}\PY{l+s+s2}{\PYZdq{}}\PY{l+s+s2}{Here is the list in alphabetical order:}\PY{l+s+s2}{\PYZdq{}}\PY{p}{)}
\PY{k}{for} \PY{n}{student} \PY{o+ow}{in} \PY{n+nb}{sorted}\PY{p}{(}\PY{n}{students}\PY{p}{)}\PY{p}{:}
    \PY{n+nb}{print}\PY{p}{(}\PY{n}{student}\PY{o}{.}\PY{n}{title}\PY{p}{(}\PY{p}{)}\PY{p}{)}

\PY{c+c1}{\PYZsh{} Display students in reverse alphabetical order, but keep the original order.}
\PY{n+nb}{print}\PY{p}{(}\PY{l+s+s2}{\PYZdq{}}\PY{l+s+se}{\PYZbs{}n}\PY{l+s+s2}{Here is the list in reverse alphabetical order:}\PY{l+s+s2}{\PYZdq{}}\PY{p}{)}
\PY{k}{for} \PY{n}{student} \PY{o+ow}{in} \PY{n+nb}{sorted}\PY{p}{(}\PY{n}{students}\PY{p}{,} \PY{n}{reverse}\PY{o}{=}\PY{k+kc}{True}\PY{p}{)}\PY{p}{:}
    \PY{n+nb}{print}\PY{p}{(}\PY{n}{student}\PY{o}{.}\PY{n}{title}\PY{p}{(}\PY{p}{)}\PY{p}{)}

\PY{n+nb}{print}\PY{p}{(}\PY{l+s+s2}{\PYZdq{}}\PY{l+s+se}{\PYZbs{}n}\PY{l+s+s2}{Here is the list in its original order:}\PY{l+s+s2}{\PYZdq{}}\PY{p}{)}
\PY{c+c1}{\PYZsh{} Show that the list is still in its original order.}
\PY{k}{for} \PY{n}{student} \PY{o+ow}{in} \PY{n}{students}\PY{p}{:}
    \PY{n+nb}{print}\PY{p}{(}\PY{n}{student}\PY{o}{.}\PY{n}{title}\PY{p}{(}\PY{p}{)}\PY{p}{)}
\end{Verbatim}
\end{tcolorbox}

    \hypertarget{reversing-a-list}{%
\subsubsection{Reversing a list}\label{reversing-a-list}}

We have seen three possible orders for a list: - The original order in
which the list was created - Alphabetical order - Reverse alphabetical
order

There is one more order we can use, and that is the reverse of the
original order of the list. The \emph{reverse()} function gives us this
order.

    \begin{tcolorbox}[breakable, size=fbox, boxrule=1pt, pad at break*=1mm,colback=cellbackground, colframe=cellborder]
\prompt{In}{incolor}{ }{\boxspacing}
\begin{Verbatim}[commandchars=\\\{\}]
\PY{n}{students} \PY{o}{=} \PY{p}{[}\PY{l+s+s1}{\PYZsq{}}\PY{l+s+s1}{bernice}\PY{l+s+s1}{\PYZsq{}}\PY{p}{,} \PY{l+s+s1}{\PYZsq{}}\PY{l+s+s1}{aaron}\PY{l+s+s1}{\PYZsq{}}\PY{p}{,} \PY{l+s+s1}{\PYZsq{}}\PY{l+s+s1}{cody}\PY{l+s+s1}{\PYZsq{}}\PY{p}{]}
\PY{n}{students}\PY{o}{.}\PY{n}{reverse}\PY{p}{(}\PY{p}{)}

\PY{n+nb}{print}\PY{p}{(}\PY{n}{students}\PY{p}{)}
\end{Verbatim}
\end{tcolorbox}

    Note that reverse is permanent, although you could follow up with
another call to \emph{reverse()} and get back the original order of the
list.

    \hypertarget{sorting-a-numerical-list}{%
\subsubsection{Sorting a numerical
list}\label{sorting-a-numerical-list}}

All of the sorting functions work for numerical lists as well.

    \begin{tcolorbox}[breakable, size=fbox, boxrule=1pt, pad at break*=1mm,colback=cellbackground, colframe=cellborder]
\prompt{In}{incolor}{66}{\boxspacing}
\begin{Verbatim}[commandchars=\\\{\}]
\PY{n}{numbers} \PY{o}{=} \PY{p}{[}\PY{l+m+mi}{1}\PY{p}{,} \PY{l+m+mi}{3}\PY{p}{,} \PY{l+m+mi}{4}\PY{p}{,} \PY{l+m+mi}{2}\PY{p}{]}

\PY{c+c1}{\PYZsh{} sort() puts numbers in increasing order.}
\PY{n}{numbers}\PY{o}{.}\PY{n}{sort}\PY{p}{(}\PY{p}{)}
\PY{n+nb}{print}\PY{p}{(}\PY{n}{numbers}\PY{p}{)}

\PY{c+c1}{\PYZsh{} sort(reverse=True) puts numbers in decreasing order.}
\PY{n}{numbers}\PY{o}{.}\PY{n}{sort}\PY{p}{(}\PY{n}{reverse}\PY{o}{=}\PY{k+kc}{True}\PY{p}{)}
\PY{n+nb}{print}\PY{p}{(}\PY{n}{numbers}\PY{p}{)}
\end{Verbatim}
\end{tcolorbox}

    \begin{Verbatim}[commandchars=\\\{\}]
[1, 2, 3, 4]
[4, 3, 2, 1]
    \end{Verbatim}

    \begin{tcolorbox}[breakable, size=fbox, boxrule=1pt, pad at break*=1mm,colback=cellbackground, colframe=cellborder]
\prompt{In}{incolor}{89}{\boxspacing}
\begin{Verbatim}[commandchars=\\\{\}]
\PY{n}{numbers} \PY{o}{=} \PY{p}{[}\PY{l+m+mi}{1}\PY{p}{,} \PY{l+m+mi}{3}\PY{p}{,} \PY{l+m+mi}{4}\PY{p}{,} \PY{l+m+mi}{2}\PY{p}{]}

\PY{c+c1}{\PYZsh{} sorted() preserves the original order of the list:}
\PY{n+nb}{print}\PY{p}{(}\PY{n+nb}{sorted}\PY{p}{(}\PY{n}{numbers}\PY{p}{)}\PY{p}{)}
\PY{n+nb}{print}\PY{p}{(}\PY{n}{numbers}\PY{p}{)}
\end{Verbatim}
\end{tcolorbox}

    \begin{Verbatim}[commandchars=\\\{\}]
[1, 2, 3, 4]
[1, 3, 4, 2]
    \end{Verbatim}

    \begin{tcolorbox}[breakable, size=fbox, boxrule=1pt, pad at break*=1mm,colback=cellbackground, colframe=cellborder]
\prompt{In}{incolor}{ }{\boxspacing}
\begin{Verbatim}[commandchars=\\\{\}]
\PY{n}{numbers} \PY{o}{=} \PY{p}{[}\PY{l+m+mi}{1}\PY{p}{,} \PY{l+m+mi}{3}\PY{p}{,} \PY{l+m+mi}{4}\PY{p}{,} \PY{l+m+mi}{2}\PY{p}{]}

\PY{c+c1}{\PYZsh{} The reverse() function also works for numerical lists.}
\PY{n}{numbers}\PY{o}{.}\PY{n}{reverse}\PY{p}{(}\PY{p}{)}
\PY{n+nb}{print}\PY{p}{(}\PY{n}{numbers}\PY{p}{)}
\end{Verbatim}
\end{tcolorbox}

    \hypertarget{finding-the-length-of-a-list}{%
\subsection{Finding the length of a
list}\label{finding-the-length-of-a-list}}

You can find the length of a list using the \emph{len()} function.

    \begin{tcolorbox}[breakable, size=fbox, boxrule=1pt, pad at break*=1mm,colback=cellbackground, colframe=cellborder]
\prompt{In}{incolor}{ }{\boxspacing}
\begin{Verbatim}[commandchars=\\\{\}]
\PY{n}{usernames} \PY{o}{=} \PY{p}{[}\PY{l+s+s1}{\PYZsq{}}\PY{l+s+s1}{bernice}\PY{l+s+s1}{\PYZsq{}}\PY{p}{,} \PY{l+s+s1}{\PYZsq{}}\PY{l+s+s1}{cody}\PY{l+s+s1}{\PYZsq{}}\PY{p}{,} \PY{l+s+s1}{\PYZsq{}}\PY{l+s+s1}{aaron}\PY{l+s+s1}{\PYZsq{}}\PY{p}{]}
\PY{n}{user\PYZus{}count} \PY{o}{=} \PY{n+nb}{len}\PY{p}{(}\PY{n}{usernames}\PY{p}{)}

\PY{n+nb}{print}\PY{p}{(}\PY{n}{user\PYZus{}count}\PY{p}{)}
\end{Verbatim}
\end{tcolorbox}

    There are many situations where you might want to know how many items in
a list. If you have a list that stores your users, you can find the
length of your list at any time, and know how many users you have.

    \begin{tcolorbox}[breakable, size=fbox, boxrule=1pt, pad at break*=1mm,colback=cellbackground, colframe=cellborder]
\prompt{In}{incolor}{ }{\boxspacing}
\begin{Verbatim}[commandchars=\\\{\}]
\PY{c+c1}{\PYZsh{} Create an empty list to hold our users.}
\PY{n}{usernames} \PY{o}{=} \PY{p}{[}\PY{p}{]}

\PY{c+c1}{\PYZsh{} Add some users, and report on how many users we have.}
\PY{n}{usernames}\PY{o}{.}\PY{n}{append}\PY{p}{(}\PY{l+s+s1}{\PYZsq{}}\PY{l+s+s1}{bernice}\PY{l+s+s1}{\PYZsq{}}\PY{p}{)}
\PY{n}{user\PYZus{}count} \PY{o}{=} \PY{n+nb}{len}\PY{p}{(}\PY{n}{usernames}\PY{p}{)}

\PY{n+nb}{print}\PY{p}{(}\PY{l+s+s2}{\PYZdq{}}\PY{l+s+s2}{We have }\PY{l+s+s2}{\PYZdq{}} \PY{o}{+} \PY{n+nb}{str}\PY{p}{(}\PY{n}{user\PYZus{}count}\PY{p}{)} \PY{o}{+} \PY{l+s+s2}{\PYZdq{}}\PY{l+s+s2}{ user!}\PY{l+s+s2}{\PYZdq{}}\PY{p}{)}

\PY{n}{usernames}\PY{o}{.}\PY{n}{append}\PY{p}{(}\PY{l+s+s1}{\PYZsq{}}\PY{l+s+s1}{cody}\PY{l+s+s1}{\PYZsq{}}\PY{p}{)}
\PY{n}{usernames}\PY{o}{.}\PY{n}{append}\PY{p}{(}\PY{l+s+s1}{\PYZsq{}}\PY{l+s+s1}{aaron}\PY{l+s+s1}{\PYZsq{}}\PY{p}{)}
\PY{n}{user\PYZus{}count} \PY{o}{=} \PY{n+nb}{len}\PY{p}{(}\PY{n}{usernames}\PY{p}{)}

\PY{n+nb}{print}\PY{p}{(}\PY{l+s+s2}{\PYZdq{}}\PY{l+s+s2}{We have }\PY{l+s+s2}{\PYZdq{}} \PY{o}{+} \PY{n+nb}{str}\PY{p}{(}\PY{n}{user\PYZus{}count}\PY{p}{)} \PY{o}{+} \PY{l+s+s2}{\PYZdq{}}\PY{l+s+s2}{ users!}\PY{l+s+s2}{\PYZdq{}}\PY{p}{)}
\end{Verbatim}
\end{tcolorbox}

    On a technical note, the \emph{len()} function returns an integer, which
can't be printed directly with strings. We use the \emph{str()} function
to turn the integer into a string so that it prints nicely:

    \begin{tcolorbox}[breakable, size=fbox, boxrule=1pt, pad at break*=1mm,colback=cellbackground, colframe=cellborder]
\prompt{In}{incolor}{67}{\boxspacing}
\begin{Verbatim}[commandchars=\\\{\}]
\PY{n}{usernames} \PY{o}{=} \PY{p}{[}\PY{l+s+s1}{\PYZsq{}}\PY{l+s+s1}{bernice}\PY{l+s+s1}{\PYZsq{}}\PY{p}{,} \PY{l+s+s1}{\PYZsq{}}\PY{l+s+s1}{cody}\PY{l+s+s1}{\PYZsq{}}\PY{p}{,} \PY{l+s+s1}{\PYZsq{}}\PY{l+s+s1}{aaron}\PY{l+s+s1}{\PYZsq{}}\PY{p}{]}
\PY{n}{user\PYZus{}count} \PY{o}{=} \PY{n+nb}{len}\PY{p}{(}\PY{n}{usernames}\PY{p}{)}

\PY{n+nb}{print}\PY{p}{(}\PY{l+s+s2}{\PYZdq{}}\PY{l+s+s2}{This will cause an error: }\PY{l+s+s2}{\PYZdq{}} \PY{o}{+} \PY{n}{user\PYZus{}count}\PY{p}{)}
\end{Verbatim}
\end{tcolorbox}

    \begin{Verbatim}[commandchars=\\\{\}]

        ---------------------------------------------------------------------------

        TypeError                                 Traceback (most recent call last)

        /var/folders/1\_/2rbdt5292zdb57b821k90d5r0000gn/T/ipykernel\_56860/3778417429.py in <module>
          2 user\_count = len(usernames)
          3 
    ----> 4 print("This will cause an error: " + user\_count)
    

        TypeError: can only concatenate str (not "int") to str

    \end{Verbatim}

    \begin{tcolorbox}[breakable, size=fbox, boxrule=1pt, pad at break*=1mm,colback=cellbackground, colframe=cellborder]
\prompt{In}{incolor}{68}{\boxspacing}
\begin{Verbatim}[commandchars=\\\{\}]
\PY{c+c1}{\PYZsh{}\PYZsh{}\PYZsh{}highlight=[5]}
\PY{n}{usernames} \PY{o}{=} \PY{p}{[}\PY{l+s+s1}{\PYZsq{}}\PY{l+s+s1}{bernice}\PY{l+s+s1}{\PYZsq{}}\PY{p}{,} \PY{l+s+s1}{\PYZsq{}}\PY{l+s+s1}{cody}\PY{l+s+s1}{\PYZsq{}}\PY{p}{,} \PY{l+s+s1}{\PYZsq{}}\PY{l+s+s1}{aaron}\PY{l+s+s1}{\PYZsq{}}\PY{p}{]}
\PY{n}{user\PYZus{}count} \PY{o}{=} \PY{n+nb}{len}\PY{p}{(}\PY{n}{usernames}\PY{p}{)}

\PY{n+nb}{print}\PY{p}{(}\PY{l+s+s2}{\PYZdq{}}\PY{l+s+s2}{This will work: }\PY{l+s+s2}{\PYZdq{}} \PY{o}{+} \PY{n+nb}{str}\PY{p}{(}\PY{n}{user\PYZus{}count}\PY{p}{)}\PY{p}{)}
\end{Verbatim}
\end{tcolorbox}

    \begin{Verbatim}[commandchars=\\\{\}]
This will work: 3
    \end{Verbatim}

    \hypertarget{working-list}{%
\paragraph{Working List}\label{working-list}}

\begin{itemize}
\tightlist
\item
  Make a list that includes four careers, such as `programmer' and
  `truck driver'.
\item
  Use the \emph{list.index()} function to find the index of one career
  in your list.
\item
  Use the \emph{in} function to show that this career is in your list.
\item
  Use the \emph{append()} function to add a new career to your list.
\item
  Use the \emph{insert()} function to add a new career at the beginning
  of the list.
\item
  Use a loop to show all the careers in your list.
\end{itemize}

    \begin{tcolorbox}[breakable, size=fbox, boxrule=1pt, pad at break*=1mm,colback=cellbackground, colframe=cellborder]
\prompt{In}{incolor}{75}{\boxspacing}
\begin{Verbatim}[commandchars=\\\{\}]
\PY{n}{careers} \PY{o}{=} \PY{p}{[}\PY{l+s+s1}{\PYZsq{}}\PY{l+s+s1}{programmer}\PY{l+s+s1}{\PYZsq{}}\PY{p}{,} \PY{l+s+s1}{\PYZsq{}}\PY{l+s+s1}{teacher}\PY{l+s+s1}{\PYZsq{}}\PY{p}{,} \PY{l+s+s1}{\PYZsq{}}\PY{l+s+s1}{doctor}\PY{l+s+s1}{\PYZsq{}}\PY{p}{,} \PY{l+s+s1}{\PYZsq{}}\PY{l+s+s1}{cleaner}\PY{l+s+s1}{\PYZsq{}}\PY{p}{]}

\PY{n+nb}{print}\PY{p}{(}\PY{n}{careers}\PY{o}{.}\PY{n}{index}\PY{p}{(}\PY{l+s+s1}{\PYZsq{}}\PY{l+s+s1}{cleaner}\PY{l+s+s1}{\PYZsq{}}\PY{p}{)}\PY{p}{)}

\PY{n+nb}{print}\PY{p}{(}\PY{l+s+s1}{\PYZsq{}}\PY{l+s+s1}{cleaner}\PY{l+s+s1}{\PYZsq{}} \PY{o+ow}{in} \PY{n}{careers}\PY{p}{)}
\PY{n+nb}{print}\PY{p}{(}\PY{l+s+s1}{\PYZsq{}}\PY{l+s+s1}{sweeper}\PY{l+s+s1}{\PYZsq{}} \PY{o+ow}{in} \PY{n}{careers}\PY{p}{)}

\PY{n}{careers}\PY{o}{.}\PY{n}{append}\PY{p}{(}\PY{l+s+s1}{\PYZsq{}}\PY{l+s+s1}{ceo}\PY{l+s+s1}{\PYZsq{}}\PY{p}{)}
\PY{n}{careers}

\PY{n}{careers}\PY{o}{.}\PY{n}{insert}\PY{p}{(}\PY{l+m+mi}{0}\PY{p}{,} \PY{l+s+s1}{\PYZsq{}}\PY{l+s+s1}{developer}\PY{l+s+s1}{\PYZsq{}}\PY{p}{)}
\PY{n}{careers}

\PY{k}{for} \PY{n}{career} \PY{o+ow}{in} \PY{n}{careers}\PY{p}{:}
    \PY{n+nb}{print}\PY{p}{(}\PY{n}{career}\PY{p}{)}
\end{Verbatim}
\end{tcolorbox}

    \begin{Verbatim}[commandchars=\\\{\}]
3
True
False
developer
programmer
teacher
doctor
cleaner
ceo
    \end{Verbatim}

    \hypertarget{starting-from-empty}{%
\paragraph{Starting From Empty}\label{starting-from-empty}}

\begin{itemize}
\tightlist
\item
  Create the list you ended up with in \emph{Working List}, but this
  time start your file with an empty list and fill it up using
  \emph{append()} statements.
\item
  Print a statement that tells us what the first career you thought of
  was.
\item
  Print a statement that tells us what the last career you thought of
  was.
\end{itemize}

    \begin{tcolorbox}[breakable, size=fbox, boxrule=1pt, pad at break*=1mm,colback=cellbackground, colframe=cellborder]
\prompt{In}{incolor}{79}{\boxspacing}
\begin{Verbatim}[commandchars=\\\{\}]
\PY{n}{careers} \PY{o}{=} \PY{p}{[}\PY{l+s+s1}{\PYZsq{}}\PY{l+s+s1}{programmer}\PY{l+s+s1}{\PYZsq{}}\PY{p}{,} \PY{l+s+s1}{\PYZsq{}}\PY{l+s+s1}{teacher}\PY{l+s+s1}{\PYZsq{}}\PY{p}{,} \PY{l+s+s1}{\PYZsq{}}\PY{l+s+s1}{doctor}\PY{l+s+s1}{\PYZsq{}}\PY{p}{,} \PY{l+s+s1}{\PYZsq{}}\PY{l+s+s1}{cleaner}\PY{l+s+s1}{\PYZsq{}}\PY{p}{]}
\PY{n+nb}{print}\PY{p}{(}\PY{l+s+sa}{f}\PY{l+s+s1}{\PYZsq{}}\PY{l+s+s1}{Last career I found was }\PY{l+s+si}{\PYZob{}}\PY{n}{careers}\PY{p}{[}\PY{o}{\PYZhy{}}\PY{l+m+mi}{1}\PY{p}{]}\PY{l+s+si}{\PYZcb{}}\PY{l+s+s1}{\PYZsq{}}\PY{p}{)}
\PY{n+nb}{print}\PY{p}{(}\PY{l+s+sa}{f}\PY{l+s+s1}{\PYZsq{}}\PY{l+s+s1}{First career I found was }\PY{l+s+si}{\PYZob{}}\PY{n}{careers}\PY{p}{[}\PY{l+m+mi}{0}\PY{p}{]}\PY{l+s+si}{\PYZcb{}}\PY{l+s+s1}{\PYZsq{}}\PY{p}{)}
\end{Verbatim}
\end{tcolorbox}

    \begin{Verbatim}[commandchars=\\\{\}]
Last career I found was cleaner
First career I found was programmer
    \end{Verbatim}

    \hypertarget{ordered-working-list}{%
\paragraph{Ordered Working List}\label{ordered-working-list}}

\begin{itemize}
\tightlist
\item
  Start with the list you created in \emph{Working List}.
\item
  You are going to print out the list in a number of different orders.
\item
  Each time you print the list, use a for loop rather than printing the
  raw list.
\item
  Print a message each time telling us what order we should see the list
  in.

  \begin{itemize}
  \tightlist
  \item
    Print the list in its original order.
  \item
    Print the list in alphabetical order.
  \item
    Print the list in its original order.
  \item
    Print the list in reverse alphabetical order.
  \item
    Print the list in its original order.
  \item
    Print the list in the reverse order from what it started.
  \item
    Print the list in its original order
  \item
    Permanently sort the list in alphabetical order, and then print it
    out.
  \item
    Permanently sort the list in reverse alphabetical order, and then
    print it out.
  \end{itemize}
\end{itemize}

    \begin{tcolorbox}[breakable, size=fbox, boxrule=1pt, pad at break*=1mm,colback=cellbackground, colframe=cellborder]
\prompt{In}{incolor}{92}{\boxspacing}
\begin{Verbatim}[commandchars=\\\{\}]
\PY{n}{careers} \PY{o}{=} \PY{p}{[}\PY{l+s+s1}{\PYZsq{}}\PY{l+s+s1}{programmer}\PY{l+s+s1}{\PYZsq{}}\PY{p}{,} \PY{l+s+s1}{\PYZsq{}}\PY{l+s+s1}{teacher}\PY{l+s+s1}{\PYZsq{}}\PY{p}{,} \PY{l+s+s1}{\PYZsq{}}\PY{l+s+s1}{doctor}\PY{l+s+s1}{\PYZsq{}}\PY{p}{,} \PY{l+s+s1}{\PYZsq{}}\PY{l+s+s1}{cleaner}\PY{l+s+s1}{\PYZsq{}}\PY{p}{]}
\PY{n+nb}{print}\PY{p}{(}\PY{n}{careers}\PY{p}{)}

\PY{n+nb}{print}\PY{p}{(}\PY{n+nb}{sorted}\PY{p}{(}\PY{n}{careers}\PY{p}{)}\PY{p}{)} 
\PY{n+nb}{print}\PY{p}{(}\PY{n}{careers}\PY{p}{)}

\PY{n+nb}{print}\PY{p}{(}\PY{n+nb}{sorted}\PY{p}{(}\PY{n}{careers}\PY{p}{,} \PY{n}{reverse}\PY{o}{=}\PY{k+kc}{True}\PY{p}{)}\PY{p}{)}
\PY{n+nb}{print}\PY{p}{(}\PY{n}{careers}\PY{p}{)}

\PY{n}{careers}\PY{o}{.}\PY{n}{sort}\PY{p}{(}\PY{p}{)}
\PY{n+nb}{print}\PY{p}{(}\PY{n}{careers}\PY{p}{)}

\PY{n}{careers}\PY{o}{.}\PY{n}{reverse}\PY{p}{(}\PY{p}{)}
\PY{n+nb}{print}\PY{p}{(}\PY{n}{careers}\PY{p}{)}
\end{Verbatim}
\end{tcolorbox}

    \begin{Verbatim}[commandchars=\\\{\}]
['programmer', 'teacher', 'doctor', 'cleaner']
['cleaner', 'doctor', 'programmer', 'teacher']
['programmer', 'teacher', 'doctor', 'cleaner']
['teacher', 'programmer', 'doctor', 'cleaner']
['programmer', 'teacher', 'doctor', 'cleaner']
['cleaner', 'doctor', 'programmer', 'teacher']
['teacher', 'programmer', 'doctor', 'cleaner']
    \end{Verbatim}

    \hypertarget{ordered-numbers}{%
\paragraph{Ordered Numbers}\label{ordered-numbers}}

\begin{itemize}
\tightlist
\item
  Make a list of 5 numbers, in a random order.
\item
  You are going to print out the list in a number of different orders.
\item
  Each time you print the list, use a for loop rather than printing the
  raw list.
\item
  Print a message each time telling us what order we should see the list
  in.

  \begin{itemize}
  \tightlist
  \item
    Print the numbers in the original order.
  \item
    Print the numbers in increasing order.
  \item
    Print the numbers in the original order.
  \item
    Print the numbers in decreasing order.
  \item
    Print the numbers in their original order.
  \item
    Print the numbers in the reverse order from how they started.
  \item
    Print the numbers in the original order.
  \item
    Permanently sort the numbers in increasing order, and then print
    them out.
  \item
    Permanently sort the numbers in descreasing order, and then print
    them out.
  \end{itemize}
\end{itemize}

    \begin{tcolorbox}[breakable, size=fbox, boxrule=1pt, pad at break*=1mm,colback=cellbackground, colframe=cellborder]
\prompt{In}{incolor}{101}{\boxspacing}
\begin{Verbatim}[commandchars=\\\{\}]
\PY{k}{def} \PY{n+nf}{print\PYZus{}list}\PY{p}{(}\PY{n}{items}\PY{p}{)}\PY{p}{:}
    \PY{k}{for} \PY{n}{item} \PY{o+ow}{in} \PY{n}{items}\PY{p}{:}
        \PY{n+nb}{print}\PY{p}{(}\PY{n+nb}{str}\PY{p}{(}\PY{n}{item}\PY{p}{)}\PY{p}{)}
        
\PY{n}{nums} \PY{o}{=} \PY{p}{[}\PY{l+m+mi}{9}\PY{p}{,}\PY{l+m+mi}{2}\PY{p}{,}\PY{l+m+mi}{4}\PY{p}{,}\PY{l+m+mi}{10}\PY{p}{,}\PY{l+m+mi}{6}\PY{p}{]}

\PY{n+nb}{print}\PY{p}{(}\PY{l+s+s2}{\PYZdq{}}\PY{l+s+s2}{Original Order:}\PY{l+s+s2}{\PYZdq{}}\PY{p}{)}
\PY{n}{print\PYZus{}list}\PY{p}{(}\PY{n}{nums}\PY{p}{)}

\PY{n+nb}{print}\PY{p}{(}\PY{l+s+s2}{\PYZdq{}}\PY{l+s+s2}{Increasing Order:}\PY{l+s+s2}{\PYZdq{}}\PY{p}{)}
\PY{n}{print\PYZus{}list}\PY{p}{(}\PY{n+nb}{sorted}\PY{p}{(}\PY{n}{nums}\PY{p}{)}\PY{p}{)}
\PY{n+nb}{print}\PY{p}{(}\PY{l+s+s2}{\PYZdq{}}\PY{l+s+s2}{Original Order:}\PY{l+s+s2}{\PYZdq{}}\PY{p}{)}
\PY{n}{print\PYZus{}list}\PY{p}{(}\PY{n}{nums}\PY{p}{)}

\PY{n+nb}{print}\PY{p}{(}\PY{l+s+s2}{\PYZdq{}}\PY{l+s+s2}{Decreasing Order:}\PY{l+s+s2}{\PYZdq{}}\PY{p}{)}
\PY{n}{print\PYZus{}list}\PY{p}{(}\PY{n+nb}{sorted}\PY{p}{(}\PY{n}{nums}\PY{p}{,} \PY{n}{reverse}\PY{o}{=}\PY{k+kc}{True}\PY{p}{)}\PY{p}{)}
\PY{n+nb}{print}\PY{p}{(}\PY{l+s+s2}{\PYZdq{}}\PY{l+s+s2}{Original Order:}\PY{l+s+s2}{\PYZdq{}}\PY{p}{)}
\PY{n}{print\PYZus{}list}\PY{p}{(}\PY{n}{nums}\PY{p}{)}

\PY{n+nb}{print}\PY{p}{(}\PY{n+nb}{type}\PY{p}{(}\PY{n}{nums}\PY{p}{)}\PY{p}{)}
\PY{n+nb}{print}\PY{p}{(}\PY{l+s+s2}{\PYZdq{}}\PY{l+s+s2}{Increasing Order:}\PY{l+s+s2}{\PYZdq{}}\PY{p}{)}
\PY{n+nb}{print}\PY{p}{(}\PY{n}{nums}\PY{o}{.}\PY{n}{sort}\PY{p}{(}\PY{p}{)}\PY{p}{)} 

\PY{n+nb}{print}\PY{p}{(}\PY{l+s+s2}{\PYZdq{}}\PY{l+s+s2}{Decreasing Order:}\PY{l+s+s2}{\PYZdq{}}\PY{p}{)}
\PY{n+nb}{print}\PY{p}{(}\PY{n}{nums}\PY{o}{.}\PY{n}{sort}\PY{p}{(}\PY{n}{reverse}\PY{o}{=}\PY{k+kc}{True}\PY{p}{)}\PY{p}{)}
\end{Verbatim}
\end{tcolorbox}

    \begin{Verbatim}[commandchars=\\\{\}]
Original Order:
9
2
4
10
6
Increasing Order:
2
4
6
9
10
Original Order:
9
2
4
10
6
Decreasing Order:
10
9
6
4
2
Original Order:
9
2
4
10
6
<class 'list'>
Increasing Order:
None
Decreasing Order:
None
    \end{Verbatim}

    Exercises --- \#\#\#\# Working List - Make a list that includes four
careers, such as `programmer' and `truck driver'. - Use the
\emph{list.index()} function to find the index of one career in your
list. - Use the \emph{in} function to show that this career is in your
list. - Use the \emph{append()} function to add a new career to your
list. - Use the \emph{insert()} function to add a new career at the
beginning of the list. - Use a loop to show all the careers in your
list.

\hypertarget{starting-from-empty}{%
\paragraph{Starting From Empty}\label{starting-from-empty}}

\begin{itemize}
\tightlist
\item
  Create the list you ended up with in \emph{Working List}, but this
  time start your file with an empty list and fill it up using
  \emph{append()} statements.
\item
  Print a statement that tells us what the first career you thought of
  was.
\item
  Print a statement that tells us what the last career you thought of
  was.
\end{itemize}

\hypertarget{ordered-working-list}{%
\paragraph{Ordered Working List}\label{ordered-working-list}}

\begin{itemize}
\tightlist
\item
  Start with the list you created in \emph{Working List}.
\item
  You are going to print out the list in a number of different orders.
\item
  Each time you print the list, use a for loop rather than printing the
  raw list.
\item
  Print a message each time telling us what order we should see the list
  in.

  \begin{itemize}
  \tightlist
  \item
    Print the list in its original order.
  \item
    Print the list in alphabetical order.
  \item
    Print the list in its original order.
  \item
    Print the list in reverse alphabetical order.
  \item
    Print the list in its original order.
  \item
    Print the list in the reverse order from what it started.
  \item
    Print the list in its original order
  \item
    Permanently sort the list in alphabetical order, and then print it
    out.
  \item
    Permanently sort the list in reverse alphabetical order, and then
    print it out.
  \end{itemize}
\end{itemize}

\hypertarget{ordered-numbers}{%
\paragraph{Ordered Numbers}\label{ordered-numbers}}

\begin{itemize}
\tightlist
\item
  Make a list of 5 numbers, in a random order.
\item
  You are going to print out the list in a number of different orders.
\item
  Each time you print the list, use a for loop rather than printing the
  raw list.
\item
  Print a message each time telling us what order we should see the list
  in.

  \begin{itemize}
  \tightlist
  \item
    Print the numbers in the original order.
  \item
    Print the numbers in increasing order.
  \item
    Print the numbers in the original order.
  \item
    Print the numbers in decreasing order.
  \item
    Print the numbers in their original order.
  \item
    Print the numbers in the reverse order from how they started.
  \item
    Print the numbers in the original order.
  \item
    Permanently sort the numbers in increasing order, and then print
    them out.
  \item
    Permanently sort the numbers in descreasing order, and then print
    them out.
  \end{itemize}
\end{itemize}

\hypertarget{list-lengths}{%
\paragraph{List Lengths}\label{list-lengths}}

\begin{itemize}
\tightlist
\item
  Copy two or three of the lists you made from the previous exercises,
  or make up two or three new lists.
\item
  Print out a series of statements that tell us how long each list is.
\end{itemize}

    \protect\hyperlink{}{top}

    \hypertarget{removing-items-from-a-list}{%
\section{Removing Items from a List}\label{removing-items-from-a-list}}

Hopefully you can see by now that lists are a dynamic structure. We can
define an empty list and then fill it up as information comes into our
program. To become really dynamic, we need some ways to remove items
from a list when we no longer need them. You can remove items from a
list through their position, or through their value.

    \hypertarget{removing-items-by-position}{%
\subsection{Removing items by
position}\label{removing-items-by-position}}

If you know the position of an item in a list, you can remove that item
using the \emph{del} command. To use this approach, give the command
\emph{del} and the name of your list, with the index of the item you
want to move in square brackets:

    \begin{tcolorbox}[breakable, size=fbox, boxrule=1pt, pad at break*=1mm,colback=cellbackground, colframe=cellborder]
\prompt{In}{incolor}{ }{\boxspacing}
\begin{Verbatim}[commandchars=\\\{\}]
\PY{n}{dogs} \PY{o}{=} \PY{p}{[}\PY{l+s+s1}{\PYZsq{}}\PY{l+s+s1}{border collie}\PY{l+s+s1}{\PYZsq{}}\PY{p}{,} \PY{l+s+s1}{\PYZsq{}}\PY{l+s+s1}{australian cattle dog}\PY{l+s+s1}{\PYZsq{}}\PY{p}{,} \PY{l+s+s1}{\PYZsq{}}\PY{l+s+s1}{labrador retriever}\PY{l+s+s1}{\PYZsq{}}\PY{p}{]}
\PY{c+c1}{\PYZsh{} Remove the first dog from the list.}
\PY{k}{del} \PY{n}{dogs}\PY{p}{[}\PY{l+m+mi}{0}\PY{p}{]}

\PY{n+nb}{print}\PY{p}{(}\PY{n}{dogs}\PY{p}{)}
\end{Verbatim}
\end{tcolorbox}

    \hypertarget{removing-items-by-value}{%
\subsection{Removing items by value}\label{removing-items-by-value}}

You can also remove an item from a list if you know its value. To do
this, we use the \emph{remove()} function. Give the name of the list,
followed by the word remove with the value of the item you want to
remove in parentheses. Python looks through your list, finds the first
item with this value, and removes it.

    \begin{tcolorbox}[breakable, size=fbox, boxrule=1pt, pad at break*=1mm,colback=cellbackground, colframe=cellborder]
\prompt{In}{incolor}{ }{\boxspacing}
\begin{Verbatim}[commandchars=\\\{\}]
\PY{n}{dogs} \PY{o}{=} \PY{p}{[}\PY{l+s+s1}{\PYZsq{}}\PY{l+s+s1}{border collie}\PY{l+s+s1}{\PYZsq{}}\PY{p}{,} \PY{l+s+s1}{\PYZsq{}}\PY{l+s+s1}{australian cattle dog}\PY{l+s+s1}{\PYZsq{}}\PY{p}{,} \PY{l+s+s1}{\PYZsq{}}\PY{l+s+s1}{labrador retriever}\PY{l+s+s1}{\PYZsq{}}\PY{p}{]}
\PY{c+c1}{\PYZsh{} Remove australian cattle dog from the list.}
\PY{n}{dogs}\PY{o}{.}\PY{n}{remove}\PY{p}{(}\PY{l+s+s1}{\PYZsq{}}\PY{l+s+s1}{australian cattle dog}\PY{l+s+s1}{\PYZsq{}}\PY{p}{)}

\PY{n+nb}{print}\PY{p}{(}\PY{n}{dogs}\PY{p}{)}
\end{Verbatim}
\end{tcolorbox}

    Be careful to note, however, that \emph{only} the first item with this
value is removed. If you have multiple items with the same value, you
will have some items with this value left in your list.

    \begin{tcolorbox}[breakable, size=fbox, boxrule=1pt, pad at break*=1mm,colback=cellbackground, colframe=cellborder]
\prompt{In}{incolor}{ }{\boxspacing}
\begin{Verbatim}[commandchars=\\\{\}]
\PY{n}{letters} \PY{o}{=} \PY{p}{[}\PY{l+s+s1}{\PYZsq{}}\PY{l+s+s1}{a}\PY{l+s+s1}{\PYZsq{}}\PY{p}{,} \PY{l+s+s1}{\PYZsq{}}\PY{l+s+s1}{b}\PY{l+s+s1}{\PYZsq{}}\PY{p}{,} \PY{l+s+s1}{\PYZsq{}}\PY{l+s+s1}{c}\PY{l+s+s1}{\PYZsq{}}\PY{p}{,} \PY{l+s+s1}{\PYZsq{}}\PY{l+s+s1}{a}\PY{l+s+s1}{\PYZsq{}}\PY{p}{,} \PY{l+s+s1}{\PYZsq{}}\PY{l+s+s1}{b}\PY{l+s+s1}{\PYZsq{}}\PY{p}{,} \PY{l+s+s1}{\PYZsq{}}\PY{l+s+s1}{c}\PY{l+s+s1}{\PYZsq{}}\PY{p}{]}
\PY{c+c1}{\PYZsh{} Remove the letter a from the list.}
\PY{n}{letters}\PY{o}{.}\PY{n}{remove}\PY{p}{(}\PY{l+s+s1}{\PYZsq{}}\PY{l+s+s1}{a}\PY{l+s+s1}{\PYZsq{}}\PY{p}{)}

\PY{n+nb}{print}\PY{p}{(}\PY{n}{letters}\PY{p}{)}
\end{Verbatim}
\end{tcolorbox}

    \hypertarget{popping-items-from-a-list}{%
\subsection{Popping items from a list}\label{popping-items-from-a-list}}

There is a cool concept in programming called ``popping'' items from a
collection. Every programming language has some sort of data structure
similar to Python's lists. All of these structures can be used as
queues, and there are various ways of processing the items in a queue.

One simple approach is to start with an empty list, and then add items
to that list. When you want to work with the items in the list, you
always take the last item from the list, do something with it, and then
remove that item. The \emph{pop()} function makes this easy. It removes
the last item from the list, and gives it to us so we can work with it.
This is easier to show with an example:

    \begin{tcolorbox}[breakable, size=fbox, boxrule=1pt, pad at break*=1mm,colback=cellbackground, colframe=cellborder]
\prompt{In}{incolor}{8}{\boxspacing}
\begin{Verbatim}[commandchars=\\\{\}]
\PY{n}{dogs} \PY{o}{=} \PY{p}{[}\PY{l+s+s1}{\PYZsq{}}\PY{l+s+s1}{border collie}\PY{l+s+s1}{\PYZsq{}}\PY{p}{,} \PY{l+s+s1}{\PYZsq{}}\PY{l+s+s1}{australian cattle dog}\PY{l+s+s1}{\PYZsq{}}\PY{p}{,} \PY{l+s+s1}{\PYZsq{}}\PY{l+s+s1}{labrador retriever}\PY{l+s+s1}{\PYZsq{}}\PY{p}{]}
\PY{n}{last\PYZus{}dog} \PY{o}{=} \PY{n}{dogs}\PY{o}{.}\PY{n}{pop}\PY{p}{(}\PY{p}{)}

\PY{n+nb}{print}\PY{p}{(}\PY{n}{last\PYZus{}dog}\PY{p}{)}
\PY{n+nb}{print}\PY{p}{(}\PY{n}{dogs}\PY{p}{)}
\end{Verbatim}
\end{tcolorbox}

    \begin{Verbatim}[commandchars=\\\{\}]
labrador retriever
['border collie', 'australian cattle dog']
    \end{Verbatim}

    This is an example of a first-in, last-out approach. The first item in
the list would be the last item processed if you kept using this
approach. We will see a full implementation of this approach later on,
when we learn about \emph{while} loops.

You can actually pop any item you want from a list, by giving the index
of the item you want to pop. So we could do a first-in, first-out
approach by popping the first iem in the list:

    \begin{tcolorbox}[breakable, size=fbox, boxrule=1pt, pad at break*=1mm,colback=cellbackground, colframe=cellborder]
\prompt{In}{incolor}{9}{\boxspacing}
\begin{Verbatim}[commandchars=\\\{\}]
\PY{c+c1}{\PYZsh{}\PYZsh{}\PYZsh{}highlight=[3]}
\PY{n}{dogs} \PY{o}{=} \PY{p}{[}\PY{l+s+s1}{\PYZsq{}}\PY{l+s+s1}{border collie}\PY{l+s+s1}{\PYZsq{}}\PY{p}{,} \PY{l+s+s1}{\PYZsq{}}\PY{l+s+s1}{australian cattle dog}\PY{l+s+s1}{\PYZsq{}}\PY{p}{,} \PY{l+s+s1}{\PYZsq{}}\PY{l+s+s1}{labrador retriever}\PY{l+s+s1}{\PYZsq{}}\PY{p}{]}
\PY{n}{first\PYZus{}dog} \PY{o}{=} \PY{n}{dogs}\PY{o}{.}\PY{n}{pop}\PY{p}{(}\PY{l+m+mi}{0}\PY{p}{)}

\PY{n+nb}{print}\PY{p}{(}\PY{n}{first\PYZus{}dog}\PY{p}{)}
\PY{n+nb}{print}\PY{p}{(}\PY{n}{dogs}\PY{p}{)}
\end{Verbatim}
\end{tcolorbox}

    \begin{Verbatim}[commandchars=\\\{\}]
border collie
['australian cattle dog', 'labrador retriever']
    \end{Verbatim}

    Exercises --- \#\#\#\# Famous People - Make a list that includes the
names of four famous people. - Remove each person from the list, one at
a time, using each of the four methods we have just seen: - Pop the last
item from the list, and pop any item except the last item. - Remove one
item by its position, and one item by its value. - Print out a message
that there are no famous people left in your list, and print your list
to prove that it is empty.

    \protect\hyperlink{}{top}

    \hypertarget{want-to-see-what-functions-are}{%
\section{Want to see what functions
are?}\label{want-to-see-what-functions-are}}

At this point, you might have noticed we have a fair bit of repetetive
code in some of our examples. This repetition will disappear once we
learn how to use functions. If this repetition is bothering you already,
you might want to go look at
\href{http://nbviewer.ipython.org/urls/raw.github.com/ehmatthes/intro_programming/master/notebooks/introducing_functions.ipynb}{Introducing
Functions} before you do any more exercises in this section.

    \hypertarget{slicing-a-list}{%
\section{Slicing a List}\label{slicing-a-list}}

Since a list is a collection of items, we should be able to get any
subset of those items. For example, if we want to get just the first
three items from the list, we should be able to do so easily. The same
should be true for any three items in the middle of the list, or the
last three items, or any x items from anywhere in the list. These
subsets of a list are called \emph{slices}.

To get a subset of a list, we give the position of the first item we
want, and the position of the first item we do \emph{not} want to
include in the subset. So the slice \emph{list{[}0:3{]}} will return a
list containing items 0, 1, and 2, but not item 3. Here is how you get a
batch containing the first three items.

    \begin{tcolorbox}[breakable, size=fbox, boxrule=1pt, pad at break*=1mm,colback=cellbackground, colframe=cellborder]
\prompt{In}{incolor}{10}{\boxspacing}
\begin{Verbatim}[commandchars=\\\{\}]
\PY{n}{usernames} \PY{o}{=} \PY{p}{[}\PY{l+s+s1}{\PYZsq{}}\PY{l+s+s1}{bernice}\PY{l+s+s1}{\PYZsq{}}\PY{p}{,} \PY{l+s+s1}{\PYZsq{}}\PY{l+s+s1}{cody}\PY{l+s+s1}{\PYZsq{}}\PY{p}{,} \PY{l+s+s1}{\PYZsq{}}\PY{l+s+s1}{aaron}\PY{l+s+s1}{\PYZsq{}}\PY{p}{,} \PY{l+s+s1}{\PYZsq{}}\PY{l+s+s1}{ever}\PY{l+s+s1}{\PYZsq{}}\PY{p}{,} \PY{l+s+s1}{\PYZsq{}}\PY{l+s+s1}{dalia}\PY{l+s+s1}{\PYZsq{}}\PY{p}{]}

\PY{c+c1}{\PYZsh{} Grab the first three users in the list.}
\PY{n}{first\PYZus{}batch} \PY{o}{=} \PY{n}{usernames}\PY{p}{[}\PY{l+m+mi}{0}\PY{p}{:}\PY{l+m+mi}{3}\PY{p}{]}

\PY{k}{for} \PY{n}{user} \PY{o+ow}{in} \PY{n}{first\PYZus{}batch}\PY{p}{:}
    \PY{n+nb}{print}\PY{p}{(}\PY{n}{user}\PY{o}{.}\PY{n}{title}\PY{p}{(}\PY{p}{)}\PY{p}{)}
\end{Verbatim}
\end{tcolorbox}

    \begin{Verbatim}[commandchars=\\\{\}]
Bernice
Cody
Aaron
    \end{Verbatim}

    If you want to grab everything up to a certain position in the list, you
can also leave the first index blank:

    \begin{tcolorbox}[breakable, size=fbox, boxrule=1pt, pad at break*=1mm,colback=cellbackground, colframe=cellborder]
\prompt{In}{incolor}{107}{\boxspacing}
\begin{Verbatim}[commandchars=\\\{\}]
\PY{c+c1}{\PYZsh{}\PYZsh{}\PYZsh{}highlight=[5]}
\PY{n}{usernames} \PY{o}{=} \PY{p}{[}\PY{l+s+s1}{\PYZsq{}}\PY{l+s+s1}{bernice}\PY{l+s+s1}{\PYZsq{}}\PY{p}{,} \PY{l+s+s1}{\PYZsq{}}\PY{l+s+s1}{cody}\PY{l+s+s1}{\PYZsq{}}\PY{p}{,} \PY{l+s+s1}{\PYZsq{}}\PY{l+s+s1}{aaron}\PY{l+s+s1}{\PYZsq{}}\PY{p}{,} \PY{l+s+s1}{\PYZsq{}}\PY{l+s+s1}{ever}\PY{l+s+s1}{\PYZsq{}}\PY{p}{,} \PY{l+s+s1}{\PYZsq{}}\PY{l+s+s1}{dalia}\PY{l+s+s1}{\PYZsq{}}\PY{p}{]}

\PY{c+c1}{\PYZsh{} Grab the first three users in the list.}
\PY{n}{first\PYZus{}batch} \PY{o}{=} \PY{n}{usernames}\PY{p}{[}\PY{l+m+mi}{0}\PY{p}{:}\PY{l+m+mi}{3}\PY{p}{]}

\PY{k}{for} \PY{n}{user} \PY{o+ow}{in} \PY{n}{usernames}\PY{p}{:}
    \PY{n+nb}{print}\PY{p}{(}\PY{n}{user}\PY{o}{.}\PY{n}{title}\PY{p}{(}\PY{p}{)}\PY{p}{)}
\end{Verbatim}
\end{tcolorbox}

    \begin{Verbatim}[commandchars=\\\{\}]
Bernice
Cody
Aaron
Ever
Dalia
    \end{Verbatim}

    When we grab a slice from a list, the original list is not affected:

    \begin{tcolorbox}[breakable, size=fbox, boxrule=1pt, pad at break*=1mm,colback=cellbackground, colframe=cellborder]
\prompt{In}{incolor}{12}{\boxspacing}
\begin{Verbatim}[commandchars=\\\{\}]
\PY{c+c1}{\PYZsh{}\PYZsh{}\PYZsh{}highlight=[7,8,9]}
\PY{n}{usernames} \PY{o}{=} \PY{p}{[}\PY{l+s+s1}{\PYZsq{}}\PY{l+s+s1}{bernice}\PY{l+s+s1}{\PYZsq{}}\PY{p}{,} \PY{l+s+s1}{\PYZsq{}}\PY{l+s+s1}{cody}\PY{l+s+s1}{\PYZsq{}}\PY{p}{,} \PY{l+s+s1}{\PYZsq{}}\PY{l+s+s1}{aaron}\PY{l+s+s1}{\PYZsq{}}\PY{p}{,} \PY{l+s+s1}{\PYZsq{}}\PY{l+s+s1}{ever}\PY{l+s+s1}{\PYZsq{}}\PY{p}{,} \PY{l+s+s1}{\PYZsq{}}\PY{l+s+s1}{dalia}\PY{l+s+s1}{\PYZsq{}}\PY{p}{]}

\PY{c+c1}{\PYZsh{} Grab the first three users in the list.}
\PY{n}{first\PYZus{}batch} \PY{o}{=} \PY{n}{usernames}\PY{p}{[}\PY{l+m+mi}{0}\PY{p}{:}\PY{l+m+mi}{3}\PY{p}{]}

\PY{c+c1}{\PYZsh{} The original list is unaffected.}
\PY{k}{for} \PY{n}{user} \PY{o+ow}{in} \PY{n}{usernames}\PY{p}{:}
    \PY{n+nb}{print}\PY{p}{(}\PY{n}{user}\PY{o}{.}\PY{n}{title}\PY{p}{(}\PY{p}{)}\PY{p}{)}
\end{Verbatim}
\end{tcolorbox}

    \begin{Verbatim}[commandchars=\\\{\}]
Bernice
Cody
Aaron
Ever
Dalia
    \end{Verbatim}

    We can get any segment of a list we want, using the slice method:

    \begin{tcolorbox}[breakable, size=fbox, boxrule=1pt, pad at break*=1mm,colback=cellbackground, colframe=cellborder]
\prompt{In}{incolor}{13}{\boxspacing}
\begin{Verbatim}[commandchars=\\\{\}]
\PY{n}{usernames} \PY{o}{=} \PY{p}{[}\PY{l+s+s1}{\PYZsq{}}\PY{l+s+s1}{bernice}\PY{l+s+s1}{\PYZsq{}}\PY{p}{,} \PY{l+s+s1}{\PYZsq{}}\PY{l+s+s1}{cody}\PY{l+s+s1}{\PYZsq{}}\PY{p}{,} \PY{l+s+s1}{\PYZsq{}}\PY{l+s+s1}{aaron}\PY{l+s+s1}{\PYZsq{}}\PY{p}{,} \PY{l+s+s1}{\PYZsq{}}\PY{l+s+s1}{ever}\PY{l+s+s1}{\PYZsq{}}\PY{p}{,} \PY{l+s+s1}{\PYZsq{}}\PY{l+s+s1}{dalia}\PY{l+s+s1}{\PYZsq{}}\PY{p}{]}

\PY{c+c1}{\PYZsh{} Grab a batch from the middle of the list.}
\PY{n}{middle\PYZus{}batch} \PY{o}{=} \PY{n}{usernames}\PY{p}{[}\PY{l+m+mi}{1}\PY{p}{:}\PY{l+m+mi}{4}\PY{p}{]}

\PY{k}{for} \PY{n}{user} \PY{o+ow}{in} \PY{n}{middle\PYZus{}batch}\PY{p}{:}
    \PY{n+nb}{print}\PY{p}{(}\PY{n}{user}\PY{o}{.}\PY{n}{title}\PY{p}{(}\PY{p}{)}\PY{p}{)}
\end{Verbatim}
\end{tcolorbox}

    \begin{Verbatim}[commandchars=\\\{\}]
Cody
Aaron
Ever
    \end{Verbatim}

    To get all items from one position in the list to the end of the list,
we can leave off the second index:

    \begin{tcolorbox}[breakable, size=fbox, boxrule=1pt, pad at break*=1mm,colback=cellbackground, colframe=cellborder]
\prompt{In}{incolor}{14}{\boxspacing}
\begin{Verbatim}[commandchars=\\\{\}]
\PY{n}{usernames} \PY{o}{=} \PY{p}{[}\PY{l+s+s1}{\PYZsq{}}\PY{l+s+s1}{bernice}\PY{l+s+s1}{\PYZsq{}}\PY{p}{,} \PY{l+s+s1}{\PYZsq{}}\PY{l+s+s1}{cody}\PY{l+s+s1}{\PYZsq{}}\PY{p}{,} \PY{l+s+s1}{\PYZsq{}}\PY{l+s+s1}{aaron}\PY{l+s+s1}{\PYZsq{}}\PY{p}{,} \PY{l+s+s1}{\PYZsq{}}\PY{l+s+s1}{ever}\PY{l+s+s1}{\PYZsq{}}\PY{p}{,} \PY{l+s+s1}{\PYZsq{}}\PY{l+s+s1}{dalia}\PY{l+s+s1}{\PYZsq{}}\PY{p}{]}

\PY{c+c1}{\PYZsh{} Grab all users from the third to the end.}
\PY{n}{end\PYZus{}batch} \PY{o}{=} \PY{n}{usernames}\PY{p}{[}\PY{l+m+mi}{2}\PY{p}{:}\PY{p}{]}

\PY{k}{for} \PY{n}{user} \PY{o+ow}{in} \PY{n}{end\PYZus{}batch}\PY{p}{:}
    \PY{n+nb}{print}\PY{p}{(}\PY{n}{user}\PY{o}{.}\PY{n}{title}\PY{p}{(}\PY{p}{)}\PY{p}{)}
\end{Verbatim}
\end{tcolorbox}

    \begin{Verbatim}[commandchars=\\\{\}]
Aaron
Ever
Dalia
    \end{Verbatim}

    \hypertarget{copying-a-list}{%
\subsubsection{Copying a list}\label{copying-a-list}}

You can use the slice notation to make a copy of a list, by leaving out
both the starting and the ending index. This causes the slice to consist
of everything from the first item to the last, which is the entire list.

    \begin{tcolorbox}[breakable, size=fbox, boxrule=1pt, pad at break*=1mm,colback=cellbackground, colframe=cellborder]
\prompt{In}{incolor}{15}{\boxspacing}
\begin{Verbatim}[commandchars=\\\{\}]
\PY{n}{usernames} \PY{o}{=} \PY{p}{[}\PY{l+s+s1}{\PYZsq{}}\PY{l+s+s1}{bernice}\PY{l+s+s1}{\PYZsq{}}\PY{p}{,} \PY{l+s+s1}{\PYZsq{}}\PY{l+s+s1}{cody}\PY{l+s+s1}{\PYZsq{}}\PY{p}{,} \PY{l+s+s1}{\PYZsq{}}\PY{l+s+s1}{aaron}\PY{l+s+s1}{\PYZsq{}}\PY{p}{,} \PY{l+s+s1}{\PYZsq{}}\PY{l+s+s1}{ever}\PY{l+s+s1}{\PYZsq{}}\PY{p}{,} \PY{l+s+s1}{\PYZsq{}}\PY{l+s+s1}{dalia}\PY{l+s+s1}{\PYZsq{}}\PY{p}{]}

\PY{c+c1}{\PYZsh{} Make a copy of the list.}
\PY{n}{copied\PYZus{}usernames} \PY{o}{=} \PY{n}{usernames}\PY{p}{[}\PY{p}{:}\PY{p}{]}
\PY{n+nb}{print}\PY{p}{(}\PY{l+s+s2}{\PYZdq{}}\PY{l+s+s2}{The full copied list:}\PY{l+s+se}{\PYZbs{}n}\PY{l+s+se}{\PYZbs{}t}\PY{l+s+s2}{\PYZdq{}}\PY{p}{,} \PY{n}{copied\PYZus{}usernames}\PY{p}{)}

\PY{c+c1}{\PYZsh{} Remove the first two users from the copied list.}
\PY{k}{del} \PY{n}{copied\PYZus{}usernames}\PY{p}{[}\PY{l+m+mi}{0}\PY{p}{]}
\PY{k}{del} \PY{n}{copied\PYZus{}usernames}\PY{p}{[}\PY{l+m+mi}{0}\PY{p}{]}
\PY{n+nb}{print}\PY{p}{(}\PY{l+s+s2}{\PYZdq{}}\PY{l+s+se}{\PYZbs{}n}\PY{l+s+s2}{Two users removed from copied list:}\PY{l+s+se}{\PYZbs{}n}\PY{l+s+se}{\PYZbs{}t}\PY{l+s+s2}{\PYZdq{}}\PY{p}{,} \PY{n}{copied\PYZus{}usernames}\PY{p}{)}

\PY{c+c1}{\PYZsh{} The original list is unaffected.}
\PY{n+nb}{print}\PY{p}{(}\PY{l+s+s2}{\PYZdq{}}\PY{l+s+se}{\PYZbs{}n}\PY{l+s+s2}{The original list:}\PY{l+s+se}{\PYZbs{}n}\PY{l+s+se}{\PYZbs{}t}\PY{l+s+s2}{\PYZdq{}}\PY{p}{,} \PY{n}{usernames}\PY{p}{)}
\end{Verbatim}
\end{tcolorbox}

    \begin{Verbatim}[commandchars=\\\{\}]
The full copied list:
         ['bernice', 'cody', 'aaron', 'ever', 'dalia']

Two users removed from copied list:
         ['aaron', 'ever', 'dalia']

The original list:
         ['bernice', 'cody', 'aaron', 'ever', 'dalia']
    \end{Verbatim}

    Exercises --- \#\#\#\# Alphabet Slices - Store the first ten letters of
the alphabet in a list. - Use a slice to print out the first three
letters of the alphabet. - Use a slice to print out any three letters
from the middle of your list. - Use a slice to print out the letters
from any point in the middle of your list, to the end.

\hypertarget{protected-list}{%
\paragraph{Protected List}\label{protected-list}}

\begin{itemize}
\tightlist
\item
  Your goal in this exercise is to prove that copying a list protects
  the original list.
\item
  Make a list with three people's names in it.
\item
  Use a slice to make a copy of the entire list.
\item
  Add at least two new names to the new copy of the list.
\item
  Make a loop that prints out all of the names in the original list,
  along with a message that this is the original list.
\item
  Make a loop that prints out all of the names in the copied list, along
  with a message that this is the copied list.
\end{itemize}

    \protect\hyperlink{}{top}

    \hypertarget{numerical-lists}{%
\section{Numerical Lists}\label{numerical-lists}}

There is nothing special about lists of numbers, but there are some
functions you can use to make working with numerical lists more
efficient. Let's make a list of the first ten numbers, and start working
with it to see how we can use numbers in a list.

    \begin{tcolorbox}[breakable, size=fbox, boxrule=1pt, pad at break*=1mm,colback=cellbackground, colframe=cellborder]
\prompt{In}{incolor}{16}{\boxspacing}
\begin{Verbatim}[commandchars=\\\{\}]
\PY{c+c1}{\PYZsh{} Print out the first ten numbers.}
\PY{n}{numbers} \PY{o}{=} \PY{p}{[}\PY{l+m+mi}{1}\PY{p}{,} \PY{l+m+mi}{2}\PY{p}{,} \PY{l+m+mi}{3}\PY{p}{,} \PY{l+m+mi}{4}\PY{p}{,} \PY{l+m+mi}{5}\PY{p}{,} \PY{l+m+mi}{6}\PY{p}{,} \PY{l+m+mi}{7}\PY{p}{,} \PY{l+m+mi}{8}\PY{p}{,} \PY{l+m+mi}{9}\PY{p}{,} \PY{l+m+mi}{10}\PY{p}{]}

\PY{k}{for} \PY{n}{number} \PY{o+ow}{in} \PY{n}{numbers}\PY{p}{:}
    \PY{n+nb}{print}\PY{p}{(}\PY{n}{number}\PY{p}{)}
\end{Verbatim}
\end{tcolorbox}

    \begin{Verbatim}[commandchars=\\\{\}]
1
2
3
4
5
6
7
8
9
10
    \end{Verbatim}

    \hypertarget{the-range-function}{%
\subsection{\texorpdfstring{The \emph{range()}
function}{The range() function}}\label{the-range-function}}

This works, but it is not very efficient if we want to work with a large
set of numbers. The \emph{range()} function helps us generate long lists
of numbers. Here are two ways to do the same thing, using the
\emph{range} function.

    \begin{tcolorbox}[breakable, size=fbox, boxrule=1pt, pad at break*=1mm,colback=cellbackground, colframe=cellborder]
\prompt{In}{incolor}{17}{\boxspacing}
\begin{Verbatim}[commandchars=\\\{\}]
\PY{c+c1}{\PYZsh{} Print the first ten numbers.}
\PY{k}{for} \PY{n}{number} \PY{o+ow}{in} \PY{n+nb}{range}\PY{p}{(}\PY{l+m+mi}{1}\PY{p}{,}\PY{l+m+mi}{11}\PY{p}{)}\PY{p}{:}
    \PY{n+nb}{print}\PY{p}{(}\PY{n}{number}\PY{p}{)}
\end{Verbatim}
\end{tcolorbox}

    \begin{Verbatim}[commandchars=\\\{\}]
1
2
3
4
5
6
7
8
9
10
    \end{Verbatim}

    The range function takes in a starting number, and an end number. You
get all integers, up to but not including the end number. You can also
add a \emph{step} value, which tells the \emph{range} function how big
of a step to take between numbers:

    \begin{tcolorbox}[breakable, size=fbox, boxrule=1pt, pad at break*=1mm,colback=cellbackground, colframe=cellborder]
\prompt{In}{incolor}{18}{\boxspacing}
\begin{Verbatim}[commandchars=\\\{\}]
\PY{c+c1}{\PYZsh{} Print the first ten odd numbers.}
\PY{k}{for} \PY{n}{number} \PY{o+ow}{in} \PY{n+nb}{range}\PY{p}{(}\PY{l+m+mi}{1}\PY{p}{,}\PY{l+m+mi}{21}\PY{p}{,}\PY{l+m+mi}{2}\PY{p}{)}\PY{p}{:}
    \PY{n+nb}{print}\PY{p}{(}\PY{n}{number}\PY{p}{)}
\end{Verbatim}
\end{tcolorbox}

    \begin{Verbatim}[commandchars=\\\{\}]
1
3
5
7
9
11
13
15
17
19
    \end{Verbatim}

    If we want to store these numbers in a list, we can use the
\emph{list()} function. This function takes in a range, and turns it
into a list:

    \begin{tcolorbox}[breakable, size=fbox, boxrule=1pt, pad at break*=1mm,colback=cellbackground, colframe=cellborder]
\prompt{In}{incolor}{19}{\boxspacing}
\begin{Verbatim}[commandchars=\\\{\}]
\PY{c+c1}{\PYZsh{} Create a list of the first ten numbers.}
\PY{n}{numbers} \PY{o}{=} \PY{n+nb}{list}\PY{p}{(}\PY{n+nb}{range}\PY{p}{(}\PY{l+m+mi}{1}\PY{p}{,}\PY{l+m+mi}{11}\PY{p}{)}\PY{p}{)}
\PY{n+nb}{print}\PY{p}{(}\PY{n}{numbers}\PY{p}{)}
\end{Verbatim}
\end{tcolorbox}

    \begin{Verbatim}[commandchars=\\\{\}]
[1, 2, 3, 4, 5, 6, 7, 8, 9, 10]
    \end{Verbatim}

    This is incredibly powerful; we can now create a list of the first
million numbers, just as easily as we made a list of the first ten
numbers. It doesn't really make sense to print the million numbers here,
but we can show that the list really does have one million items in it,
and we can print the last ten items to show that the list is correct.

    \begin{tcolorbox}[breakable, size=fbox, boxrule=1pt, pad at break*=1mm,colback=cellbackground, colframe=cellborder]
\prompt{In}{incolor}{20}{\boxspacing}
\begin{Verbatim}[commandchars=\\\{\}]
\PY{c+c1}{\PYZsh{} Store the first million numbers in a list.}
\PY{n}{numbers} \PY{o}{=} \PY{n+nb}{list}\PY{p}{(}\PY{n+nb}{range}\PY{p}{(}\PY{l+m+mi}{1}\PY{p}{,}\PY{l+m+mi}{1000001}\PY{p}{)}\PY{p}{)}

\PY{c+c1}{\PYZsh{} Show the length of the list:}
\PY{n+nb}{print}\PY{p}{(}\PY{l+s+s2}{\PYZdq{}}\PY{l+s+s2}{The list }\PY{l+s+s2}{\PYZsq{}}\PY{l+s+s2}{numbers}\PY{l+s+s2}{\PYZsq{}}\PY{l+s+s2}{ has }\PY{l+s+s2}{\PYZdq{}} \PY{o}{+} \PY{n+nb}{str}\PY{p}{(}\PY{n+nb}{len}\PY{p}{(}\PY{n}{numbers}\PY{p}{)}\PY{p}{)} \PY{o}{+} \PY{l+s+s2}{\PYZdq{}}\PY{l+s+s2}{ numbers in it.}\PY{l+s+s2}{\PYZdq{}}\PY{p}{)}

\PY{c+c1}{\PYZsh{} Show the last ten numbers:}
\PY{n+nb}{print}\PY{p}{(}\PY{l+s+s2}{\PYZdq{}}\PY{l+s+se}{\PYZbs{}n}\PY{l+s+s2}{The last ten numbers in the list are:}\PY{l+s+s2}{\PYZdq{}}\PY{p}{)}
\PY{k}{for} \PY{n}{number} \PY{o+ow}{in} \PY{n}{numbers}\PY{p}{[}\PY{o}{\PYZhy{}}\PY{l+m+mi}{10}\PY{p}{:}\PY{p}{]}\PY{p}{:}
    \PY{n+nb}{print}\PY{p}{(}\PY{n}{number}\PY{p}{)}
\end{Verbatim}
\end{tcolorbox}

    \begin{Verbatim}[commandchars=\\\{\}]
The list 'numbers' has 1000000 numbers in it.

The last ten numbers in the list are:
999991
999992
999993
999994
999995
999996
999997
999998
999999
1000000
    \end{Verbatim}

    There are two things here that might be a little unclear. The expression

\begin{verbatim}
str(len(numbers))
\end{verbatim}

takes the length of the \emph{numbers} list, and turns it into a string
that can be printed.

The expression

\begin{verbatim}
numbers[-10:]
\end{verbatim}

gives us a \emph{slice} of the list. The index \texttt{-1} is the last
item in the list, and the index \texttt{-10} is the item ten places from
the end of the list. So the slice \texttt{numbers{[}-10:{]}} gives us
everything from that item to the end of the list.

    \hypertarget{the-min-max-and-sum-functions}{%
\subsection{\texorpdfstring{The \emph{min()}, \emph{max()}, and
\emph{sum()}
functions}{The min(), max(), and sum() functions}}\label{the-min-max-and-sum-functions}}

There are three functions you can easily use with numerical lists. As
you might expect, the \emph{min()} function returns the smallest number
in the list, the \emph{max()} function returns the largest number in the
list, and the \emph{sum()} function returns the total of all numbers in
the list.

    \begin{tcolorbox}[breakable, size=fbox, boxrule=1pt, pad at break*=1mm,colback=cellbackground, colframe=cellborder]
\prompt{In}{incolor}{21}{\boxspacing}
\begin{Verbatim}[commandchars=\\\{\}]
\PY{n}{ages} \PY{o}{=} \PY{p}{[}\PY{l+m+mi}{23}\PY{p}{,} \PY{l+m+mi}{16}\PY{p}{,} \PY{l+m+mi}{14}\PY{p}{,} \PY{l+m+mi}{28}\PY{p}{,} \PY{l+m+mi}{19}\PY{p}{,} \PY{l+m+mi}{11}\PY{p}{,} \PY{l+m+mi}{38}\PY{p}{]}

\PY{n}{youngest} \PY{o}{=} \PY{n+nb}{min}\PY{p}{(}\PY{n}{ages}\PY{p}{)}
\PY{n}{oldest} \PY{o}{=} \PY{n+nb}{max}\PY{p}{(}\PY{n}{ages}\PY{p}{)}
\PY{n}{total\PYZus{}years} \PY{o}{=} \PY{n+nb}{sum}\PY{p}{(}\PY{n}{ages}\PY{p}{)}

\PY{n+nb}{print}\PY{p}{(}\PY{l+s+s2}{\PYZdq{}}\PY{l+s+s2}{Our youngest reader is }\PY{l+s+s2}{\PYZdq{}} \PY{o}{+} \PY{n+nb}{str}\PY{p}{(}\PY{n}{youngest}\PY{p}{)} \PY{o}{+} \PY{l+s+s2}{\PYZdq{}}\PY{l+s+s2}{ years old.}\PY{l+s+s2}{\PYZdq{}}\PY{p}{)}
\PY{n+nb}{print}\PY{p}{(}\PY{l+s+s2}{\PYZdq{}}\PY{l+s+s2}{Our oldest reader is }\PY{l+s+s2}{\PYZdq{}} \PY{o}{+} \PY{n+nb}{str}\PY{p}{(}\PY{n}{oldest}\PY{p}{)} \PY{o}{+} \PY{l+s+s2}{\PYZdq{}}\PY{l+s+s2}{ years old.}\PY{l+s+s2}{\PYZdq{}}\PY{p}{)}
\PY{n+nb}{print}\PY{p}{(}\PY{l+s+s2}{\PYZdq{}}\PY{l+s+s2}{Together, we have }\PY{l+s+s2}{\PYZdq{}} \PY{o}{+} \PY{n+nb}{str}\PY{p}{(}\PY{n}{total\PYZus{}years}\PY{p}{)} \PY{o}{+} \PY{l+s+s2}{\PYZdq{}}\PY{l+s+s2}{ years worth of life experience.}\PY{l+s+s2}{\PYZdq{}}\PY{p}{)}
\end{Verbatim}
\end{tcolorbox}

    \begin{Verbatim}[commandchars=\\\{\}]
Our youngest reader is 11 years old.
Our oldest reader is 38 years old.
Together, we have 149 years worth of life experience.
    \end{Verbatim}

    Exercises --- \#\#\#\# First Twenty - Use the \emph{range()} function to
store the first twenty numbers (1-20) in a list, and print them out.

\hypertarget{larger-sets}{%
\paragraph{Larger Sets}\label{larger-sets}}

\begin{itemize}
\tightlist
\item
  Take the \emph{first\_twenty.py} program you just wrote. Change your
  end number to a much larger number. How long does it take your
  computer to print out the first million numbers? (Most people will
  never see a million numbers scroll before their eyes. You can now see
  this!)
\end{itemize}

\hypertarget{five-wallets}{%
\paragraph{Five Wallets}\label{five-wallets}}

\begin{itemize}
\tightlist
\item
  Imagine five wallets with different amounts of cash in them. Store
  these five values in a list, and print out the following sentences:

  \begin{itemize}
  \tightlist
  \item
    ``The fattest wallet has \$ \emph{value} in it.''
  \item
    ``The skinniest wallet has \$ \emph{value} in it.''
  \item
    ``All together, these wallets have \$ \emph{value} in them.''
  \end{itemize}
\end{itemize}

    \protect\hyperlink{}{top}

    \hypertarget{list-comprehensions}{%
\section{List Comprehensions}\label{list-comprehensions}}

I thought carefully before including this section. If you are brand new
to programming, list comprehensions may look confusing at first. They
are a shorthand way of creating and working with lists. It is good to be
aware of list comprehensions, because you will see them in other
people's code, and they are really useful when you understand how to use
them. That said, if they don't make sense to you yet, don't worry about
using them right away. When you have worked with enough lists, you will
want to use comprehensions. For now, it is good enough to know they
exist, and to recognize them when you see them. If you like them, go
ahead and start trying to use them now.

\hypertarget{numerical-comprehensions}{%
\subsection{Numerical Comprehensions}\label{numerical-comprehensions}}

Let's consider how we might make a list of the first ten square numbers.
We could do it like this:

    \begin{tcolorbox}[breakable, size=fbox, boxrule=1pt, pad at break*=1mm,colback=cellbackground, colframe=cellborder]
\prompt{In}{incolor}{22}{\boxspacing}
\begin{Verbatim}[commandchars=\\\{\}]
\PY{c+c1}{\PYZsh{} Store the first ten square numbers in a list.}
\PY{c+c1}{\PYZsh{} Make an empty list that will hold our square numbers.}
\PY{n}{squares} \PY{o}{=} \PY{p}{[}\PY{p}{]}

\PY{c+c1}{\PYZsh{} Go through the first ten numbers, square them, and add them to our list.}
\PY{k}{for} \PY{n}{number} \PY{o+ow}{in} \PY{n+nb}{range}\PY{p}{(}\PY{l+m+mi}{1}\PY{p}{,}\PY{l+m+mi}{11}\PY{p}{)}\PY{p}{:}
    \PY{n}{new\PYZus{}square} \PY{o}{=} \PY{n}{number}\PY{o}{*}\PY{o}{*}\PY{l+m+mi}{2}
    \PY{n}{squares}\PY{o}{.}\PY{n}{append}\PY{p}{(}\PY{n}{new\PYZus{}square}\PY{p}{)}
    
\PY{c+c1}{\PYZsh{} Show that our list is correct.}
\PY{k}{for} \PY{n}{square} \PY{o+ow}{in} \PY{n}{squares}\PY{p}{:}
    \PY{n+nb}{print}\PY{p}{(}\PY{n}{square}\PY{p}{)}
\end{Verbatim}
\end{tcolorbox}

    \begin{Verbatim}[commandchars=\\\{\}]
1
4
9
16
25
36
49
64
81
100
    \end{Verbatim}

    This should make sense at this point. If it doesn't, go over the code
with these thoughts in mind: - We make an empty list called
\emph{squares} that will hold the values we are interested in. - Using
the \emph{range()} function, we start a loop that will go through the
numbers 1-10. - Each time we pass through the loop, we find the square
of the current number by raising it to the second power. - We add this
new value to our list \emph{squares}. - We go through our newly-defined
list and print out each square.

Now let's make this code more efficient. We don't really need to store
the new square in its own variable \emph{new\_square}; we can just add
it directly to the list of squares. The line

\begin{verbatim}
new_square = number**2
\end{verbatim}

is taken out, and the next line takes care of the squaring:

    \begin{tcolorbox}[breakable, size=fbox, boxrule=1pt, pad at break*=1mm,colback=cellbackground, colframe=cellborder]
\prompt{In}{incolor}{23}{\boxspacing}
\begin{Verbatim}[commandchars=\\\{\}]
\PY{c+c1}{\PYZsh{}\PYZsh{}\PYZsh{}highlight=[8]}
\PY{c+c1}{\PYZsh{} Store the first ten square numbers in a list.}
\PY{c+c1}{\PYZsh{} Make an empty list that will hold our square numbers.}
\PY{n}{squares} \PY{o}{=} \PY{p}{[}\PY{p}{]}

\PY{c+c1}{\PYZsh{} Go through the first ten numbers, square them, and add them to our list.}
\PY{k}{for} \PY{n}{number} \PY{o+ow}{in} \PY{n+nb}{range}\PY{p}{(}\PY{l+m+mi}{1}\PY{p}{,}\PY{l+m+mi}{11}\PY{p}{)}\PY{p}{:}
    \PY{n}{squares}\PY{o}{.}\PY{n}{append}\PY{p}{(}\PY{n}{number}\PY{o}{*}\PY{o}{*}\PY{l+m+mi}{2}\PY{p}{)}
    
\PY{c+c1}{\PYZsh{} Show that our list is correct.}
\PY{k}{for} \PY{n}{square} \PY{o+ow}{in} \PY{n}{squares}\PY{p}{:}
    \PY{n+nb}{print}\PY{p}{(}\PY{n}{square}\PY{p}{)}
\end{Verbatim}
\end{tcolorbox}

    \begin{Verbatim}[commandchars=\\\{\}]
1
4
9
16
25
36
49
64
81
100
    \end{Verbatim}

    List comprehensions allow us to collapse the first three lines of code
into one line. Here's what it looks like:

    \begin{tcolorbox}[breakable, size=fbox, boxrule=1pt, pad at break*=1mm,colback=cellbackground, colframe=cellborder]
\prompt{In}{incolor}{24}{\boxspacing}
\begin{Verbatim}[commandchars=\\\{\}]
\PY{c+c1}{\PYZsh{}\PYZsh{}\PYZsh{}highlight=[2,3]}
\PY{c+c1}{\PYZsh{} Store the first ten square numbers in a list.}
\PY{n}{squares} \PY{o}{=} \PY{p}{[}\PY{n}{number}\PY{o}{*}\PY{o}{*}\PY{l+m+mi}{2} \PY{k}{for} \PY{n}{number} \PY{o+ow}{in} \PY{n+nb}{range}\PY{p}{(}\PY{l+m+mi}{1}\PY{p}{,}\PY{l+m+mi}{11}\PY{p}{)}\PY{p}{]}

\PY{c+c1}{\PYZsh{} Show that our list is correct.}
\PY{k}{for} \PY{n}{square} \PY{o+ow}{in} \PY{n}{squares}\PY{p}{:}
    \PY{n+nb}{print}\PY{p}{(}\PY{n}{square}\PY{p}{)}
\end{Verbatim}
\end{tcolorbox}

    \begin{Verbatim}[commandchars=\\\{\}]
1
4
9
16
25
36
49
64
81
100
    \end{Verbatim}

    It should be pretty clear that this code is more efficient than our
previous approach, but it may not be clear what is happening. Let's take
a look at everything that is happening in that first line:

We define a list called \emph{squares}.

Look at the second part of what's in square brackets:

\begin{verbatim}
for number in range(1,11)
\end{verbatim}

This sets up a loop that goes through the numbers 1-10, storing each
value in the variable \emph{number}. Now we can see what happens to each
\emph{number} in the loop:

\begin{verbatim}
number**2
\end{verbatim}

Each number is raised to the second power, and this is the value that is
stored in the list we defined. We might read this line in the following
way:

squares = {[}raise \emph{number} to the second power, for each
\emph{number} in the range 1-10{]}

    \hypertarget{another-example}{%
\subsubsection{Another example}\label{another-example}}

It is probably helpful to see a few more examples of how comprehensions
can be used. Let's try to make the first ten even numbers, the longer
way:

    \begin{tcolorbox}[breakable, size=fbox, boxrule=1pt, pad at break*=1mm,colback=cellbackground, colframe=cellborder]
\prompt{In}{incolor}{25}{\boxspacing}
\begin{Verbatim}[commandchars=\\\{\}]
\PY{c+c1}{\PYZsh{} Make an empty list that will hold the even numbers.}
\PY{n}{evens} \PY{o}{=} \PY{p}{[}\PY{p}{]}

\PY{c+c1}{\PYZsh{} Loop through the numbers 1\PYZhy{}10, double each one, and add it to our list.}
\PY{k}{for} \PY{n}{number} \PY{o+ow}{in} \PY{n+nb}{range}\PY{p}{(}\PY{l+m+mi}{1}\PY{p}{,}\PY{l+m+mi}{11}\PY{p}{)}\PY{p}{:}
    \PY{n}{evens}\PY{o}{.}\PY{n}{append}\PY{p}{(}\PY{n}{number}\PY{o}{*}\PY{l+m+mi}{2}\PY{p}{)}
    
\PY{c+c1}{\PYZsh{} Show that our list is correct:}
\PY{k}{for} \PY{n}{even} \PY{o+ow}{in} \PY{n}{evens}\PY{p}{:}
    \PY{n+nb}{print}\PY{p}{(}\PY{n}{even}\PY{p}{)}
\end{Verbatim}
\end{tcolorbox}

    \begin{Verbatim}[commandchars=\\\{\}]
2
4
6
8
10
12
14
16
18
20
    \end{Verbatim}

    Here's how we might think of doing the same thing, using a list
comprehension:

evens = {[}multiply each \emph{number} by 2, for each \emph{number} in
the range 1-10{]}

Here is the same line in code:

    \begin{tcolorbox}[breakable, size=fbox, boxrule=1pt, pad at break*=1mm,colback=cellbackground, colframe=cellborder]
\prompt{In}{incolor}{26}{\boxspacing}
\begin{Verbatim}[commandchars=\\\{\}]
\PY{c+c1}{\PYZsh{}\PYZsh{}\PYZsh{}highlight=[2,3]}
\PY{c+c1}{\PYZsh{} Make a list of the first ten even numbers.}
\PY{n}{evens} \PY{o}{=} \PY{p}{[}\PY{n}{number}\PY{o}{*}\PY{l+m+mi}{2} \PY{k}{for} \PY{n}{number} \PY{o+ow}{in} \PY{n+nb}{range}\PY{p}{(}\PY{l+m+mi}{1}\PY{p}{,}\PY{l+m+mi}{11}\PY{p}{)}\PY{p}{]}

\PY{k}{for} \PY{n}{even} \PY{o+ow}{in} \PY{n}{evens}\PY{p}{:}
    \PY{n+nb}{print}\PY{p}{(}\PY{n}{even}\PY{p}{)}
\end{Verbatim}
\end{tcolorbox}

    \begin{Verbatim}[commandchars=\\\{\}]
2
4
6
8
10
12
14
16
18
20
    \end{Verbatim}

    \hypertarget{non-numerical-comprehensions}{%
\subsection{Non-numerical
comprehensions}\label{non-numerical-comprehensions}}

We can use comprehensions with non-numerical lists as well. In this
case, we will create an initial list, and then use a comprehension to
make a second list from the first one. Here is a simple example, without
using comprehensions:

    \begin{tcolorbox}[breakable, size=fbox, boxrule=1pt, pad at break*=1mm,colback=cellbackground, colframe=cellborder]
\prompt{In}{incolor}{27}{\boxspacing}
\begin{Verbatim}[commandchars=\\\{\}]
\PY{c+c1}{\PYZsh{} Consider some students.}
\PY{n}{students} \PY{o}{=} \PY{p}{[}\PY{l+s+s1}{\PYZsq{}}\PY{l+s+s1}{bernice}\PY{l+s+s1}{\PYZsq{}}\PY{p}{,} \PY{l+s+s1}{\PYZsq{}}\PY{l+s+s1}{aaron}\PY{l+s+s1}{\PYZsq{}}\PY{p}{,} \PY{l+s+s1}{\PYZsq{}}\PY{l+s+s1}{cody}\PY{l+s+s1}{\PYZsq{}}\PY{p}{]}

\PY{c+c1}{\PYZsh{} Let\PYZsq{}s turn them into great students.}
\PY{n}{great\PYZus{}students} \PY{o}{=} \PY{p}{[}\PY{p}{]}
\PY{k}{for} \PY{n}{student} \PY{o+ow}{in} \PY{n}{students}\PY{p}{:}
    \PY{n}{great\PYZus{}students}\PY{o}{.}\PY{n}{append}\PY{p}{(}\PY{n}{student}\PY{o}{.}\PY{n}{title}\PY{p}{(}\PY{p}{)} \PY{o}{+} \PY{l+s+s2}{\PYZdq{}}\PY{l+s+s2}{ the great!}\PY{l+s+s2}{\PYZdq{}}\PY{p}{)}

\PY{c+c1}{\PYZsh{} Let\PYZsq{}s greet each great student.}
\PY{k}{for} \PY{n}{great\PYZus{}student} \PY{o+ow}{in} \PY{n}{great\PYZus{}students}\PY{p}{:}
    \PY{n+nb}{print}\PY{p}{(}\PY{l+s+s2}{\PYZdq{}}\PY{l+s+s2}{Hello, }\PY{l+s+s2}{\PYZdq{}} \PY{o}{+} \PY{n}{great\PYZus{}student}\PY{p}{)}
\end{Verbatim}
\end{tcolorbox}

    \begin{Verbatim}[commandchars=\\\{\}]
Hello, Bernice the great!
Hello, Aaron the great!
Hello, Cody the great!
    \end{Verbatim}

    To use a comprehension in this code, we want to write something like
this:

great\_students = {[}add `the great' to each \emph{student}, for each
\emph{student} in the list of \emph{students}{]}

Here's what it looks like:

    \begin{tcolorbox}[breakable, size=fbox, boxrule=1pt, pad at break*=1mm,colback=cellbackground, colframe=cellborder]
\prompt{In}{incolor}{28}{\boxspacing}
\begin{Verbatim}[commandchars=\\\{\}]
\PY{c+c1}{\PYZsh{}\PYZsh{}\PYZsh{}highlight=[5,6]}
\PY{c+c1}{\PYZsh{} Consider some students.}
\PY{n}{students} \PY{o}{=} \PY{p}{[}\PY{l+s+s1}{\PYZsq{}}\PY{l+s+s1}{bernice}\PY{l+s+s1}{\PYZsq{}}\PY{p}{,} \PY{l+s+s1}{\PYZsq{}}\PY{l+s+s1}{aaron}\PY{l+s+s1}{\PYZsq{}}\PY{p}{,} \PY{l+s+s1}{\PYZsq{}}\PY{l+s+s1}{cody}\PY{l+s+s1}{\PYZsq{}}\PY{p}{]}

\PY{c+c1}{\PYZsh{} Let\PYZsq{}s turn them into great students.}
\PY{n}{great\PYZus{}students} \PY{o}{=} \PY{p}{[}\PY{n}{student}\PY{o}{.}\PY{n}{title}\PY{p}{(}\PY{p}{)} \PY{o}{+} \PY{l+s+s2}{\PYZdq{}}\PY{l+s+s2}{ the great!}\PY{l+s+s2}{\PYZdq{}} \PY{k}{for} \PY{n}{student} \PY{o+ow}{in} \PY{n}{students}\PY{p}{]}

\PY{c+c1}{\PYZsh{} Let\PYZsq{}s greet each great student.}
\PY{k}{for} \PY{n}{great\PYZus{}student} \PY{o+ow}{in} \PY{n}{great\PYZus{}students}\PY{p}{:}
    \PY{n+nb}{print}\PY{p}{(}\PY{l+s+s2}{\PYZdq{}}\PY{l+s+s2}{Hello, }\PY{l+s+s2}{\PYZdq{}} \PY{o}{+} \PY{n}{great\PYZus{}student}\PY{p}{)}
\end{Verbatim}
\end{tcolorbox}

    \begin{Verbatim}[commandchars=\\\{\}]
Hello, Bernice the great!
Hello, Aaron the great!
Hello, Cody the great!
    \end{Verbatim}

    Exercises --- If these examples are making sense, go ahead and try to do
the following exercises using comprehensions. If not, try the exercises
without comprehensions. You may figure out how to use comprehensions
after you have solved each exercise the longer way.

\hypertarget{multiples-of-ten}{%
\paragraph{Multiples of Ten}\label{multiples-of-ten}}

\begin{itemize}
\tightlist
\item
  Make a list of the first ten multiples of ten (10, 20, 30\ldots{} 90,
  100). There are a number of ways to do this, but try to do it using a
  list comprehension. Print out your list.
\end{itemize}

\hypertarget{cubes}{%
\paragraph{Cubes}\label{cubes}}

\begin{itemize}
\tightlist
\item
  We saw how to make a list of the first ten squares. Make a list of the
  first ten cubes (1, 8, 27\ldots{} 1000) using a list comprehension,
  and print them out.
\end{itemize}

\hypertarget{awesomeness}{%
\paragraph{Awesomeness}\label{awesomeness}}

\begin{itemize}
\tightlist
\item
  Store five names in a list. Make a second list that adds the phrase
  ``is awesome!'' to each name, using a list comprehension. Print out
  the awesome version of the names.
\end{itemize}

\hypertarget{working-backwards}{%
\paragraph{Working Backwards}\label{working-backwards}}

\begin{itemize}
\item
  Write out the following code without using a list comprehension:

  plus\_thirteen = {[}number + 13 for number in range(1,11){]}
\end{itemize}

    \protect\hyperlink{}{top}

    \hypertarget{strings-as-lists}{%
\section{Strings as Lists}\label{strings-as-lists}}

Now that you have some familiarity with lists, we can take a second look
at strings. A string is really a list of characters, so many of the
concepts from working with lists behave the same with strings.

    \hypertarget{strings-as-a-list-of-characters}{%
\subsection{Strings as a list of
characters}\label{strings-as-a-list-of-characters}}

We can loop through a string using a \emph{for} loop, just like we loop
through a list:

    \begin{tcolorbox}[breakable, size=fbox, boxrule=1pt, pad at break*=1mm,colback=cellbackground, colframe=cellborder]
\prompt{In}{incolor}{29}{\boxspacing}
\begin{Verbatim}[commandchars=\\\{\}]
\PY{n}{message} \PY{o}{=} \PY{l+s+s2}{\PYZdq{}}\PY{l+s+s2}{Hello!}\PY{l+s+s2}{\PYZdq{}}

\PY{k}{for} \PY{n}{letter} \PY{o+ow}{in} \PY{n}{message}\PY{p}{:}
    \PY{n+nb}{print}\PY{p}{(}\PY{n}{letter}\PY{p}{)}
\end{Verbatim}
\end{tcolorbox}

    \begin{Verbatim}[commandchars=\\\{\}]
H
e
l
l
o
!
    \end{Verbatim}

    We can create a list from a string. The list will have one element for
each character in the string:

    \begin{tcolorbox}[breakable, size=fbox, boxrule=1pt, pad at break*=1mm,colback=cellbackground, colframe=cellborder]
\prompt{In}{incolor}{30}{\boxspacing}
\begin{Verbatim}[commandchars=\\\{\}]
\PY{n}{message} \PY{o}{=} \PY{l+s+s2}{\PYZdq{}}\PY{l+s+s2}{Hello world!}\PY{l+s+s2}{\PYZdq{}}

\PY{n}{message\PYZus{}list} \PY{o}{=} \PY{n+nb}{list}\PY{p}{(}\PY{n}{message}\PY{p}{)}
\PY{n+nb}{print}\PY{p}{(}\PY{n}{message\PYZus{}list}\PY{p}{)}
\end{Verbatim}
\end{tcolorbox}

    \begin{Verbatim}[commandchars=\\\{\}]
['H', 'e', 'l', 'l', 'o', ' ', 'w', 'o', 'r', 'l', 'd', '!']
    \end{Verbatim}

    \hypertarget{slicing-strings}{%
\subsection{Slicing strings}\label{slicing-strings}}

We can access any character in a string by its position, just as we
access individual items in a list:

    \begin{tcolorbox}[breakable, size=fbox, boxrule=1pt, pad at break*=1mm,colback=cellbackground, colframe=cellborder]
\prompt{In}{incolor}{31}{\boxspacing}
\begin{Verbatim}[commandchars=\\\{\}]
\PY{n}{message} \PY{o}{=} \PY{l+s+s2}{\PYZdq{}}\PY{l+s+s2}{Hello World!}\PY{l+s+s2}{\PYZdq{}}
\PY{n}{first\PYZus{}char} \PY{o}{=} \PY{n}{message}\PY{p}{[}\PY{l+m+mi}{0}\PY{p}{]}
\PY{n}{last\PYZus{}char} \PY{o}{=} \PY{n}{message}\PY{p}{[}\PY{o}{\PYZhy{}}\PY{l+m+mi}{1}\PY{p}{]}

\PY{n+nb}{print}\PY{p}{(}\PY{n}{first\PYZus{}char}\PY{p}{,} \PY{n}{last\PYZus{}char}\PY{p}{)}
\end{Verbatim}
\end{tcolorbox}

    \begin{Verbatim}[commandchars=\\\{\}]
H !
    \end{Verbatim}

    We can extend this to take slices of a string:

    \begin{tcolorbox}[breakable, size=fbox, boxrule=1pt, pad at break*=1mm,colback=cellbackground, colframe=cellborder]
\prompt{In}{incolor}{32}{\boxspacing}
\begin{Verbatim}[commandchars=\\\{\}]
\PY{n}{message} \PY{o}{=} \PY{l+s+s2}{\PYZdq{}}\PY{l+s+s2}{Hello World!}\PY{l+s+s2}{\PYZdq{}}
\PY{n}{first\PYZus{}three} \PY{o}{=} \PY{n}{message}\PY{p}{[}\PY{p}{:}\PY{l+m+mi}{3}\PY{p}{]}
\PY{n}{last\PYZus{}three} \PY{o}{=} \PY{n}{message}\PY{p}{[}\PY{o}{\PYZhy{}}\PY{l+m+mi}{3}\PY{p}{:}\PY{p}{]}

\PY{n+nb}{print}\PY{p}{(}\PY{n}{first\PYZus{}three}\PY{p}{,} \PY{n}{last\PYZus{}three}\PY{p}{)}
\end{Verbatim}
\end{tcolorbox}

    \begin{Verbatim}[commandchars=\\\{\}]
Hel ld!
    \end{Verbatim}

    \hypertarget{finding-substrings}{%
\subsection{Finding substrings}\label{finding-substrings}}

Now that you have seen what indexes mean for strings, we can search for
\emph{substrings}. A substring is a series of characters that appears in
a string.

You can use the \emph{in} keyword to find out whether a particular
substring appears in a string:

    \begin{tcolorbox}[breakable, size=fbox, boxrule=1pt, pad at break*=1mm,colback=cellbackground, colframe=cellborder]
\prompt{In}{incolor}{33}{\boxspacing}
\begin{Verbatim}[commandchars=\\\{\}]
\PY{n}{message} \PY{o}{=} \PY{l+s+s2}{\PYZdq{}}\PY{l+s+s2}{I like cats and dogs.}\PY{l+s+s2}{\PYZdq{}}
\PY{n}{dog\PYZus{}present} \PY{o}{=} \PY{l+s+s1}{\PYZsq{}}\PY{l+s+s1}{dog}\PY{l+s+s1}{\PYZsq{}} \PY{o+ow}{in} \PY{n}{message}
\PY{n+nb}{print}\PY{p}{(}\PY{n}{dog\PYZus{}present}\PY{p}{)}
\end{Verbatim}
\end{tcolorbox}

    \begin{Verbatim}[commandchars=\\\{\}]
True
    \end{Verbatim}

    If you want to know where a substring appears in a string, you can use
the \emph{find()} method. The \emph{find()} method tells you the index
at which the substring begins.

    \begin{tcolorbox}[breakable, size=fbox, boxrule=1pt, pad at break*=1mm,colback=cellbackground, colframe=cellborder]
\prompt{In}{incolor}{34}{\boxspacing}
\begin{Verbatim}[commandchars=\\\{\}]
\PY{n}{message} \PY{o}{=} \PY{l+s+s2}{\PYZdq{}}\PY{l+s+s2}{I like cats and dogs.}\PY{l+s+s2}{\PYZdq{}}
\PY{n}{dog\PYZus{}index} \PY{o}{=} \PY{n}{message}\PY{o}{.}\PY{n}{find}\PY{p}{(}\PY{l+s+s1}{\PYZsq{}}\PY{l+s+s1}{dog}\PY{l+s+s1}{\PYZsq{}}\PY{p}{)}
\PY{n+nb}{print}\PY{p}{(}\PY{n}{dog\PYZus{}index}\PY{p}{)}
\end{Verbatim}
\end{tcolorbox}

    \begin{Verbatim}[commandchars=\\\{\}]
16
    \end{Verbatim}

    Note, however, that this function only returns the index of the first
appearance of the substring you are looking for. If the substring
appears more than once, you will miss the other substrings.

    \begin{tcolorbox}[breakable, size=fbox, boxrule=1pt, pad at break*=1mm,colback=cellbackground, colframe=cellborder]
\prompt{In}{incolor}{35}{\boxspacing}
\begin{Verbatim}[commandchars=\\\{\}]
\PY{c+c1}{\PYZsh{}\PYZsh{}\PYZsh{}highlight=[2]}
\PY{n}{message} \PY{o}{=} \PY{l+s+s2}{\PYZdq{}}\PY{l+s+s2}{I like cats and dogs, but I}\PY{l+s+s2}{\PYZsq{}}\PY{l+s+s2}{d much rather own a dog.}\PY{l+s+s2}{\PYZdq{}}
\PY{n}{dog\PYZus{}index} \PY{o}{=} \PY{n}{message}\PY{o}{.}\PY{n}{find}\PY{p}{(}\PY{l+s+s1}{\PYZsq{}}\PY{l+s+s1}{dog}\PY{l+s+s1}{\PYZsq{}}\PY{p}{)}
\PY{n+nb}{print}\PY{p}{(}\PY{n}{dog\PYZus{}index}\PY{p}{)}
\end{Verbatim}
\end{tcolorbox}

    \begin{Verbatim}[commandchars=\\\{\}]
16
    \end{Verbatim}

    If you want to find the last appearance of a substring, you can use the
\emph{rfind()} function:

    \begin{tcolorbox}[breakable, size=fbox, boxrule=1pt, pad at break*=1mm,colback=cellbackground, colframe=cellborder]
\prompt{In}{incolor}{36}{\boxspacing}
\begin{Verbatim}[commandchars=\\\{\}]
\PY{c+c1}{\PYZsh{}\PYZsh{}\PYZsh{}highlight=[3,4]}
\PY{n}{message} \PY{o}{=} \PY{l+s+s2}{\PYZdq{}}\PY{l+s+s2}{I like cats and dogs, but I}\PY{l+s+s2}{\PYZsq{}}\PY{l+s+s2}{d much rather own a dog.}\PY{l+s+s2}{\PYZdq{}}
\PY{n}{last\PYZus{}dog\PYZus{}index} \PY{o}{=} \PY{n}{message}\PY{o}{.}\PY{n}{rfind}\PY{p}{(}\PY{l+s+s1}{\PYZsq{}}\PY{l+s+s1}{dog}\PY{l+s+s1}{\PYZsq{}}\PY{p}{)}
\PY{n+nb}{print}\PY{p}{(}\PY{n}{last\PYZus{}dog\PYZus{}index}\PY{p}{)}
\end{Verbatim}
\end{tcolorbox}

    \begin{Verbatim}[commandchars=\\\{\}]
48
    \end{Verbatim}

    \hypertarget{replacing-substrings}{%
\subsection{Replacing substrings}\label{replacing-substrings}}

You can use the \emph{replace()} function to replace any substring with
another substring. To use the \emph{replace()} function, give the
substring you want to replace, and then the substring you want to
replace it with. You also need to store the new string, either in the
same string variable or in a new variable.

    \begin{tcolorbox}[breakable, size=fbox, boxrule=1pt, pad at break*=1mm,colback=cellbackground, colframe=cellborder]
\prompt{In}{incolor}{37}{\boxspacing}
\begin{Verbatim}[commandchars=\\\{\}]
\PY{n}{message} \PY{o}{=} \PY{l+s+s2}{\PYZdq{}}\PY{l+s+s2}{I like cats and dogs, but I}\PY{l+s+s2}{\PYZsq{}}\PY{l+s+s2}{d much rather own a dog.}\PY{l+s+s2}{\PYZdq{}}
\PY{n}{message} \PY{o}{=} \PY{n}{message}\PY{o}{.}\PY{n}{replace}\PY{p}{(}\PY{l+s+s1}{\PYZsq{}}\PY{l+s+s1}{dog}\PY{l+s+s1}{\PYZsq{}}\PY{p}{,} \PY{l+s+s1}{\PYZsq{}}\PY{l+s+s1}{snake}\PY{l+s+s1}{\PYZsq{}}\PY{p}{)}
\PY{n+nb}{print}\PY{p}{(}\PY{n}{message}\PY{p}{)}
\end{Verbatim}
\end{tcolorbox}

    \begin{Verbatim}[commandchars=\\\{\}]
I like cats and snakes, but I'd much rather own a snake.
    \end{Verbatim}

    \hypertarget{counting-substrings}{%
\subsection{Counting substrings}\label{counting-substrings}}

If you want to know how many times a substring appears within a string,
you can use the \emph{count()} method.

    \begin{tcolorbox}[breakable, size=fbox, boxrule=1pt, pad at break*=1mm,colback=cellbackground, colframe=cellborder]
\prompt{In}{incolor}{38}{\boxspacing}
\begin{Verbatim}[commandchars=\\\{\}]
\PY{n}{message} \PY{o}{=} \PY{l+s+s2}{\PYZdq{}}\PY{l+s+s2}{I like cats and dogs, but I}\PY{l+s+s2}{\PYZsq{}}\PY{l+s+s2}{d much rather own a dog.}\PY{l+s+s2}{\PYZdq{}}
\PY{n}{number\PYZus{}dogs} \PY{o}{=} \PY{n}{message}\PY{o}{.}\PY{n}{count}\PY{p}{(}\PY{l+s+s1}{\PYZsq{}}\PY{l+s+s1}{dog}\PY{l+s+s1}{\PYZsq{}}\PY{p}{)}
\PY{n+nb}{print}\PY{p}{(}\PY{n}{number\PYZus{}dogs}\PY{p}{)}
\end{Verbatim}
\end{tcolorbox}

    \begin{Verbatim}[commandchars=\\\{\}]
2
    \end{Verbatim}

    \hypertarget{splitting-strings}{%
\subsection{Splitting strings}\label{splitting-strings}}

Strings can be split into a set of substrings when they are separated by
a repeated character. If a string consists of a simple sentence, the
string can be split based on spaces. The \emph{split()} function returns
a list of substrings. The \emph{split()} function takes one argument,
the character that separates the parts of the string.

    \begin{tcolorbox}[breakable, size=fbox, boxrule=1pt, pad at break*=1mm,colback=cellbackground, colframe=cellborder]
\prompt{In}{incolor}{39}{\boxspacing}
\begin{Verbatim}[commandchars=\\\{\}]
\PY{n}{message} \PY{o}{=} \PY{l+s+s2}{\PYZdq{}}\PY{l+s+s2}{I like cats and dogs, but I}\PY{l+s+s2}{\PYZsq{}}\PY{l+s+s2}{d much rather own a dog.}\PY{l+s+s2}{\PYZdq{}}
\PY{n}{words} \PY{o}{=} \PY{n}{message}\PY{o}{.}\PY{n}{split}\PY{p}{(}\PY{l+s+s1}{\PYZsq{}}\PY{l+s+s1}{ }\PY{l+s+s1}{\PYZsq{}}\PY{p}{)}
\PY{n+nb}{print}\PY{p}{(}\PY{n}{words}\PY{p}{)}
\end{Verbatim}
\end{tcolorbox}

    \begin{Verbatim}[commandchars=\\\{\}]
['I', 'like', 'cats', 'and', 'dogs,', 'but', "I'd", 'much', 'rather', 'own',
'a', 'dog.']
    \end{Verbatim}

    Notice that the punctuation is left in the substrings.

It is more common to split strings that are really lists, separated by
something like a comma. The \emph{split()} function gives you an easy
way to turn comma-separated strings, which you can't do much with in
Python, into lists. Once you have your data in a list, you can work with
it in much more powerful ways.

    \begin{tcolorbox}[breakable, size=fbox, boxrule=1pt, pad at break*=1mm,colback=cellbackground, colframe=cellborder]
\prompt{In}{incolor}{40}{\boxspacing}
\begin{Verbatim}[commandchars=\\\{\}]
\PY{n}{animals} \PY{o}{=} \PY{l+s+s2}{\PYZdq{}}\PY{l+s+s2}{dog, cat, tiger, mouse, liger, bear}\PY{l+s+s2}{\PYZdq{}}

\PY{c+c1}{\PYZsh{} Rewrite the string as a list, and store it in the same variable}
\PY{n}{animals} \PY{o}{=} \PY{n}{animals}\PY{o}{.}\PY{n}{split}\PY{p}{(}\PY{l+s+s1}{\PYZsq{}}\PY{l+s+s1}{,}\PY{l+s+s1}{\PYZsq{}}\PY{p}{)}
\PY{n+nb}{print}\PY{p}{(}\PY{n}{animals}\PY{p}{)}
\end{Verbatim}
\end{tcolorbox}

    \begin{Verbatim}[commandchars=\\\{\}]
['dog', ' cat', ' tiger', ' mouse', ' liger', ' bear']
    \end{Verbatim}

    Notice that in this case, the spaces are also ignored. It is a good idea
to test the output of the \emph{split()} function and make sure it is
doing what you want with the data you are interested in.

One use of this is to work with spreadsheet data in your Python
programs. Most spreadsheet applications allow you to dump your data into
a comma-separated text file. You can read this file into your Python
program, or even copy and paste from the text file into your program
file, and then turn the data into a list. You can then process your
spreadsheet data using a \emph{for} loop.

    \hypertarget{other-string-methods}{%
\subsection{Other string methods}\label{other-string-methods}}

There are a number of
\href{http://docs.python.org/3.3/library/stdtypes.html\#string-methods}{other
string methods} that we won't go into right here, but you might want to
take a look at them. Most of these methods should make sense to you at
this point. You might not have use for any of them right now, but it is
good to know what you can do with strings. This way you will have a
sense of how to solve certain problems, even if it means referring back
to the list of methods to remind yourself how to write the correct
syntax when you need it.

    Exercises --- \#\#\#\# Listing a Sentence - Store a single sentence in a
variable. Use a for loop to print each character from your sentence on a
separate line.

\hypertarget{sentence-list}{%
\paragraph{Sentence List}\label{sentence-list}}

\begin{itemize}
\tightlist
\item
  Store a single sentence in a variable. Create a list from your
  sentence. Print your raw list (don't use a loop, just print the list).
\end{itemize}

\hypertarget{sentence-slices}{%
\paragraph{Sentence Slices}\label{sentence-slices}}

\begin{itemize}
\tightlist
\item
  Store a sentence in a variable. Using slices, print out the first five
  characters, any five consecutive characters from the middle of the
  sentence, and the last five characters of the sentence.
\end{itemize}

\hypertarget{finding-python}{%
\paragraph{Finding Python}\label{finding-python}}

\begin{itemize}
\tightlist
\item
  Store a sentence in a variable, making sure you use the word
  \emph{Python} at least twice in the sentence.
\item
  Use the \emph{in} keyword to prove that the word \emph{Python} is
  actually in the sentence.
\item
  Use the \emph{find()} function to show where the word \emph{Python}
  first appears in the sentence.
\item
  Use the \emph{rfind()} function to show the last place \emph{Python}
  appears in the sentence.
\item
  Use the \emph{count()} function to show how many times the word
  \emph{Python} appears in your sentence.
\item
  Use the \emph{split()} function to break your sentence into a list of
  words. Print the raw list, and use a loop to print each word on its
  own line.
\item
  Use the \emph{replace()} function to change \emph{Python} to
  \emph{Ruby} in your sentence.
\end{itemize}

    Challenges --- \#\#\#\# Counting DNA Nucleotides -
\href{http://rosalind.info/problems/locations/}{Project Rosalind} is a
\href{http://rosalind.info/problems/list-view/}{problem set} based on
biotechnology concepts. It is meant to show how programming skills can
help solve problems in genetics and biology. - If you have understood
this section on strings, you have enough information to solve the first
problem in Project Rosalind,
\href{http://rosalind.info/problems/dna/}{Counting DNA Nucleotides}.
Give the sample problem a try. - If you get the sample problem correct,
log in and try the full version of the problem!

\hypertarget{transcribing-dna-into-rna}{%
\paragraph{Transcribing DNA into RNA}\label{transcribing-dna-into-rna}}

\begin{itemize}
\tightlist
\item
  You also have enough information to try the second problem,
  \href{http://rosalind.info/problems/rna/}{Transcribing DNA into RNA}.
  Solve the sample problem.
\item
  If you solved the sample problem, log in and try the full version!
\end{itemize}

\hypertarget{complementing-a-strand-of-dna}{%
\paragraph{Complementing a Strand of
DNA}\label{complementing-a-strand-of-dna}}

\begin{itemize}
\tightlist
\item
  You guessed it, you can now try the third problem as well:
  \href{http://rosalind.info/problems/revc/}{Complementing a Strand of
  DNA}. Try the sample problem, and then try the full version if you are
  successful.
\end{itemize}

    \protect\hyperlink{}{top}

    \hypertarget{tuples}{%
\section{Tuples}\label{tuples}}

Tuples are basically lists that can never be changed. Lists are quite
dynamic; they can grow as you append and insert items, and they can
shrink as you remove items. You can modify any element you want to in a
list. Sometimes we like this behavior, but other times we may want to
ensure that no user or no part of a program can change a list. That's
what tuples are for.

Technically, lists are \emph{mutable} objects and tuples are
\emph{immutable} objects. Mutable objects can change (think of
\emph{mutations}), and immutable objects can not change.

\hypertarget{defining-tuples-and-accessing-elements}{%
\subsection{Defining tuples, and accessing
elements}\label{defining-tuples-and-accessing-elements}}

You define a tuple just like you define a list, except you use
parentheses instead of square brackets. Once you have a tuple, you can
access individual elements just like you can with a list, and you can
loop through the tuple with a \emph{for} loop:

    \begin{tcolorbox}[breakable, size=fbox, boxrule=1pt, pad at break*=1mm,colback=cellbackground, colframe=cellborder]
\prompt{In}{incolor}{41}{\boxspacing}
\begin{Verbatim}[commandchars=\\\{\}]
\PY{n}{colors} \PY{o}{=} \PY{p}{(}\PY{l+s+s1}{\PYZsq{}}\PY{l+s+s1}{red}\PY{l+s+s1}{\PYZsq{}}\PY{p}{,} \PY{l+s+s1}{\PYZsq{}}\PY{l+s+s1}{green}\PY{l+s+s1}{\PYZsq{}}\PY{p}{,} \PY{l+s+s1}{\PYZsq{}}\PY{l+s+s1}{blue}\PY{l+s+s1}{\PYZsq{}}\PY{p}{)}
\PY{n+nb}{print}\PY{p}{(}\PY{l+s+s2}{\PYZdq{}}\PY{l+s+s2}{The first color is: }\PY{l+s+s2}{\PYZdq{}} \PY{o}{+} \PY{n}{colors}\PY{p}{[}\PY{l+m+mi}{0}\PY{p}{]}\PY{p}{)}

\PY{n+nb}{print}\PY{p}{(}\PY{l+s+s2}{\PYZdq{}}\PY{l+s+se}{\PYZbs{}n}\PY{l+s+s2}{The available colors are:}\PY{l+s+s2}{\PYZdq{}}\PY{p}{)}
\PY{k}{for} \PY{n}{color} \PY{o+ow}{in} \PY{n}{colors}\PY{p}{:}
    \PY{n+nb}{print}\PY{p}{(}\PY{l+s+s2}{\PYZdq{}}\PY{l+s+s2}{\PYZhy{} }\PY{l+s+s2}{\PYZdq{}} \PY{o}{+} \PY{n}{color}\PY{p}{)}
\end{Verbatim}
\end{tcolorbox}

    \begin{Verbatim}[commandchars=\\\{\}]
The first color is: red

The available colors are:
- red
- green
- blue
    \end{Verbatim}

    If you try to add something to a tuple, you will get an error:

    \begin{tcolorbox}[breakable, size=fbox, boxrule=1pt, pad at break*=1mm,colback=cellbackground, colframe=cellborder]
\prompt{In}{incolor}{42}{\boxspacing}
\begin{Verbatim}[commandchars=\\\{\}]
\PY{n}{colors} \PY{o}{=} \PY{p}{(}\PY{l+s+s1}{\PYZsq{}}\PY{l+s+s1}{red}\PY{l+s+s1}{\PYZsq{}}\PY{p}{,} \PY{l+s+s1}{\PYZsq{}}\PY{l+s+s1}{green}\PY{l+s+s1}{\PYZsq{}}\PY{p}{,} \PY{l+s+s1}{\PYZsq{}}\PY{l+s+s1}{blue}\PY{l+s+s1}{\PYZsq{}}\PY{p}{)}
\PY{n}{colors}\PY{o}{.}\PY{n}{append}\PY{p}{(}\PY{l+s+s1}{\PYZsq{}}\PY{l+s+s1}{purple}\PY{l+s+s1}{\PYZsq{}}\PY{p}{)}
\end{Verbatim}
\end{tcolorbox}

    \begin{Verbatim}[commandchars=\\\{\}]

        ---------------------------------------------------------------------------

        AttributeError                            Traceback (most recent call last)

        /var/folders/1\_/2rbdt5292zdb57b821k90d5r0000gn/T/ipykernel\_56860/1652377402.py in <module>
          1 colors = ('red', 'green', 'blue')
    ----> 2 colors.append('purple')
    

        AttributeError: 'tuple' object has no attribute 'append'

    \end{Verbatim}

    The same kind of thing happens when you try to remove something from a
tuple, or modify one of its elements. Once you define a tuple, you can
be confident that its values will not change.

    \hypertarget{using-tuples-to-make-strings}{%
\subsection{Using tuples to make
strings}\label{using-tuples-to-make-strings}}

We have seen that it is pretty useful to be able to mix raw English
strings with values that are stored in variables, as in the following:

    \begin{tcolorbox}[breakable, size=fbox, boxrule=1pt, pad at break*=1mm,colback=cellbackground, colframe=cellborder]
\prompt{In}{incolor}{ }{\boxspacing}
\begin{Verbatim}[commandchars=\\\{\}]
\PY{n}{animal} \PY{o}{=} \PY{l+s+s1}{\PYZsq{}}\PY{l+s+s1}{dog}\PY{l+s+s1}{\PYZsq{}}
\PY{n+nb}{print}\PY{p}{(}\PY{l+s+s2}{\PYZdq{}}\PY{l+s+s2}{I have a }\PY{l+s+s2}{\PYZdq{}} \PY{o}{+} \PY{n}{animal} \PY{o}{+} \PY{l+s+s2}{\PYZdq{}}\PY{l+s+s2}{.}\PY{l+s+s2}{\PYZdq{}}\PY{p}{)}
\end{Verbatim}
\end{tcolorbox}

    This was especially useful when we had a series of similar statements to
make:

    \begin{tcolorbox}[breakable, size=fbox, boxrule=1pt, pad at break*=1mm,colback=cellbackground, colframe=cellborder]
\prompt{In}{incolor}{ }{\boxspacing}
\begin{Verbatim}[commandchars=\\\{\}]
\PY{n}{animals} \PY{o}{=} \PY{p}{[}\PY{l+s+s1}{\PYZsq{}}\PY{l+s+s1}{dog}\PY{l+s+s1}{\PYZsq{}}\PY{p}{,} \PY{l+s+s1}{\PYZsq{}}\PY{l+s+s1}{cat}\PY{l+s+s1}{\PYZsq{}}\PY{p}{,} \PY{l+s+s1}{\PYZsq{}}\PY{l+s+s1}{bear}\PY{l+s+s1}{\PYZsq{}}\PY{p}{]}
\PY{k}{for} \PY{n}{animal} \PY{o+ow}{in} \PY{n}{animals}\PY{p}{:}
    \PY{n+nb}{print}\PY{p}{(}\PY{l+s+s2}{\PYZdq{}}\PY{l+s+s2}{I have a }\PY{l+s+s2}{\PYZdq{}} \PY{o}{+} \PY{n}{animal} \PY{o}{+} \PY{l+s+s2}{\PYZdq{}}\PY{l+s+s2}{.}\PY{l+s+s2}{\PYZdq{}}\PY{p}{)}
\end{Verbatim}
\end{tcolorbox}

    I like this approach of using the plus sign to build strings because it
is fairly intuitive. We can see that we are adding several smaller
strings together to make one longer string. This is intuitive, but it is
a lot of typing. There is a shorter way to do this, using
\emph{placeholders}.

Python ignores most of the characters we put inside of strings. There
are a few characters that Python pays attention to, as we saw with
strings such as ``\t" and''\n``. Python also pays attention to''\%s''
and ``\%d''. These are placeholders. When Python sees the ``\%s''
placeholder, it looks ahead and pulls in the first argument after the \%
sign:

    \begin{tcolorbox}[breakable, size=fbox, boxrule=1pt, pad at break*=1mm,colback=cellbackground, colframe=cellborder]
\prompt{In}{incolor}{ }{\boxspacing}
\begin{Verbatim}[commandchars=\\\{\}]
\PY{n}{animal} \PY{o}{=} \PY{l+s+s1}{\PYZsq{}}\PY{l+s+s1}{dog}\PY{l+s+s1}{\PYZsq{}}
\PY{n+nb}{print}\PY{p}{(}\PY{l+s+s2}{\PYZdq{}}\PY{l+s+s2}{I have a }\PY{l+s+si}{\PYZpc{}s}\PY{l+s+s2}{.}\PY{l+s+s2}{\PYZdq{}} \PY{o}{\PYZpc{}} \PY{n}{animal}\PY{p}{)}
\end{Verbatim}
\end{tcolorbox}

    This is a much cleaner way of generating strings that include values. We
compose our sentence all in one string, and then tell Python what values
to pull into the string, in the appropriate places.

This is called \emph{string formatting}, and it looks the same when you
use a list:

    \begin{tcolorbox}[breakable, size=fbox, boxrule=1pt, pad at break*=1mm,colback=cellbackground, colframe=cellborder]
\prompt{In}{incolor}{ }{\boxspacing}
\begin{Verbatim}[commandchars=\\\{\}]
\PY{n}{animals} \PY{o}{=} \PY{p}{[}\PY{l+s+s1}{\PYZsq{}}\PY{l+s+s1}{dog}\PY{l+s+s1}{\PYZsq{}}\PY{p}{,} \PY{l+s+s1}{\PYZsq{}}\PY{l+s+s1}{cat}\PY{l+s+s1}{\PYZsq{}}\PY{p}{,} \PY{l+s+s1}{\PYZsq{}}\PY{l+s+s1}{bear}\PY{l+s+s1}{\PYZsq{}}\PY{p}{]}
\PY{k}{for} \PY{n}{animal} \PY{o+ow}{in} \PY{n}{animals}\PY{p}{:}
    \PY{n+nb}{print}\PY{p}{(}\PY{l+s+s2}{\PYZdq{}}\PY{l+s+s2}{I have a }\PY{l+s+si}{\PYZpc{}s}\PY{l+s+s2}{.}\PY{l+s+s2}{\PYZdq{}} \PY{o}{\PYZpc{}} \PY{n}{animal}\PY{p}{)}
\end{Verbatim}
\end{tcolorbox}

    If you have more than one value to put into the string you are
composing, you have to pack the values into a tuple:

    \begin{tcolorbox}[breakable, size=fbox, boxrule=1pt, pad at break*=1mm,colback=cellbackground, colframe=cellborder]
\prompt{In}{incolor}{ }{\boxspacing}
\begin{Verbatim}[commandchars=\\\{\}]
\PY{n}{animals} \PY{o}{=} \PY{p}{[}\PY{l+s+s1}{\PYZsq{}}\PY{l+s+s1}{dog}\PY{l+s+s1}{\PYZsq{}}\PY{p}{,} \PY{l+s+s1}{\PYZsq{}}\PY{l+s+s1}{cat}\PY{l+s+s1}{\PYZsq{}}\PY{p}{,} \PY{l+s+s1}{\PYZsq{}}\PY{l+s+s1}{bear}\PY{l+s+s1}{\PYZsq{}}\PY{p}{]}
\PY{n+nb}{print}\PY{p}{(}\PY{l+s+s2}{\PYZdq{}}\PY{l+s+s2}{I have a }\PY{l+s+si}{\PYZpc{}s}\PY{l+s+s2}{, a }\PY{l+s+si}{\PYZpc{}s}\PY{l+s+s2}{, and a }\PY{l+s+si}{\PYZpc{}s}\PY{l+s+s2}{.}\PY{l+s+s2}{\PYZdq{}} \PY{o}{\PYZpc{}} \PY{p}{(}\PY{n}{animals}\PY{p}{[}\PY{l+m+mi}{0}\PY{p}{]}\PY{p}{,} \PY{n}{animals}\PY{p}{[}\PY{l+m+mi}{1}\PY{p}{]}\PY{p}{,} \PY{n}{animals}\PY{p}{[}\PY{l+m+mi}{2}\PY{p}{]}\PY{p}{)}\PY{p}{)}
\end{Verbatim}
\end{tcolorbox}

    \hypertarget{string-formatting-with-numbers}{%
\subsubsection{String formatting with
numbers}\label{string-formatting-with-numbers}}

If you recall, printing a number with a string can cause an error:

    \begin{tcolorbox}[breakable, size=fbox, boxrule=1pt, pad at break*=1mm,colback=cellbackground, colframe=cellborder]
\prompt{In}{incolor}{ }{\boxspacing}
\begin{Verbatim}[commandchars=\\\{\}]
\PY{n}{number} \PY{o}{=} \PY{l+m+mi}{23}
\PY{n+nb}{print}\PY{p}{(}\PY{l+s+s2}{\PYZdq{}}\PY{l+s+s2}{My favorite number is }\PY{l+s+s2}{\PYZdq{}} \PY{o}{+} \PY{n}{number} \PY{o}{+} \PY{l+s+s2}{\PYZdq{}}\PY{l+s+s2}{.}\PY{l+s+s2}{\PYZdq{}}\PY{p}{)}
\end{Verbatim}
\end{tcolorbox}

    Python knows that you could be talking about the value 23, or the
characters `23'. So it throws an error, forcing us to clarify that we
want Python to treat the number as a string. We do this by
\emph{casting} the number into a string using the \emph{str()} function:

    \begin{tcolorbox}[breakable, size=fbox, boxrule=1pt, pad at break*=1mm,colback=cellbackground, colframe=cellborder]
\prompt{In}{incolor}{ }{\boxspacing}
\begin{Verbatim}[commandchars=\\\{\}]
\PY{c+c1}{\PYZsh{}\PYZsh{}\PYZsh{}highlight=[3]}
\PY{n}{number} \PY{o}{=} \PY{l+m+mi}{23}
\PY{n+nb}{print}\PY{p}{(}\PY{l+s+s2}{\PYZdq{}}\PY{l+s+s2}{My favorite number is }\PY{l+s+s2}{\PYZdq{}} \PY{o}{+} \PY{n+nb}{str}\PY{p}{(}\PY{n}{number}\PY{p}{)} \PY{o}{+} \PY{l+s+s2}{\PYZdq{}}\PY{l+s+s2}{.}\PY{l+s+s2}{\PYZdq{}}\PY{p}{)}
\end{Verbatim}
\end{tcolorbox}

    The format string ``\%d'' takes care of this for us. Watch how clean
this code is:

    \begin{tcolorbox}[breakable, size=fbox, boxrule=1pt, pad at break*=1mm,colback=cellbackground, colframe=cellborder]
\prompt{In}{incolor}{ }{\boxspacing}
\begin{Verbatim}[commandchars=\\\{\}]
\PY{c+c1}{\PYZsh{}\PYZsh{}\PYZsh{}highlight=[3]}
\PY{n}{number} \PY{o}{=} \PY{l+m+mi}{23}
\PY{n+nb}{print}\PY{p}{(}\PY{l+s+s2}{\PYZdq{}}\PY{l+s+s2}{My favorite number is }\PY{l+s+si}{\PYZpc{}d}\PY{l+s+s2}{.}\PY{l+s+s2}{\PYZdq{}} \PY{o}{\PYZpc{}} \PY{n}{number}\PY{p}{)}
\end{Verbatim}
\end{tcolorbox}

    If you want to use a series of numbers, you pack them into a tuple just
like we saw with strings:

    \begin{tcolorbox}[breakable, size=fbox, boxrule=1pt, pad at break*=1mm,colback=cellbackground, colframe=cellborder]
\prompt{In}{incolor}{ }{\boxspacing}
\begin{Verbatim}[commandchars=\\\{\}]
\PY{n}{numbers} \PY{o}{=} \PY{p}{[}\PY{l+m+mi}{7}\PY{p}{,} \PY{l+m+mi}{23}\PY{p}{,} \PY{l+m+mi}{42}\PY{p}{]}
\PY{n+nb}{print}\PY{p}{(}\PY{l+s+s2}{\PYZdq{}}\PY{l+s+s2}{My favorite numbers are }\PY{l+s+si}{\PYZpc{}d}\PY{l+s+s2}{, }\PY{l+s+si}{\PYZpc{}d}\PY{l+s+s2}{, and }\PY{l+s+si}{\PYZpc{}d}\PY{l+s+s2}{.}\PY{l+s+s2}{\PYZdq{}} \PY{o}{\PYZpc{}} \PY{p}{(}\PY{n}{numbers}\PY{p}{[}\PY{l+m+mi}{0}\PY{p}{]}\PY{p}{,} \PY{n}{numbers}\PY{p}{[}\PY{l+m+mi}{1}\PY{p}{]}\PY{p}{,} \PY{n}{numbers}\PY{p}{[}\PY{l+m+mi}{2}\PY{p}{]}\PY{p}{)}\PY{p}{)}
\end{Verbatim}
\end{tcolorbox}

    Just for clarification, look at how much longer the code is if you use
concatenation instead of string formatting:

    \begin{tcolorbox}[breakable, size=fbox, boxrule=1pt, pad at break*=1mm,colback=cellbackground, colframe=cellborder]
\prompt{In}{incolor}{ }{\boxspacing}
\begin{Verbatim}[commandchars=\\\{\}]
\PY{c+c1}{\PYZsh{}\PYZsh{}\PYZsh{}highlight=[3]}
\PY{n}{numbers} \PY{o}{=} \PY{p}{[}\PY{l+m+mi}{7}\PY{p}{,} \PY{l+m+mi}{23}\PY{p}{,} \PY{l+m+mi}{42}\PY{p}{]}
\PY{n+nb}{print}\PY{p}{(}\PY{l+s+s2}{\PYZdq{}}\PY{l+s+s2}{My favorite numbers are }\PY{l+s+s2}{\PYZdq{}} \PY{o}{+} \PY{n+nb}{str}\PY{p}{(}\PY{n}{numbers}\PY{p}{[}\PY{l+m+mi}{0}\PY{p}{]}\PY{p}{)} \PY{o}{+} \PY{l+s+s2}{\PYZdq{}}\PY{l+s+s2}{, }\PY{l+s+s2}{\PYZdq{}} \PY{o}{+} \PY{n+nb}{str}\PY{p}{(}\PY{n}{numbers}\PY{p}{[}\PY{l+m+mi}{1}\PY{p}{]}\PY{p}{)} \PY{o}{+} \PY{l+s+s2}{\PYZdq{}}\PY{l+s+s2}{, and }\PY{l+s+s2}{\PYZdq{}} \PY{o}{+} \PY{n+nb}{str}\PY{p}{(}\PY{n}{numbers}\PY{p}{[}\PY{l+m+mi}{2}\PY{p}{]}\PY{p}{)} \PY{o}{+} \PY{l+s+s2}{\PYZdq{}}\PY{l+s+s2}{.}\PY{l+s+s2}{\PYZdq{}}\PY{p}{)}
\end{Verbatim}
\end{tcolorbox}

    You can mix string and numerical placeholders in any order you want.

    \begin{tcolorbox}[breakable, size=fbox, boxrule=1pt, pad at break*=1mm,colback=cellbackground, colframe=cellborder]
\prompt{In}{incolor}{ }{\boxspacing}
\begin{Verbatim}[commandchars=\\\{\}]
\PY{n}{names} \PY{o}{=} \PY{p}{[}\PY{l+s+s1}{\PYZsq{}}\PY{l+s+s1}{eric}\PY{l+s+s1}{\PYZsq{}}\PY{p}{,} \PY{l+s+s1}{\PYZsq{}}\PY{l+s+s1}{ever}\PY{l+s+s1}{\PYZsq{}}\PY{p}{]}
\PY{n}{numbers} \PY{o}{=} \PY{p}{[}\PY{l+m+mi}{23}\PY{p}{,} \PY{l+m+mi}{2}\PY{p}{]}
\PY{n+nb}{print}\PY{p}{(}\PY{l+s+s2}{\PYZdq{}}\PY{l+s+si}{\PYZpc{}s}\PY{l+s+s2}{\PYZsq{}}\PY{l+s+s2}{s favorite number is }\PY{l+s+si}{\PYZpc{}d}\PY{l+s+s2}{, and }\PY{l+s+si}{\PYZpc{}s}\PY{l+s+s2}{\PYZsq{}}\PY{l+s+s2}{s favorite number is }\PY{l+s+si}{\PYZpc{}d}\PY{l+s+s2}{.}\PY{l+s+s2}{\PYZdq{}} \PY{o}{\PYZpc{}} \PY{p}{(}\PY{n}{names}\PY{p}{[}\PY{l+m+mi}{0}\PY{p}{]}\PY{o}{.}\PY{n}{title}\PY{p}{(}\PY{p}{)}\PY{p}{,} \PY{n}{numbers}\PY{p}{[}\PY{l+m+mi}{0}\PY{p}{]}\PY{p}{,} \PY{n}{names}\PY{p}{[}\PY{l+m+mi}{1}\PY{p}{]}\PY{o}{.}\PY{n}{title}\PY{p}{(}\PY{p}{)}\PY{p}{,} \PY{n}{numbers}\PY{p}{[}\PY{l+m+mi}{1}\PY{p}{]}\PY{p}{)}\PY{p}{)}
\end{Verbatim}
\end{tcolorbox}

    There are more sophisticated ways to do string formatting in Python 3,
but we will save that for later because it's a bit less intuitive than
this approach. For now, you can use whichever approach consistently gets
you the output that you want to see.

    Exercises ---

\hypertarget{gymnast-scores}{%
\paragraph{Gymnast Scores}\label{gymnast-scores}}

\begin{itemize}
\tightlist
\item
  A gymnast can earn a score between 1 and 10 from each judge; nothing
  lower, nothing higher. All scores are integer values; there are no
  decimal scores from a single judge.
\item
  Store the possible scores a gymnast can earn from one judge in a
  tuple.
\item
  Print out the sentence, ``The lowest possible score is \_\_\_, and the
  highest possible score is \_\_\_.'' Use the values from your tuple.
\item
  Print out a series of sentences, ``A judge can give a gymnast \_\_\_
  points.''

  \begin{itemize}
  \tightlist
  \item
    Don't worry if your first sentence reads ``A judge can give a
    gymnast 1 points.''
  \item
    However, you get 1000 bonus internet points if you can use a for
    loop, and have correct grammar. Section \ref{hints_gymnast_scores}
  \end{itemize}
\end{itemize}

\hypertarget{revision-with-tuples}{%
\paragraph{Revision with Tuples}\label{revision-with-tuples}}

\begin{itemize}
\tightlist
\item
  Choose a program you have already written that uses string
  concatenation.
\item
  Save the program with the same filename, but add \emph{\_tuple.py} to
  the end. For example, \emph{gymnast\_scores.py} becomes
  \emph{gymnast\_scores\_tuple.py}.
\item
  Rewrite your string sections using \emph{\%s} and \emph{\%d} instead
  of concatenation.
\item
  Repeat this with two other programs you have already written.
\end{itemize}

    \protect\hyperlink{}{top}

    \hypertarget{coding-style-pep-8}{%
\section{Coding Style: PEP 8}\label{coding-style-pep-8}}

You are now starting to write Python programs that have a little
substance. Your programs are growing a little longer, and there is a
little more structure to your programs. This is a really good time to
consider your overall style in writing code.

\hypertarget{why-do-we-need-style-conventions}{%
\subsection{Why do we need style
conventions?}\label{why-do-we-need-style-conventions}}

The people who originally developed Python made some of their decisions
based on the realization that code is read much more often than it is
written. The original developers paid as much attention to making the
language easy to read, as well as easy to write. Python has gained a lot
of respect as a programming language because of how readable the code
is. You have seen that Python uses indentation to show which lines in a
program are grouped together. This makes the structure of your code
visible to anyone who reads it. There are, however, some styling
decisions we get to make as programmers that can make our programs more
readable for ourselves, and for others.

There are several audiences to consider when you think about how
readable your code is.

\hypertarget{yourself-6-months-from-now}{%
\paragraph{Yourself, 6 months from
now}\label{yourself-6-months-from-now}}

\begin{itemize}
\tightlist
\item
  You know what you are thinking when you write code for the first time.
  But how easily will you recall what you were thinking when you come
  back to that code tomorrow, next week, or six months from now? We want
  our code to be as easy to read as possible six months from now, so we
  can jump back into our projects when we want to.
\end{itemize}

\hypertarget{other-programmers-you-might-want-to-collaborate-with}{%
\paragraph{Other programmers you might want to collaborate
with}\label{other-programmers-you-might-want-to-collaborate-with}}

\begin{itemize}
\tightlist
\item
  Every significant project is the result of collaboration these days.
  If you stay in programming, you will work with others in jobs and in
  open source projects. If you write readable code with good commments,
  people will be happy to work with you in any setting.
\end{itemize}

\hypertarget{potential-employers}{%
\paragraph{Potential employers}\label{potential-employers}}

\begin{itemize}
\tightlist
\item
  Most people who hire programmers will ask to see some code you have
  written, and they will probably ask you to write some code during your
  interview. If you are in the habit of writing code that is easy to
  read, you will do well in these situations.
\end{itemize}

\hypertarget{what-is-a-pep}{%
\subsection{What is a PEP?}\label{what-is-a-pep}}

A PEP is a \emph{Python Enhancement Proposal}. When people want to
suggest changes to the actual Python language, someone drafts a Python
Enhancement Proposal. One of the earliest PEPs was a collection of
guidelines for writing code that is easy to read. It was PEP 8, the
\href{http://www.python.org/dev/peps/pep-0008/}{Style Guide for Python
Code}. There is a lot in there that won't make sense to you for some
time yet, but there are some suggestions that you should be aware of
from the beginning. Starting with good style habits now will help you
write clean code from the beginning, which will help you make sense of
your code as well.

\hypertarget{basic-python-style-guidelines}{%
\subsection{Basic Python style
guidelines}\label{basic-python-style-guidelines}}

\hypertarget{indentation}{%
\paragraph{Indentation}\label{indentation}}

\begin{itemize}
\tightlist
\item
  Use 4 spaces for indentation. This is enough space to give your code
  some visual structure, while leaving room for multiple indentation
  levels. There are configuration settings in most editors to
  automatically convert tabs to 4 spaces, and it is a good idea to check
  this setting. On Geany, this is under
  Edit\textgreater Preferences\textgreater Editor\textgreater Indentation;
  set Width to 4, and Type to \emph{Spaces}.
\end{itemize}

\hypertarget{line-length}{%
\paragraph{Line Length}\label{line-length}}

\begin{itemize}
\item
  Use up to 79 characters per line of code, and 72 characters for
  comments. This is a style guideline that some people adhere to and
  others completely ignore. This used to relate to a limit on the
  display size of most monitors. Now almost every monitor is capable of
  showing much more than 80 characters per line. But we often work in
  terminals, which are not always high-resolution. We also like to have
  multiple code files open, next to each other. It turns out this is
  still a useful guideline to follow in most cases. There is a secondary
  guideline of sticking to 99 characters per line, if you want longer
  lines.

  Many editors have a setting that shows a vertical line that helps you
  keep your lines to a certain length. In Geany, you can find this
  setting under
  Edit\textgreater Preferences\textgreater Editor\textgreater Display.
  Make sure ``Long Line Marker'' is enabled, and set ``Column'' to 79.
\end{itemize}

\hypertarget{blank-lines}{%
\paragraph{Blank Lines}\label{blank-lines}}

\begin{itemize}
\tightlist
\item
  Use single blank lines to break up your code into meaningful blocks.
  You have seen this in many examples so far. You can use two blank
  lines in longer programs, but don't get excessive with blank lines.
\end{itemize}

\hypertarget{comments}{%
\paragraph{Comments}\label{comments}}

\begin{itemize}
\tightlist
\item
  Use a single space after the pound sign at the beginning of a line. If
  you are writing more than one paragraph, use an empty line with a
  pound sign between paragraphs.
\end{itemize}

\hypertarget{naming-variables}{%
\paragraph{Naming Variables}\label{naming-variables}}

\begin{itemize}
\tightlist
\item
  Name variables and program files using only lowercase letters,
  underscores, and numbers. Python won't complain or throw errors if you
  use capitalization, but you will mislead other programmers if you use
  capital letters in variables at this point.
\end{itemize}

That's all for now. We will go over more style guidelines as we
introduce more complicated programming structures. If you follow these
guidelines for now, you will be well on your way to writing readable
code that professionals will respect.

    Exercises --- \#\#\#\# Skim PEP 8 - If you haven't done so already, skim
\href{http://www.python.org/dev/peps/pep-0008/\#block-comments}{PEP 8 -
Style Guide for Python Code}. As you continue to learn Python, go back
and look at this every once in a while. I can't stress enough that many
good programmers will take you much more seriously from the start if you
are following community-wide conventions as you write your code.

\hypertarget{implement-pep-8}{%
\paragraph{Implement PEP 8}\label{implement-pep-8}}

\begin{itemize}
\tightlist
\item
  Take three of your longest programs, and add the extension
  \emph{\_pep8.py} to the filename of each program. Revise your code so
  that it meets the styling conventions listed above.
\end{itemize}

    \protect\hyperlink{}{top}

    \hypertarget{overall-challenges}{%
\section{Overall Challenges}\label{overall-challenges}}

\hypertarget{programming-words}{%
\paragraph{Programming Words}\label{programming-words}}

\begin{itemize}
\tightlist
\item
  Make a list of the most important words you have learned in
  programming so far. You should have terms such as list,
\item
  Make a corresponding list of definitions. Fill your list with
  `definition'.
\item
  Use a for loop to print out each word and its corresponding
  definition.
\item
  Maintain this program until you get to the section on Python's
  Dictionaries.
\end{itemize}

    \protect\hyperlink{}{top}

    \begin{center}\rule{0.5\linewidth}{0.5pt}\end{center}

\href{http://nbviewer.ipython.org/urls/raw.github.com/ehmatthes/intro_programming/master/notebooks/var_string_num.ipynb}{Previous:
Variables, Strings, and Numbers} \textbar{}
\href{http://nbviewer.ipython.org/urls/raw.github.com/ehmatthes/intro_programming/master/notebooks/index.ipynb}{Home}
\textbar{}
\href{http://nbviewer.ipython.org/urls/raw.github.com/ehmatthes/intro_programming/master/notebooks/introducing_functions.ipynb}{Next:
Introducing Functions}

    \hypertarget{hints}{%
\section{Hints}\label{hints}}

These are placed at the bottom, so you can have a chance to solve
exercises without seeing any hints.

\hypertarget{gymnast-scores}{%
\paragraph{Gymnast Scores}\label{gymnast-scores}}

\begin{itemize}
\tightlist
\item
  Hint: Use a slice.
\end{itemize}

    \protect\hyperlink{}{top}


    % Add a bibliography block to the postdoc
    
    
    
\end{document}
