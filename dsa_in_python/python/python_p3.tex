\documentclass[11pt]{article}

    \usepackage[breakable]{tcolorbox}
    \usepackage{parskip} % Stop auto-indenting (to mimic markdown behaviour)
    
    \usepackage{iftex}
    \ifPDFTeX
    	\usepackage[T1]{fontenc}
    	\usepackage{mathpazo}
    \else
    	\usepackage{fontspec}
    \fi

    % Basic figure setup, for now with no caption control since it's done
    % automatically by Pandoc (which extracts ![](path) syntax from Markdown).
    \usepackage{graphicx}
    % Maintain compatibility with old templates. Remove in nbconvert 6.0
    \let\Oldincludegraphics\includegraphics
    % Ensure that by default, figures have no caption (until we provide a
    % proper Figure object with a Caption API and a way to capture that
    % in the conversion process - todo).
    \usepackage{caption}
    \DeclareCaptionFormat{nocaption}{}
    \captionsetup{format=nocaption,aboveskip=0pt,belowskip=0pt}

    \usepackage[Export]{adjustbox} % Used to constrain images to a maximum size
    \adjustboxset{max size={0.9\linewidth}{0.9\paperheight}}
    \usepackage{float}
    \floatplacement{figure}{H} % forces figures to be placed at the correct location
    \usepackage{xcolor} % Allow colors to be defined
    \usepackage{enumerate} % Needed for markdown enumerations to work
    \usepackage{geometry} % Used to adjust the document margins
    \usepackage{amsmath} % Equations
    \usepackage{amssymb} % Equations
    \usepackage{textcomp} % defines textquotesingle
    % Hack from http://tex.stackexchange.com/a/47451/13684:
    \AtBeginDocument{%
        \def\PYZsq{\textquotesingle}% Upright quotes in Pygmentized code
    }
    \usepackage{upquote} % Upright quotes for verbatim code
    \usepackage{eurosym} % defines \euro
    \usepackage[mathletters]{ucs} % Extended unicode (utf-8) support
    \usepackage{fancyvrb} % verbatim replacement that allows latex
    \usepackage{grffile} % extends the file name processing of package graphics 
                         % to support a larger range
    \makeatletter % fix for grffile with XeLaTeX
    \def\Gread@@xetex#1{%
      \IfFileExists{"\Gin@base".bb}%
      {\Gread@eps{\Gin@base.bb}}%
      {\Gread@@xetex@aux#1}%
    }
    \makeatother

    % The hyperref package gives us a pdf with properly built
    % internal navigation ('pdf bookmarks' for the table of contents,
    % internal cross-reference links, web links for URLs, etc.)
    \usepackage{hyperref}
    % The default LaTeX title has an obnoxious amount of whitespace. By default,
    % titling removes some of it. It also provides customization options.
    \usepackage{titling}
    \usepackage{longtable} % longtable support required by pandoc >1.10
    \usepackage{booktabs}  % table support for pandoc > 1.12.2
    \usepackage[inline]{enumitem} % IRkernel/repr support (it uses the enumerate* environment)
    \usepackage[normalem]{ulem} % ulem is needed to support strikethroughs (\sout)
                                % normalem makes italics be italics, not underlines
    \usepackage{mathrsfs}
    

    
    % Colors for the hyperref package
    \definecolor{urlcolor}{rgb}{0,.145,.698}
    \definecolor{linkcolor}{rgb}{.71,0.21,0.01}
    \definecolor{citecolor}{rgb}{.12,.54,.11}

    % ANSI colors
    \definecolor{ansi-black}{HTML}{3E424D}
    \definecolor{ansi-black-intense}{HTML}{282C36}
    \definecolor{ansi-red}{HTML}{E75C58}
    \definecolor{ansi-red-intense}{HTML}{B22B31}
    \definecolor{ansi-green}{HTML}{00A250}
    \definecolor{ansi-green-intense}{HTML}{007427}
    \definecolor{ansi-yellow}{HTML}{DDB62B}
    \definecolor{ansi-yellow-intense}{HTML}{B27D12}
    \definecolor{ansi-blue}{HTML}{208FFB}
    \definecolor{ansi-blue-intense}{HTML}{0065CA}
    \definecolor{ansi-magenta}{HTML}{D160C4}
    \definecolor{ansi-magenta-intense}{HTML}{A03196}
    \definecolor{ansi-cyan}{HTML}{60C6C8}
    \definecolor{ansi-cyan-intense}{HTML}{258F8F}
    \definecolor{ansi-white}{HTML}{C5C1B4}
    \definecolor{ansi-white-intense}{HTML}{A1A6B2}
    \definecolor{ansi-default-inverse-fg}{HTML}{FFFFFF}
    \definecolor{ansi-default-inverse-bg}{HTML}{000000}

    % commands and environments needed by pandoc snippets
    % extracted from the output of `pandoc -s`
    \providecommand{\tightlist}{%
      \setlength{\itemsep}{0pt}\setlength{\parskip}{0pt}}
    \DefineVerbatimEnvironment{Highlighting}{Verbatim}{commandchars=\\\{\}}
    % Add ',fontsize=\small' for more characters per line
    \newenvironment{Shaded}{}{}
    \newcommand{\KeywordTok}[1]{\textcolor[rgb]{0.00,0.44,0.13}{\textbf{{#1}}}}
    \newcommand{\DataTypeTok}[1]{\textcolor[rgb]{0.56,0.13,0.00}{{#1}}}
    \newcommand{\DecValTok}[1]{\textcolor[rgb]{0.25,0.63,0.44}{{#1}}}
    \newcommand{\BaseNTok}[1]{\textcolor[rgb]{0.25,0.63,0.44}{{#1}}}
    \newcommand{\FloatTok}[1]{\textcolor[rgb]{0.25,0.63,0.44}{{#1}}}
    \newcommand{\CharTok}[1]{\textcolor[rgb]{0.25,0.44,0.63}{{#1}}}
    \newcommand{\StringTok}[1]{\textcolor[rgb]{0.25,0.44,0.63}{{#1}}}
    \newcommand{\CommentTok}[1]{\textcolor[rgb]{0.38,0.63,0.69}{\textit{{#1}}}}
    \newcommand{\OtherTok}[1]{\textcolor[rgb]{0.00,0.44,0.13}{{#1}}}
    \newcommand{\AlertTok}[1]{\textcolor[rgb]{1.00,0.00,0.00}{\textbf{{#1}}}}
    \newcommand{\FunctionTok}[1]{\textcolor[rgb]{0.02,0.16,0.49}{{#1}}}
    \newcommand{\RegionMarkerTok}[1]{{#1}}
    \newcommand{\ErrorTok}[1]{\textcolor[rgb]{1.00,0.00,0.00}{\textbf{{#1}}}}
    \newcommand{\NormalTok}[1]{{#1}}
    
    % Additional commands for more recent versions of Pandoc
    \newcommand{\ConstantTok}[1]{\textcolor[rgb]{0.53,0.00,0.00}{{#1}}}
    \newcommand{\SpecialCharTok}[1]{\textcolor[rgb]{0.25,0.44,0.63}{{#1}}}
    \newcommand{\VerbatimStringTok}[1]{\textcolor[rgb]{0.25,0.44,0.63}{{#1}}}
    \newcommand{\SpecialStringTok}[1]{\textcolor[rgb]{0.73,0.40,0.53}{{#1}}}
    \newcommand{\ImportTok}[1]{{#1}}
    \newcommand{\DocumentationTok}[1]{\textcolor[rgb]{0.73,0.13,0.13}{\textit{{#1}}}}
    \newcommand{\AnnotationTok}[1]{\textcolor[rgb]{0.38,0.63,0.69}{\textbf{\textit{{#1}}}}}
    \newcommand{\CommentVarTok}[1]{\textcolor[rgb]{0.38,0.63,0.69}{\textbf{\textit{{#1}}}}}
    \newcommand{\VariableTok}[1]{\textcolor[rgb]{0.10,0.09,0.49}{{#1}}}
    \newcommand{\ControlFlowTok}[1]{\textcolor[rgb]{0.00,0.44,0.13}{\textbf{{#1}}}}
    \newcommand{\OperatorTok}[1]{\textcolor[rgb]{0.40,0.40,0.40}{{#1}}}
    \newcommand{\BuiltInTok}[1]{{#1}}
    \newcommand{\ExtensionTok}[1]{{#1}}
    \newcommand{\PreprocessorTok}[1]{\textcolor[rgb]{0.74,0.48,0.00}{{#1}}}
    \newcommand{\AttributeTok}[1]{\textcolor[rgb]{0.49,0.56,0.16}{{#1}}}
    \newcommand{\InformationTok}[1]{\textcolor[rgb]{0.38,0.63,0.69}{\textbf{\textit{{#1}}}}}
    \newcommand{\WarningTok}[1]{\textcolor[rgb]{0.38,0.63,0.69}{\textbf{\textit{{#1}}}}}
    
    
    % Define a nice break command that doesn't care if a line doesn't already
    % exist.
    \def\br{\hspace*{\fill} \\* }
    % Math Jax compatibility definitions
    \def\gt{>}
    \def\lt{<}
    \let\Oldtex\TeX
    \let\Oldlatex\LaTeX
    \renewcommand{\TeX}{\textrm{\Oldtex}}
    \renewcommand{\LaTeX}{\textrm{\Oldlatex}}
    % Document parameters
    % Document title
    \title{python\_p3}
    
    
    
    
    
% Pygments definitions
\makeatletter
\def\PY@reset{\let\PY@it=\relax \let\PY@bf=\relax%
    \let\PY@ul=\relax \let\PY@tc=\relax%
    \let\PY@bc=\relax \let\PY@ff=\relax}
\def\PY@tok#1{\csname PY@tok@#1\endcsname}
\def\PY@toks#1+{\ifx\relax#1\empty\else%
    \PY@tok{#1}\expandafter\PY@toks\fi}
\def\PY@do#1{\PY@bc{\PY@tc{\PY@ul{%
    \PY@it{\PY@bf{\PY@ff{#1}}}}}}}
\def\PY#1#2{\PY@reset\PY@toks#1+\relax+\PY@do{#2}}

\@namedef{PY@tok@w}{\def\PY@tc##1{\textcolor[rgb]{0.73,0.73,0.73}{##1}}}
\@namedef{PY@tok@c}{\let\PY@it=\textit\def\PY@tc##1{\textcolor[rgb]{0.25,0.50,0.50}{##1}}}
\@namedef{PY@tok@cp}{\def\PY@tc##1{\textcolor[rgb]{0.74,0.48,0.00}{##1}}}
\@namedef{PY@tok@k}{\let\PY@bf=\textbf\def\PY@tc##1{\textcolor[rgb]{0.00,0.50,0.00}{##1}}}
\@namedef{PY@tok@kp}{\def\PY@tc##1{\textcolor[rgb]{0.00,0.50,0.00}{##1}}}
\@namedef{PY@tok@kt}{\def\PY@tc##1{\textcolor[rgb]{0.69,0.00,0.25}{##1}}}
\@namedef{PY@tok@o}{\def\PY@tc##1{\textcolor[rgb]{0.40,0.40,0.40}{##1}}}
\@namedef{PY@tok@ow}{\let\PY@bf=\textbf\def\PY@tc##1{\textcolor[rgb]{0.67,0.13,1.00}{##1}}}
\@namedef{PY@tok@nb}{\def\PY@tc##1{\textcolor[rgb]{0.00,0.50,0.00}{##1}}}
\@namedef{PY@tok@nf}{\def\PY@tc##1{\textcolor[rgb]{0.00,0.00,1.00}{##1}}}
\@namedef{PY@tok@nc}{\let\PY@bf=\textbf\def\PY@tc##1{\textcolor[rgb]{0.00,0.00,1.00}{##1}}}
\@namedef{PY@tok@nn}{\let\PY@bf=\textbf\def\PY@tc##1{\textcolor[rgb]{0.00,0.00,1.00}{##1}}}
\@namedef{PY@tok@ne}{\let\PY@bf=\textbf\def\PY@tc##1{\textcolor[rgb]{0.82,0.25,0.23}{##1}}}
\@namedef{PY@tok@nv}{\def\PY@tc##1{\textcolor[rgb]{0.10,0.09,0.49}{##1}}}
\@namedef{PY@tok@no}{\def\PY@tc##1{\textcolor[rgb]{0.53,0.00,0.00}{##1}}}
\@namedef{PY@tok@nl}{\def\PY@tc##1{\textcolor[rgb]{0.63,0.63,0.00}{##1}}}
\@namedef{PY@tok@ni}{\let\PY@bf=\textbf\def\PY@tc##1{\textcolor[rgb]{0.60,0.60,0.60}{##1}}}
\@namedef{PY@tok@na}{\def\PY@tc##1{\textcolor[rgb]{0.49,0.56,0.16}{##1}}}
\@namedef{PY@tok@nt}{\let\PY@bf=\textbf\def\PY@tc##1{\textcolor[rgb]{0.00,0.50,0.00}{##1}}}
\@namedef{PY@tok@nd}{\def\PY@tc##1{\textcolor[rgb]{0.67,0.13,1.00}{##1}}}
\@namedef{PY@tok@s}{\def\PY@tc##1{\textcolor[rgb]{0.73,0.13,0.13}{##1}}}
\@namedef{PY@tok@sd}{\let\PY@it=\textit\def\PY@tc##1{\textcolor[rgb]{0.73,0.13,0.13}{##1}}}
\@namedef{PY@tok@si}{\let\PY@bf=\textbf\def\PY@tc##1{\textcolor[rgb]{0.73,0.40,0.53}{##1}}}
\@namedef{PY@tok@se}{\let\PY@bf=\textbf\def\PY@tc##1{\textcolor[rgb]{0.73,0.40,0.13}{##1}}}
\@namedef{PY@tok@sr}{\def\PY@tc##1{\textcolor[rgb]{0.73,0.40,0.53}{##1}}}
\@namedef{PY@tok@ss}{\def\PY@tc##1{\textcolor[rgb]{0.10,0.09,0.49}{##1}}}
\@namedef{PY@tok@sx}{\def\PY@tc##1{\textcolor[rgb]{0.00,0.50,0.00}{##1}}}
\@namedef{PY@tok@m}{\def\PY@tc##1{\textcolor[rgb]{0.40,0.40,0.40}{##1}}}
\@namedef{PY@tok@gh}{\let\PY@bf=\textbf\def\PY@tc##1{\textcolor[rgb]{0.00,0.00,0.50}{##1}}}
\@namedef{PY@tok@gu}{\let\PY@bf=\textbf\def\PY@tc##1{\textcolor[rgb]{0.50,0.00,0.50}{##1}}}
\@namedef{PY@tok@gd}{\def\PY@tc##1{\textcolor[rgb]{0.63,0.00,0.00}{##1}}}
\@namedef{PY@tok@gi}{\def\PY@tc##1{\textcolor[rgb]{0.00,0.63,0.00}{##1}}}
\@namedef{PY@tok@gr}{\def\PY@tc##1{\textcolor[rgb]{1.00,0.00,0.00}{##1}}}
\@namedef{PY@tok@ge}{\let\PY@it=\textit}
\@namedef{PY@tok@gs}{\let\PY@bf=\textbf}
\@namedef{PY@tok@gp}{\let\PY@bf=\textbf\def\PY@tc##1{\textcolor[rgb]{0.00,0.00,0.50}{##1}}}
\@namedef{PY@tok@go}{\def\PY@tc##1{\textcolor[rgb]{0.53,0.53,0.53}{##1}}}
\@namedef{PY@tok@gt}{\def\PY@tc##1{\textcolor[rgb]{0.00,0.27,0.87}{##1}}}
\@namedef{PY@tok@err}{\def\PY@bc##1{{\setlength{\fboxsep}{\string -\fboxrule}\fcolorbox[rgb]{1.00,0.00,0.00}{1,1,1}{\strut ##1}}}}
\@namedef{PY@tok@kc}{\let\PY@bf=\textbf\def\PY@tc##1{\textcolor[rgb]{0.00,0.50,0.00}{##1}}}
\@namedef{PY@tok@kd}{\let\PY@bf=\textbf\def\PY@tc##1{\textcolor[rgb]{0.00,0.50,0.00}{##1}}}
\@namedef{PY@tok@kn}{\let\PY@bf=\textbf\def\PY@tc##1{\textcolor[rgb]{0.00,0.50,0.00}{##1}}}
\@namedef{PY@tok@kr}{\let\PY@bf=\textbf\def\PY@tc##1{\textcolor[rgb]{0.00,0.50,0.00}{##1}}}
\@namedef{PY@tok@bp}{\def\PY@tc##1{\textcolor[rgb]{0.00,0.50,0.00}{##1}}}
\@namedef{PY@tok@fm}{\def\PY@tc##1{\textcolor[rgb]{0.00,0.00,1.00}{##1}}}
\@namedef{PY@tok@vc}{\def\PY@tc##1{\textcolor[rgb]{0.10,0.09,0.49}{##1}}}
\@namedef{PY@tok@vg}{\def\PY@tc##1{\textcolor[rgb]{0.10,0.09,0.49}{##1}}}
\@namedef{PY@tok@vi}{\def\PY@tc##1{\textcolor[rgb]{0.10,0.09,0.49}{##1}}}
\@namedef{PY@tok@vm}{\def\PY@tc##1{\textcolor[rgb]{0.10,0.09,0.49}{##1}}}
\@namedef{PY@tok@sa}{\def\PY@tc##1{\textcolor[rgb]{0.73,0.13,0.13}{##1}}}
\@namedef{PY@tok@sb}{\def\PY@tc##1{\textcolor[rgb]{0.73,0.13,0.13}{##1}}}
\@namedef{PY@tok@sc}{\def\PY@tc##1{\textcolor[rgb]{0.73,0.13,0.13}{##1}}}
\@namedef{PY@tok@dl}{\def\PY@tc##1{\textcolor[rgb]{0.73,0.13,0.13}{##1}}}
\@namedef{PY@tok@s2}{\def\PY@tc##1{\textcolor[rgb]{0.73,0.13,0.13}{##1}}}
\@namedef{PY@tok@sh}{\def\PY@tc##1{\textcolor[rgb]{0.73,0.13,0.13}{##1}}}
\@namedef{PY@tok@s1}{\def\PY@tc##1{\textcolor[rgb]{0.73,0.13,0.13}{##1}}}
\@namedef{PY@tok@mb}{\def\PY@tc##1{\textcolor[rgb]{0.40,0.40,0.40}{##1}}}
\@namedef{PY@tok@mf}{\def\PY@tc##1{\textcolor[rgb]{0.40,0.40,0.40}{##1}}}
\@namedef{PY@tok@mh}{\def\PY@tc##1{\textcolor[rgb]{0.40,0.40,0.40}{##1}}}
\@namedef{PY@tok@mi}{\def\PY@tc##1{\textcolor[rgb]{0.40,0.40,0.40}{##1}}}
\@namedef{PY@tok@il}{\def\PY@tc##1{\textcolor[rgb]{0.40,0.40,0.40}{##1}}}
\@namedef{PY@tok@mo}{\def\PY@tc##1{\textcolor[rgb]{0.40,0.40,0.40}{##1}}}
\@namedef{PY@tok@ch}{\let\PY@it=\textit\def\PY@tc##1{\textcolor[rgb]{0.25,0.50,0.50}{##1}}}
\@namedef{PY@tok@cm}{\let\PY@it=\textit\def\PY@tc##1{\textcolor[rgb]{0.25,0.50,0.50}{##1}}}
\@namedef{PY@tok@cpf}{\let\PY@it=\textit\def\PY@tc##1{\textcolor[rgb]{0.25,0.50,0.50}{##1}}}
\@namedef{PY@tok@c1}{\let\PY@it=\textit\def\PY@tc##1{\textcolor[rgb]{0.25,0.50,0.50}{##1}}}
\@namedef{PY@tok@cs}{\let\PY@it=\textit\def\PY@tc##1{\textcolor[rgb]{0.25,0.50,0.50}{##1}}}

\def\PYZbs{\char`\\}
\def\PYZus{\char`\_}
\def\PYZob{\char`\{}
\def\PYZcb{\char`\}}
\def\PYZca{\char`\^}
\def\PYZam{\char`\&}
\def\PYZlt{\char`\<}
\def\PYZgt{\char`\>}
\def\PYZsh{\char`\#}
\def\PYZpc{\char`\%}
\def\PYZdl{\char`\$}
\def\PYZhy{\char`\-}
\def\PYZsq{\char`\'}
\def\PYZdq{\char`\"}
\def\PYZti{\char`\~}
% for compatibility with earlier versions
\def\PYZat{@}
\def\PYZlb{[}
\def\PYZrb{]}
\makeatother


    % For linebreaks inside Verbatim environment from package fancyvrb. 
    \makeatletter
        \newbox\Wrappedcontinuationbox 
        \newbox\Wrappedvisiblespacebox 
        \newcommand*\Wrappedvisiblespace {\textcolor{red}{\textvisiblespace}} 
        \newcommand*\Wrappedcontinuationsymbol {\textcolor{red}{\llap{\tiny$\m@th\hookrightarrow$}}} 
        \newcommand*\Wrappedcontinuationindent {3ex } 
        \newcommand*\Wrappedafterbreak {\kern\Wrappedcontinuationindent\copy\Wrappedcontinuationbox} 
        % Take advantage of the already applied Pygments mark-up to insert 
        % potential linebreaks for TeX processing. 
        %        {, <, #, %, $, ' and ": go to next line. 
        %        _, }, ^, &, >, - and ~: stay at end of broken line. 
        % Use of \textquotesingle for straight quote. 
        \newcommand*\Wrappedbreaksatspecials {% 
            \def\PYGZus{\discretionary{\char`\_}{\Wrappedafterbreak}{\char`\_}}% 
            \def\PYGZob{\discretionary{}{\Wrappedafterbreak\char`\{}{\char`\{}}% 
            \def\PYGZcb{\discretionary{\char`\}}{\Wrappedafterbreak}{\char`\}}}% 
            \def\PYGZca{\discretionary{\char`\^}{\Wrappedafterbreak}{\char`\^}}% 
            \def\PYGZam{\discretionary{\char`\&}{\Wrappedafterbreak}{\char`\&}}% 
            \def\PYGZlt{\discretionary{}{\Wrappedafterbreak\char`\<}{\char`\<}}% 
            \def\PYGZgt{\discretionary{\char`\>}{\Wrappedafterbreak}{\char`\>}}% 
            \def\PYGZsh{\discretionary{}{\Wrappedafterbreak\char`\#}{\char`\#}}% 
            \def\PYGZpc{\discretionary{}{\Wrappedafterbreak\char`\%}{\char`\%}}% 
            \def\PYGZdl{\discretionary{}{\Wrappedafterbreak\char`\$}{\char`\$}}% 
            \def\PYGZhy{\discretionary{\char`\-}{\Wrappedafterbreak}{\char`\-}}% 
            \def\PYGZsq{\discretionary{}{\Wrappedafterbreak\textquotesingle}{\textquotesingle}}% 
            \def\PYGZdq{\discretionary{}{\Wrappedafterbreak\char`\"}{\char`\"}}% 
            \def\PYGZti{\discretionary{\char`\~}{\Wrappedafterbreak}{\char`\~}}% 
        } 
        % Some characters . , ; ? ! / are not pygmentized. 
        % This macro makes them "active" and they will insert potential linebreaks 
        \newcommand*\Wrappedbreaksatpunct {% 
            \lccode`\~`\.\lowercase{\def~}{\discretionary{\hbox{\char`\.}}{\Wrappedafterbreak}{\hbox{\char`\.}}}% 
            \lccode`\~`\,\lowercase{\def~}{\discretionary{\hbox{\char`\,}}{\Wrappedafterbreak}{\hbox{\char`\,}}}% 
            \lccode`\~`\;\lowercase{\def~}{\discretionary{\hbox{\char`\;}}{\Wrappedafterbreak}{\hbox{\char`\;}}}% 
            \lccode`\~`\:\lowercase{\def~}{\discretionary{\hbox{\char`\:}}{\Wrappedafterbreak}{\hbox{\char`\:}}}% 
            \lccode`\~`\?\lowercase{\def~}{\discretionary{\hbox{\char`\?}}{\Wrappedafterbreak}{\hbox{\char`\?}}}% 
            \lccode`\~`\!\lowercase{\def~}{\discretionary{\hbox{\char`\!}}{\Wrappedafterbreak}{\hbox{\char`\!}}}% 
            \lccode`\~`\/\lowercase{\def~}{\discretionary{\hbox{\char`\/}}{\Wrappedafterbreak}{\hbox{\char`\/}}}% 
            \catcode`\.\active
            \catcode`\,\active 
            \catcode`\;\active
            \catcode`\:\active
            \catcode`\?\active
            \catcode`\!\active
            \catcode`\/\active 
            \lccode`\~`\~ 	
        }
    \makeatother

    \let\OriginalVerbatim=\Verbatim
    \makeatletter
    \renewcommand{\Verbatim}[1][1]{%
        %\parskip\z@skip
        \sbox\Wrappedcontinuationbox {\Wrappedcontinuationsymbol}%
        \sbox\Wrappedvisiblespacebox {\FV@SetupFont\Wrappedvisiblespace}%
        \def\FancyVerbFormatLine ##1{\hsize\linewidth
            \vtop{\raggedright\hyphenpenalty\z@\exhyphenpenalty\z@
                \doublehyphendemerits\z@\finalhyphendemerits\z@
                \strut ##1\strut}%
        }%
        % If the linebreak is at a space, the latter will be displayed as visible
        % space at end of first line, and a continuation symbol starts next line.
        % Stretch/shrink are however usually zero for typewriter font.
        \def\FV@Space {%
            \nobreak\hskip\z@ plus\fontdimen3\font minus\fontdimen4\font
            \discretionary{\copy\Wrappedvisiblespacebox}{\Wrappedafterbreak}
            {\kern\fontdimen2\font}%
        }%
        
        % Allow breaks at special characters using \PYG... macros.
        \Wrappedbreaksatspecials
        % Breaks at punctuation characters . , ; ? ! and / need catcode=\active 	
        \OriginalVerbatim[#1,codes*=\Wrappedbreaksatpunct]%
    }
    \makeatother

    % Exact colors from NB
    \definecolor{incolor}{HTML}{303F9F}
    \definecolor{outcolor}{HTML}{D84315}
    \definecolor{cellborder}{HTML}{CFCFCF}
    \definecolor{cellbackground}{HTML}{F7F7F7}
    
    % prompt
    \makeatletter
    \newcommand{\boxspacing}{\kern\kvtcb@left@rule\kern\kvtcb@boxsep}
    \makeatother
    \newcommand{\prompt}[4]{
        \ttfamily\llap{{\color{#2}[#3]:\hspace{3pt}#4}}\vspace{-\baselineskip}
    }
    

    
    % Prevent overflowing lines due to hard-to-break entities
    \sloppy 
    % Setup hyperref package
    \hypersetup{
      breaklinks=true,  % so long urls are correctly broken across lines
      colorlinks=true,
      urlcolor=urlcolor,
      linkcolor=linkcolor,
      citecolor=citecolor,
      }
    % Slightly bigger margins than the latex defaults
    
    \geometry{verbose,tmargin=1in,bmargin=1in,lmargin=1in,rmargin=1in}
    
    

\begin{document}
    
    \maketitle
    
    

    
    \hypertarget{naming-conventions}{%
\subsection{Naming conventions}\label{naming-conventions}}

\begin{itemize}
\tightlist
\item
  \_single\_leading\_underscore: weak ``internal use'' indicator
\item
  e.g., from M import * does not import objects whose names start with
  an underscore
\item
  single\_trailing\_underscore\_: use to avoid conflicts with keywords
\item
  \_\_double\_leading\_underscore: used to name a class attribute
\item
  invokes name mangling (inside class Class \_\_par is accessible,
  outside it becomes \_Class\_\_par)
\item
  \_\emph{double\_leading\_and\_trailing\_underscore\_}: ``magic''
  objects or attributes such as \_\emph{init\_}: should be never
  invented by user
\end{itemize}

    \begin{tcolorbox}[breakable, size=fbox, boxrule=1pt, pad at break*=1mm,colback=cellbackground, colframe=cellborder]
\prompt{In}{incolor}{10}{\boxspacing}
\begin{Verbatim}[commandchars=\\\{\}]
\PY{o}{\PYZpc{}}\PY{k}{reset} \PYZhy{}f
\PY{k}{class} \PY{n+nc}{A}\PY{p}{(}\PY{n+nb}{object}\PY{p}{)}\PY{p}{:} \PY{n}{\PYZus{}\PYZus{}id} \PY{o}{=} \PY{l+s+s2}{\PYZdq{}}\PY{l+s+s2}{id of class A}\PY{l+s+s2}{\PYZdq{}} \PY{c+c1}{\PYZsh{} not part of A\PYZsq{}s API}
\PY{k}{class} \PY{n+nc}{B}\PY{p}{(}\PY{n}{A}\PY{p}{)}\PY{p}{:}      \PY{n}{\PYZus{}\PYZus{}id} \PY{o}{=} \PY{l+s+s2}{\PYZdq{}}\PY{l+s+s2}{id of class B}\PY{l+s+s2}{\PYZdq{}} \PY{c+c1}{\PYZsh{} not part of B\PYZsq{}s API, the name is the same for coincidence}
                                        \PY{c+c1}{\PYZsh{} thanks to name mangling, \PYZus{}\PYZus{}id is not overwritten }

\PY{n}{a} \PY{o}{=} \PY{n}{A}\PY{p}{(}\PY{p}{)}
\PY{n}{b} \PY{o}{=} \PY{n}{B}\PY{p}{(}\PY{p}{)}
\PY{n+nb}{print}\PY{p}{(}\PY{l+s+s1}{\PYZsq{}}\PY{l+s+s1}{\PYZus{}\PYZus{}id}\PY{l+s+s1}{\PYZsq{}} \PY{o+ow}{in} \PY{n+nb}{dir}\PY{p}{(}\PY{n}{a}\PY{p}{)}\PY{p}{)}    \PY{c+c1}{\PYZsh{} False \PYZus{}\PYZus{}id is hidden}
\PY{n+nb}{print}\PY{p}{(}\PY{l+s+s1}{\PYZsq{}}\PY{l+s+s1}{\PYZus{}A\PYZus{}\PYZus{}id}\PY{l+s+s1}{\PYZsq{}} \PY{o+ow}{in} \PY{n+nb}{dir}\PY{p}{(}\PY{n}{a}\PY{p}{)}\PY{p}{)}  \PY{c+c1}{\PYZsh{} True  \PYZus{}\PYZus{}id can be exposed through name mangling}

\PY{n+nb}{print}\PY{p}{(}\PY{n}{a}\PY{o}{.}\PY{n}{\PYZus{}A\PYZus{}\PYZus{}id}\PY{p}{)} \PY{c+c1}{\PYZsh{} relative to A}
\PY{n+nb}{print}\PY{p}{(}\PY{n}{b}\PY{o}{.}\PY{n}{\PYZus{}B\PYZus{}\PYZus{}id}\PY{p}{)} \PY{c+c1}{\PYZsh{} relative to B}
\end{Verbatim}
\end{tcolorbox}

    \begin{Verbatim}[commandchars=\\\{\}]
False
True
id of class A
id of class B
    \end{Verbatim}

    \hypertarget{introduction-to-decorators}{%
\subsection{Introduction to
decorators}\label{introduction-to-decorators}}

\begin{itemize}
\tightlist
\item
  functions in python (as we have seen)
\item
  can be assigned to names
\item
  can be defined inside other functions
\item
  can be passed as parameters to other functions
\item
  can return other functions
\item
  decorators are wrappers that modify the behavior of the code (e.g., of
  a function)
\item
  i.e., decorators target an object and increase its original
  functionality, so ``decorating'' it
\item
  a decorator is passed the original object being defined and returns a
  modified object
\end{itemize}

\begin{Shaded}
\begin{Highlighting}[]
\KeywordTok{def}\NormalTok{ myFun():  }
    \CommentTok{\textquotesingle{}\textquotesingle{}\textquotesingle{}whatever behavior\textquotesingle{}\textquotesingle{}\textquotesingle{}}
    \ControlFlowTok{pass}
\KeywordTok{def}\NormalTok{ decoration():  }
    \CommentTok{\textquotesingle{}\textquotesingle{}\textquotesingle{}whatever decoration\textquotesingle{}\textquotesingle{}\textquotesingle{}}
    \ControlFlowTok{pass}
\NormalTok{myFun }\OperatorTok{=}\NormalTok{ decoration(myFun)}
\end{Highlighting}
\end{Shaded}

\hypertarget{the-syntactic-sugar-to-make-decorators-cleaner-to-read-python-use-the-keyword}{%
\paragraph{the ``syntactic sugar'' to make decorators cleaner to read,
python use the keyword
@}\label{the-syntactic-sugar-to-make-decorators-cleaner-to-read-python-use-the-keyword}}

\begin{Shaded}
\begin{Highlighting}[]
\KeywordTok{def}\NormalTok{ decoration():  }
    \CommentTok{\textquotesingle{}\textquotesingle{}\textquotesingle{}whatever decoration\textquotesingle{}\textquotesingle{}\textquotesingle{}}
    \ControlFlowTok{pass}
\AttributeTok{@decoration}
\KeywordTok{def}\NormalTok{ myFun():  }
    \CommentTok{\textquotesingle{}\textquotesingle{}\textquotesingle{}whatever behavior\textquotesingle{}\textquotesingle{}\textquotesingle{}}
    \ControlFlowTok{pass}    
\end{Highlighting}
\end{Shaded}

\hypertarget{multiple-decorations-can-be-chained}{%
\paragraph{multiple decorations can be
chained}\label{multiple-decorations-can-be-chained}}

    \begin{tcolorbox}[breakable, size=fbox, boxrule=1pt, pad at break*=1mm,colback=cellbackground, colframe=cellborder]
\prompt{In}{incolor}{187}{\boxspacing}
\begin{Verbatim}[commandchars=\\\{\}]
\PY{o}{\PYZpc{}}\PY{k}{reset} \PYZhy{}f
\PY{k}{def} \PY{n+nf}{who\PYZus{}just\PYZus{}arrived}\PY{p}{(}\PY{p}{)}\PY{p}{:} \PY{k}{return} \PY{l+s+s2}{\PYZdq{}}\PY{l+s+s2}{jacopo}\PY{l+s+s2}{\PYZdq{}}   \PY{c+c1}{\PYZsh{} this is a function}
\PY{k}{def} \PY{n+nf}{greet}\PY{p}{(}\PY{n}{whoever\PYZus{}I\PYZus{}want\PYZus{}to\PYZus{}greet}\PY{p}{)}\PY{p}{:}       \PY{c+c1}{\PYZsh{} functions can be passed as parameters}
    \PY{k}{def} \PY{n+nf}{hi}\PY{p}{(}\PY{p}{)}\PY{p}{:} \PY{k}{return} \PY{l+s+s2}{\PYZdq{}}\PY{l+s+s2}{hi }\PY{l+s+s2}{\PYZdq{}}                \PY{c+c1}{\PYZsh{} functions can be defined inside other functions}
    \PY{k}{return} \PY{n}{hi}\PY{p}{(}\PY{p}{)}\PY{o}{+}\PY{n}{whoever\PYZus{}I\PYZus{}want\PYZus{}to\PYZus{}greet}\PY{p}{(}\PY{p}{)} \PY{c+c1}{\PYZsh{} functions can return other functions           }

\PY{n}{say\PYZus{}hi\PYZus{}to\PYZus{}} \PY{o}{=} \PY{n}{greet}\PY{p}{(}\PY{n}{who\PYZus{}just\PYZus{}arrived}\PY{p}{)}      \PY{c+c1}{\PYZsh{} functions can be assigned to names}

\PY{n+nb}{print}\PY{p}{(}\PY{n}{say\PYZus{}hi\PYZus{}to\PYZus{}}\PY{p}{)}
\end{Verbatim}
\end{tcolorbox}

    \begin{Verbatim}[commandchars=\\\{\}]
hi jacopo
    \end{Verbatim}

    \begin{tcolorbox}[breakable, size=fbox, boxrule=1pt, pad at break*=1mm,colback=cellbackground, colframe=cellborder]
\prompt{In}{incolor}{19}{\boxspacing}
\begin{Verbatim}[commandchars=\\\{\}]
\PY{o}{\PYZpc{}}\PY{k}{reset} \PYZhy{}f
\PY{k}{def} \PY{n+nf}{decorate}\PY{p}{(}\PY{n}{func}\PY{p}{)}\PY{p}{:}         \PY{c+c1}{\PYZsh{} decoration inputs the function!}
    \PY{k}{def} \PY{n+nf}{func\PYZus{}wrapper}\PY{p}{(}\PY{n}{name}\PY{p}{)}\PY{p}{:} \PY{c+c1}{\PYZsh{} wrappers inputs the function\PYZsq{}s parameter(s)}
        \PY{c+c1}{\PYZsh{} check http://ascii\PYZhy{}table.com/ansi\PYZhy{}escape\PYZhy{}sequences.php}
        \PY{k}{return}   \PY{l+s+s2}{\PYZdq{}}\PY{l+s+se}{\PYZbs{}033}\PY{l+s+s2}{[1m}\PY{l+s+si}{\PYZob{}0\PYZcb{}}\PY{l+s+se}{\PYZbs{}033}\PY{l+s+s2}{[0m}\PY{l+s+s2}{\PYZdq{}}\PY{o}{.}\PY{n}{format}\PY{p}{(}\PY{n}{func}\PY{p}{(}\PY{n}{name}\PY{p}{)}\PY{p}{)}
        \PY{c+c1}{\PYZsh{} equals \PYZdq{}\PYZbs{}033[1m\PYZdq{}+func(name)+\PYZdq{}\PYZbs{}033[0m\PYZdq{} }
    \PY{k}{return} \PY{n}{func\PYZus{}wrapper}

\PY{k}{def} \PY{n+nf}{greet}\PY{p}{(}\PY{n}{name}\PY{p}{)}\PY{p}{:}
    \PY{k}{return} \PY{l+s+s2}{\PYZdq{}}\PY{l+s+s2}{hi }\PY{l+s+si}{\PYZob{}0\PYZcb{}}\PY{l+s+s2}{\PYZdq{}}\PY{o}{.}\PY{n}{format}\PY{p}{(}\PY{n}{name}\PY{p}{)}

\PY{n}{my\PYZus{}greet} \PY{o}{=} \PY{n}{decorate}\PY{p}{(}\PY{n}{greet}\PY{p}{)}
\PY{n+nb}{print}\PY{p}{(}\PY{n}{greet}\PY{p}{(}\PY{l+s+s2}{\PYZdq{}}\PY{l+s+s2}{jacopo}\PY{l+s+s2}{\PYZdq{}}\PY{p}{)}\PY{p}{)}
\PY{n+nb}{print}\PY{p}{(}\PY{n}{my\PYZus{}greet}\PY{p}{(}\PY{l+s+s2}{\PYZdq{}}\PY{l+s+s2}{jacopo}\PY{l+s+s2}{\PYZdq{}}\PY{p}{)}\PY{p}{)}
\end{Verbatim}
\end{tcolorbox}

    \begin{Verbatim}[commandchars=\\\{\}]
hi jacopo
\textbf{hi jacopo}
    \end{Verbatim}

    \begin{tcolorbox}[breakable, size=fbox, boxrule=1pt, pad at break*=1mm,colback=cellbackground, colframe=cellborder]
\prompt{In}{incolor}{13}{\boxspacing}
\begin{Verbatim}[commandchars=\\\{\}]
\PY{o}{\PYZpc{}}\PY{k}{reset} \PYZhy{}f
\PY{k}{def} \PY{n+nf}{decorate}\PY{p}{(}\PY{n}{func}\PY{p}{)}\PY{p}{:}
    \PY{k}{def} \PY{n+nf}{func\PYZus{}wrapper}\PY{p}{(}\PY{n}{name}\PY{p}{)}\PY{p}{:}
        \PY{c+c1}{\PYZsh{} check http://ascii\PYZhy{}table.com/ansi\PYZhy{}escape\PYZhy{}sequences.php}
        \PY{k}{return} \PY{l+s+s2}{\PYZdq{}}\PY{l+s+se}{\PYZbs{}033}\PY{l+s+s2}{[1m}\PY{l+s+si}{\PYZob{}0\PYZcb{}}\PY{l+s+se}{\PYZbs{}033}\PY{l+s+s2}{[0m}\PY{l+s+s2}{\PYZdq{}}\PY{o}{.}\PY{n}{format}\PY{p}{(}\PY{n}{func}\PY{p}{(}\PY{n}{name}\PY{p}{)}\PY{p}{)}
    \PY{k}{return} \PY{n}{func\PYZus{}wrapper}

\PY{n+nd}{@decorate}  
\PY{k}{def} \PY{n+nf}{greet}\PY{p}{(}\PY{n}{name}\PY{p}{)}\PY{p}{:} \PY{c+c1}{\PYZsh{} \PYZhy{}\PYZgt{} greet = decorate(greet)}
    \PY{k}{return} \PY{l+s+s2}{\PYZdq{}}\PY{l+s+s2}{hi }\PY{l+s+si}{\PYZob{}0\PYZcb{}}\PY{l+s+s2}{\PYZdq{}}\PY{o}{.}\PY{n}{format}\PY{p}{(}\PY{n}{name}\PY{p}{)}

\PY{n+nb}{print}\PY{p}{(}\PY{n}{greet}\PY{p}{(}\PY{l+s+s2}{\PYZdq{}}\PY{l+s+s2}{jacopo}\PY{l+s+s2}{\PYZdq{}}\PY{p}{)}\PY{p}{)}
\end{Verbatim}
\end{tcolorbox}

    \begin{Verbatim}[commandchars=\\\{\}]
\textbf{hi jacopo}
    \end{Verbatim}

    \begin{tcolorbox}[breakable, size=fbox, boxrule=1pt, pad at break*=1mm,colback=cellbackground, colframe=cellborder]
\prompt{In}{incolor}{225}{\boxspacing}
\begin{Verbatim}[commandchars=\\\{\}]
\PY{o}{\PYZpc{}}\PY{k}{reset} \PYZhy{}f
\PY{k}{def} \PY{n+nf}{decorate\PYZus{}red}\PY{p}{(}\PY{n}{func}\PY{p}{)}\PY{p}{:} \PY{c+c1}{\PYZsh{} makes greet red}
    \PY{k}{def} \PY{n+nf}{func\PYZus{}wrapper}\PY{p}{(}\PY{n}{name}\PY{p}{)}\PY{p}{:}
        \PY{k}{return} \PY{l+s+s2}{\PYZdq{}}\PY{l+s+se}{\PYZbs{}033}\PY{l+s+s2}{[91m}\PY{l+s+si}{\PYZob{}0\PYZcb{}}\PY{l+s+se}{\PYZbs{}033}\PY{l+s+s2}{[0m}\PY{l+s+s2}{\PYZdq{}}\PY{o}{.}\PY{n}{format}\PY{p}{(}\PY{n}{func}\PY{p}{(}\PY{n}{name}\PY{p}{)}\PY{p}{)}
    \PY{k}{return} \PY{n}{func\PYZus{}wrapper}

\PY{k}{def} \PY{n+nf}{decorate\PYZus{}bold}\PY{p}{(}\PY{n}{func}\PY{p}{)}\PY{p}{:} \PY{c+c1}{\PYZsh{} makes greet bold}
    \PY{k}{def} \PY{n+nf}{func\PYZus{}wrapper}\PY{p}{(}\PY{n}{name}\PY{p}{)}\PY{p}{:}
        \PY{c+c1}{\PYZsh{} check http://ascii\PYZhy{}table.com/ansi\PYZhy{}escape\PYZhy{}sequences.php}
        \PY{k}{return} \PY{l+s+s2}{\PYZdq{}}\PY{l+s+se}{\PYZbs{}033}\PY{l+s+s2}{[1m}\PY{l+s+si}{\PYZob{}0\PYZcb{}}\PY{l+s+se}{\PYZbs{}033}\PY{l+s+s2}{[0m}\PY{l+s+s2}{\PYZdq{}}\PY{o}{.}\PY{n}{format}\PY{p}{(}\PY{n}{func}\PY{p}{(}\PY{n}{name}\PY{p}{)}\PY{p}{)}
    \PY{k}{return} \PY{n}{func\PYZus{}wrapper}

\PY{n+nd}{@decorate\PYZus{}bold}
\PY{n+nd}{@decorate\PYZus{}red}
\PY{k}{def} \PY{n+nf}{greet}\PY{p}{(}\PY{n}{name}\PY{p}{)}\PY{p}{:} \PY{c+c1}{\PYZsh{} \PYZhy{}\PYZgt{} greet = decorate\PYZus{}bold(decorate\PYZus{}red(greet))}
    \PY{k}{return} \PY{l+s+s2}{\PYZdq{}}\PY{l+s+s2}{hi }\PY{l+s+si}{\PYZob{}0\PYZcb{}}\PY{l+s+s2}{\PYZdq{}}\PY{o}{.}\PY{n}{format}\PY{p}{(}\PY{n}{name}\PY{p}{)}

\PY{n+nb}{print}\PY{p}{(}\PY{n}{greet}\PY{p}{(}\PY{l+s+s2}{\PYZdq{}}\PY{l+s+s2}{jacopo}\PY{l+s+s2}{\PYZdq{}}\PY{p}{)}\PY{p}{)}
\end{Verbatim}
\end{tcolorbox}

    \begin{Verbatim}[commandchars=\\\{\}]
\textcolor{ansi-red-intense}{\textbf{hi jacopo}}
    \end{Verbatim}

    \begin{tcolorbox}[breakable, size=fbox, boxrule=1pt, pad at break*=1mm,colback=cellbackground, colframe=cellborder]
\prompt{In}{incolor}{17}{\boxspacing}
\begin{Verbatim}[commandchars=\\\{\}]
\PY{o}{\PYZpc{}}\PY{k}{reset} \PYZhy{}f
\PY{k}{def} \PY{n+nf}{formatting}\PY{p}{(}\PY{n}{how\PYZus{}to\PYZus{}format}\PY{p}{)}\PY{p}{:}
    \PY{k}{def} \PY{n+nf}{etiquette\PYZus{}decorator}\PY{p}{(}\PY{n}{func}\PY{p}{)}\PY{p}{:}
        \PY{k}{def} \PY{n+nf}{func\PYZus{}wrapper}\PY{p}{(}\PY{n}{name}\PY{p}{)}\PY{p}{:} 
            \PY{c+c1}{\PYZsh{} check http://ascii\PYZhy{}table.com/ansi\PYZhy{}escape\PYZhy{}sequences.php}
            \PY{k}{if} \PY{n}{how\PYZus{}to\PYZus{}format}\PY{o}{==}\PY{l+s+s1}{\PYZsq{}}\PY{l+s+s1}{RED}\PY{l+s+s1}{\PYZsq{}}\PY{p}{:} 
                \PY{k}{return}   \PY{l+s+s2}{\PYZdq{}}\PY{l+s+se}{\PYZbs{}033}\PY{l+s+s2}{[91m}\PY{l+s+si}{\PYZob{}0\PYZcb{}}\PY{l+s+se}{\PYZbs{}033}\PY{l+s+s2}{[0m}\PY{l+s+s2}{\PYZdq{}}\PY{o}{.}\PY{n}{format}\PY{p}{(}\PY{n}{func}\PY{p}{(}\PY{n}{name}\PY{p}{)}\PY{p}{)}                
            \PY{k}{if} \PY{n}{how\PYZus{}to\PYZus{}format}\PY{o}{==}\PY{l+s+s1}{\PYZsq{}}\PY{l+s+s1}{BOLD}\PY{l+s+s1}{\PYZsq{}}\PY{p}{:} 
                \PY{k}{return} \PY{l+s+s2}{\PYZdq{}}\PY{l+s+se}{\PYZbs{}033}\PY{l+s+s2}{[1m}\PY{l+s+si}{\PYZob{}0\PYZcb{}}\PY{l+s+se}{\PYZbs{}033}\PY{l+s+s2}{[0m}\PY{l+s+s2}{\PYZdq{}}\PY{o}{.}\PY{n}{format}\PY{p}{(}\PY{n}{func}\PY{p}{(}\PY{n}{name}\PY{p}{)}\PY{p}{)}
            \PY{k}{if} \PY{n}{how\PYZus{}to\PYZus{}format}\PY{o}{==}\PY{l+s+s1}{\PYZsq{}}\PY{l+s+s1}{UNDERLINE}\PY{l+s+s1}{\PYZsq{}}\PY{p}{:} 
                \PY{k}{return} \PY{l+s+s2}{\PYZdq{}}\PY{l+s+se}{\PYZbs{}033}\PY{l+s+s2}{[4m}\PY{l+s+si}{\PYZob{}0\PYZcb{}}\PY{l+s+se}{\PYZbs{}033}\PY{l+s+s2}{[0m}\PY{l+s+s2}{\PYZdq{}}\PY{o}{.}\PY{n}{format}\PY{p}{(}\PY{n}{func}\PY{p}{(}\PY{n}{name}\PY{p}{)}\PY{p}{)}            
        \PY{k}{return} \PY{n}{func\PYZus{}wrapper}
    \PY{k}{return} \PY{n}{etiquette\PYZus{}decorator}

\PY{n+nd}{@formatting}\PY{p}{(}\PY{l+s+s2}{\PYZdq{}}\PY{l+s+s2}{RED}\PY{l+s+s2}{\PYZdq{}}\PY{p}{)}        \PY{c+c1}{\PYZsh{} you can pass arguments to decorators functions}
\PY{n+nd}{@formatting}\PY{p}{(}\PY{l+s+s2}{\PYZdq{}}\PY{l+s+s2}{BOLD}\PY{l+s+s2}{\PYZdq{}}\PY{p}{)}
\PY{n+nd}{@formatting}\PY{p}{(}\PY{l+s+s2}{\PYZdq{}}\PY{l+s+s2}{UNDERLINE}\PY{l+s+s2}{\PYZdq{}}\PY{p}{)}
\PY{k}{def} \PY{n+nf}{greet}\PY{p}{(}\PY{n}{name}\PY{p}{)}\PY{p}{:}
    \PY{k}{return} \PY{l+s+s2}{\PYZdq{}}\PY{l+s+s2}{hi }\PY{l+s+si}{\PYZob{}0\PYZcb{}}\PY{l+s+s2}{\PYZdq{}}\PY{o}{.}\PY{n}{format}\PY{p}{(}\PY{n}{name}\PY{p}{)}

\PY{n+nb}{print}\PY{p}{(}\PY{n}{greet}\PY{p}{(}\PY{l+s+s1}{\PYZsq{}}\PY{l+s+s1}{jacopo}\PY{l+s+s1}{\PYZsq{}}\PY{p}{)}\PY{p}{)}
\end{Verbatim}
\end{tcolorbox}

    \begin{Verbatim}[commandchars=\\\{\}]
\textcolor{ansi-red-intense}{\textbf{\underline{hi jacopo}}}
    \end{Verbatim}

    \hypertarget{data-encapsulation}{%
\subsection{Data encapsulation}\label{data-encapsulation}}

\begin{itemize}
\tightlist
\item
  many object-oriented languages stress to have getter, setter and
  deleter methods to access private attributes
\item
  this is called data encapsulation
\item
  it does not hide attributes (nothing in python can do that)
\item
  code loses readability: why should data be encapsulated?
\item
  to isolate code changes that depends on a specific attribute's value
\end{itemize}

    \begin{tcolorbox}[breakable, size=fbox, boxrule=1pt, pad at break*=1mm,colback=cellbackground, colframe=cellborder]
\prompt{In}{incolor}{61}{\boxspacing}
\begin{Verbatim}[commandchars=\\\{\}]
\PY{o}{\PYZpc{}}\PY{k}{reset} \PYZhy{}f 
\PY{k}{class} \PY{n+nc}{Fruit}\PY{p}{(}\PY{n+nb}{object}\PY{p}{)}\PY{p}{:}  \PY{c+c1}{\PYZsh{} implementation using setter,    }
                      \PY{c+c1}{\PYZsh{}                      getter, }
                      \PY{c+c1}{\PYZsh{}                      deleter}
    \PY{k}{def} \PY{n+nf+fm}{\PYZus{}\PYZus{}init\PYZus{}\PYZus{}}\PY{p}{(}\PY{n+nb+bp}{self}\PY{p}{,}\PY{n}{name}\PY{p}{)}\PY{p}{:} 
        \PY{n+nb+bp}{self}\PY{o}{.}\PY{n}{\PYZus{}set\PYZus{}name}\PY{p}{(}\PY{n}{name}\PY{p}{)}
    \PY{k}{def} \PY{n+nf}{\PYZus{}set\PYZus{}name}\PY{p}{(}\PY{n+nb+bp}{self}\PY{p}{,}\PY{n}{name}\PY{p}{)}\PY{p}{:}     \PY{c+c1}{\PYZsh{} setter}
        \PY{n+nb+bp}{self}\PY{o}{.}\PY{n}{\PYZus{}name} \PY{o}{=} \PY{n}{name}
    \PY{k}{def} \PY{n+nf}{\PYZus{}get\PYZus{}name}\PY{p}{(}\PY{n+nb+bp}{self}\PY{p}{)}\PY{p}{:}          \PY{c+c1}{\PYZsh{} getter}
        \PY{k}{return} \PY{n+nb+bp}{self}\PY{o}{.}\PY{n}{\PYZus{}name}
    \PY{k}{def} \PY{n+nf}{\PYZus{}del\PYZus{}name}\PY{p}{(}\PY{n+nb+bp}{self}\PY{p}{)}\PY{p}{:}          \PY{c+c1}{\PYZsh{} deleter}
        \PY{k}{del} \PY{n+nb+bp}{self}\PY{o}{.}\PY{n}{\PYZus{}colour}                        
    \PY{k}{def} \PY{n+nf+fm}{\PYZus{}\PYZus{}str\PYZus{}\PYZus{}}\PY{p}{(}\PY{n+nb+bp}{self}\PY{p}{)}\PY{p}{:} 
        \PY{k}{return} \PY{l+s+s2}{\PYZdq{}}\PY{l+s+s2}{this fruit is an }\PY{l+s+s2}{\PYZdq{}}\PY{o}{+}\PY{n+nb+bp}{self}\PY{o}{.}\PY{n}{\PYZus{}name}

\PY{n+nb}{print}\PY{p}{(}\PY{n}{Fruit}\PY{p}{(}\PY{l+s+s1}{\PYZsq{}}\PY{l+s+s1}{apple}\PY{l+s+s1}{\PYZsq{}}\PY{p}{)}\PY{p}{)}
\end{Verbatim}
\end{tcolorbox}

    \begin{Verbatim}[commandchars=\\\{\}]
this fruit is an apple
    \end{Verbatim}

    \hypertarget{using-property-to-encapsulate-data}{%
\subsection{Using property() to encapsulate
data}\label{using-property-to-encapsulate-data}}

\begin{itemize}
\tightlist
\item
  property(getter,setter{[},deleter,docstring{]}):
\item
  built-in function
\item
  proxy any attribute access to getter, setter, deleter and docstring
\item
  properties can embed customized actions when they are accessed
\end{itemize}

    \begin{tcolorbox}[breakable, size=fbox, boxrule=1pt, pad at break*=1mm,colback=cellbackground, colframe=cellborder]
\prompt{In}{incolor}{20}{\boxspacing}
\begin{Verbatim}[commandchars=\\\{\}]
\PY{o}{\PYZpc{}}\PY{k}{reset} \PYZhy{}f 
\PY{k}{class} \PY{n+nc}{Fruit}\PY{p}{(}\PY{n+nb}{object}\PY{p}{)}\PY{p}{:}
    \PY{n}{\PYZus{}name}\PY{o}{=}\PY{l+s+s1}{\PYZsq{}}\PY{l+s+s1}{\PYZsq{}}
    \PY{k}{def} \PY{n+nf+fm}{\PYZus{}\PYZus{}init\PYZus{}\PYZus{}}\PY{p}{(}\PY{n+nb+bp}{self}\PY{p}{,}\PY{n}{name}\PY{p}{)}\PY{p}{:} 
        \PY{n+nb+bp}{self}\PY{o}{.}\PY{n}{\PYZus{}set\PYZus{}name}\PY{p}{(}\PY{n}{name}\PY{p}{)}
    \PY{k}{def} \PY{n+nf}{\PYZus{}set\PYZus{}name}\PY{p}{(}\PY{n+nb+bp}{self}\PY{p}{,}\PY{n}{name}\PY{p}{)}\PY{p}{:}        
        \PY{k}{if} \PY{n+nb+bp}{self}\PY{o}{.}\PY{n}{\PYZus{}name} \PY{o}{==} \PY{l+s+s1}{\PYZsq{}}\PY{l+s+s1}{\PYZsq{}}\PY{p}{:} 
            \PY{n+nb+bp}{self}\PY{o}{.}\PY{n}{\PYZus{}name} \PY{o}{=} \PY{n}{name}
        \PY{k}{else}\PY{p}{:} 
            \PY{k}{raise} \PY{n+ne}{TypeError}\PY{p}{(}\PY{l+s+s1}{\PYZsq{}}\PY{l+s+s1}{Fruit name cannot be changed once set}\PY{l+s+s1}{\PYZsq{}}\PY{p}{)}
    \PY{k}{def} \PY{n+nf}{\PYZus{}get\PYZus{}name}\PY{p}{(}\PY{n+nb+bp}{self}\PY{p}{)}\PY{p}{:}          
        \PY{k}{return} \PY{n+nb+bp}{self}\PY{o}{.}\PY{n}{\PYZus{}name}
    \PY{k}{def} \PY{n+nf}{\PYZus{}del\PYZus{}name}\PY{p}{(}\PY{n+nb+bp}{self}\PY{p}{)}\PY{p}{:}          
        \PY{k}{del} \PY{n+nb+bp}{self}\PY{o}{.}\PY{n}{\PYZus{}name}
    \PY{k}{def} \PY{n+nf+fm}{\PYZus{}\PYZus{}str\PYZus{}\PYZus{}}\PY{p}{(}\PY{n+nb+bp}{self}\PY{p}{)}\PY{p}{:} 
        \PY{k}{return} \PY{l+s+s2}{\PYZdq{}}\PY{l+s+s2}{this fruit is an }\PY{l+s+s2}{\PYZdq{}}\PY{o}{+}\PY{n+nb+bp}{self}\PY{o}{.}\PY{n}{\PYZus{}name}
        
    \PY{c+c1}{\PYZsh{} assign properties}
    \PY{n}{name} \PY{o}{=} \PY{n+nb}{property}\PY{p}{(}\PY{n}{\PYZus{}get\PYZus{}name}\PY{p}{,}\PYZbs{}
                    \PY{n}{\PYZus{}set\PYZus{}name}\PY{p}{,}\PYZbs{}
                    \PY{n}{\PYZus{}del\PYZus{}name}\PY{p}{,}\PYZbs{}
                    \PY{l+s+s1}{\PYZsq{}}\PY{l+s+s1}{name is property of Fruit}\PY{l+s+s1}{\PYZsq{}}\PY{p}{)}
    

\PY{n}{f} \PY{o}{=} \PY{n}{Fruit}\PY{p}{(}\PY{l+s+s1}{\PYZsq{}}\PY{l+s+s1}{apple}\PY{l+s+s1}{\PYZsq{}}\PY{p}{)}
\PY{n+nb}{print}\PY{p}{(}\PY{n}{f}\PY{p}{)}
\PY{k}{try}\PY{p}{:} 
    \PY{n}{f}\PY{o}{.}\PY{n}{name} \PY{o}{=} \PY{l+s+s1}{\PYZsq{}}\PY{l+s+s1}{pear}\PY{l+s+s1}{\PYZsq{}}     
\PY{k}{except} \PY{n+ne}{TypeError} \PY{k}{as} \PY{n}{err}\PY{p}{:} 
    \PY{n+nb}{print}\PY{p}{(}\PY{l+s+s1}{\PYZsq{}}\PY{l+s+s1}{ \PYZhy{}\PYZgt{} err: }\PY{l+s+s1}{\PYZsq{}}\PY{p}{,} \PY{n}{err}\PY{p}{)}
\PY{n+nb}{print}\PY{p}{(}\PY{n}{f}\PY{o}{.}\PY{n}{name}\PY{p}{)}
\end{Verbatim}
\end{tcolorbox}

    \begin{Verbatim}[commandchars=\\\{\}]
this fruit is an apple
 -> err:  Fruit name cannot be changed once set
apple
    \end{Verbatim}

    \hypertarget{property-decorations}{%
\subsection{Property decorations}\label{property-decorations}}

\begin{itemize}
\tightlist
\item
  you can use decorators to invoke property()
\item
  you mark a function instead of call property() at the end of the class
  definition
\item
  you can avoid underscore prefixes
\item
  (more readable)
\item
  @property decorates the getter of the attribute \$foo
\item
  @\$foo.setter decorates the setter
\item
  @\$foo.deleter decorates the deleter
\end{itemize}

    \begin{tcolorbox}[breakable, size=fbox, boxrule=1pt, pad at break*=1mm,colback=cellbackground, colframe=cellborder]
\prompt{In}{incolor}{22}{\boxspacing}
\begin{Verbatim}[commandchars=\\\{\}]
\PY{o}{\PYZpc{}}\PY{k}{reset} \PYZhy{}f 
\PY{k}{class} \PY{n+nc}{Fruit}\PY{p}{(}\PY{n+nb}{object}\PY{p}{)}\PY{p}{:} \PY{c+c1}{\PYZsh{} re\PYZhy{}implementation using property decorators}
    \PY{k}{def} \PY{n+nf+fm}{\PYZus{}\PYZus{}init\PYZus{}\PYZus{}}\PY{p}{(}\PY{n+nb+bp}{self}\PY{p}{,}\PY{n}{name}\PY{o}{=}\PY{k+kc}{None}\PY{p}{)}\PY{p}{:} 
        \PY{n+nb+bp}{self}\PY{o}{.}\PY{n}{\PYZus{}name} \PY{o}{=} \PY{n}{name}
    \PY{n+nd}{@property}    
    \PY{k}{def} \PY{n+nf}{name}\PY{p}{(}\PY{n+nb+bp}{self}\PY{p}{)}\PY{p}{:}     
        \PY{l+s+sd}{\PYZsq{}\PYZsq{}\PYZsq{}name is property of Fruit\PYZsq{}\PYZsq{}\PYZsq{}}
        \PY{k}{return} \PY{n+nb+bp}{self}\PY{o}{.}\PY{n}{\PYZus{}name}
    \PY{n+nd}{@name}\PY{o}{.}\PY{n}{setter}
    \PY{k}{def} \PY{n+nf}{name}\PY{p}{(}\PY{n+nb+bp}{self}\PY{p}{,}\PY{n}{name}\PY{p}{)}\PY{p}{:}
        \PY{k}{if} \PY{o+ow}{not} \PY{n+nb+bp}{self}\PY{o}{.}\PY{n}{\PYZus{}name}\PY{p}{:} 
            \PY{n+nb+bp}{self}\PY{o}{.}\PY{n}{\PYZus{}name} \PY{o}{=} \PY{n}{name}
        \PY{k}{else}\PY{p}{:} 
            \PY{k}{raise} \PY{n+ne}{TypeError}\PY{p}{(}\PY{l+s+s1}{\PYZsq{}}\PY{l+s+s1}{Fruit name cannot be changed once set}\PY{l+s+s1}{\PYZsq{}}\PY{p}{)}        
    \PY{n+nd}{@name}\PY{o}{.}\PY{n}{deleter}
    \PY{k}{def} \PY{n+nf}{name}\PY{p}{(}\PY{n+nb+bp}{self}\PY{p}{)}\PY{p}{:}          
        \PY{k}{del} \PY{n+nb+bp}{self}\PY{o}{.}\PY{n}{\PYZus{}name}
    \PY{k}{def} \PY{n+nf+fm}{\PYZus{}\PYZus{}str\PYZus{}\PYZus{}}\PY{p}{(}\PY{n+nb+bp}{self}\PY{p}{)}\PY{p}{:} 
        \PY{k}{return} \PY{l+s+s2}{\PYZdq{}}\PY{l+s+s2}{this fruit is an }\PY{l+s+s2}{\PYZdq{}}\PY{o}{+}\PY{n+nb+bp}{self}\PY{o}{.}\PY{n}{\PYZus{}name}        

\PY{n}{f} \PY{o}{=} \PY{n}{Fruit}\PY{p}{(}\PY{l+s+s1}{\PYZsq{}}\PY{l+s+s1}{apple}\PY{l+s+s1}{\PYZsq{}}\PY{p}{)}
\PY{n+nb}{print}\PY{p}{(}\PY{n}{f}\PY{p}{)}
\PY{k}{try}\PY{p}{:} 
    \PY{n}{f}\PY{o}{.}\PY{n}{name} \PY{o}{=} \PY{l+s+s1}{\PYZsq{}}\PY{l+s+s1}{pear}\PY{l+s+s1}{\PYZsq{}}
\PY{k}{except} \PY{n+ne}{TypeError} \PY{k}{as} \PY{n}{err}\PY{p}{:} 
    \PY{n+nb}{print}\PY{p}{(}\PY{l+s+s1}{\PYZsq{}}\PY{l+s+s1}{ \PYZhy{}\PYZgt{} err: }\PY{l+s+s1}{\PYZsq{}}\PY{p}{,} \PY{n}{err}\PY{p}{)}
\PY{n+nb}{print}\PY{p}{(}\PY{n}{f}\PY{o}{.}\PY{n}{name}\PY{p}{)}
\end{Verbatim}
\end{tcolorbox}

    \begin{Verbatim}[commandchars=\\\{\}]
this fruit is an apple
 -> err:  Fruit name cannot be changed once set
apple
    \end{Verbatim}

    \hypertarget{why-properties-are-useful}{%
\subsection{Why properties are useful}\label{why-properties-are-useful}}

\begin{itemize}
\tightlist
\item
  methods should implement an action
\item
  (commonly named with verbs)
\item
  data attributes do not not implement an action
\item
  (commonly named with a noun)
\item
  properties differs from data attributes because\ldots{}
\item
  they can embed customized actions when they are accessed
\end{itemize}

    \begin{tcolorbox}[breakable, size=fbox, boxrule=1pt, pad at break*=1mm,colback=cellbackground, colframe=cellborder]
\prompt{In}{incolor}{26}{\boxspacing}
\begin{Verbatim}[commandchars=\\\{\}]
\PY{o}{\PYZpc{}}\PY{k}{reset} \PYZhy{}f
\PY{k}{class} \PY{n+nc}{Values}\PY{p}{(}\PY{n+nb}{list}\PY{p}{)}\PY{p}{:}     
    \PY{n}{\PYZus{}positive}\PY{o}{=}\PY{p}{[}\PY{p}{]}
    \PY{k}{def} \PY{n+nf+fm}{\PYZus{}\PYZus{}init\PYZus{}\PYZus{}}\PY{p}{(}\PY{n+nb+bp}{self}\PY{p}{,}\PY{n}{data}\PY{o}{=}\PY{k+kc}{None}\PY{p}{)}\PY{p}{:} 
        \PY{n+nb+bp}{self}\PY{o}{.}\PY{n}{\PYZus{}data}    \PY{o}{=} \PY{n}{data}   
        \PY{n+nb+bp}{self}\PY{o}{.}\PY{n}{positive} \PY{o}{=} \PY{n}{data} \PY{c+c1}{\PYZsh{} calls the setter}
    \PY{n+nd}{@property}
    \PY{k}{def} \PY{n+nf}{positive}\PY{p}{(}\PY{n+nb+bp}{self}\PY{p}{)}\PY{p}{:} 
        \PY{n+nb}{print}\PY{p}{(}\PY{l+s+s2}{\PYZdq{}}\PY{l+s+s2}{getting self.\PYZus{}positive}\PY{l+s+s2}{\PYZdq{}}\PY{p}{)}
        \PY{k}{return} \PY{n+nb+bp}{self}\PY{o}{.}\PY{n}{\PYZus{}positive}
    \PY{n+nd}{@positive}\PY{o}{.}\PY{n}{setter}
    \PY{k}{def} \PY{n+nf}{positive}\PY{p}{(}\PY{n+nb+bp}{self}\PY{p}{,}\PY{n}{data}\PY{p}{)}\PY{p}{:} 
        \PY{n+nb}{print}\PY{p}{(}\PY{l+s+s2}{\PYZdq{}}\PY{l+s+s2}{setting self.\PYZus{}positive}\PY{l+s+s2}{\PYZdq{}}\PY{p}{)}
        \PY{k}{if} \PY{o+ow}{not} \PY{n+nb+bp}{self}\PY{o}{.}\PY{n}{\PYZus{}positive}\PY{p}{:}  \PY{c+c1}{\PYZsh{} prevents the use of this setter outside of this class}
            \PY{p}{[}\PY{n+nb+bp}{self}\PY{o}{.}\PY{n}{\PYZus{}positive}\PY{o}{.}\PY{n}{append}\PY{p}{(}\PY{n}{d}\PY{p}{)} \PY{k}{for} \PY{n}{d} \PY{o+ow}{in} \PY{n}{data} \PY{k}{if} \PY{n}{d}\PY{o}{\PYZgt{}}\PY{l+m+mi}{0}\PY{p}{]}
        
\PY{n}{v} \PY{o}{=} \PY{n}{Values}\PY{p}{(}\PY{p}{[}\PY{o}{\PYZhy{}}\PY{l+m+mi}{1}\PY{p}{,}\PY{l+m+mi}{1}\PY{p}{,}\PY{o}{\PYZhy{}}\PY{l+m+mi}{2}\PY{p}{,}\PY{l+m+mi}{2}\PY{p}{,}\PY{o}{\PYZhy{}}\PY{l+m+mi}{3}\PY{p}{,}\PY{l+m+mi}{3}\PY{p}{]} \PY{p}{)}  
\PY{k}{for} \PY{n}{i} \PY{o+ow}{in} \PY{n+nb}{range}\PY{p}{(}\PY{l+m+mi}{3}\PY{p}{)}\PY{p}{:}
    \PY{n+nb}{print}\PY{p}{(}\PY{n}{v}\PY{o}{.}\PY{n}{positive}\PY{p}{)}  \PY{c+c1}{\PYZsh{} calls the setter only the first time, then only the getter}
\end{Verbatim}
\end{tcolorbox}

    \begin{Verbatim}[commandchars=\\\{\}]
setting self.\_positive
getting self.\_positive
[1, 2, 3]
getting self.\_positive
[1, 2, 3]
getting self.\_positive
[1, 2, 3]
    \end{Verbatim}

    \hypertarget{inheritance}{%
\subsection{Inheritance}\label{inheritance}}

\begin{itemize}
\tightlist
\item
  a new class can inherit attributes from one or more ``base'' classes
\item
  this new class is called derived class
\item
  the base class must be in a scope containing the derived class
\item
  if a requested attribute is not found in the class, the search
  proceeds to look in the base class\\
\item
  this rule is applied recursively if base classe is derived itself
\item
  a derived class can override methods of a base class
\item
  a method of a base class may end up calling a method of a derived
  class
\item
  this behaviour is equivalent to C++ virtual classes
\item
  the overriden base class method can be invoked explicitly
  (baseClass.methosName(self,args))
\item
  sometimes overriden methods have a lot in commons with the relative
  base definition
\item
  super() is a built-in function that can be used to extend the base
  class method
\end{itemize}

\hypertarget{isinstanceinstclass-true-if-inst-is-an-instance-of-of-class-or-derived}{%
\paragraph{isinstance(inst,class): True if inst is an instance of of
class or
derived}\label{isinstanceinstclass-true-if-inst-is-an-instance-of-of-class-or-derived}}

\hypertarget{issubclasssubclassclass-true-if-subclass-derived-from-class}{%
\paragraph{issubclass(subClass,class): True if subClass derived from
class}\label{issubclasssubclassclass-true-if-subclass-derived-from-class}}

    \begin{tcolorbox}[breakable, size=fbox, boxrule=1pt, pad at break*=1mm,colback=cellbackground, colframe=cellborder]
\prompt{In}{incolor}{2}{\boxspacing}
\begin{Verbatim}[commandchars=\\\{\}]
\PY{o}{\PYZpc{}}\PY{k}{reset} \PYZhy{}f
\PY{k}{class} \PY{n+nc}{A}\PY{p}{(}\PY{n+nb}{object}\PY{p}{)}\PY{p}{:}
    \PY{k}{def} \PY{n+nf}{f}\PY{p}{(}\PY{n+nb+bp}{self}\PY{p}{)}\PY{p}{:} \PY{k}{return} \PY{n+nb+bp}{self}\PY{o}{.}\PY{n}{f0}\PY{p}{(}\PY{p}{)} \PY{c+c1}{\PYZsh{} calls A.f0()}
    \PY{k}{def} \PY{n+nf}{f0}\PY{p}{(}\PY{n+nb+bp}{self}\PY{p}{)}\PY{p}{:} \PY{k}{return} \PY{l+s+s2}{\PYZdq{}}\PY{l+s+s2}{f0 is a method of class A}\PY{l+s+s2}{\PYZdq{}}
\PY{k}{class} \PY{n+nc}{B}\PY{p}{(}\PY{n}{A}\PY{p}{)}\PY{p}{:} 
    \PY{k}{def} \PY{n+nf}{f0}\PY{p}{(}\PY{n+nb+bp}{self}\PY{p}{)}\PY{p}{:} \PY{k}{return} \PY{l+s+s2}{\PYZdq{}}\PY{l+s+s2}{f0 is a method of class B}\PY{l+s+s2}{\PYZdq{}}

\PY{n}{a} \PY{o}{=} \PY{n}{A}\PY{p}{(}\PY{p}{)}    
\PY{n}{b} \PY{o}{=} \PY{n}{B}\PY{p}{(}\PY{p}{)}
\PY{n+nb}{print}\PY{p}{(}\PY{n}{a}\PY{o}{.}\PY{n}{f}\PY{p}{(}\PY{p}{)}\PY{p}{)} 
\PY{n+nb}{print}\PY{p}{(}\PY{n}{b}\PY{o}{.}\PY{n}{f}\PY{p}{(}\PY{p}{)}\PY{p}{)} \PY{c+c1}{\PYZsh{} b.f()\PYZhy{}\PYZgt{}a.f()\PYZhy{}\PYZgt{}(self).f0() == b.f0()}
             \PY{c+c1}{\PYZsh{} b inherits b from class A and calls A.f0()}
             \PY{c+c1}{\PYZsh{} however, A.f0() was overriden by B.f0()! }
\end{Verbatim}
\end{tcolorbox}

    \begin{Verbatim}[commandchars=\\\{\}]
f0 is a method of class A
f0 is a method of class B
    \end{Verbatim}

    \begin{tcolorbox}[breakable, size=fbox, boxrule=1pt, pad at break*=1mm,colback=cellbackground, colframe=cellborder]
\prompt{In}{incolor}{7}{\boxspacing}
\begin{Verbatim}[commandchars=\\\{\}]
\PY{o}{\PYZpc{}}\PY{k}{reset} \PYZhy{}f
\PY{k}{class} \PY{n+nc}{A}\PY{p}{(}\PY{n+nb}{object}\PY{p}{)}\PY{p}{:}
    \PY{k}{def} \PY{n+nf}{f}\PY{p}{(}\PY{n+nb+bp}{self}\PY{p}{)}\PY{p}{:} \PY{k}{pass}
\PY{k}{class} \PY{n+nc}{B}\PY{p}{(}\PY{n}{A}\PY{p}{)}\PY{p}{:}       \PY{c+c1}{\PYZsh{} inherits from A}
    \PY{k}{pass}
\PY{k}{class} \PY{n+nc}{C}\PY{p}{(}\PY{n+nb}{object}\PY{p}{)}\PY{p}{:}  \PY{c+c1}{\PYZsh{} does not inherit from A}
    \PY{k}{pass}
\PY{k}{class} \PY{n+nc}{D}\PY{p}{(}\PY{n}{B}\PY{p}{)}\PY{p}{:}       \PY{c+c1}{\PYZsh{} inherits from B, which inherits from A}
    \PY{k}{pass}

\PY{n+nb}{print}\PY{p}{(}\PY{n+nb}{issubclass}\PY{p}{(}\PY{n}{B}\PY{p}{,}\PY{n}{A}\PY{p}{)}\PY{p}{)} \PY{c+c1}{\PYZsh{} True}
\PY{n+nb}{print}\PY{p}{(}\PY{n+nb}{issubclass}\PY{p}{(}\PY{n}{C}\PY{p}{,}\PY{n}{A}\PY{p}{)}\PY{p}{)} \PY{c+c1}{\PYZsh{} False}
\PY{n+nb}{print}\PY{p}{(}\PY{n+nb}{issubclass}\PY{p}{(}\PY{n}{D}\PY{p}{,}\PY{n}{A}\PY{p}{)}\PY{p}{)} \PY{c+c1}{\PYZsh{} True}
\end{Verbatim}
\end{tcolorbox}

    \begin{Verbatim}[commandchars=\\\{\}]
True
False
True
    \end{Verbatim}

    \begin{tcolorbox}[breakable, size=fbox, boxrule=1pt, pad at break*=1mm,colback=cellbackground, colframe=cellborder]
\prompt{In}{incolor}{8}{\boxspacing}
\begin{Verbatim}[commandchars=\\\{\}]
\PY{n}{a} \PY{o}{=} \PY{n}{A}\PY{p}{(}\PY{p}{)}
\PY{n}{b} \PY{o}{=} \PY{n}{B}\PY{p}{(}\PY{p}{)}   \PY{c+c1}{\PYZsh{} B inherits from A}

\PY{n+nb}{print}\PY{p}{(}\PY{n+nb}{isinstance}\PY{p}{(}\PY{n}{b}\PY{p}{,}\PY{n}{B}\PY{p}{)}\PY{p}{)} \PY{c+c1}{\PYZsh{} True }
\PY{n+nb}{print}\PY{p}{(}\PY{n+nb}{isinstance}\PY{p}{(}\PY{n}{b}\PY{p}{,}\PY{n}{A}\PY{p}{)}\PY{p}{)} \PY{c+c1}{\PYZsh{} True }
\PY{n+nb}{print}\PY{p}{(}\PY{n+nb}{isinstance}\PY{p}{(}\PY{n}{a}\PY{p}{,}\PY{n}{B}\PY{p}{)}\PY{p}{)} \PY{c+c1}{\PYZsh{} False }
\end{Verbatim}
\end{tcolorbox}

    \begin{Verbatim}[commandchars=\\\{\}]
True
True
False
    \end{Verbatim}

    \hypertarget{has-a-is-a}{%
\subsection{HAS A != IS A}\label{has-a-is-a}}

\begin{itemize}
\tightlist
\item
  a class HAS attributes that define a specific behavior
\item
  DO NOT USE INHERITANCE JUST TO REUSE IMPLEMENTATION OF A METHOD OR AN
  ATTRIBUTE
\item
  conceptually Inheritance implements IS A
\item
  Inheritance should never be used to implement HAS A
\item
  the derived class IS the same kind of the base class with additional
  features
\end{itemize}

    \hypertarget{multiple-inheritance}{%
\subsection{Multiple inheritance}\label{multiple-inheritance}}

\begin{itemize}
\tightlist
\item
  a class can be derived by multiple base class
\item
  search for attributes is depth-first, left-to-right
\item
  Method Resolution Order algorithm
\item
  base class should be accessed only once, otherwise TypeError is raised
\item
  type(obj).mro(): returns the order in which the methods accessed by
  obj are searched
\end{itemize}

    \begin{tcolorbox}[breakable, size=fbox, boxrule=1pt, pad at break*=1mm,colback=cellbackground, colframe=cellborder]
\prompt{In}{incolor}{36}{\boxspacing}
\begin{Verbatim}[commandchars=\\\{\}]
\PY{o}{\PYZpc{}}\PY{k}{reset} \PYZhy{}f
\PY{k}{class} \PY{n+nc}{C}\PY{p}{(}\PY{n+nb}{object}\PY{p}{)}\PY{p}{:} 
    \PY{k}{def} \PY{n+nf}{f\PYZus{}c}\PY{p}{(}\PY{n+nb+bp}{self}\PY{p}{)}\PY{p}{:} \PY{k}{return} \PY{l+s+s2}{\PYZdq{}}\PY{l+s+s2}{C has f\PYZus{}c}\PY{l+s+s2}{\PYZdq{}}
\PY{k}{class} \PY{n+nc}{B}\PY{p}{(}\PY{n+nb}{object}\PY{p}{)}\PY{p}{:} 
    \PY{k}{def} \PY{n+nf}{f\PYZus{}b}\PY{p}{(}\PY{n+nb+bp}{self}\PY{p}{)}\PY{p}{:} \PY{k}{return} \PY{l+s+s2}{\PYZdq{}}\PY{l+s+s2}{B has f\PYZus{}b}\PY{l+s+s2}{\PYZdq{}}
\PY{k}{class} \PY{n+nc}{A}\PY{p}{(}\PY{n}{B}\PY{p}{,}\PY{n}{C}\PY{p}{)}\PY{p}{:} 
    \PY{k}{def} \PY{n+nf+fm}{\PYZus{}\PYZus{}str\PYZus{}\PYZus{}}\PY{p}{(}\PY{n+nb+bp}{self}\PY{p}{)}\PY{p}{:} \PY{k}{return} \PY{l+s+s2}{\PYZdq{}}\PY{l+s+s2}{A IS both B and C}\PY{l+s+s2}{\PYZdq{}}
    \PY{k}{pass}

\PY{n}{a} \PY{o}{=} \PY{n}{A}\PY{p}{(}\PY{p}{)} 
\PY{n+nb}{print}\PY{p}{(}\PY{n}{a}\PY{p}{)}
\PY{n+nb}{print}\PY{p}{(}\PY{n}{a}\PY{o}{.}\PY{n}{f\PYZus{}b}\PY{p}{(}\PY{p}{)}\PY{p}{)} \PY{c+c1}{\PYZsh{} inherited from B}
\PY{n+nb}{print}\PY{p}{(}\PY{n}{a}\PY{o}{.}\PY{n}{f\PYZus{}c}\PY{p}{(}\PY{p}{)}\PY{p}{)} \PY{c+c1}{\PYZsh{} inherited from C}
\end{Verbatim}
\end{tcolorbox}

    \begin{Verbatim}[commandchars=\\\{\}]
A IS both B and C
B has f\_b
C has f\_c
    \end{Verbatim}

    \hypertarget{super-and-method-resolution-order}{%
\subsection{super() and Method Resolution
Order}\label{super-and-method-resolution-order}}

\begin{itemize}
\tightlist
\item
  super() means super(NextBaseClass,self)
\item
  MRO rules what means ``next''
\item
  diamond structure: what is method order?
\end{itemize}

\begin{Shaded}
\begin{Highlighting}[]
\NormalTok{|               A.f()}
\NormalTok{|              / \textbackslash{}}
\NormalTok{|             /   \textbackslash{}}
\NormalTok{|            /     \textbackslash{}}
\NormalTok{|           B.f()   C.f()}
\NormalTok{|            \textbackslash{}     / }
\NormalTok{|             \textbackslash{}   /  }
\NormalTok{|              \textbackslash{} /   }
\NormalTok{|               D.f()}
\end{Highlighting}
\end{Shaded}

    \begin{tcolorbox}[breakable, size=fbox, boxrule=1pt, pad at break*=1mm,colback=cellbackground, colframe=cellborder]
\prompt{In}{incolor}{64}{\boxspacing}
\begin{Verbatim}[commandchars=\\\{\}]
\PY{o}{\PYZpc{}}\PY{k}{reset} \PYZhy{}f
\PY{k}{class} \PY{n+nc}{D}\PY{p}{(}\PY{n+nb}{object}\PY{p}{)}\PY{p}{:} 
    \PY{k}{def} \PY{n+nf}{f}\PY{p}{(}\PY{n+nb+bp}{self}\PY{p}{)}\PY{p}{:} \PY{k}{return} \PY{l+s+s2}{\PYZdq{}}\PY{l+s+s2}{D}\PY{l+s+s2}{\PYZdq{}}
\PY{k}{class} \PY{n+nc}{C}\PY{p}{(}\PY{n}{D}\PY{p}{)}\PY{p}{:} 
    \PY{k}{def} \PY{n+nf}{f}\PY{p}{(}\PY{n+nb+bp}{self}\PY{p}{)}\PY{p}{:} \PY{k}{return} \PY{l+s+s2}{\PYZdq{}}\PY{l+s+s2}{C}\PY{l+s+s2}{\PYZdq{}}    
\PY{k}{class} \PY{n+nc}{B}\PY{p}{(}\PY{n}{D}\PY{p}{)}\PY{p}{:} 
    \PY{k}{def} \PY{n+nf}{f}\PY{p}{(}\PY{n+nb+bp}{self}\PY{p}{)}\PY{p}{:} \PY{k}{return} \PY{l+s+s2}{\PYZdq{}}\PY{l+s+s2}{B}\PY{l+s+s2}{\PYZdq{}}
\PY{k}{class} \PY{n+nc}{A}\PY{p}{(}\PY{n}{B}\PY{p}{,}\PY{n}{C}\PY{p}{)}\PY{p}{:}    \PY{k}{pass}
\PY{k}{class} \PY{n+nc}{A2}\PY{p}{(}\PY{n}{C}\PY{p}{,}\PY{n}{B}\PY{p}{)}\PY{p}{:}   \PY{k}{pass}


\PY{n}{a} \PY{o}{=} \PY{n}{A}\PY{p}{(}\PY{p}{)}
\PY{n+nb}{print}\PY{p}{(}\PY{n}{a}\PY{o}{.}\PY{n}{f}\PY{p}{(}\PY{p}{)}\PY{p}{)} \PY{c+c1}{\PYZsh{} B or C?  B}
\PY{n+nb}{print}\PY{p}{(}\PY{n}{A}\PY{o}{.}\PY{n}{mro}\PY{p}{(}\PY{p}{)}\PY{p}{)}

\PY{n}{a2} \PY{o}{=} \PY{n}{A2}\PY{p}{(}\PY{p}{)}
\PY{n+nb}{print}\PY{p}{(}\PY{n}{a2}\PY{o}{.}\PY{n}{f}\PY{p}{(}\PY{p}{)}\PY{p}{)} \PY{c+c1}{\PYZsh{} B or C? C }
\PY{n+nb}{print}\PY{p}{(}\PY{n}{A2}\PY{o}{.}\PY{n}{mro}\PY{p}{(}\PY{p}{)}\PY{p}{)}
\end{Verbatim}
\end{tcolorbox}

    \begin{Verbatim}[commandchars=\\\{\}]
B
[<class '\_\_main\_\_.A'>, <class '\_\_main\_\_.B'>, <class '\_\_main\_\_.C'>, <class
'\_\_main\_\_.D'>, <class 'object'>]
C
[<class '\_\_main\_\_.A2'>, <class '\_\_main\_\_.C'>, <class '\_\_main\_\_.B'>, <class
'\_\_main\_\_.D'>, <class 'object'>]
    \end{Verbatim}

    \begin{tcolorbox}[breakable, size=fbox, boxrule=1pt, pad at break*=1mm,colback=cellbackground, colframe=cellborder]
\prompt{In}{incolor}{61}{\boxspacing}
\begin{Verbatim}[commandchars=\\\{\}]
\PY{n}{b} \PY{o}{=} \PY{n}{B}\PY{p}{(}\PY{p}{)}
\PY{k}{for} \PY{n}{i} \PY{o+ow}{in} \PY{p}{[}\PY{n}{a}\PY{p}{,}\PY{n}{a2}\PY{p}{,}\PY{n}{b}\PY{p}{]}\PY{p}{:}
    \PY{n+nb}{print}\PY{p}{(}\PY{n+nb}{super}\PY{p}{(}\PY{n}{B}\PY{p}{,}\PY{n}{i}\PY{p}{)}\PY{o}{.}\PY{n}{f}\PY{p}{(}\PY{p}{)}\PY{p}{)} \PY{c+c1}{\PYZsh{} gets the MRO for object i, }
                          \PY{c+c1}{\PYZsh{} call the f() for the class next to the MRO with respect to B}
\end{Verbatim}
\end{tcolorbox}

    \begin{Verbatim}[commandchars=\\\{\}]
C
D
D
    \end{Verbatim}

    \hypertarget{polymorphism}{%
\subsection{Polymorphism}\label{polymorphism}}

\begin{itemize}
\tightlist
\item
  polymorphism is the provision of a single interface to entities of
  different types:
\item
  subtyping: supported with inheritance
\item
  parametric polymorphism: supported with generic programming (e.g.,
  templates)\\
\item
  ad hoc polymorphism: supported in many languages using function
  overloading
\item
  implementation aspects
\item
  the implementation is static if selected at compile time (e.g.~C++
  templates, macros or function overloading)
\item
  the implementation is dynamic if selected at run time (e.g.~virtual
  functions, abstract base classes)
\end{itemize}

\hypertarget{polymorphism-in-python}{%
\subsection{Polymorphism in python}\label{polymorphism-in-python}}

\begin{itemize}
\tightlist
\item
  subtyping (via inheritance)
\item
  duck typing
\item
  abstract base classes (optional)
\end{itemize}

    \hypertarget{subtyping-and-duck-typing}{%
\subsection{Subtyping and duck typing}\label{subtyping-and-duck-typing}}

\begin{itemize}
\tightlist
\item
  a subtype is a datatype that is related to another datatype, i.e.~the
  supertype, i.e.~the interface
\item
  multiple subtype can refere to the same interface
\item
  polymorphic subtyping is a good reason to use inheritance
\item
  duck typing mimics subtyping without implementing inheritance
\item
  ``If it walks like a duck and it quacks like a duck, then it must be a
  duck.''
\end{itemize}

\hypertarget{in-normal-programming-style-suitability-is-assumed-to-be-determined-by-an-objects-type-only}{%
\paragraph{in normal programming style, suitability is assumed to be
determined by an object's type
only}\label{in-normal-programming-style-suitability-is-assumed-to-be-determined-by-an-objects-type-only}}

\hypertarget{in-duck-typings-style-suitability-is-determined-by-the-interface-i.e.-the-presence-of-certain-methods-and-properties}{%
\paragraph{in duck typing's style, suitability is determined by the
interface (i.e., the presence of certain methods and
properties)}\label{in-duck-typings-style-suitability-is-determined-by-the-interface-i.e.-the-presence-of-certain-methods-and-properties}}

    \begin{tcolorbox}[breakable, size=fbox, boxrule=1pt, pad at break*=1mm,colback=cellbackground, colframe=cellborder]
\prompt{In}{incolor}{2}{\boxspacing}
\begin{Verbatim}[commandchars=\\\{\}]
\PY{o}{\PYZpc{}}\PY{k}{reset} \PYZhy{}f
\PY{k}{class} \PY{n+nc}{Animal}\PY{p}{(}\PY{n+nb}{object}\PY{p}{)}\PY{p}{:}  \PY{c+c1}{\PYZsh{} interface class}
    \PY{k}{def} \PY{n+nf+fm}{\PYZus{}\PYZus{}init\PYZus{}\PYZus{}}\PY{p}{(}\PY{n+nb+bp}{self}\PY{p}{)}\PY{p}{:} \PY{k}{pass}
    \PY{k}{def} \PY{n+nf}{call}\PY{p}{(}\PY{n+nb+bp}{self}\PY{p}{)}\PY{p}{:} \PY{k}{return} \PY{n+nb+bp}{self}\PY{o}{.}\PY{n}{animalSound}
\PY{k}{class} \PY{n+nc}{Tiger}\PY{p}{(}\PY{n}{Animal}\PY{p}{)}\PY{p}{:}
    \PY{n}{animalSound}\PY{o}{=}\PY{l+s+s2}{\PYZdq{}}\PY{l+s+s2}{roarrrrs}\PY{l+s+s2}{\PYZdq{}} 
\PY{k}{class} \PY{n+nc}{Dog}\PY{p}{(}\PY{n}{Animal}\PY{p}{)}\PY{p}{:}
    \PY{n}{animalSound}\PY{o}{=}\PY{l+s+s2}{\PYZdq{}}\PY{l+s+s2}{whoooofs}\PY{l+s+s2}{\PYZdq{}}
\PY{k}{class} \PY{n+nc}{Bird}\PY{p}{(}\PY{n}{Animal}\PY{p}{)}\PY{p}{:}
    \PY{n}{animalSound}\PY{o}{=}\PY{l+s+s2}{\PYZdq{}}\PY{l+s+s2}{tweets}\PY{l+s+s2}{\PYZdq{}}  

\PY{k}{def} \PY{n+nf}{I\PYZus{}hear\PYZus{}}\PY{p}{(}\PY{n}{something}\PY{p}{)}\PY{p}{:}
    \PY{n+nb}{print}\PY{p}{(}\PY{l+s+s1}{\PYZsq{}}\PY{l+s+s1}{I hear... a }\PY{l+s+s1}{\PYZsq{}}\PY{o}{+}\PYZbs{}
          \PY{n}{something}\PY{o}{.}\PY{n+nv+vm}{\PYZus{}\PYZus{}class\PYZus{}\PYZus{}}\PY{o}{.}\PY{n+nv+vm}{\PYZus{}\PYZus{}name\PYZus{}\PYZus{}}\PY{p}{,}\PYZbs{}
          \PY{l+s+s1}{\PYZsq{}}\PY{l+s+s1}{which}\PY{l+s+s1}{\PYZsq{}}\PY{p}{,}\PY{n}{something}\PY{o}{.}\PY{n}{call}\PY{p}{(}\PY{p}{)}\PY{p}{)}     
\PY{n}{al}\PY{o}{=}\PY{p}{[}\PY{p}{]}
\PY{n}{al}\PY{o}{.}\PY{n}{append}\PY{p}{(}\PY{n}{Tiger}\PY{p}{(}\PY{p}{)}\PY{p}{)}
\PY{n}{al}\PY{o}{.}\PY{n}{append}\PY{p}{(}\PY{n}{Dog}\PY{p}{(}\PY{p}{)}\PY{p}{)}
\PY{n}{al}\PY{o}{.}\PY{n}{append}\PY{p}{(}\PY{n}{Bird}\PY{p}{(}\PY{p}{)}\PY{p}{)}

\PY{k}{for} \PY{n}{this\PYZus{}animal} \PY{o+ow}{in} \PY{n}{al}\PY{p}{:}
    \PY{c+c1}{\PYZsh{} the interface method is invoked dynamically}
    \PY{n}{I\PYZus{}hear\PYZus{}}\PY{p}{(}\PY{n}{this\PYZus{}animal}\PY{p}{)} 
\end{Verbatim}
\end{tcolorbox}

    \begin{Verbatim}[commandchars=\\\{\}]
I hear{\ldots} a Tiger which roarrrrs
I hear{\ldots} a Dog which whoooofs
I hear{\ldots} a Bird which tweets
    \end{Verbatim}

    \begin{tcolorbox}[breakable, size=fbox, boxrule=1pt, pad at break*=1mm,colback=cellbackground, colframe=cellborder]
\prompt{In}{incolor}{1}{\boxspacing}
\begin{Verbatim}[commandchars=\\\{\}]
\PY{o}{\PYZpc{}}\PY{k}{reset} \PYZhy{}f
\PY{k}{class} \PY{n+nc}{Animal}\PY{p}{(}\PY{n+nb}{object}\PY{p}{)}\PY{p}{:}  \PY{c+c1}{\PYZsh{} interface class}
    \PY{k}{def} \PY{n+nf+fm}{\PYZus{}\PYZus{}init\PYZus{}\PYZus{}}\PY{p}{(}\PY{n+nb+bp}{self}\PY{p}{)}\PY{p}{:} \PY{k}{pass}
    \PY{k}{def} \PY{n+nf}{call}\PY{p}{(}\PY{n+nb+bp}{self}\PY{p}{)}\PY{p}{:} \PY{k}{return} \PY{n+nb+bp}{self}\PY{o}{.}\PY{n}{animalSound}
    
\PY{c+c1}{\PYZsh{} duck typing example: }
\PY{c+c1}{\PYZsh{} Duck is not subclass of Animal, but has the same interface}
\PY{k}{class} \PY{n+nc}{Duck}\PY{p}{(}\PY{n+nb}{object}\PY{p}{)}\PY{p}{:} 
    \PY{n}{animalSound}\PY{o}{=}\PY{l+s+s2}{\PYZdq{}}\PY{l+s+s2}{definitely quacks like a duck}\PY{l+s+s2}{\PYZdq{}}
    \PY{k}{def} \PY{n+nf+fm}{\PYZus{}\PYZus{}init\PYZus{}\PYZus{}}\PY{p}{(}\PY{n+nb+bp}{self}\PY{p}{)}\PY{p}{:} \PY{k}{pass}
    \PY{k}{def} \PY{n+nf}{call}\PY{p}{(}\PY{n+nb+bp}{self}\PY{p}{)}\PY{p}{:} \PY{k}{return} \PY{n+nb+bp}{self}\PY{o}{.}\PY{n}{animalSound}    
    
\PY{k}{def} \PY{n+nf}{I\PYZus{}hear\PYZus{}}\PY{p}{(}\PY{n}{something}\PY{p}{)}\PY{p}{:}
    \PY{n+nb}{print}\PY{p}{(}\PY{l+s+s1}{\PYZsq{}}\PY{l+s+s1}{I hear... a }\PY{l+s+s1}{\PYZsq{}}\PY{o}{+}\PYZbs{}
          \PY{n}{something}\PY{o}{.}\PY{n+nv+vm}{\PYZus{}\PYZus{}class\PYZus{}\PYZus{}}\PY{o}{.}\PY{n+nv+vm}{\PYZus{}\PYZus{}name\PYZus{}\PYZus{}}\PY{p}{,}\PYZbs{}
          \PY{l+s+s1}{\PYZsq{}}\PY{l+s+s1}{which}\PY{l+s+s1}{\PYZsq{}}\PY{p}{,}\PY{n}{something}\PY{o}{.}\PY{n}{call}\PY{p}{(}\PY{p}{)}\PY{p}{)} 
    
\PY{n}{I\PYZus{}hear\PYZus{}}\PY{p}{(}\PY{n}{Duck}\PY{p}{(}\PY{p}{)}\PY{p}{)}
\end{Verbatim}
\end{tcolorbox}

    \begin{Verbatim}[commandchars=\\\{\}]
I hear{\ldots} a Duck which definitely quacks like a duck
    \end{Verbatim}

    \begin{tcolorbox}[breakable, size=fbox, boxrule=1pt, pad at break*=1mm,colback=cellbackground, colframe=cellborder]
\prompt{In}{incolor}{78}{\boxspacing}
\begin{Verbatim}[commandchars=\\\{\}]
\PY{o}{\PYZpc{}}\PY{k}{reset} \PYZhy{}f
\PY{k}{class} \PY{n+nc}{multiple\PYZus{}of\PYZus{}}\PY{p}{(}\PY{n+nb}{object}\PY{p}{)}\PY{p}{:}                        \PY{c+c1}{\PYZsh{} I am duck\PYZhy{}typing the class Container }
    \PY{k}{def} \PY{n+nf+fm}{\PYZus{}\PYZus{}init\PYZus{}\PYZus{}}\PY{p}{(}\PY{n+nb+bp}{self}\PY{p}{,}\PY{n}{number}\PY{p}{)}\PY{p}{:} \PY{n+nb+bp}{self}\PY{o}{.}\PY{n}{number}\PY{o}{=}\PY{n}{number}  \PY{c+c1}{\PYZsh{} constructor to indroduce an attribute number}
    \PY{k}{def} \PY{n+nf+fm}{\PYZus{}\PYZus{}contains\PYZus{}\PYZus{}}\PY{p}{(}\PY{n+nb+bp}{self}\PY{p}{,}\PY{n}{x}\PY{p}{)}\PY{p}{:}                      \PY{c+c1}{\PYZsh{} method that defines which number belongs to this class}
        \PY{k}{if} \PY{n}{x}\PY{o}{\PYZlt{}}\PY{n+nb+bp}{self}\PY{o}{.}\PY{n}{number}\PY{p}{:} \PY{k}{return} \PY{k+kc}{False}             \PY{c+c1}{\PYZsh{} by definition, multiple of x should not be smaller than x }
        \PY{k}{if} \PY{n}{x}\PY{o}{\PYZpc{}}\PY{k}{self}.number==0: return True           \PYZsh{} if multiple of number, then it belongs to this class
        \PY{k}{else}\PY{p}{:} \PY{k}{return} \PY{k+kc}{False}                         \PY{c+c1}{\PYZsh{} otherwise it does not}
    \PY{k}{def} \PY{n+nf+fm}{\PYZus{}\PYZus{}str\PYZus{}\PYZus{}}\PY{p}{(}\PY{n+nb+bp}{self}\PY{p}{)}\PY{p}{:} 
        \PY{k}{return} \PY{l+s+s1}{\PYZsq{}}\PY{l+s+s1}{is multiple of }\PY{l+s+s1}{\PYZsq{}}\PY{o}{+}\PY{n+nb}{str}\PY{p}{(}\PY{n+nb+bp}{self}\PY{o}{.}\PY{n}{number}\PY{p}{)}

\PY{n}{m10} \PY{o}{=} \PY{n}{multiple\PYZus{}of\PYZus{}}\PY{p}{(}\PY{l+m+mi}{10}\PY{p}{)}
\PY{n}{m9} \PY{o}{=} \PY{n}{multiple\PYZus{}of\PYZus{}}\PY{p}{(}\PY{l+m+mi}{9}\PY{p}{)}
\PY{n}{m8} \PY{o}{=} \PY{n}{multiple\PYZus{}of\PYZus{}}\PY{p}{(}\PY{l+m+mi}{8}\PY{p}{)}
\PY{n}{m7} \PY{o}{=} \PY{n}{multiple\PYZus{}of\PYZus{}}\PY{p}{(}\PY{l+m+mi}{7}\PY{p}{)}
\PY{k}{for} \PY{n}{i} \PY{o+ow}{in} \PY{n+nb}{range}\PY{p}{(}\PY{l+m+mi}{31}\PY{p}{)}\PY{p}{:}
    \PY{k}{for} \PY{n}{j} \PY{o+ow}{in} \PY{p}{[}\PY{n}{m7}\PY{p}{,}\PY{n}{m8}\PY{p}{,}\PY{n}{m9}\PY{p}{,}\PY{n}{m10}\PY{p}{]}\PY{p}{:}             
        \PY{k}{if} \PY{n}{i} \PY{o+ow}{in} \PY{n}{j}\PY{p}{:}                
            \PY{n+nb}{print}\PY{p}{(}\PY{n}{i}\PY{p}{,}\PY{n}{j}\PY{p}{)}
\end{Verbatim}
\end{tcolorbox}

    \begin{Verbatim}[commandchars=\\\{\}]
7 is multiple of 7
8 is multiple of 8
9 is multiple of 9
10 is multiple of 10
14 is multiple of 7
16 is multiple of 8
18 is multiple of 9
20 is multiple of 10
21 is multiple of 7
24 is multiple of 8
27 is multiple of 9
28 is multiple of 7
30 is multiple of 10
    \end{Verbatim}

    \hypertarget{removed-abc-from-this-lecture}{%
\subsubsection{removed abc from this
lecture}\label{removed-abc-from-this-lecture}}

\hypertarget{duck-typing-limits-and-abstract-base-classes-abcs}{%
\subsection{Duck typing limits and abstract base classes
(ABCs)}\label{duck-typing-limits-and-abstract-base-classes-abcs}}

\begin{itemize}
\tightlist
\item
  duck typing does not verify whether the class satisfies your
  requirements or not
\item
  the concept of abstract classes extends duck typing
\item
  an abstract class defines a set of attributes that a class must
  implement (i.e., limits the duck-type flexibility)
\item
  collections module contains many ABCs implementations
\item
  e.g., the Container class
\item
  to duck-type an ABC some methods must be defines: check this using
  \_\emph{abstractmethods\_}
\item
  abc module provides the infrastructure to define new ABC classes
\item
  uses decorators
\end{itemize}

    \hypertarget{how-to-define-a-context-manager-class}{%
\subsection{How to define a context manager
class}\label{how-to-define-a-context-manager-class}}

\begin{itemize}
\tightlist
\item
  a context manager class must define two methods:
\item
  \_\_enter\_\_(self): enter the runtime context
\item
  \_\_exit\_\_(self, exc\_type, exc\_value, traceback): exit the runtime
  context
\item
  useful to embed a specific protocol
\end{itemize}

    \begin{tcolorbox}[breakable, size=fbox, boxrule=1pt, pad at break*=1mm,colback=cellbackground, colframe=cellborder]
\prompt{In}{incolor}{35}{\boxspacing}
\begin{Verbatim}[commandchars=\\\{\}]
\PY{o}{\PYZpc{}}\PY{k}{reset} \PYZhy{}f 
\PY{k}{class} \PY{n+nc}{myErr}\PY{p}{(}\PY{n+ne}{Exception}\PY{p}{)}\PY{p}{:} \PY{k}{pass}

\PY{k}{class} \PY{n+nc}{CManager}\PY{p}{(}\PY{n+nb}{object}\PY{p}{)}\PY{p}{:}
    \PY{k}{def} \PY{n+nf+fm}{\PYZus{}\PYZus{}enter\PYZus{}\PYZus{}}\PY{p}{(}\PY{n+nb+bp}{self}\PY{p}{)}\PY{p}{:}
        \PY{n+nb}{print}\PY{p}{(}\PY{l+s+s1}{\PYZsq{}}\PY{l+s+s1}{\PYZus{}\PYZus{}enter\PYZus{}\PYZus{}}\PY{l+s+s1}{\PYZsq{}}\PY{p}{)}
        \PY{k}{return} \PY{n+nb+bp}{self}    
    \PY{k}{def} \PY{n+nf+fm}{\PYZus{}\PYZus{}exit\PYZus{}\PYZus{}}\PY{p}{(}\PY{n+nb+bp}{self}\PY{p}{,} \PY{n+nb}{type}\PY{p}{,} \PY{n}{value}\PY{p}{,} \PY{n}{traceback}\PY{p}{)}\PY{p}{:}
        \PY{n+nb}{print}\PY{p}{(}\PY{l+s+s1}{\PYZsq{}}\PY{l+s+s1}{\PYZus{}\PYZus{}exit\PYZus{}\PYZus{}:}\PY{l+s+s1}{\PYZsq{}}\PY{p}{,} \PY{n+nb}{type}\PY{p}{,} \PY{n}{value}\PY{p}{)}
        \PY{k}{return} \PY{k+kc}{True} \PY{c+c1}{\PYZsh{} if False no exception is considered}

\PY{n+nb}{print}\PY{p}{(}\PY{l+s+s1}{\PYZsq{}}\PY{l+s+s1}{\PYZsh{} with no interrupt}\PY{l+s+s1}{\PYZsq{}}\PY{p}{)}
\PY{k}{with} \PY{n}{CManager}\PY{p}{(}\PY{p}{)} \PY{k}{as} \PY{n}{c}\PY{p}{:}
    \PY{n+nb}{print}\PY{p}{(}\PY{l+s+s1}{\PYZsq{}}\PY{l+s+s1}{doing something with }\PY{l+s+s1}{\PYZsq{}}\PY{p}{,} \PY{n}{c}\PY{p}{)}
\PY{n+nb}{print}\PY{p}{(}\PY{l+s+s1}{\PYZsq{}}\PY{l+s+s1}{done something}\PY{l+s+se}{\PYZbs{}n}\PY{l+s+s1}{\PYZsq{}}\PY{p}{)}    
    
\PY{n+nb}{print}\PY{p}{(}\PY{l+s+s1}{\PYZsq{}}\PY{l+s+s1}{\PYZsh{} with interrupt}\PY{l+s+s1}{\PYZsq{}}\PY{p}{)}
\PY{k}{with} \PY{n}{CManager}\PY{p}{(}\PY{p}{)} \PY{k}{as} \PY{n}{c}\PY{p}{:}
    \PY{n+nb}{print}\PY{p}{(}\PY{l+s+s1}{\PYZsq{}}\PY{l+s+s1}{doing something with }\PY{l+s+s1}{\PYZsq{}}\PY{p}{,} \PY{n}{c}\PY{p}{)}
    \PY{k}{raise} \PY{n}{myErr}\PY{p}{(}\PY{l+s+s1}{\PYZsq{}}\PY{l+s+s1}{err msg}\PY{l+s+s1}{\PYZsq{}}\PY{p}{)}
    \PY{n+nb}{print}\PY{p}{(}\PY{l+s+s1}{\PYZsq{}}\PY{l+s+s1}{unreachable}\PY{l+s+s1}{\PYZsq{}}\PY{p}{)}
\PY{n+nb}{print}\PY{p}{(}\PY{l+s+s1}{\PYZsq{}}\PY{l+s+s1}{done something}\PY{l+s+s1}{\PYZsq{}}\PY{p}{)}
\end{Verbatim}
\end{tcolorbox}

    \begin{Verbatim}[commandchars=\\\{\}]
\# with no interrupt
\_\_enter\_\_
doing something with  <\_\_main\_\_.CManager object at 0x7ffb0c46b470>
\_\_exit\_\_: None None
done something

\# with interrupt
\_\_enter\_\_
doing something with  <\_\_main\_\_.CManager object at 0x7ffb0c46b2b0>
\_\_exit\_\_: <class '\_\_main\_\_.myErr'> err msg
done something
    \end{Verbatim}

    \hypertarget{contextlib-module}{%
\subsection{contextlib module}\label{contextlib-module}}

\begin{itemize}
\tightlist
\item
  decorator @contextmanager: combined with yield, can be used to define
  context manager even without defining a class
\item
  ContextDecorator: base class to enable a context manager to be used as
  a decorator
\end{itemize}

    \begin{tcolorbox}[breakable, size=fbox, boxrule=1pt, pad at break*=1mm,colback=cellbackground, colframe=cellborder]
\prompt{In}{incolor}{131}{\boxspacing}
\begin{Verbatim}[commandchars=\\\{\}]
\PY{o}{\PYZpc{}}\PY{k}{reset} \PYZhy{}f
\PY{k+kn}{from} \PY{n+nn}{contextlib} \PY{k+kn}{import} \PY{n}{contextmanager}

\PY{n+nd}{@contextmanager}
\PY{k}{def} \PY{n+nf}{tag}\PY{p}{(}\PY{n}{name}\PY{p}{)}\PY{p}{:} 
    \PY{n+nb}{print}\PY{p}{(}\PY{l+s+s2}{\PYZdq{}}\PY{l+s+s2}{\PYZlt{}}\PY{l+s+si}{\PYZob{}\PYZcb{}}\PY{l+s+s2}{\PYZgt{} }\PY{l+s+s2}{\PYZdq{}}\PY{o}{.}\PY{n}{format}\PY{p}{(}\PY{n}{name}\PY{p}{)}\PY{p}{,}\PY{n}{end}\PY{o}{=}\PY{l+s+s2}{\PYZdq{}}\PY{l+s+s2}{\PYZdq{}}\PY{p}{)}
    \PY{k}{yield}
    \PY{n+nb}{print}\PY{p}{(}\PY{l+s+s2}{\PYZdq{}}\PY{l+s+s2}{ \PYZlt{}/}\PY{l+s+si}{\PYZob{}\PYZcb{}}\PY{l+s+s2}{\PYZgt{}}\PY{l+s+s2}{\PYZdq{}}\PY{o}{.}\PY{n}{format}\PY{p}{(}\PY{n}{name}\PY{p}{)}\PY{p}{,}\PY{n}{end}\PY{o}{=}\PY{l+s+s2}{\PYZdq{}}\PY{l+s+s2}{\PYZdq{}}\PY{p}{)}
    
\PY{k}{with} \PY{n}{tag}\PY{p}{(}\PY{l+s+s2}{\PYZdq{}}\PY{l+s+s2}{thisTag}\PY{l+s+s2}{\PYZdq{}}\PY{p}{)}\PY{p}{:} 
    \PY{n+nb}{print}\PY{p}{(}\PY{l+s+s2}{\PYZdq{}}\PY{l+s+s2}{something}\PY{l+s+s2}{\PYZdq{}}\PY{p}{,}\PY{n}{end}\PY{o}{=}\PY{l+s+s2}{\PYZdq{}}\PY{l+s+s2}{\PYZdq{}}\PY{p}{)}    
\end{Verbatim}
\end{tcolorbox}

    \begin{Verbatim}[commandchars=\\\{\}]
<thisTag> something </thisTag>
    \end{Verbatim}

    \begin{tcolorbox}[breakable, size=fbox, boxrule=1pt, pad at break*=1mm,colback=cellbackground, colframe=cellborder]
\prompt{In}{incolor}{4}{\boxspacing}
\begin{Verbatim}[commandchars=\\\{\}]
\PY{o}{\PYZpc{}}\PY{k}{reset} \PYZhy{}f
\PY{k+kn}{from} \PY{n+nn}{contextlib} \PY{k+kn}{import} \PY{n}{ContextDecorator}
\PY{k+kn}{import} \PY{n+nn}{time}
\PY{k}{class} \PY{n+nc}{Timer}\PY{p}{(}\PY{n+nb}{object}\PY{p}{)}\PY{p}{:}
    \PY{k}{def} \PY{n+nf+fm}{\PYZus{}\PYZus{}enter\PYZus{}\PYZus{}}\PY{p}{(}\PY{n+nb+bp}{self}\PY{p}{)}\PY{p}{:}
        \PY{n+nb+bp}{self}\PY{o}{.}\PY{n}{\PYZus{}\PYZus{}t1} \PY{o}{=} \PY{n}{time}\PY{o}{.}\PY{n}{time}\PY{p}{(}\PY{p}{)}
        \PY{k}{return} \PY{n+nb+bp}{self}
    \PY{k}{def} \PY{n+nf+fm}{\PYZus{}\PYZus{}exit\PYZus{}\PYZus{}}\PY{p}{(}\PY{n+nb+bp}{self}\PY{p}{,}\PY{o}{*}\PY{n}{exc}\PY{p}{)}\PY{p}{:}
        \PY{n+nb+bp}{self}\PY{o}{.}\PY{n}{\PYZus{}\PYZus{}t2} \PY{o}{=} \PY{n}{time}\PY{o}{.}\PY{n}{time}\PY{p}{(}\PY{p}{)}
        \PY{n+nb}{print}\PY{p}{(}\PY{l+m+mi}{1000}\PY{o}{*}\PY{p}{(}\PY{n+nb+bp}{self}\PY{o}{.}\PY{n}{\PYZus{}\PYZus{}t2}\PY{o}{\PYZhy{}}\PY{n+nb+bp}{self}\PY{o}{.}\PY{n}{\PYZus{}\PYZus{}t1}\PY{p}{)}\PY{p}{,}\PY{l+s+s1}{\PYZsq{}}\PY{l+s+s1}{ms}\PY{l+s+s1}{\PYZsq{}}\PY{p}{)}
        \PY{k}{return} \PY{k+kc}{False}
    
\PY{k}{with} \PY{n}{Timer}\PY{p}{(}\PY{p}{)}\PY{p}{:}
    \PY{k}{for} \PY{n}{i} \PY{o+ow}{in} \PY{n+nb}{range}\PY{p}{(}\PY{l+m+mi}{10}\PY{p}{)}\PY{p}{:}         
        \PY{k}{with} \PY{n}{Timer}\PY{p}{(}\PY{p}{)}\PY{p}{:} 
            \PY{n+nb}{print}\PY{p}{(}\PY{l+s+s1}{\PYZsq{}}\PY{l+s+s1}{\PYZsq{}}\PY{p}{,}\PY{n}{end}\PY{o}{=}\PY{l+s+s1}{\PYZsq{}}\PY{l+s+s1}{\PYZsq{}}\PY{p}{)}
    \PY{n+nb}{print}\PY{p}{(}\PY{l+s+s2}{\PYZdq{}}\PY{l+s+s2}{whole time: }\PY{l+s+s2}{\PYZdq{}}\PY{p}{)}
\end{Verbatim}
\end{tcolorbox}

    \begin{Verbatim}[commandchars=\\\{\}]
0.18143653869628906 ms
0.028133392333984375 ms
0.3943443298339844 ms
0.10967254638671875 ms
0.10013580322265625 ms
0.09894371032714844 ms
0.10061264038085938 ms
0.09894371032714844 ms
0.09870529174804688 ms
0.09918212890625 ms
whole time:
3.642559051513672 ms
    \end{Verbatim}

    \begin{tcolorbox}[breakable, size=fbox, boxrule=1pt, pad at break*=1mm,colback=cellbackground, colframe=cellborder]
\prompt{In}{incolor}{7}{\boxspacing}
\begin{Verbatim}[commandchars=\\\{\}]
\PY{o}{\PYZpc{}}\PY{k}{reset} \PYZhy{}f
\PY{k+kn}{from} \PY{n+nn}{contextlib} \PY{k+kn}{import} \PY{n}{ContextDecorator}
\PY{k+kn}{import} \PY{n+nn}{time}
\PY{k}{class} \PY{n+nc}{Timer}\PY{p}{(}\PY{n}{ContextDecorator}\PY{p}{)}\PY{p}{:}
    \PY{k}{def} \PY{n+nf+fm}{\PYZus{}\PYZus{}enter\PYZus{}\PYZus{}}\PY{p}{(}\PY{n+nb+bp}{self}\PY{p}{)}\PY{p}{:}
        \PY{n+nb+bp}{self}\PY{o}{.}\PY{n}{\PYZus{}\PYZus{}t1} \PY{o}{=} \PY{n}{time}\PY{o}{.}\PY{n}{time}\PY{p}{(}\PY{p}{)}
        \PY{k}{return} \PY{n+nb+bp}{self}
    \PY{k}{def} \PY{n+nf+fm}{\PYZus{}\PYZus{}exit\PYZus{}\PYZus{}}\PY{p}{(}\PY{n+nb+bp}{self}\PY{p}{,}\PY{o}{*}\PY{n}{exc}\PY{p}{)}\PY{p}{:}
        \PY{n+nb+bp}{self}\PY{o}{.}\PY{n}{\PYZus{}\PYZus{}t2} \PY{o}{=} \PY{n}{time}\PY{o}{.}\PY{n}{time}\PY{p}{(}\PY{p}{)}
        \PY{n+nb}{print}\PY{p}{(}\PY{l+m+mi}{1000}\PY{o}{*}\PY{p}{(}\PY{n+nb+bp}{self}\PY{o}{.}\PY{n}{\PYZus{}\PYZus{}t2}\PY{o}{\PYZhy{}}\PY{n+nb+bp}{self}\PY{o}{.}\PY{n}{\PYZus{}\PYZus{}t1}\PY{p}{)}\PY{p}{,}\PY{l+s+s1}{\PYZsq{}}\PY{l+s+s1}{ms}\PY{l+s+s1}{\PYZsq{}}\PY{p}{)}
        \PY{k}{return} \PY{k+kc}{False}
\PY{n+nd}{@Timer}\PY{p}{(}\PY{p}{)}
\PY{k}{def} \PY{n+nf}{fun}\PY{p}{(}\PY{p}{)}\PY{p}{:} 
    \PY{n+nb}{print}\PY{p}{(}\PY{l+s+s1}{\PYZsq{}}\PY{l+s+s1}{\PYZsq{}}\PY{p}{,}\PY{n}{end}\PY{o}{=}\PY{l+s+s1}{\PYZsq{}}\PY{l+s+s1}{\PYZsq{}}\PY{p}{)}
\PY{n+nd}{@Timer}\PY{p}{(}\PY{p}{)}
\PY{k}{def} \PY{n+nf}{ten\PYZus{}times\PYZus{}fun}\PY{p}{(}\PY{p}{)}\PY{p}{:}
    \PY{k}{for} \PY{n}{i} \PY{o+ow}{in} \PY{n+nb}{range}\PY{p}{(}\PY{l+m+mi}{10}\PY{p}{)}\PY{p}{:} 
        \PY{n}{fun}\PY{p}{(}\PY{p}{)}
    \PY{n+nb}{print}\PY{p}{(}\PY{l+s+s2}{\PYZdq{}}\PY{l+s+s2}{whole time: }\PY{l+s+s2}{\PYZdq{}}\PY{p}{)}
    
\PY{n}{ten\PYZus{}times\PYZus{}fun}\PY{p}{(}\PY{p}{)}        
    
\end{Verbatim}
\end{tcolorbox}

    \begin{Verbatim}[commandchars=\\\{\}]
0.11539459228515625 ms
0.030517578125 ms
0.2498626708984375 ms
0.15091896057128906 ms
0.12159347534179688 ms
0.12111663818359375 ms
0.12254714965820312 ms
0.12087821960449219 ms
0.12087821960449219 ms
0.12111663818359375 ms
whole time:
4.659891128540039 ms
    \end{Verbatim}

    \hypertarget{iterable-sequences-and-iterators}{%
\subsection{Iterable, sequences and
iterators}\label{iterable-sequences-and-iterators}}

\begin{itemize}
\tightlist
\item
  an iterable is an object with \_\_iter\_\_() method
\item
  \_\_iter\_\_() normally return the class itself
\item
  a sequence is an iterable with \_\_getitem\_\_() and \_\_len\_\_()
  methods
\item
  e.g., list, str, tuple are sequences
\item
  an iterator is an iterable with \_\_next\_\_() method
\item
  \_\_next\_\_() contains the logic to obtain the next element
\item
  \_\_next\_\_() should raise StopIteration() when no elements are left
\end{itemize}

    \begin{tcolorbox}[breakable, size=fbox, boxrule=1pt, pad at break*=1mm,colback=cellbackground, colframe=cellborder]
\prompt{In}{incolor}{17}{\boxspacing}
\begin{Verbatim}[commandchars=\\\{\}]
\PY{o}{\PYZpc{}}\PY{k}{reset} \PYZhy{}f 
\PY{k}{class} \PY{n+nc}{CapitalChars\PYZus{}Iterator}\PY{p}{(}\PY{n+nb}{object}\PY{p}{)}\PY{p}{:}     
    \PY{k}{def} \PY{n+nf+fm}{\PYZus{}\PYZus{}init\PYZus{}\PYZus{}}\PY{p}{(}\PY{n+nb+bp}{self}\PY{p}{,}\PY{n}{string}\PY{p}{)}\PY{p}{:} 
        \PY{n+nb+bp}{self}\PY{o}{.}\PY{n}{\PYZus{}string} \PY{o}{=} \PY{n}{string}
        \PY{n+nb+bp}{self}\PY{o}{.}\PY{n}{\PYZus{}index} \PY{o}{=} \PY{o}{\PYZhy{}}\PY{l+m+mi}{1}
    \PY{k}{def} \PY{n+nf+fm}{\PYZus{}\PYZus{}next\PYZus{}\PYZus{}}\PY{p}{(}\PY{n+nb+bp}{self}\PY{p}{)}\PY{p}{:} 
        \PY{n+nb+bp}{self}\PY{o}{.}\PY{n}{\PYZus{}index} \PY{o}{+}\PY{o}{=} \PY{l+m+mi}{1}
        \PY{k}{if} \PY{n+nb+bp}{self}\PY{o}{.}\PY{n}{\PYZus{}index} \PY{o}{\PYZgt{}}\PY{o}{=} \PY{n+nb}{len}\PY{p}{(}\PY{n+nb+bp}{self}\PY{o}{.}\PY{n}{\PYZus{}string}\PY{p}{)}\PY{p}{:} 
            \PY{k}{raise} \PY{n+ne}{StopIteration}\PY{p}{(}\PY{p}{)}
        \PY{k}{if} \PY{n+nb+bp}{self}\PY{o}{.}\PY{n}{\PYZus{}string}\PY{p}{[}\PY{n+nb+bp}{self}\PY{o}{.}\PY{n}{\PYZus{}index}\PY{p}{]}\PY{o}{.}\PY{n}{isupper}\PY{p}{(}\PY{p}{)}\PY{p}{:}            
            \PY{k}{return} \PY{n+nb+bp}{self}\PY{o}{.}\PY{n}{\PYZus{}index}
        \PY{c+c1}{\PYZsh{} else returns None        }
    \PY{k}{def} \PY{n+nf+fm}{\PYZus{}\PYZus{}iter\PYZus{}\PYZus{}}\PY{p}{(}\PY{n+nb+bp}{self}\PY{p}{)}\PY{p}{:}        
        \PY{k}{return} \PY{n+nb+bp}{self}

\PY{n+nb}{print}\PY{p}{(}\PY{p}{[}\PY{n}{i} \PY{k}{for} \PY{n}{i} \PY{o+ow}{in} \PY{n}{CapitalChars\PYZus{}Iterator}\PY{p}{(}\PY{l+s+s1}{\PYZsq{}}\PY{l+s+s1}{SiSsA}\PY{l+s+s1}{\PYZsq{}}\PY{p}{)} \PY{k}{if} \PY{n}{i} \PY{o}{!=} \PY{k+kc}{None}\PY{p}{]}\PY{p}{)}
\end{Verbatim}
\end{tcolorbox}

    \begin{Verbatim}[commandchars=\\\{\}]
[0, 2, 4]
    \end{Verbatim}

    \hypertarget{unit-testing-framework}{%
\subsection{Unit testing framework}\label{unit-testing-framework}}

Why test? * testing is a crucial aspects of software development * it
ensures code works as the developer thinks * it ensures code still works
after changes * in teamwork, it helps planning, understanding and
sharing requirements

\begin{itemize}
\tightlist
\item
  tests should be devised before the software development
\item
  python has a built-in test library: unittest
\end{itemize}

Workflow

\begin{itemize}
\item
  \begin{enumerate}
  \def\labelenumi{\arabic{enumi})}
  \setcounter{enumi}{-1}
  \tightlist
  \item
    import the module to be tested
  \end{enumerate}
\item
  \begin{enumerate}
  \def\labelenumi{\arabic{enumi})}
  \tightlist
  \item
    define your class derived from unittest: the TestCase
  \end{enumerate}
\item
  \begin{enumerate}
  \def\labelenumi{\arabic{enumi})}
  \setcounter{enumi}{1}
  \tightlist
  \item
    fill it with test functions whose names start with `test\_'
  \end{enumerate}
\item
  \begin{enumerate}
  \def\labelenumi{\arabic{enumi})}
  \setcounter{enumi}{2}
  \tightlist
  \item
    run tests using unittest.main({[}argv,exit{]}): argv and SystemExit
    in case an exception is raised
  \end{enumerate}
\end{itemize}

    \hypertarget{testcase-methods-to-run-the-test}{%
\subsection{TestCase methods to run the
test}\label{testcase-methods-to-run-the-test}}

\begin{itemize}
\tightlist
\item
  setUpModule(), tearDownModule(): called at first and at last in
  module, run once
\item
  setUpClass(), tearDownClass: called at first and at last in class, run
  once, must be decorated @classmethod
\item
  setUp(), tearDown(): execute before/after any TestCase method
\end{itemize}

    \begin{tcolorbox}[breakable, size=fbox, boxrule=1pt, pad at break*=1mm,colback=cellbackground, colframe=cellborder]
\prompt{In}{incolor}{30}{\boxspacing}
\begin{Verbatim}[commandchars=\\\{\}]
\PY{o}{\PYZpc{}}\PY{k}{reset} \PYZhy{}f
\PY{k+kn}{import} \PY{n+nn}{unittest}

\PY{k}{class} \PY{n+nc}{Err}\PY{p}{(}\PY{n+nb}{object}\PY{p}{)}\PY{p}{:} \PY{k}{pass}

\PY{k}{def} \PY{n+nf}{setUpModule}\PY{p}{(}\PY{p}{)}\PY{p}{:} \PY{n+nb}{print}\PY{p}{(}\PY{l+s+s2}{\PYZdq{}}\PY{l+s+s2}{setUpModule}\PY{l+s+s2}{\PYZdq{}}\PY{p}{)}
\PY{k}{def} \PY{n+nf}{tearDownModule}\PY{p}{(}\PY{p}{)}\PY{p}{:} \PY{n+nb}{print}\PY{p}{(}\PY{l+s+s2}{\PYZdq{}}\PY{l+s+s2}{tearDownModule}\PY{l+s+s2}{\PYZdq{}}\PY{p}{)}

\PY{k}{class} \PY{n+nc}{Example1}\PY{p}{(}\PY{n}{unittest}\PY{o}{.}\PY{n}{TestCase}\PY{p}{)}\PY{p}{:}
    \PY{n+nd}{@classmethod}
    \PY{k}{def} \PY{n+nf}{setUpClass}\PY{p}{(}\PY{n+nb+bp}{self}\PY{p}{)}\PY{p}{:} \PY{n+nb}{print}\PY{p}{(}\PY{l+s+s2}{\PYZdq{}}\PY{l+s+s2}{ setUpClass Example1}\PY{l+s+s2}{\PYZdq{}}\PY{p}{)}
    \PY{k}{def} \PY{n+nf}{setUp}\PY{p}{(}\PY{n+nb+bp}{self}\PY{p}{)}\PY{p}{:} \PY{n+nb}{print}\PY{p}{(}\PY{l+s+s2}{\PYZdq{}}\PY{l+s+s2}{  setUp}\PY{l+s+s2}{\PYZdq{}}\PY{p}{)}
    \PY{k}{def} \PY{n+nf}{test1}\PY{p}{(}\PY{n+nb+bp}{self}\PY{p}{)}\PY{p}{:} \PY{n+nb}{print}\PY{p}{(}\PY{l+s+s2}{\PYZdq{}}\PY{l+s+s2}{   test1}\PY{l+s+s2}{\PYZdq{}}\PY{p}{)}        
    \PY{k}{def} \PY{n+nf}{test2}\PY{p}{(}\PY{n+nb+bp}{self}\PY{p}{)}\PY{p}{:} 
        \PY{n+nb}{print}\PY{p}{(}\PY{l+s+s2}{\PYZdq{}}\PY{l+s+s2}{   test2}\PY{l+s+s2}{\PYZdq{}}\PY{p}{)}
        \PY{k}{raise} \PY{n}{Err}\PY{p}{(}\PY{l+s+s1}{\PYZsq{}}\PY{l+s+s1}{error}\PY{l+s+s1}{\PYZsq{}}\PY{p}{)}
    \PY{k}{def} \PY{n+nf}{tearDown}\PY{p}{(}\PY{n+nb+bp}{self}\PY{p}{)}\PY{p}{:} \PY{n+nb}{print}\PY{p}{(}\PY{l+s+s2}{\PYZdq{}}\PY{l+s+s2}{  tearDown}\PY{l+s+s2}{\PYZdq{}}\PY{p}{)}
    \PY{n+nd}{@classmethod}
    \PY{k}{def} \PY{n+nf}{tearDownClass}\PY{p}{(}\PY{n+nb+bp}{self}\PY{p}{)}\PY{p}{:} \PY{n+nb}{print}\PY{p}{(}\PY{l+s+s2}{\PYZdq{}}\PY{l+s+s2}{ tearDownClass Example1}\PY{l+s+s2}{\PYZdq{}}\PY{p}{)}

\PY{k}{class} \PY{n+nc}{Example2}\PY{p}{(}\PY{n}{unittest}\PY{o}{.}\PY{n}{TestCase}\PY{p}{)}\PY{p}{:} 
    \PY{n+nd}{@classmethod}
    \PY{k}{def} \PY{n+nf}{setUpClass}\PY{p}{(}\PY{n+nb+bp}{cls}\PY{p}{)}\PY{p}{:} \PY{n+nb}{print}\PY{p}{(}\PY{l+s+s2}{\PYZdq{}}\PY{l+s+s2}{ setUpClass Example2}\PY{l+s+s2}{\PYZdq{}}\PY{p}{)}
    \PY{n+nd}{@classmethod}
    \PY{k}{def} \PY{n+nf}{tearDownClass}\PY{p}{(}\PY{n+nb+bp}{cls}\PY{p}{)}\PY{p}{:} \PY{n+nb}{print}\PY{p}{(}\PY{l+s+s2}{\PYZdq{}}\PY{l+s+s2}{ tearDownClass Example2}\PY{l+s+s2}{\PYZdq{}}\PY{p}{)}
    \PY{k}{def} \PY{n+nf}{test3}\PY{p}{(}\PY{n+nb+bp}{self}\PY{p}{)}\PY{p}{:} \PY{n+nb}{print}\PY{p}{(}\PY{l+s+s2}{\PYZdq{}}\PY{l+s+s2}{   test3}\PY{l+s+s2}{\PYZdq{}}\PY{p}{)}        
    
\PY{n+nb}{print}\PY{p}{(}\PY{n}{unittest}\PY{o}{.}\PY{n}{main}\PY{p}{(}\PY{n}{argv}\PY{o}{=}\PY{p}{[}\PY{l+s+s1}{\PYZsq{}}\PY{l+s+s1}{first\PYZhy{}arg\PYZhy{}is\PYZhy{}ignored}\PY{l+s+s1}{\PYZsq{}}\PY{p}{]}\PY{p}{,} \PY{n}{exit}\PY{o}{=}\PY{k+kc}{False}\PY{p}{)}\PY{p}{)}
\end{Verbatim}
\end{tcolorbox}

    \begin{Verbatim}[commandchars=\\\{\}]
.E.
    \end{Verbatim}

    \begin{Verbatim}[commandchars=\\\{\}]
setUpModule
 setUpClass Example1
  setUp
   test1
  tearDown
  setUp
   test2
  tearDown
 tearDownClass Example1
 setUpClass Example2
   test3
 tearDownClass Example2
tearDownModule
<unittest.main.TestProgram object at 0x7fa7580837b8>
    \end{Verbatim}

    \begin{Verbatim}[commandchars=\\\{\}]

======================================================================
ERROR: test2 (\_\_main\_\_.Example1)
----------------------------------------------------------------------
Traceback (most recent call last):
  File "<ipython-input-30-9b56c873b00a>", line 16, in test2
    raise Err('error')
TypeError: object() takes no parameters

----------------------------------------------------------------------
Ran 3 tests in 0.004s

FAILED (errors=1)
    \end{Verbatim}

    \hypertarget{testcase-assert-methods}{%
\subsection{TestCase assert methods}\label{testcase-assert-methods}}

Most common asserts * assertFalse(x{[},msg{]}), assertTrue(x{[},msg{]})
* assertEqual(a,b{[},msg{]}),
assertNotEqual(a,b{[},msg{]}),assertGreater(a,b{[},msg{]}),
assertGreaterEqual(a,b{[},msg{]}),assertLess(a,b{[},msg{]}),
assertLessEqual(a,b{[},msg{]})\\
* assertIs(a,b{[},msg{]}), assertIsNot(a,b{[},msg{]}),
assertIsNone(x{[},msg{]}), assertIsNotNone(x{[},msg{]}), *
assertIn(a,b{[},msg{]}), assertNotIn(a,b{[},msg{]}) *
assertIsInstance(a,b{[},msg{]}), assertNotIsInstance(a,b{[},msg{]})

Type-specific asserts * assertMultiLineEqual(a,b{[},msg{]}),
assertSequenceEqual(a,b{[},msg{]}), assertListEqual(a,b{[},msg{]}),
assertTupleEqual(a,b{[},msg{]}), assertSetEqual(a,b{[},msg{]}),
assertDictEqual(a,b{[},msg{]})

It is possible to check if Exceptions, Warnings or Logs were invoked *
assertRaises(exc, fun, *args, **kwds) * fun(*args,**kwds) raises
exception exc * assertRaisesRegex(exc, fun, *args, **kwds) *
fun(*args,**kwds) raises exception exc and message matches regex r *
assertWarns(warn, fun, *args, **kwds) * fun(*args,**kwds) raises warning
warn * assertWarnsRegex() * fun(*args,**kwds) raises warning warn and
message matches regex r * assertLogs(logger,level) -- TODO * the with
block logs on logger with minimum level

    \begin{tcolorbox}[breakable, size=fbox, boxrule=1pt, pad at break*=1mm,colback=cellbackground, colframe=cellborder]
\prompt{In}{incolor}{65}{\boxspacing}
\begin{Verbatim}[commandchars=\\\{\}]
\PY{o}{\PYZpc{}}\PY{k}{reset} \PYZhy{}f 
\PY{k+kn}{import} \PY{n+nn}{unittest}

\PY{k}{def} \PY{n+nf}{fun\PYZus{}raising\PYZus{}TypeError}\PY{p}{(}\PY{p}{)}\PY{p}{:} \PY{k}{raise} \PY{n+ne}{TypeError}\PY{p}{(}\PY{l+s+s1}{\PYZsq{}}\PY{l+s+s1}{type error}\PY{l+s+s1}{\PYZsq{}}\PY{p}{)}
\PY{k}{def} \PY{n+nf}{fun\PYZus{}raising\PYZus{}WarningMsg}\PY{p}{(}\PY{p}{)}\PY{p}{:} \PY{k}{raise} \PY{n+ne}{Warning}\PY{p}{(}\PY{l+s+s1}{\PYZsq{}}\PY{l+s+s1}{warning message}\PY{l+s+s1}{\PYZsq{}}\PY{p}{)}

\PY{k}{class} \PY{n+nc}{Test}\PY{p}{(}\PY{n}{unittest}\PY{o}{.}\PY{n}{TestCase}\PY{p}{)}\PY{p}{:}
    \PY{k}{def} \PY{n+nf}{test1}\PY{p}{(}\PY{n+nb+bp}{self}\PY{p}{)}\PY{p}{:} 
        \PY{n+nb+bp}{self}\PY{o}{.}\PY{n}{assertRaises}\PY{p}{(}\PY{n+ne}{TypeError}\PY{p}{,}\PY{n}{fun\PYZus{}raising\PYZus{}TypeError}\PY{p}{)}
    \PY{k}{def} \PY{n+nf}{test2}\PY{p}{(}\PY{n+nb+bp}{self}\PY{p}{)}\PY{p}{:}
        \PY{n+nb+bp}{self}\PY{o}{.}\PY{n}{assertRaises}\PY{p}{(}\PY{n+ne}{Warning}\PY{p}{,}\PY{n}{fun\PYZus{}raising\PYZus{}WarningMsg}\PY{p}{)}
    \PY{k}{def} \PY{n+nf}{test3}\PY{p}{(}\PY{n+nb+bp}{self}\PY{p}{)}\PY{p}{:} \PY{c+c1}{\PYZsh{} should fail}
        \PY{n+nb+bp}{self}\PY{o}{.}\PY{n}{assertRaises}\PY{p}{(}\PY{n+ne}{TypeError}\PY{p}{,}\PY{n}{fun\PYZus{}raising\PYZus{}WarningMsg}\PY{p}{)}

\PY{n+nb}{print}\PY{p}{(}\PY{n}{unittest}\PY{o}{.}\PY{n}{main}\PY{p}{(}\PY{n}{argv}\PY{o}{=}\PY{p}{[}\PY{l+s+s1}{\PYZsq{}}\PY{l+s+s1}{first\PYZhy{}arg\PYZhy{}is\PYZhy{}ignored}\PY{l+s+s1}{\PYZsq{}}\PY{p}{]}\PY{p}{,} \PY{n}{exit}\PY{o}{=}\PY{k+kc}{False}\PY{p}{)}\PY{p}{)}                        
\end{Verbatim}
\end{tcolorbox}

    \begin{Verbatim}[commandchars=\\\{\}]
..E
    \end{Verbatim}

    \begin{Verbatim}[commandchars=\\\{\}]
<unittest.main.TestProgram object at 0x7f796036d278>
    \end{Verbatim}

    \begin{Verbatim}[commandchars=\\\{\}]

======================================================================
ERROR: test3 (\_\_main\_\_.Test)
----------------------------------------------------------------------
Traceback (most recent call last):
  File "<ipython-input-65-a467d41f55bd>", line 13, in test3
    self.assertRaises(TypeError,fun\_raising\_WarningMsg)
  File "/home/jaky/miniconda3/envs/py36/lib/python3.6/unittest/case.py", line
733, in assertRaises
    return context.handle('assertRaises', args, kwargs)
  File "/home/jaky/miniconda3/envs/py36/lib/python3.6/unittest/case.py", line
178, in handle
    callable\_obj(*args, **kwargs)
  File "<ipython-input-65-a467d41f55bd>", line 5, in fun\_raising\_WarningMsg
    def fun\_raising\_WarningMsg(): raise Warning('warning message')
Warning: warning message

----------------------------------------------------------------------
Ran 3 tests in 0.015s

FAILED (errors=1)
    \end{Verbatim}

    \hypertarget{skip-decorators}{%
\subsection{Skip decorators}\label{skip-decorators}}

Tests usefulness could be conditioned to conditions (e.g., platforms,
libraries' version, \ldots) * @unittest.skip(reason): skips the test *
@unittest.skipIf(condition, reason): skips if condition is True *
@unittest.skipUnless(condition, reason): skips unless condition is True
* @unittest.expectedFailure: marks (but does not count) as failure *
exception unittest.SkipTest(reason): to skip from inside the test

    \begin{tcolorbox}[breakable, size=fbox, boxrule=1pt, pad at break*=1mm,colback=cellbackground, colframe=cellborder]
\prompt{In}{incolor}{678}{\boxspacing}
\begin{Verbatim}[commandchars=\\\{\}]
\PY{o}{\PYZpc{}}\PY{k}{reset} \PYZhy{}f
\PY{k+kn}{import} \PY{n+nn}{unittest}

\PY{k}{class} \PY{n+nc}{Example}\PY{p}{(}\PY{n}{unittest}\PY{o}{.}\PY{n}{TestCase}\PY{p}{)}\PY{p}{:}
    \PY{n}{trueCondition} \PY{o}{=} \PY{k+kc}{True}
    \PY{n}{falseCondition} \PY{o}{=} \PY{k+kc}{False}
    \PY{n+nd}{@unittest}\PY{o}{.}\PY{n}{skipIf}\PY{p}{(}\PY{n}{trueCondition}\PY{p}{,}\PYZbs{}
                     \PY{l+s+s2}{\PYZdq{}}\PY{l+s+s2}{good reason to skip test1}\PY{l+s+s2}{\PYZdq{}}\PY{p}{)}
    \PY{k}{def} \PY{n+nf}{test1}\PY{p}{(}\PY{n+nb+bp}{self}\PY{p}{)}\PY{p}{:} \PY{k}{pass}
    \PY{n+nd}{@unittest}\PY{o}{.}\PY{n}{skipUnless}\PY{p}{(}\PY{n}{falseCondition}\PY{p}{,}\PYZbs{}
                         \PY{l+s+s2}{\PYZdq{}}\PY{l+s+s2}{good reason to skip test2}\PY{l+s+s2}{\PYZdq{}}\PY{p}{)}
    \PY{k}{def} \PY{n+nf}{test2}\PY{p}{(}\PY{n+nb+bp}{self}\PY{p}{)}\PY{p}{:} \PY{k}{pass}    
    \PY{k}{def} \PY{n+nf}{test3}\PY{p}{(}\PY{n+nb+bp}{self}\PY{p}{)}\PY{p}{:} 
        \PY{n+nb}{print}\PY{p}{(}\PY{l+s+s2}{\PYZdq{}}\PY{l+s+s2}{test3}\PY{l+s+s2}{\PYZdq{}}\PY{p}{)}
        \PY{k}{if} \PY{k+kc}{True}\PY{p}{:} \PY{k}{raise} \PY{n}{unittest}\PY{o}{.}\PY{n}{SkipTest}\PY{p}{(}\PY{l+s+s2}{\PYZdq{}}\PY{l+s+s2}{this makes test3 useless}\PY{l+s+s2}{\PYZdq{}}\PY{p}{)}
        \PY{n+nb}{print}\PY{p}{(}\PY{l+s+s2}{\PYZdq{}}\PY{l+s+s2}{not going to be printed}\PY{l+s+s2}{\PYZdq{}}\PY{p}{)}
    
\PY{n+nb}{print}\PY{p}{(}\PY{n}{unittest}\PY{o}{.}\PY{n}{main}\PY{p}{(}\PY{n}{argv}\PY{o}{=}\PY{p}{[}\PY{l+s+s1}{\PYZsq{}}\PY{l+s+s1}{first\PYZhy{}arg\PYZhy{}is\PYZhy{}ignored}\PY{l+s+s1}{\PYZsq{}}\PY{p}{]}\PY{p}{,} \PY{n}{exit}\PY{o}{=}\PY{k+kc}{False}\PY{p}{)}\PY{p}{)}
\end{Verbatim}
\end{tcolorbox}

    \begin{Verbatim}[commandchars=\\\{\}]
sss
    \end{Verbatim}

    \begin{Verbatim}[commandchars=\\\{\}]
test3
<unittest.main.TestProgram object at 0x7f4b5ff04668>
    \end{Verbatim}

    \begin{Verbatim}[commandchars=\\\{\}]

----------------------------------------------------------------------
Ran 3 tests in 0.002s

OK (skipped=3)
    \end{Verbatim}

    \hypertarget{iterate-tests-using-subtests}{%
\subsection{Iterate tests using
subtests}\label{iterate-tests-using-subtests}}

\begin{itemize}
\tightlist
\item
  you can distinguish similar tests inside the body of the same test
  method using subTest
\item
  subTest({[}msg=None,**params{]}): test can be nested as context
  manager
\end{itemize}

    \begin{tcolorbox}[breakable, size=fbox, boxrule=1pt, pad at break*=1mm,colback=cellbackground, colframe=cellborder]
\prompt{In}{incolor}{6}{\boxspacing}
\begin{Verbatim}[commandchars=\\\{\}]
\PY{o}{\PYZpc{}}\PY{k}{reset} \PYZhy{}f 
\PY{k+kn}{import} \PY{n+nn}{unittest}

\PY{k}{global} \PY{n}{ListOfIntegers} 
\PY{n}{ListOfIntegers} \PY{o}{=} \PY{p}{[}\PY{l+m+mi}{1}\PY{p}{,}\PY{l+m+mi}{2}\PY{p}{,}\PY{l+m+mi}{3}\PY{p}{,}\PY{l+m+mi}{4}\PY{p}{,}\PY{l+m+mi}{5}\PY{p}{,}\PY{l+m+mi}{6}\PY{p}{,}\PY{l+s+s1}{\PYZsq{}}\PY{l+s+s1}{7}\PY{l+s+s1}{\PYZsq{}}\PY{p}{,}\PY{l+s+s1}{\PYZsq{}}\PY{l+s+s1}{8}\PY{l+s+s1}{\PYZsq{}}\PY{p}{,}\PY{l+s+s1}{\PYZsq{}}\PY{l+s+s1}{9}\PY{l+s+s1}{\PYZsq{}}\PY{p}{]} \PYZbs{}
                            \PY{c+c1}{\PYZsh{} \PYZsq{}7\PYZsq{} and}
                            \PY{c+c1}{\PYZsh{}     \PYZsq{}8\PYZsq{} and }
                            \PY{c+c1}{\PYZsh{}         \PYZsq{}9\PYZsq{} are not integers}

\PY{k}{class} \PY{n+nc}{Test}\PY{p}{(}\PY{n}{unittest}\PY{o}{.}\PY{n}{TestCase}\PY{p}{)}\PY{p}{:}
    \PY{k}{def} \PY{n+nf}{test\PYZus{}list}\PY{p}{(}\PY{n+nb+bp}{self}\PY{p}{)}\PY{p}{:} 
        \PY{k}{for} \PY{n}{i} \PY{o+ow}{in} \PY{n}{ListOfIntegers}\PY{p}{:}
            \PY{k}{with} \PY{n+nb+bp}{self}\PY{o}{.}\PY{n}{subTest}\PY{p}{(}\PY{n}{value}\PY{o}{=}\PY{n}{i}\PY{p}{)}\PY{p}{:}
                \PY{n+nb+bp}{self}\PY{o}{.}\PY{n}{assertEqual}\PY{p}{(}\PY{n+nb}{type}\PY{p}{(}\PY{n}{i}\PY{p}{)}\PY{p}{,} \PY{n+nb}{type}\PY{p}{(}\PY{l+m+mi}{1}\PY{p}{)}\PY{p}{)}

\PY{n+nb}{print}\PY{p}{(}\PY{n}{unittest}\PY{o}{.}\PY{n}{main}\PY{p}{(}\PY{n}{argv}\PY{o}{=}\PY{p}{[}\PY{l+s+s1}{\PYZsq{}}\PY{l+s+s1}{first\PYZhy{}arg\PYZhy{}is\PYZhy{}ignored}\PY{l+s+s1}{\PYZsq{}}\PY{p}{]}\PY{p}{,} \PY{n}{exit}\PY{o}{=}\PY{k+kc}{False}\PY{p}{)}\PY{p}{)}                
\end{Verbatim}
\end{tcolorbox}

    \begin{Verbatim}[commandchars=\\\{\}]
<unittest.main.TestProgram object at 0x7f99940d9128>
    \end{Verbatim}

    \begin{Verbatim}[commandchars=\\\{\}]

======================================================================
FAIL: test\_list (\_\_main\_\_.Test) (value='7')
----------------------------------------------------------------------
Traceback (most recent call last):
  File "<ipython-input-6-4c54590060b2>", line 13, in test\_list
    self.assertEqual(type(i), type(1))
AssertionError: <class 'str'> != <class 'int'>

======================================================================
FAIL: test\_list (\_\_main\_\_.Test) (value='8')
----------------------------------------------------------------------
Traceback (most recent call last):
  File "<ipython-input-6-4c54590060b2>", line 13, in test\_list
    self.assertEqual(type(i), type(1))
AssertionError: <class 'str'> != <class 'int'>

======================================================================
FAIL: test\_list (\_\_main\_\_.Test) (value='9')
----------------------------------------------------------------------
Traceback (most recent call last):
  File "<ipython-input-6-4c54590060b2>", line 13, in test\_list
    self.assertEqual(type(i), type(1))
AssertionError: <class 'str'> != <class 'int'>

----------------------------------------------------------------------
Ran 1 test in 0.001s

FAILED (failures=3)
    \end{Verbatim}

    \hypertarget{exercises}{%
\subsection{Exercises}\label{exercises}}


    % Add a bibliography block to the postdoc
    
    
    
\end{document}
