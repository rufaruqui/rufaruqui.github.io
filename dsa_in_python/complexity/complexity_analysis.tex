\documentclass[11pt]{article}

    \usepackage[breakable]{tcolorbox}
    \usepackage{parskip} % Stop auto-indenting (to mimic markdown behaviour)
    
    \usepackage{iftex}
    \ifPDFTeX
    	\usepackage[T1]{fontenc}
    	\usepackage{mathpazo}
    \else
    	\usepackage{fontspec}
    \fi

    % Basic figure setup, for now with no caption control since it's done
    % automatically by Pandoc (which extracts ![](path) syntax from Markdown).
    \usepackage{graphicx}
    % Maintain compatibility with old templates. Remove in nbconvert 6.0
    \let\Oldincludegraphics\includegraphics
    % Ensure that by default, figures have no caption (until we provide a
    % proper Figure object with a Caption API and a way to capture that
    % in the conversion process - todo).
    \usepackage{caption}
    \DeclareCaptionFormat{nocaption}{}
    \captionsetup{format=nocaption,aboveskip=0pt,belowskip=0pt}

    \usepackage[Export]{adjustbox} % Used to constrain images to a maximum size
    \adjustboxset{max size={0.9\linewidth}{0.9\paperheight}}
    \usepackage{float}
    \floatplacement{figure}{H} % forces figures to be placed at the correct location
    \usepackage{xcolor} % Allow colors to be defined
    \usepackage{enumerate} % Needed for markdown enumerations to work
    \usepackage{geometry} % Used to adjust the document margins
    \usepackage{amsmath} % Equations
    \usepackage{amssymb} % Equations
    \usepackage{textcomp} % defines textquotesingle
    % Hack from http://tex.stackexchange.com/a/47451/13684:
    \AtBeginDocument{%
        \def\PYZsq{\textquotesingle}% Upright quotes in Pygmentized code
    }
    \usepackage{upquote} % Upright quotes for verbatim code
    \usepackage{eurosym} % defines \euro
    \usepackage[mathletters]{ucs} % Extended unicode (utf-8) support
    \usepackage{fancyvrb} % verbatim replacement that allows latex
    \usepackage{grffile} % extends the file name processing of package graphics 
                         % to support a larger range
    \makeatletter % fix for grffile with XeLaTeX
    \def\Gread@@xetex#1{%
      \IfFileExists{"\Gin@base".bb}%
      {\Gread@eps{\Gin@base.bb}}%
      {\Gread@@xetex@aux#1}%
    }
    \makeatother

    % The hyperref package gives us a pdf with properly built
    % internal navigation ('pdf bookmarks' for the table of contents,
    % internal cross-reference links, web links for URLs, etc.)
    \usepackage{hyperref}
    % The default LaTeX title has an obnoxious amount of whitespace. By default,
    % titling removes some of it. It also provides customization options.
    \usepackage{titling}
    \usepackage{longtable} % longtable support required by pandoc >1.10
    \usepackage{booktabs}  % table support for pandoc > 1.12.2
    \usepackage[inline]{enumitem} % IRkernel/repr support (it uses the enumerate* environment)
    \usepackage[normalem]{ulem} % ulem is needed to support strikethroughs (\sout)
                                % normalem makes italics be italics, not underlines
    \usepackage{mathrsfs}
    

    
    % Colors for the hyperref package
    \definecolor{urlcolor}{rgb}{0,.145,.698}
    \definecolor{linkcolor}{rgb}{.71,0.21,0.01}
    \definecolor{citecolor}{rgb}{.12,.54,.11}

    % ANSI colors
    \definecolor{ansi-black}{HTML}{3E424D}
    \definecolor{ansi-black-intense}{HTML}{282C36}
    \definecolor{ansi-red}{HTML}{E75C58}
    \definecolor{ansi-red-intense}{HTML}{B22B31}
    \definecolor{ansi-green}{HTML}{00A250}
    \definecolor{ansi-green-intense}{HTML}{007427}
    \definecolor{ansi-yellow}{HTML}{DDB62B}
    \definecolor{ansi-yellow-intense}{HTML}{B27D12}
    \definecolor{ansi-blue}{HTML}{208FFB}
    \definecolor{ansi-blue-intense}{HTML}{0065CA}
    \definecolor{ansi-magenta}{HTML}{D160C4}
    \definecolor{ansi-magenta-intense}{HTML}{A03196}
    \definecolor{ansi-cyan}{HTML}{60C6C8}
    \definecolor{ansi-cyan-intense}{HTML}{258F8F}
    \definecolor{ansi-white}{HTML}{C5C1B4}
    \definecolor{ansi-white-intense}{HTML}{A1A6B2}
    \definecolor{ansi-default-inverse-fg}{HTML}{FFFFFF}
    \definecolor{ansi-default-inverse-bg}{HTML}{000000}

    % commands and environments needed by pandoc snippets
    % extracted from the output of `pandoc -s`
    \providecommand{\tightlist}{%
      \setlength{\itemsep}{0pt}\setlength{\parskip}{0pt}}
    \DefineVerbatimEnvironment{Highlighting}{Verbatim}{commandchars=\\\{\}}
    % Add ',fontsize=\small' for more characters per line
    \newenvironment{Shaded}{}{}
    \newcommand{\KeywordTok}[1]{\textcolor[rgb]{0.00,0.44,0.13}{\textbf{{#1}}}}
    \newcommand{\DataTypeTok}[1]{\textcolor[rgb]{0.56,0.13,0.00}{{#1}}}
    \newcommand{\DecValTok}[1]{\textcolor[rgb]{0.25,0.63,0.44}{{#1}}}
    \newcommand{\BaseNTok}[1]{\textcolor[rgb]{0.25,0.63,0.44}{{#1}}}
    \newcommand{\FloatTok}[1]{\textcolor[rgb]{0.25,0.63,0.44}{{#1}}}
    \newcommand{\CharTok}[1]{\textcolor[rgb]{0.25,0.44,0.63}{{#1}}}
    \newcommand{\StringTok}[1]{\textcolor[rgb]{0.25,0.44,0.63}{{#1}}}
    \newcommand{\CommentTok}[1]{\textcolor[rgb]{0.38,0.63,0.69}{\textit{{#1}}}}
    \newcommand{\OtherTok}[1]{\textcolor[rgb]{0.00,0.44,0.13}{{#1}}}
    \newcommand{\AlertTok}[1]{\textcolor[rgb]{1.00,0.00,0.00}{\textbf{{#1}}}}
    \newcommand{\FunctionTok}[1]{\textcolor[rgb]{0.02,0.16,0.49}{{#1}}}
    \newcommand{\RegionMarkerTok}[1]{{#1}}
    \newcommand{\ErrorTok}[1]{\textcolor[rgb]{1.00,0.00,0.00}{\textbf{{#1}}}}
    \newcommand{\NormalTok}[1]{{#1}}
    
    % Additional commands for more recent versions of Pandoc
    \newcommand{\ConstantTok}[1]{\textcolor[rgb]{0.53,0.00,0.00}{{#1}}}
    \newcommand{\SpecialCharTok}[1]{\textcolor[rgb]{0.25,0.44,0.63}{{#1}}}
    \newcommand{\VerbatimStringTok}[1]{\textcolor[rgb]{0.25,0.44,0.63}{{#1}}}
    \newcommand{\SpecialStringTok}[1]{\textcolor[rgb]{0.73,0.40,0.53}{{#1}}}
    \newcommand{\ImportTok}[1]{{#1}}
    \newcommand{\DocumentationTok}[1]{\textcolor[rgb]{0.73,0.13,0.13}{\textit{{#1}}}}
    \newcommand{\AnnotationTok}[1]{\textcolor[rgb]{0.38,0.63,0.69}{\textbf{\textit{{#1}}}}}
    \newcommand{\CommentVarTok}[1]{\textcolor[rgb]{0.38,0.63,0.69}{\textbf{\textit{{#1}}}}}
    \newcommand{\VariableTok}[1]{\textcolor[rgb]{0.10,0.09,0.49}{{#1}}}
    \newcommand{\ControlFlowTok}[1]{\textcolor[rgb]{0.00,0.44,0.13}{\textbf{{#1}}}}
    \newcommand{\OperatorTok}[1]{\textcolor[rgb]{0.40,0.40,0.40}{{#1}}}
    \newcommand{\BuiltInTok}[1]{{#1}}
    \newcommand{\ExtensionTok}[1]{{#1}}
    \newcommand{\PreprocessorTok}[1]{\textcolor[rgb]{0.74,0.48,0.00}{{#1}}}
    \newcommand{\AttributeTok}[1]{\textcolor[rgb]{0.49,0.56,0.16}{{#1}}}
    \newcommand{\InformationTok}[1]{\textcolor[rgb]{0.38,0.63,0.69}{\textbf{\textit{{#1}}}}}
    \newcommand{\WarningTok}[1]{\textcolor[rgb]{0.38,0.63,0.69}{\textbf{\textit{{#1}}}}}
    
    
    % Define a nice break command that doesn't care if a line doesn't already
    % exist.
    \def\br{\hspace*{\fill} \\* }
    % Math Jax compatibility definitions
    \def\gt{>}
    \def\lt{<}
    \let\Oldtex\TeX
    \let\Oldlatex\LaTeX
    \renewcommand{\TeX}{\textrm{\Oldtex}}
    \renewcommand{\LaTeX}{\textrm{\Oldlatex}}
    % Document parameters
    % Document title
    \title{complexity\_analysis}
    
    
    
    
    
% Pygments definitions
\makeatletter
\def\PY@reset{\let\PY@it=\relax \let\PY@bf=\relax%
    \let\PY@ul=\relax \let\PY@tc=\relax%
    \let\PY@bc=\relax \let\PY@ff=\relax}
\def\PY@tok#1{\csname PY@tok@#1\endcsname}
\def\PY@toks#1+{\ifx\relax#1\empty\else%
    \PY@tok{#1}\expandafter\PY@toks\fi}
\def\PY@do#1{\PY@bc{\PY@tc{\PY@ul{%
    \PY@it{\PY@bf{\PY@ff{#1}}}}}}}
\def\PY#1#2{\PY@reset\PY@toks#1+\relax+\PY@do{#2}}

\@namedef{PY@tok@w}{\def\PY@tc##1{\textcolor[rgb]{0.73,0.73,0.73}{##1}}}
\@namedef{PY@tok@c}{\let\PY@it=\textit\def\PY@tc##1{\textcolor[rgb]{0.25,0.50,0.50}{##1}}}
\@namedef{PY@tok@cp}{\def\PY@tc##1{\textcolor[rgb]{0.74,0.48,0.00}{##1}}}
\@namedef{PY@tok@k}{\let\PY@bf=\textbf\def\PY@tc##1{\textcolor[rgb]{0.00,0.50,0.00}{##1}}}
\@namedef{PY@tok@kp}{\def\PY@tc##1{\textcolor[rgb]{0.00,0.50,0.00}{##1}}}
\@namedef{PY@tok@kt}{\def\PY@tc##1{\textcolor[rgb]{0.69,0.00,0.25}{##1}}}
\@namedef{PY@tok@o}{\def\PY@tc##1{\textcolor[rgb]{0.40,0.40,0.40}{##1}}}
\@namedef{PY@tok@ow}{\let\PY@bf=\textbf\def\PY@tc##1{\textcolor[rgb]{0.67,0.13,1.00}{##1}}}
\@namedef{PY@tok@nb}{\def\PY@tc##1{\textcolor[rgb]{0.00,0.50,0.00}{##1}}}
\@namedef{PY@tok@nf}{\def\PY@tc##1{\textcolor[rgb]{0.00,0.00,1.00}{##1}}}
\@namedef{PY@tok@nc}{\let\PY@bf=\textbf\def\PY@tc##1{\textcolor[rgb]{0.00,0.00,1.00}{##1}}}
\@namedef{PY@tok@nn}{\let\PY@bf=\textbf\def\PY@tc##1{\textcolor[rgb]{0.00,0.00,1.00}{##1}}}
\@namedef{PY@tok@ne}{\let\PY@bf=\textbf\def\PY@tc##1{\textcolor[rgb]{0.82,0.25,0.23}{##1}}}
\@namedef{PY@tok@nv}{\def\PY@tc##1{\textcolor[rgb]{0.10,0.09,0.49}{##1}}}
\@namedef{PY@tok@no}{\def\PY@tc##1{\textcolor[rgb]{0.53,0.00,0.00}{##1}}}
\@namedef{PY@tok@nl}{\def\PY@tc##1{\textcolor[rgb]{0.63,0.63,0.00}{##1}}}
\@namedef{PY@tok@ni}{\let\PY@bf=\textbf\def\PY@tc##1{\textcolor[rgb]{0.60,0.60,0.60}{##1}}}
\@namedef{PY@tok@na}{\def\PY@tc##1{\textcolor[rgb]{0.49,0.56,0.16}{##1}}}
\@namedef{PY@tok@nt}{\let\PY@bf=\textbf\def\PY@tc##1{\textcolor[rgb]{0.00,0.50,0.00}{##1}}}
\@namedef{PY@tok@nd}{\def\PY@tc##1{\textcolor[rgb]{0.67,0.13,1.00}{##1}}}
\@namedef{PY@tok@s}{\def\PY@tc##1{\textcolor[rgb]{0.73,0.13,0.13}{##1}}}
\@namedef{PY@tok@sd}{\let\PY@it=\textit\def\PY@tc##1{\textcolor[rgb]{0.73,0.13,0.13}{##1}}}
\@namedef{PY@tok@si}{\let\PY@bf=\textbf\def\PY@tc##1{\textcolor[rgb]{0.73,0.40,0.53}{##1}}}
\@namedef{PY@tok@se}{\let\PY@bf=\textbf\def\PY@tc##1{\textcolor[rgb]{0.73,0.40,0.13}{##1}}}
\@namedef{PY@tok@sr}{\def\PY@tc##1{\textcolor[rgb]{0.73,0.40,0.53}{##1}}}
\@namedef{PY@tok@ss}{\def\PY@tc##1{\textcolor[rgb]{0.10,0.09,0.49}{##1}}}
\@namedef{PY@tok@sx}{\def\PY@tc##1{\textcolor[rgb]{0.00,0.50,0.00}{##1}}}
\@namedef{PY@tok@m}{\def\PY@tc##1{\textcolor[rgb]{0.40,0.40,0.40}{##1}}}
\@namedef{PY@tok@gh}{\let\PY@bf=\textbf\def\PY@tc##1{\textcolor[rgb]{0.00,0.00,0.50}{##1}}}
\@namedef{PY@tok@gu}{\let\PY@bf=\textbf\def\PY@tc##1{\textcolor[rgb]{0.50,0.00,0.50}{##1}}}
\@namedef{PY@tok@gd}{\def\PY@tc##1{\textcolor[rgb]{0.63,0.00,0.00}{##1}}}
\@namedef{PY@tok@gi}{\def\PY@tc##1{\textcolor[rgb]{0.00,0.63,0.00}{##1}}}
\@namedef{PY@tok@gr}{\def\PY@tc##1{\textcolor[rgb]{1.00,0.00,0.00}{##1}}}
\@namedef{PY@tok@ge}{\let\PY@it=\textit}
\@namedef{PY@tok@gs}{\let\PY@bf=\textbf}
\@namedef{PY@tok@gp}{\let\PY@bf=\textbf\def\PY@tc##1{\textcolor[rgb]{0.00,0.00,0.50}{##1}}}
\@namedef{PY@tok@go}{\def\PY@tc##1{\textcolor[rgb]{0.53,0.53,0.53}{##1}}}
\@namedef{PY@tok@gt}{\def\PY@tc##1{\textcolor[rgb]{0.00,0.27,0.87}{##1}}}
\@namedef{PY@tok@err}{\def\PY@bc##1{{\setlength{\fboxsep}{\string -\fboxrule}\fcolorbox[rgb]{1.00,0.00,0.00}{1,1,1}{\strut ##1}}}}
\@namedef{PY@tok@kc}{\let\PY@bf=\textbf\def\PY@tc##1{\textcolor[rgb]{0.00,0.50,0.00}{##1}}}
\@namedef{PY@tok@kd}{\let\PY@bf=\textbf\def\PY@tc##1{\textcolor[rgb]{0.00,0.50,0.00}{##1}}}
\@namedef{PY@tok@kn}{\let\PY@bf=\textbf\def\PY@tc##1{\textcolor[rgb]{0.00,0.50,0.00}{##1}}}
\@namedef{PY@tok@kr}{\let\PY@bf=\textbf\def\PY@tc##1{\textcolor[rgb]{0.00,0.50,0.00}{##1}}}
\@namedef{PY@tok@bp}{\def\PY@tc##1{\textcolor[rgb]{0.00,0.50,0.00}{##1}}}
\@namedef{PY@tok@fm}{\def\PY@tc##1{\textcolor[rgb]{0.00,0.00,1.00}{##1}}}
\@namedef{PY@tok@vc}{\def\PY@tc##1{\textcolor[rgb]{0.10,0.09,0.49}{##1}}}
\@namedef{PY@tok@vg}{\def\PY@tc##1{\textcolor[rgb]{0.10,0.09,0.49}{##1}}}
\@namedef{PY@tok@vi}{\def\PY@tc##1{\textcolor[rgb]{0.10,0.09,0.49}{##1}}}
\@namedef{PY@tok@vm}{\def\PY@tc##1{\textcolor[rgb]{0.10,0.09,0.49}{##1}}}
\@namedef{PY@tok@sa}{\def\PY@tc##1{\textcolor[rgb]{0.73,0.13,0.13}{##1}}}
\@namedef{PY@tok@sb}{\def\PY@tc##1{\textcolor[rgb]{0.73,0.13,0.13}{##1}}}
\@namedef{PY@tok@sc}{\def\PY@tc##1{\textcolor[rgb]{0.73,0.13,0.13}{##1}}}
\@namedef{PY@tok@dl}{\def\PY@tc##1{\textcolor[rgb]{0.73,0.13,0.13}{##1}}}
\@namedef{PY@tok@s2}{\def\PY@tc##1{\textcolor[rgb]{0.73,0.13,0.13}{##1}}}
\@namedef{PY@tok@sh}{\def\PY@tc##1{\textcolor[rgb]{0.73,0.13,0.13}{##1}}}
\@namedef{PY@tok@s1}{\def\PY@tc##1{\textcolor[rgb]{0.73,0.13,0.13}{##1}}}
\@namedef{PY@tok@mb}{\def\PY@tc##1{\textcolor[rgb]{0.40,0.40,0.40}{##1}}}
\@namedef{PY@tok@mf}{\def\PY@tc##1{\textcolor[rgb]{0.40,0.40,0.40}{##1}}}
\@namedef{PY@tok@mh}{\def\PY@tc##1{\textcolor[rgb]{0.40,0.40,0.40}{##1}}}
\@namedef{PY@tok@mi}{\def\PY@tc##1{\textcolor[rgb]{0.40,0.40,0.40}{##1}}}
\@namedef{PY@tok@il}{\def\PY@tc##1{\textcolor[rgb]{0.40,0.40,0.40}{##1}}}
\@namedef{PY@tok@mo}{\def\PY@tc##1{\textcolor[rgb]{0.40,0.40,0.40}{##1}}}
\@namedef{PY@tok@ch}{\let\PY@it=\textit\def\PY@tc##1{\textcolor[rgb]{0.25,0.50,0.50}{##1}}}
\@namedef{PY@tok@cm}{\let\PY@it=\textit\def\PY@tc##1{\textcolor[rgb]{0.25,0.50,0.50}{##1}}}
\@namedef{PY@tok@cpf}{\let\PY@it=\textit\def\PY@tc##1{\textcolor[rgb]{0.25,0.50,0.50}{##1}}}
\@namedef{PY@tok@c1}{\let\PY@it=\textit\def\PY@tc##1{\textcolor[rgb]{0.25,0.50,0.50}{##1}}}
\@namedef{PY@tok@cs}{\let\PY@it=\textit\def\PY@tc##1{\textcolor[rgb]{0.25,0.50,0.50}{##1}}}

\def\PYZbs{\char`\\}
\def\PYZus{\char`\_}
\def\PYZob{\char`\{}
\def\PYZcb{\char`\}}
\def\PYZca{\char`\^}
\def\PYZam{\char`\&}
\def\PYZlt{\char`\<}
\def\PYZgt{\char`\>}
\def\PYZsh{\char`\#}
\def\PYZpc{\char`\%}
\def\PYZdl{\char`\$}
\def\PYZhy{\char`\-}
\def\PYZsq{\char`\'}
\def\PYZdq{\char`\"}
\def\PYZti{\char`\~}
% for compatibility with earlier versions
\def\PYZat{@}
\def\PYZlb{[}
\def\PYZrb{]}
\makeatother


    % For linebreaks inside Verbatim environment from package fancyvrb. 
    \makeatletter
        \newbox\Wrappedcontinuationbox 
        \newbox\Wrappedvisiblespacebox 
        \newcommand*\Wrappedvisiblespace {\textcolor{red}{\textvisiblespace}} 
        \newcommand*\Wrappedcontinuationsymbol {\textcolor{red}{\llap{\tiny$\m@th\hookrightarrow$}}} 
        \newcommand*\Wrappedcontinuationindent {3ex } 
        \newcommand*\Wrappedafterbreak {\kern\Wrappedcontinuationindent\copy\Wrappedcontinuationbox} 
        % Take advantage of the already applied Pygments mark-up to insert 
        % potential linebreaks for TeX processing. 
        %        {, <, #, %, $, ' and ": go to next line. 
        %        _, }, ^, &, >, - and ~: stay at end of broken line. 
        % Use of \textquotesingle for straight quote. 
        \newcommand*\Wrappedbreaksatspecials {% 
            \def\PYGZus{\discretionary{\char`\_}{\Wrappedafterbreak}{\char`\_}}% 
            \def\PYGZob{\discretionary{}{\Wrappedafterbreak\char`\{}{\char`\{}}% 
            \def\PYGZcb{\discretionary{\char`\}}{\Wrappedafterbreak}{\char`\}}}% 
            \def\PYGZca{\discretionary{\char`\^}{\Wrappedafterbreak}{\char`\^}}% 
            \def\PYGZam{\discretionary{\char`\&}{\Wrappedafterbreak}{\char`\&}}% 
            \def\PYGZlt{\discretionary{}{\Wrappedafterbreak\char`\<}{\char`\<}}% 
            \def\PYGZgt{\discretionary{\char`\>}{\Wrappedafterbreak}{\char`\>}}% 
            \def\PYGZsh{\discretionary{}{\Wrappedafterbreak\char`\#}{\char`\#}}% 
            \def\PYGZpc{\discretionary{}{\Wrappedafterbreak\char`\%}{\char`\%}}% 
            \def\PYGZdl{\discretionary{}{\Wrappedafterbreak\char`\$}{\char`\$}}% 
            \def\PYGZhy{\discretionary{\char`\-}{\Wrappedafterbreak}{\char`\-}}% 
            \def\PYGZsq{\discretionary{}{\Wrappedafterbreak\textquotesingle}{\textquotesingle}}% 
            \def\PYGZdq{\discretionary{}{\Wrappedafterbreak\char`\"}{\char`\"}}% 
            \def\PYGZti{\discretionary{\char`\~}{\Wrappedafterbreak}{\char`\~}}% 
        } 
        % Some characters . , ; ? ! / are not pygmentized. 
        % This macro makes them "active" and they will insert potential linebreaks 
        \newcommand*\Wrappedbreaksatpunct {% 
            \lccode`\~`\.\lowercase{\def~}{\discretionary{\hbox{\char`\.}}{\Wrappedafterbreak}{\hbox{\char`\.}}}% 
            \lccode`\~`\,\lowercase{\def~}{\discretionary{\hbox{\char`\,}}{\Wrappedafterbreak}{\hbox{\char`\,}}}% 
            \lccode`\~`\;\lowercase{\def~}{\discretionary{\hbox{\char`\;}}{\Wrappedafterbreak}{\hbox{\char`\;}}}% 
            \lccode`\~`\:\lowercase{\def~}{\discretionary{\hbox{\char`\:}}{\Wrappedafterbreak}{\hbox{\char`\:}}}% 
            \lccode`\~`\?\lowercase{\def~}{\discretionary{\hbox{\char`\?}}{\Wrappedafterbreak}{\hbox{\char`\?}}}% 
            \lccode`\~`\!\lowercase{\def~}{\discretionary{\hbox{\char`\!}}{\Wrappedafterbreak}{\hbox{\char`\!}}}% 
            \lccode`\~`\/\lowercase{\def~}{\discretionary{\hbox{\char`\/}}{\Wrappedafterbreak}{\hbox{\char`\/}}}% 
            \catcode`\.\active
            \catcode`\,\active 
            \catcode`\;\active
            \catcode`\:\active
            \catcode`\?\active
            \catcode`\!\active
            \catcode`\/\active 
            \lccode`\~`\~ 	
        }
    \makeatother

    \let\OriginalVerbatim=\Verbatim
    \makeatletter
    \renewcommand{\Verbatim}[1][1]{%
        %\parskip\z@skip
        \sbox\Wrappedcontinuationbox {\Wrappedcontinuationsymbol}%
        \sbox\Wrappedvisiblespacebox {\FV@SetupFont\Wrappedvisiblespace}%
        \def\FancyVerbFormatLine ##1{\hsize\linewidth
            \vtop{\raggedright\hyphenpenalty\z@\exhyphenpenalty\z@
                \doublehyphendemerits\z@\finalhyphendemerits\z@
                \strut ##1\strut}%
        }%
        % If the linebreak is at a space, the latter will be displayed as visible
        % space at end of first line, and a continuation symbol starts next line.
        % Stretch/shrink are however usually zero for typewriter font.
        \def\FV@Space {%
            \nobreak\hskip\z@ plus\fontdimen3\font minus\fontdimen4\font
            \discretionary{\copy\Wrappedvisiblespacebox}{\Wrappedafterbreak}
            {\kern\fontdimen2\font}%
        }%
        
        % Allow breaks at special characters using \PYG... macros.
        \Wrappedbreaksatspecials
        % Breaks at punctuation characters . , ; ? ! and / need catcode=\active 	
        \OriginalVerbatim[#1,codes*=\Wrappedbreaksatpunct]%
    }
    \makeatother

    % Exact colors from NB
    \definecolor{incolor}{HTML}{303F9F}
    \definecolor{outcolor}{HTML}{D84315}
    \definecolor{cellborder}{HTML}{CFCFCF}
    \definecolor{cellbackground}{HTML}{F7F7F7}
    
    % prompt
    \makeatletter
    \newcommand{\boxspacing}{\kern\kvtcb@left@rule\kern\kvtcb@boxsep}
    \makeatother
    \newcommand{\prompt}[4]{
        \ttfamily\llap{{\color{#2}[#3]:\hspace{3pt}#4}}\vspace{-\baselineskip}
    }
    

    
    % Prevent overflowing lines due to hard-to-break entities
    \sloppy 
    % Setup hyperref package
    \hypersetup{
      breaklinks=true,  % so long urls are correctly broken across lines
      colorlinks=true,
      urlcolor=urlcolor,
      linkcolor=linkcolor,
      citecolor=citecolor,
      }
    % Slightly bigger margins than the latex defaults
    
    \geometry{verbose,tmargin=1in,bmargin=1in,lmargin=1in,rmargin=1in}
    
    

\begin{document}
    
    \maketitle
    
    

    
    \hypertarget{the-analysis-of-algorithims}{%
\section{The Analysis of
Algorithims}\label{the-analysis-of-algorithims}}

\begin{itemize}
\tightlist
\item
  Primary analysis tool

  \begin{itemize}
  \tightlist
  \item
    characterizing the running times of algorithms and data structure
    operations
  \item
    Space usage\\
  \end{itemize}
\item
  Running time is a natural measure of ``goodness,''

  \begin{itemize}
  \tightlist
  \item
    time is a precious resource
  \item
    computer solutions should run as fast as possible
  \end{itemize}
\end{itemize}

    \#\# Motivation

\begin{itemize}
\item
  Sometimes it is necessary to have a precise understanding of the
  complexity of an algorithm.
\item
  In order to obtain this understanding we could proceed as follows:\\

  We implement the algorithm in a given programming language.

  We count how many additions, multiplications, assignments,
  etc.\textasciitilde are needed

  for an input of a given size. Additionally, we have to count all
  storage accesses.

  We look up the amount of time that is needed for the different
  operations in the processor handbook.

  Using the information discovered in the previous two steps we predict
  the running time of our algorithm for given input.
\end{itemize}

    \hypertarget{experimental-studies}{%
\subsubsection{Experimental Studies}\label{experimental-studies}}

\begin{Shaded}
\begin{Highlighting}[]
\ImportTok{from}\NormalTok{ time }\ImportTok{import}\NormalTok{ time}

\NormalTok{start time }\OperatorTok{=}\NormalTok{ time( ) }\CommentTok{\# record the starting time}
\NormalTok{run algorithm    }
\NormalTok{end time }\OperatorTok{=}\NormalTok{ time( )   }\CommentTok{\# record the ending time}
\NormalTok{elapsed }\OperatorTok{=}\NormalTok{ end time − start time }\CommentTok{\# compute the elapsed time}
\end{Highlighting}
\end{Shaded}

    \hypertarget{naive-approach}{%
\subsubsection{Naive Approach}\label{naive-approach}}

\(S~=~1+2+3+ ...+n\)

    \begin{tcolorbox}[breakable, size=fbox, boxrule=1pt, pad at break*=1mm,colback=cellbackground, colframe=cellborder]
\prompt{In}{incolor}{14}{\boxspacing}
\begin{Verbatim}[commandchars=\\\{\}]
\PY{k+kn}{import} \PY{n+nn}{time}


\PY{k}{def} \PY{n+nf}{sum\PYZus{}of\PYZus{}n\PYZus{}2}\PY{p}{(}\PY{n}{n}\PY{p}{)}\PY{p}{:}
    \PY{n}{start} \PY{o}{=} \PY{n}{time}\PY{o}{.}\PY{n}{time}\PY{p}{(}\PY{p}{)}

    \PY{n}{the\PYZus{}sum} \PY{o}{=} \PY{l+m+mi}{0}
    \PY{k}{for} \PY{n}{i} \PY{o+ow}{in} \PY{n+nb}{range}\PY{p}{(}\PY{l+m+mi}{1}\PY{p}{,} \PY{n}{n} \PY{o}{+} \PY{l+m+mi}{1}\PY{p}{)}\PY{p}{:}
        \PY{n}{the\PYZus{}sum} \PY{o}{=} \PY{n}{the\PYZus{}sum} \PY{o}{+} \PY{n}{i}

    \PY{n}{end} \PY{o}{=} \PY{n}{time}\PY{o}{.}\PY{n}{time}\PY{p}{(}\PY{p}{)}

    \PY{k}{return} \PY{n}{the\PYZus{}sum}\PY{p}{,} \PY{n}{end} \PY{o}{\PYZhy{}} \PY{n}{start}

\PY{k}{for} \PY{n}{i} \PY{o+ow}{in} \PY{n+nb}{range}\PY{p}{(}\PY{l+m+mi}{5}\PY{p}{)}\PY{p}{:}
    \PY{n+nb}{print}\PY{p}{(}\PY{l+s+s2}{\PYZdq{}}\PY{l+s+s2}{Sum is }\PY{l+s+si}{\PYZpc{}d}\PY{l+s+s2}{ required }\PY{l+s+si}{\PYZpc{}10.6f}\PY{l+s+s2}{ seconds}\PY{l+s+s2}{\PYZdq{}} \PY{o}{\PYZpc{}} \PY{n}{sum\PYZus{}of\PYZus{}n\PYZus{}2}\PY{p}{(}\PY{l+m+mi}{10000}\PY{p}{)}\PY{p}{)}
    
\end{Verbatim}
\end{tcolorbox}

    \begin{Verbatim}[commandchars=\\\{\}]
Sum is 50005000 required   0.000841 seconds
Sum is 50005000 required   0.001362 seconds
Sum is 50005000 required   0.001088 seconds
Sum is 50005000 required   0.001040 seconds
Sum is 50005000 required   0.001048 seconds
    \end{Verbatim}

    \hypertarget{alternative-approach}{%
\subsubsection{Alternative Approach}\label{alternative-approach}}

\(S~=~1 + 2 + 3+ ...+n\)

\(S~=~n + (n-1) + (n- 2)+ .... + 1\)

\(2S = (n+1) + (n+1)+ .... + (n+1) = n * (n +1)\)

\(S = \frac{n* (n+1)} {2}\)

    \begin{tcolorbox}[breakable, size=fbox, boxrule=1pt, pad at break*=1mm,colback=cellbackground, colframe=cellborder]
\prompt{In}{incolor}{15}{\boxspacing}
\begin{Verbatim}[commandchars=\\\{\}]
\PY{k}{def} \PY{n+nf}{sum\PYZus{}of\PYZus{}n\PYZus{}3}\PY{p}{(}\PY{n}{n}\PY{p}{)}\PY{p}{:}
    \PY{n}{start} \PY{o}{=} \PY{n}{time}\PY{o}{.}\PY{n}{time}\PY{p}{(}\PY{p}{)}

    \PY{n}{the\PYZus{}sum} \PY{o}{=} \PY{p}{(}\PY{p}{(}\PY{n}{n}\PY{o}{+}\PY{l+m+mi}{1}\PY{p}{)}\PY{o}{*}\PY{n}{n}\PY{p}{)}\PY{o}{/}\PY{l+m+mf}{2.0}

    \PY{n}{end} \PY{o}{=} \PY{n}{time}\PY{o}{.}\PY{n}{time}\PY{p}{(}\PY{p}{)}

    \PY{k}{return} \PY{n}{the\PYZus{}sum}\PY{p}{,} \PY{n}{end} \PY{o}{\PYZhy{}} \PY{n}{start}

\PY{k}{for} \PY{n}{i} \PY{o+ow}{in} \PY{n+nb}{range}\PY{p}{(}\PY{l+m+mi}{5}\PY{p}{)}\PY{p}{:}
    \PY{n+nb}{print}\PY{p}{(}\PY{l+s+s2}{\PYZdq{}}\PY{l+s+s2}{V2: Sum is }\PY{l+s+si}{\PYZpc{}d}\PY{l+s+s2}{ required }\PY{l+s+si}{\PYZpc{}10.6f}\PY{l+s+s2}{ seconds}\PY{l+s+s2}{\PYZdq{}} \PY{o}{\PYZpc{}} \PY{n}{sum\PYZus{}of\PYZus{}n\PYZus{}3}\PY{p}{(}\PY{l+m+mi}{10000}\PY{p}{)}\PY{p}{)}
\end{Verbatim}
\end{tcolorbox}

    \begin{Verbatim}[commandchars=\\\{\}]
V2: Sum is 50005000 required   0.000001 seconds
V2: Sum is 50005000 required   0.000001 seconds
V2: Sum is 50005000 required   0.000000 seconds
V2: Sum is 50005000 required   0.000001 seconds
V2: Sum is 50005000 required   0.000001 seconds
    \end{Verbatim}

    \hypertarget{moving-beyond-experimental-analysis}{%
\subsection{Moving Beyond Experimental
Analysis}\label{moving-beyond-experimental-analysis}}

\begin{itemize}
\item
  Our goal is to develop an approach to analyzing the efficiency of
  algorithms that:

  \begin{enumerate}
  \def\labelenumi{\arabic{enumi}.}
  \tightlist
  \item
    Allows us to evaluate the relative efficiency of any two algorithms
    in a way that is independent of the hardware and software
    environment.
  \item
    Is performed by studying a high-level description of the algorithm
    without need for implementation.
  \item
    Takes into account all possible inputs.
  \end{enumerate}
\item ~
  \hypertarget{counting-primitive-operations}{%
  \subsubsection{Counting Primitive
  Operations}\label{counting-primitive-operations}}
\item ~
  \hypertarget{measuring-operations-as-a-function-of-input-size}{%
  \subsubsection{Measuring Operations as a Function of Input
  Size}\label{measuring-operations-as-a-function-of-input-size}}
\item ~
  \hypertarget{focusing-on-the-worst-case-input}{%
  \subsubsection{Focusing on the Worst-Case
  Input}\label{focusing-on-the-worst-case-input}}
\end{itemize}

    \hypertarget{counting-primitive-operations}{%
\subsubsection{Counting Primitive
Operations}\label{counting-primitive-operations}}

\begin{itemize}
\tightlist
\item
  Perform analysis directly on a high-level description of the algorithm
\item
  Identify and analyse primitive operations - \textbf{a primitive
  operation is low-level instruction with an execution time that is
  constant} e.g.,

  \begin{itemize}
  \tightlist
  \item
    Assigning an identifier to an object
  \item
    Determining the object associated with an identifier
  \item
    Performing an arithmetic operation (for example, adding two numbers)
    • Comparing two numbers
  \item
    Accessing a single element of a Python list by index
  \item
    Calling a function (excluding operations executed within the
    function) • Returning from a function.
  \end{itemize}
\item
  A primitive operation is basic operation that is executed by the
  hardware
\item
  Many of our primitive operations may be translated to a small number
  of instructions.
\item
  Instead of trying to determine the specific execution time of each
  primitive operation, we will count

  \begin{itemize}
  \tightlist
  \item
    how many primitive operations are executed
  \item
    use this number \(t\) as a measure of the running time of the
    algorithm.
  \end{itemize}
\end{itemize}

    \hypertarget{measuring-operations-as-a-function-of-input-size}{%
\subsubsection{Measuring Operations as a Function of Input
Size}\label{measuring-operations-as-a-function-of-input-size}}

\begin{itemize}
\tightlist
\item
  Capture the order of growth of an algorithm's running time
\item
  A function \(f(n)\) that characterizes the number of primitive
  operations that are performed as a function of the input size n
\end{itemize}

    \hypertarget{focusing-on-the-worst-case-input}{%
\subsubsection{Focusing on the Worst-Case
Input}\label{focusing-on-the-worst-case-input}}

\begin{itemize}
\item
  \textbf{Worst-case:} \(T(n)\) The maximum steps/time/memory taken by
  an algorithm to complete the solution for inputs of size \(n\)
\item
  \textbf{Average-case:} The expected time over all inputs with the same
  characteristics (e.g.~size).
\item
  \textbf{Best-case:} The performance in ideal situations. Rarely used.
\end{itemize}

Worst-case analysis is much easier than average-case analysis, as it
requires only the ability to identify the worst-case input, which is
often simple.

When analysing algorithmic performance, we ignore computational
resources (e.g. CPU speed); we only care about the growth of \(T(n)\) as
\(n \rightarrow \infty\)

This is called the \textbf{asymptotic behaviour}

    \begin{figure}
\centering
\includegraphics{fig3.2.png}
\caption{Runtime Analysis}
\end{figure}

    \hypertarget{growth-function}{%
\subsection{Growth Function}\label{growth-function}}

\begin{enumerate}
\def\labelenumi{\arabic{enumi}.}
\tightlist
\item
  \(f(x) = n\) (constant function)
\end{enumerate}

\begin{itemize}
\tightlist
\item
  for any argument \(n\), the constant function \(f(n)\) assigns the
  value \(c\).
\end{itemize}

\begin{enumerate}
\def\labelenumi{\arabic{enumi}.}
\setcounter{enumi}{1}
\tightlist
\item
  \(f(n) = log_b(n)\) (logarithmic function)
\end{enumerate}

\begin{itemize}
\tightlist
\item
  \(x=\log_bn\) if and only if \(b^x = n\).
\end{itemize}

\begin{enumerate}
\def\labelenumi{\arabic{enumi}.}
\setcounter{enumi}{2}
\tightlist
\item
  \(f(x) = sqrt(x)\) (square root function)
\item
  \(f(x) = x\) (linear function)
\item
  \(f(x) = nlog(n)\) (log-linear function)
\item
  \(f(x) = x^2\) (quadratic function)
\item
  \(f(x) = x^3\) (cubic function)
\item
  \(f(x) = a^x, \forall a > 1\) (exponential function)
\end{enumerate}

    \hypertarget{math-review}{%
\subsection{Math Review}\label{math-review}}

    \hypertarget{exponentiation}{%
\subsubsection{Exponentiation}\label{exponentiation}}

Exponents have a number of important properties, including:

\begin{align}
        b^0 &= 1 \\
        b^1 &= b \\
        b^{1/2} &= b^{0.5} = \sqrt{b} \\
        b^{-1} &= \frac{1}{b} \\
        b^a \cdot b^c &= b^{a + c} \\
        (b^a)^c &= b^{ac}
    \end{align}

    \hypertarget{logarithms}{%
\subsubsection{Logarithms}\label{logarithms}}

\textbf{Definition}: \(\log_b a = c\) if and only if \(a = b^c\). If
\(b = 2\), we write \(\lg a\).

There are a number of log identities you should be aware of:

\begin{align}
        \log_b (a \cdot c) &= \log_b a + \log_b c \\
        \log \left( \frac{a}{c} \right) &= \log_b a - \log_b c \\
        \log(a^c) &= c \cdot \log_b a \\
        b^{\log_c a} &= a^{\log_c b} \\
        \log_b a &= \frac{\log_c a}{\log_c b}
    \end{align}

    \hypertarget{recursive-definitions}{%
\subsubsection{Recursive Definitions}\label{recursive-definitions}}

\#\#\#\#\# \textbf{Factorial} - The \textbf{factorial} of a number \(n\)
is represented by \(n!\). Informally,
\(n! = 1 \cdot 2 \cdot 3 \cdot \ldots \cdot n\). More formally:

\begin{align*}
            n! = \begin{cases}
                1 & n = 0 \\
                n \cdot (n - 1)! & n > 0
            \end{cases}
\end{align*}

    \hypertarget{recursive-definitions}{%
\subsubsection{Recursive Definitions}\label{recursive-definitions}}

\hypertarget{fibonacci-numbers}{%
\subparagraph{Fibonacci numbers}\label{fibonacci-numbers}}

The \textbf{fibonacci numbers} are defined by: \begin{align*}
            F_i = \begin{cases}
                0 & i = 0 \\
                1 & i = 1 \\
                F_{i - 2} + F_{i - 1} & i > 1
            \end{cases}
        \end{align*}

    \hypertarget{summations}{%
\subsubsection{Summations}\label{summations}}

\textbf{Arithmetic Series} \begin{align*}
        \sum_{i = 1}^{n} i &= 1+2+3+ \ldots + n = \frac{n(n+1)}{2} \\
        \sum_{i = 1}^{n} i^2 &= \frac{n(n+1)(2n+1)}{6} \\
        \sum_{i = 1}^{n} i^k &\approx \frac{n^{k + 1}}{k + 1} = \Theta(n^{k + 1})
    \end{align*}

    \hypertarget{summations}{%
\subsubsection{Summations}\label{summations}}

\begin{itemize}
\tightlist
\item
  \textbf{Geometric series}
\end{itemize}

\begin{align*}
        \sum_{i = 0}^{n} a^i &= 1 + a + a^2+a^3+ ... +a^n= \frac{a^{n + 1} - 1}{a - 1} \\
        \sum_{i = 0}^{\infty} a^i &= \frac{1}{1 - a} \text{ where } a < 1
    \end{align*}

    \hypertarget{polynomials}{%
\subsubsection{Polynomials}\label{polynomials}}

A polynomial function has the form

\begin{align*}
f(n) = a_0 +a_1n+a_2n^2 +a_3n^3 +···+a_dn^d 
\end{align*}

where \(a_0\) , \(a_1\) , . . . , \(a_d\) are constants, called the
\textbf{coefficients} of the polynomial, and \(a^d\neq0\).

Integer \(d\), which indicates the highest power in the polynomial, is
called the \textbf{degree} of the polynomial.

    \hypertarget{comparing-growth-rates}{%
\subsubsection{Comparing Growth Rates}\label{comparing-growth-rates}}

    \begin{tcolorbox}[breakable, size=fbox, boxrule=1pt, pad at break*=1mm,colback=cellbackground, colframe=cellborder]
\prompt{In}{incolor}{16}{\boxspacing}
\begin{Verbatim}[commandchars=\\\{\}]
\PY{c+c1}{\PYZsh{} remember to evaluate this cell first!}
\PY{k+kn}{import} \PY{n+nn}{matplotlib}\PY{n+nn}{.}\PY{n+nn}{pylab} \PY{k}{as} \PY{n+nn}{plt}
\PY{k+kn}{import} \PY{n+nn}{numpy} \PY{k}{as} \PY{n+nn}{np}
\PY{k+kn}{import} \PY{n+nn}{math}
\PY{k+kn}{import} \PY{n+nn}{timeit}
\PY{k+kn}{import} \PY{n+nn}{random}
\PY{o}{\PYZpc{}}\PY{k}{matplotlib} inline
\end{Verbatim}
\end{tcolorbox}

    \begin{tcolorbox}[breakable, size=fbox, boxrule=1pt, pad at break*=1mm,colback=cellbackground, colframe=cellborder]
\prompt{In}{incolor}{17}{\boxspacing}
\begin{Verbatim}[commandchars=\\\{\}]
\PY{n}{count} \PY{o}{=} \PY{l+m+mi}{50}
\PY{n}{xs} \PY{o}{=} \PY{n}{np}\PY{o}{.}\PY{n}{linspace}\PY{p}{(}\PY{l+m+mf}{0.1}\PY{p}{,} \PY{l+m+mi}{2}\PY{p}{,} \PY{n}{count}\PY{p}{)}
\PY{n}{ys\PYZus{}const}        \PY{o}{=} \PY{p}{[}\PY{l+m+mi}{1}\PY{p}{]} \PY{o}{*} \PY{n}{count}
\PY{n}{ys\PYZus{}log}          \PY{o}{=} \PY{p}{[}\PY{n}{math}\PY{o}{.}\PY{n}{log}\PY{p}{(}\PY{n}{x}\PY{p}{)} \PY{k}{for} \PY{n}{x} \PY{o+ow}{in} \PY{n}{xs}\PY{p}{]}
\PY{n}{ys\PYZus{}linear}       \PY{o}{=} \PY{p}{[}\PY{n}{x} \PY{k}{for} \PY{n}{x} \PY{o+ow}{in} \PY{n}{xs}\PY{p}{]}
\PY{n}{ys\PYZus{}linearithmic} \PY{o}{=} \PY{p}{[}\PY{n}{x} \PY{o}{*} \PY{n}{math}\PY{o}{.}\PY{n}{log}\PY{p}{(}\PY{n}{x}\PY{p}{)} \PY{k}{for} \PY{n}{x} \PY{o+ow}{in} \PY{n}{xs}\PY{p}{]}
\PY{n}{ys\PYZus{}quadratic}    \PY{o}{=} \PY{p}{[}\PY{n}{x}\PY{o}{*}\PY{o}{*}\PY{l+m+mi}{2} \PY{k}{for} \PY{n}{x} \PY{o+ow}{in} \PY{n}{xs}\PY{p}{]}
\PY{n}{ys\PYZus{}cubic}    \PY{o}{=} \PY{p}{[}\PY{n}{x}\PY{o}{*}\PY{o}{*}\PY{l+m+mi}{3} \PY{k}{for} \PY{n}{x} \PY{o+ow}{in} \PY{n}{xs}\PY{p}{]}
\PY{n}{ys\PYZus{}exp}    \PY{o}{=} \PY{p}{[}\PY{l+m+mi}{3}\PY{o}{*}\PY{o}{*}\PY{n}{x} \PY{k}{for} \PY{n}{x} \PY{o+ow}{in} \PY{n}{xs}\PY{p}{]}



\PY{n}{plt}\PY{o}{.}\PY{n}{plot}\PY{p}{(}\PY{n}{xs}\PY{p}{,} \PY{n}{ys\PYZus{}const}\PY{p}{,} \PY{l+s+s1}{\PYZsq{}}\PY{l+s+s1}{c}\PY{l+s+s1}{\PYZsq{}}\PY{p}{,} \PY{n}{label}\PY{o}{=}\PY{l+s+s1}{\PYZsq{}}\PY{l+s+s1}{Constant}\PY{l+s+s1}{\PYZsq{}}\PY{p}{)}
\PY{n}{plt}\PY{o}{.}\PY{n}{plot}\PY{p}{(}\PY{n}{xs}\PY{p}{,} \PY{n}{ys\PYZus{}log}\PY{p}{,} \PY{l+s+s1}{\PYZsq{}}\PY{l+s+s1}{r}\PY{l+s+s1}{\PYZsq{}}\PY{p}{,} \PY{n}{label} \PY{o}{=} \PY{l+s+s1}{\PYZsq{}}\PY{l+s+s1}{Logarithmic}\PY{l+s+s1}{\PYZsq{}}\PY{p}{)}
\PY{n}{plt}\PY{o}{.}\PY{n}{plot}\PY{p}{(}\PY{n}{xs}\PY{p}{,} \PY{n}{ys\PYZus{}linear}\PY{p}{,} \PY{l+s+s1}{\PYZsq{}}\PY{l+s+s1}{b}\PY{l+s+s1}{\PYZsq{}}\PY{p}{,} \PY{n}{label}\PY{o}{=}\PY{l+s+s1}{\PYZsq{}}\PY{l+s+s1}{Linear}\PY{l+s+s1}{\PYZsq{}}\PY{p}{)}
\PY{n}{plt}\PY{o}{.}\PY{n}{plot}\PY{p}{(}\PY{n}{xs}\PY{p}{,} \PY{n}{ys\PYZus{}linearithmic}\PY{p}{,} \PY{l+s+s1}{\PYZsq{}}\PY{l+s+s1}{g}\PY{l+s+s1}{\PYZsq{}}\PY{p}{,} \PY{n}{label}\PY{o}{=}\PY{l+s+s1}{\PYZsq{}}\PY{l+s+s1}{Linear Arthmatic}\PY{l+s+s1}{\PYZsq{}}\PY{p}{)}
\PY{n}{plt}\PY{o}{.}\PY{n}{plot}\PY{p}{(}\PY{n}{xs}\PY{p}{,} \PY{n}{ys\PYZus{}quadratic}\PY{p}{,} \PY{l+s+s1}{\PYZsq{}}\PY{l+s+s1}{y}\PY{l+s+s1}{\PYZsq{}}\PY{p}{,} \PY{n}{label}\PY{o}{=}\PY{l+s+s1}{\PYZsq{}}\PY{l+s+s1}{Quadratic}\PY{l+s+s1}{\PYZsq{}}\PY{p}{)}\PY{p}{;}
\PY{n}{plt}\PY{o}{.}\PY{n}{plot}\PY{p}{(}\PY{n}{xs}\PY{p}{,} \PY{n}{ys\PYZus{}cubic}\PY{p}{,} \PY{l+s+s1}{\PYZsq{}}\PY{l+s+s1}{o}\PY{l+s+s1}{\PYZsq{}}\PY{p}{,} \PY{n}{label}\PY{o}{=}\PY{l+s+s1}{\PYZsq{}}\PY{l+s+s1}{Cubic}\PY{l+s+s1}{\PYZsq{}}\PY{p}{)}\PY{p}{;}
\PY{n}{plt}\PY{o}{.}\PY{n}{plot}\PY{p}{(}\PY{n}{xs}\PY{p}{,} \PY{n}{ys\PYZus{}exp}\PY{p}{,} \PY{l+s+s1}{\PYZsq{}}\PY{l+s+s1}{\PYZhy{}p}\PY{l+s+s1}{\PYZsq{}}\PY{p}{,} \PY{n}{label}\PY{o}{=}\PY{l+s+s1}{\PYZsq{}}\PY{l+s+s1}{Exponential}\PY{l+s+s1}{\PYZsq{}}\PY{p}{)}\PY{p}{;}
\PY{n}{plt}\PY{o}{.}\PY{n}{legend}\PY{p}{(}\PY{p}{)}
\end{Verbatim}
\end{tcolorbox}

            \begin{tcolorbox}[breakable, size=fbox, boxrule=.5pt, pad at break*=1mm, opacityfill=0]
\prompt{Out}{outcolor}{17}{\boxspacing}
\begin{Verbatim}[commandchars=\\\{\}]
<matplotlib.legend.Legend at 0x10fe110d0>
\end{Verbatim}
\end{tcolorbox}
        
    \begin{center}
    \adjustimage{max size={0.9\linewidth}{0.9\paperheight}}{complexity_analysis_files/complexity_analysis_23_1.png}
    \end{center}
    { \hspace*{\fill} \\}
    
    \begin{longtable}[]{@{}lllllll@{}}
\toprule
constant & logarithm & linear & n-log-n & quadratic & cubic &
exponential \\
\midrule
\endhead
1 & \(log(n)\) & \(n\) & \(nlogn\) & \(n^2\) & \(n^3\) & \(a^n\) \\
\bottomrule
\end{longtable}

    \hypertarget{the-ceiling-and-floor-functions}{%
\subparagraph{The Ceiling and Floor
Functions}\label{the-ceiling-and-floor-functions}}

\begin{itemize}
\tightlist
\item
  \(\lceil x \rceil =\) the largest integer less than or equal to \(x\).
\item
  \(\lfloor x \rfloor =\) the smallest integer greater than or equal to
  \(x\).
\end{itemize}

    \hypertarget{asymptotic-analysis}{%
\section{3.3 Asymptotic Analysis}\label{asymptotic-analysis}}

\begin{itemize}
\tightlist
\item
  In algorithm analysis, we focus on the growth rate of the running time
  as a function of the input size \(n\), taking a ``big-picture''
  approach.
\item
  For example, it is often enough just to know that the running time of
  an algorithm \textbf{grows proportionally} to \(n\).
\end{itemize}

    \textbf{A function that returns the maximum value of a Python list.}

\begin{Shaded}
\begin{Highlighting}[]
\KeywordTok{def}\NormalTok{ find }\BuiltInTok{max}\NormalTok{(data):}
\NormalTok{”””Return the maximum element }\ImportTok{from}\NormalTok{ a nonempty Python }\BuiltInTok{list}\NormalTok{.””” }
\NormalTok{biggest }\OperatorTok{=}\NormalTok{ data[}\DecValTok{0}\NormalTok{]                       }\CommentTok{\# The initial value to beat}
\ControlFlowTok{for}\NormalTok{ val }\KeywordTok{in}\NormalTok{ data:                        }\CommentTok{\# For each value:}
 \ControlFlowTok{if}\NormalTok{ val }\OperatorTok{\textgreater{}}\NormalTok{ biggest                       }\CommentTok{\# if it is greater than the best so far,}
\NormalTok{        biggest }\OperatorTok{=}\NormalTok{ val                  }\CommentTok{\# we have found a new best (so far)         }
 \ControlFlowTok{return}\NormalTok{ biggest                        }\CommentTok{\# When loop ends, biggest is the max}
\end{Highlighting}
\end{Shaded}

    This is a classic example of an algorithm with a running time that grows
\textbf{proportional to \(n\)}, as the loop executes once for each data
element, with some fixed number of primitive operations executing for
each pass.

    \hypertarget{the-big-oh-notation}{%
\subsection{3.3.1 The ``Big-Oh'' Notation}\label{the-big-oh-notation}}

\begin{itemize}
\tightlist
\item
  Let \(f(n)\) and \(g(n)\) be functions mapping positive integers to
  positive real numbers.
\item
  We say that \(f(n)\) is \(O(g(n))\)
\item
  if there is a real constant \(c > 0\) and
\item
  an integer constant \(n_0 \ge 1\) such that \[
  f(n)\le cg(n) 
  \] for \(n\ge n_0\).
\end{itemize}

    \hypertarget{example-the-function-8n-5-is-on.}{%
\paragraph{\texorpdfstring{Example : The function \(8n + 5\) is
\(O(n)\).}{Example : The function 8n + 5 is O(n).}}\label{example-the-function-8n-5-is-on.}}

\textbf{Justification}: - By the big-Oh definition, we need to find a
real constant \(c > 0\) and an integer constant \(n0\ge 1\) such that
\(8n+5 \le cn\) for every integer \(n\ge n_0\). - It is easy to see that
a possible choice is \(c = 9\) and \(n_0 = 5\). - Indeed, this is one of
infinitely many choices available because there is a trade-off between
\(c\) and \(n_0\). - For example, we could rely on constants \(c = 13\)
and \(n_0 = 1\).

    \hypertarget{proposition-the-algorithm-find-max-for-computing-the-maximum-element-of-a-list-of-n-numbers-runs-in-on-time.}{%
\subparagraph{Proposition: The algorithm, find max, for computing the
maximum element of a list of n numbers, runs in O(n)
time.}\label{proposition-the-algorithm-find-max-for-computing-the-maximum-element-of-a-list-of-n-numbers-runs-in-on-time.}}

\textbf{Justification}

\begin{itemize}
\tightlist
\item
  The initialization before the loop begins requires only a constant
  number of primitive operations.
\item
  Each iteration of the loop also requires only a con- stant number of
  primitive operations.
\item
  The loop executes \(n\) times.
\item
  The number of primitive operations being \(c' + c".n\) for appropriate
  constants \(c'\) and \(c"\)
\end{itemize}

    \hypertarget{example-5n4-3n3-2n2-4n1-is-on4.}{%
\paragraph{\texorpdfstring{Example: \(5n^4 +3n^3 +2n^2 +4n+1\) is
\(O(n4)\).}{Example: 5n\^{}4 +3n\^{}3 +2n\^{}2 +4n+1 is O(n4).}}\label{example-5n4-3n3-2n2-4n1-is-on4.}}

\textbf{Justification}: Note that
\(5n^4+3n^3+2n^2+4n+1~~\le(5+3+2+4+1)n^4 =cn^4\), for \(c=15\),when
\(n\ge n_0 =1\).

Thus, the highest-degree term in a polynomial is the term that
determines the asymptotic growth rate of that polynomial

    \textbf{Example}: \(5n^2 +3n log n+2n+5\) is \(O(n^2)\).

\textbf{Example}: \(20n^3 +10nlogn+5\) is \(O(n^3)\).

\textbf{Example}: \(2^n+2\) is \(O(2^n)\).

    \hypertarget{big-omega-notation}{%
\subsubsection{3.3.2 Big-Omega Notation}\label{big-omega-notation}}

\begin{itemize}
\tightlist
\item
  Let \(f(n)\) and \(g(n)\) be functions mapping positive integers to
  positive real numbers.
\item
  We say that \(f(n)\) is \(\Omega(g(n))\)
\item
  if there is a real constant \(c > 0\) and
\item
  an integer constant \(n_0 \ge 1\) such that \[
  f(n)\ge cg(n) 
  \] for \(n\ge n_0\).
\end{itemize}

    \hypertarget{example-3n-log-n-2n-is-omegan-log-n.}{%
\paragraph{\texorpdfstring{Example: \(3n log n − 2n\) is
\(\Omega(n log n)\).}{Example: 3n log n − 2n is \textbackslash Omega(n log n).}}\label{example-3n-log-n-2n-is-omegan-log-n.}}

\textbf{Justification}: \(3nlogn−2n = nlogn+2n(logn−1) \ge n log n\) for
\(n \ge 2\); hence, we can take \(c=1\) and \(n_0 =2\) in this case.

    \hypertarget{big-theta-notation}{%
\subsubsection{3.3.3 Big-Theta Notation}\label{big-theta-notation}}

\begin{itemize}
\tightlist
\item
  Let \(f(n)\) and \(g(n)\) be functions mapping positive integers to
  positive real numbers.
\item
  We say that \(f(n)\) is \(\Theta(g(n))\) if \(f(n)\) is \(O(g(n))\)
  and \(f(n)\) is \(\Omega(g(n))\)
\item
  if there is two real constant \(c_1 > 0\) and \(c_2 > 0\) and
\item
  an integer constant \(n_0 \ge 1\) such that \[
  c_1g(n) \le f(n)\ge c_2g(n) 
  \] for \(n\ge n_0\).
\end{itemize}

    \hypertarget{example-3nlogn4n5logn-is-thetanlogn.}{%
\paragraph{\texorpdfstring{Example: \(3nlogn+4n+5logn\) is
\(\Theta(nlogn)\).}{Example: 3nlogn+4n+5logn is \textbackslash Theta(nlogn).}}\label{example-3nlogn4n5logn-is-thetanlogn.}}

\textbf{Justification}: \(3nlogn \le 3nlogn+4n+5logn \le (3+4+5)nlogn\)
for \(n \ge 2\)

    \begin{tcolorbox}[breakable, size=fbox, boxrule=1pt, pad at break*=1mm,colback=cellbackground, colframe=cellborder]
\prompt{In}{incolor}{ }{\boxspacing}
\begin{Verbatim}[commandchars=\\\{\}]

\end{Verbatim}
\end{tcolorbox}


    % Add a bibliography block to the postdoc
    
    
    
\end{document}
